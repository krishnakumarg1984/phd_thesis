% -*- root: ../main.tex -*-
%!TEX root = ../main.tex
% this file is called up by main.tex
% content in this file will be fed into the main document
% vim:textwidth=80 fo=cqt


An in-depth  coverage of the topic  of system identification is  well beyond the
scope of  this thesis.  However, keeping  in mind the  interests of  the battery
modelling community  who might not be  familiar with this subject  area, a brief
overview of the core ideas that  are essential for tackling the specific problem
at hand is  presented. For readers further interested in  this topic, the author
suggests the textbook by  Ljung~\cite{Ljung1999} for a comprehensive theoretical
treatment of the foundation topics in system identification.

System identification aims  to provide a mathematical model  of the input-output
mapping  of the  system\footnote{The precise  definition of  what constitutes  a
`system' is detailed  in Ljung's textbook. However, for  all practical purposes,
in this  thesis the word `system'  stands for any unknown  entity whose terminal
behavioural model  is being  sought for ---  primarily from  input-output data.}
under consideration. The three categories of system identification are:


\begin{enumdescriptnum}[leftmargin=!,itemsep=1ex,labelwidth=\widthof{$\symbf{\text{brugg}_j}\ \scriptstyle (\times 3)$abc}
    ,partopsep=0pt
    ,topsep=0pt
    ]

\item[White  box] wherein  underlying physical  equations are  completely known.
The  numerical value  of  coefficients of  governing equations  are  then to  be
parametrised from input-output data.

\item[Black box]  wherein no  governing equations are  available for  the system
under  consideration. The  model formulation  is facilitated  by a  rich set  of
system theory  which proceeds by exciting  the system with input  waveforms with
certain desirable  properties and correlating characteristics  from the response
to draw  conclusions about viable  mathematical structures capable  of emulating
the terminal behaviour of the system  under generalised inputs. Black box system
identification was employed for the specific problem under consideration in this
thesis and hence all future descriptions will pertain to this class.

\item[Grey box] is a hybrid of the  two approaches wherein a part of the model's
governing  physics is  known  a priori  \eg{} the  structure  of a  well-defined
subsystem  that is  part  of a  large,  complex  system may  be  known ahead  of
time,  where the  task  is  to characterise  the  full  system. Grey box  system
identification tasks can  often be reduced to a single  sub-problem of black box
system  identification  by  removal  of  the known  physics  and  tackling  them
separately.

\end{enumdescriptnum}

