% -*- root: ../main.tex -*-
%!TEX root = ../main.tex
% this file is called up by main.tex
% content in this file will be fed into the main document
% vim:textwidth=80 fo=cqt

\clearpage
\chapter{Implementing a New Electrolyte Model for the \glsfmtshort{spm}}\label{ch:newelectrolytemodel}
\startcontents[chapters]
\printcontents[chapters]{}{1}{\setcounter{tocdepth}{1}}

% change according to folder and file names
\graphicspath{{6/figures/}}

% A suitable  family of models from  the broad category of  reduced-order models
% is  identified  as  a  promising  candidate  for  implementation  in  controls
% applications.
\bigskip

This   chapter  presents   the   author's  attempts   towards   arriving  at   a
accurate  description  of  the  spatio-temporal  evolution  of  the  electrolyte
concentration. The  performance of the quadratic  approximation model introduced
in~\cref{ch:spmanalysis} is  thoroughly analysed  in the  context of  a symbolic
regression framework that  helped to expose the issue of  equation deficiency in
the underlying \gls{dfn} model. Although this framework did not ultimately yield
the  desired  outcomes,  it  did  facilitate a  comprehensive  analysis  of  the
strengths and weaknesses of the quadratic  approximation model which is not seen
in present literature.  Finally, the author's unique contribution to  the art of
single particle  modelling \viz{}  a novel  time-evolution model  of electrolyte
concentrations through  system identification is  presented. The results  of the
new approach  is compared  against the  baseline quadratic  approximation model.
Finally,  the novel  electrolyte time-evolution  model is  incorporated into  an
electrolyte-enhanced \gls{spm}  and the  performance of  the composite  model is
compared against the conventional \gls{spm},

\section{Performance Analysis: Quadratic Approximation Model}\label{sec:symbolicregression}

% -*- root: ../main.tex -*-
%!TEX root = ../main.tex
% this file is called up by main.tex
% content in this file will be fed into the main document
% vim:textwidth=80 fo=cqt


In the  author's analysis of the  quadratic approximation model, the  origin and
nature of  its sub-optimal  performance can  be explained  as per  the following
rationale. The quadratic  approximation model uses a  bottom-up approach wherein
the final  simplified model structure  is pre-assumed  and then the  physics are
made  to  fit  within  this  framework. Given  that  the  top-down  approach  to
electrolyte  modelling, \ie{}  accounting for  all physical  phenomena and  then
simplifying  them yields  mathematically intractable  and overly  complex models
(see~\cref{sec:electrolyteinclusion} for  a detailed discussion),  this approach
seems to be a pragmatic alternative  to enhancing the \gls{spm} with electrolyte
dynamics.


\subsection{Symbolic regression using \glsfmtlong{mggp}}

The  question  boils down  to  whether  quadratic  approximation is  indeed  the
\emph{best} model structure that can be assumed a priori. The author embarked on
a journey  to find suitable  alternate model  structures, \ie{} a  single family
of  curves  that  will  capture  both  the  transient  and  \gls{qss}  behaviour
exhibited by  the ionic  concentration. The open-source  \textsc{MATLAB} toolbox
GPTIPS2~\cite{Searson2015}  uses the  state-of-the-art  \gls{mggp} approach  for
symbolic data  mining and is ideally  suited for such symbolic  regression tasks
(fitting a  mathematical equation structure,  and not merely  obtaining best-fit
coefficients to a  pre-assumed curve as in classical numerical  regression, to a
given data).

It  is  important  to  recognize  that  the  key  criteria  that  restricts  the
choice  of gene  sequence  depth as  well  as  the choice  in  number of  parent
mutations  is  the  total  number  of  unknown  symbolic  coefficients  required
to  be   solved  in  the  assumed   model  structure.  There  are   a  total  of
\emph{seven} linear equations available from  the physics of the \gls{dfn} model
(see~\crefrange{eq:cecontinuitynegsep}{eq:Qepbyintegration}).  Thus the  assumed
family  of curves  \emph{cannot}  consist  of more  than  seven coefficients  to
guarantee a  unique solution. Yet  another restriction  on the choice  of curves
arises due to the fact that the behaviour of ionic concentration in the negative
and positive electrode  regions are similar in complexity, and  hence need to be
mathematically described by an identical family of curves.

Upon a close inspection of the  spatial concentration profile from the \gls{p2d}
simulation  results shown  in~\cref{fig:spatialionicconc1C}, it  is evident  the
electrolyte approximation  functions within the  electrode regions is  of higher
complexity than  the approximation  function suitable for  use in  the separator
region. Based on the  results of the quadratic model, it is  clear that at least
two coefficients  are required within the  electrode regions, $n_{\text{c,elec}}
\ge  2$.  There  exists  an  inhibiting  factor  that  prevents  the  use  of  a
lower  order  function within  the  separator.  As  per the  simulation  results
of  the  \gls{p2d}  model,  the  time-domain  change  of  number  of  moles  per
square  meter  in  the  separator   is  non-zero.  Therefore,  a  simple  linear
approximation is immediately  ruled out as per~\cref{eq:sepliionmolesquadratic}.
Among non-polynomial mathematical curves tried (trigonometric, hyperbolic, power
series  among  others),  none  could  obtain  the  relatively  simple  shape  of
the  separator function,  without  reducing  the contribution  from  one of  the
coefficients  to below  machine  precision.  This forces  the  retention of  the
quadratic approximation  function used thus  far (with no missing  terms), \ie{}
$n_\text{c,sep}  = 3$.  Thus, the  overall number  of coefficients  in the  best
possible approximation shall be $2  n_\text{c,elec} + n_\text{c,sep} = 2\cdot2 +
3 = 7$,  which is the total number of  electrolyte-specific physical constraints
available from the \gls{dfn} model. Thus, it can be concluded that the quadratic
approximation model does indeed make the \emph{best} use of all of the available
physical  equations.  The  final  question  remains to  answer  is,  with  these
coefficient limitations, is the quadratic  equation structure indeed the optimal
one. This was investigated through the \gls{mggp} approach.

The GPTIPS2  toolbox uses a variety  of heuristic algorithms from  the theory of
\gls{mggp} to fit a  suitable equation structure for the data  to be fitted. The
dataset  consisted  of the  simulation  results  from  the \gls{p2d}  model,  in
particular  the  values of  electrolyte  concentration  at the  three  different
cell  regions,  captured  at  various  snapshots of  time.  Both  transient  and
\gls{qss}  output  were  fed  into  this symbolic  data  mining  process  and  a
single all-encompassing  family of curves  capable of capturing  the electrolyte
concentration  behaviour  was  sought  for.  However,  with  the  aforementioned
constraints in the number of coefficients results in restriction of the depth of
gene mutations as  well as the number of unconnected  seed populations. The best
equation  set (without  hard constraints,  yet  minimizing the  distance to  the
constraint vector) that the symbolic regression approach yielded was
\begin{alignat}{2}
    c_\ensub(z,t) & = a_2(t) \cosh z^2 + a_1(t) \sinh z + a_0(t) \qquad &  & 0 \le z \le l_\text{n}   \\
    c_\essub(z,t) & = a_5(t) z^2 + a_4(t) z + a_3(t) \qquad &  & 0 \le z \le l_\text{s}   \\
    c_\epsub(z,t) & = a_8(t) \cosh z^2 + a_7(t) \sinh z + a_6(t) \qquad &  & 0 \le z \le l_\text{p}
\end{alignat}
which  although  fits  the  transient  and  \gls{qss}  profiles  well,  violates
the  constraint on  the  number of  coefficients available,  and  results in  an
underdetermined  system  of equations.  Both  the  least-squares and  least-norm
solution of this  system were tried. However, the results  were inferior to that
produced by the baseline quadratic approximation method.

An  attempt  was  made  to  obtain different  mathematical  structures  for  the
transient  phase  and  \gls{qss}  phase  which  respects  the  overall  equation
constraint.  The  symbolic  regression  output   for  this  approach  are  shown
in~\cref{tbl:symbreg}.
% -*- root: ../../main.tex -*-
%!TEX root = ../../main.tex
% this file is called up by main.tex
% content in this file will be fed into the main document
% vim:nospell textwidth=180 foldlevelstart=3 foldlevel=3 conceallevel=0

\begin{table}[!htbp]
    \centering
    \caption[Transient \& \glsfmtshort{qss} expressions for electrolyte
    concentration obtained by \glsfmtshort{mggp}]{Best fit expressions for the
        transient and \glsfmtlong{qss} approximaton functions for the
    electrolyte functions obtained by the \gls{mggp} approach.}
    \label{tbl:symbreg}
    \begingroup
    \addtolength{\jot}{0.25em}
    \begin{tabular}{@{} c c r @{}}
        \toprule
        \multicolumn{1}{l}{Transient Function} & \multicolumn{1}{c}{\glsfmtlong{qss} Function} & \multicolumn{1}{c}{Region} \\
        \midrule
        $\begin{aligned}
            c_{\text{e,n}_\text{trans}} &= a_1 z^6 \ln z^6 + a_0 \\
            c_{\text{e,s}_\text{trans}} &= a_4 z^2 + a_3 z + a_2 \\
            c_{\text{e,p}_\text{trans}} &= a_6 z^6 \ln z^6 + a_5 \\
        \end{aligned}$ &
        $\begin{aligned}
            c_{\text{e,n}_\text{QSS}} &= a_1 \sinh z^2 + a_0 \\
            c_{\text{e,s}_\text{QSS}} &= a_4 z^2 + a_3 z + a_2 \\
            c_{\text{e,p}_\text{QSS}} &= a_6 \sinh z^2 + a_5
        \end{aligned}$ &
        $\begin{aligned}
            &0 \le z \le l_\text{n} \\
            &0 \le z \le l_\text{s} \\
            &0 \le z \le l_\text{p}
        \end{aligned}$
        \\
        \bottomrule
    \end{tabular}
    \endgroup
\end{table}



Although  the equations  from~\cref{tbl:symbreg}  produced  a markedly  improved
response during  transient phase, the  performance in  the \gls{qss} was  at par
with respect to the baseline quadratic  approximation model. This brought up the
prospect of a blended approach, wherein a model changeover between the transient
and \gls{qss}  was contemplated. However,  since there is no  precise definition
of  what  constitutes  the  transient  and  dynamic  phase  of  the  electrolyte
concentration  response,  this  approach  required ad-hoc  timing  criteria  for
transition between the two \gls{mggp} equation sets. Further complications arise
during dynamic input  conditions, wherein the concentration  profiles are mostly
in a  state of  flux and  could linger  for longer  durations at  the continuous
boundary between  transient-like and \gls{qss}-like behaviours.  In the interest
of reproducibility across different cell-chemistries and corresponding parameter
sets, the proposed blended model transition approach was not further pursued.

Overall,  the  long  and  arduous  process of  symbolic  regression  was  not  a
definitive  success in  this  case  mainly due  to  the  limitation of  equation
deficiency of physical constraints. Perhaps  if yet another physics-based model,
\ie{} an  alternative to the  widely prevalent \gls{dfn}  model, can be  used as
the  baseline,  a  few  more  physical governing  equations  could  possibly  be
made  available. This  can perhaps  result in  a less  restrictive gene-set  for
coefficient  determination  and  consequently  pave the  way  for  a  successful
implementation  through  this  hitherto  unexplored  route  of  \gls{mggp}-based
equation synthesis.

In conclusion,  the quadratic approximation model  for electrolyte concentration
approximation makes the best use of  the available physical equations. Given the
constraints  with  respect to  physical  equations  discussed in  this  section,
it  also  seems   to  be  the  optimum  family  of   curves  for  modelling  the
\emph{spatial}  profile of  ionic concentration.  Notwithstanding these  merits,
the  \emph{temporal}  performance of  the  quadratic  approximation approach  is
sub-optimal  as seen  in~\cref{fig:temporalcequadratic} and  the author  of this
thesis addresses this specific issue next in~\cref{sec:newelectrolytemodel}.



\section{A New Electrolyte Model through System Identification}\label{sec:newelectrolytemodel}
% -*- root: ../main.tex -*-
%!TEX root = ../main.tex
% this file is called up by main.tex
% content in this file will be fed into the main document
% vim:textwidth=80 fo=cqt

Having performed a  comprehensive analysis of the state of  the art in \gls{spm}
modelling with electrolyte  dynamics, this section presents  the author's unique
contribution to the field. Firstly, the scope of the contribution is identified.
The methodology adopted and corresponding results are presented thereafter.

\subsection{Scope and motivation}\label{subsec:scopenewelectrolyte}

This subsection is intended as a capstone summary helping to briefly recount the
discussion so far  and to provide a  context for the author's work  in the wider
realm  of the  \gls{spm}  modelling art.  In  the same  vein  as the  discussion
in~\cref{sec:electrolyteinclusion}, the scope of the proposed enhancement to the
\gls{spm}  concerns  entirely with  improving  the  electrolyte subsystem  since
it  has already  been established  in~\cref{subsec:simresultsbasicspm} that  the
simplified representation  of the solid-phase  subsystem through a  fourth order
polynomial approximation method for diffusion of \ch{Li^0} in the solid particle
is of sufficiently high accuracy.

Inspecting the electrolyte domain, the contribution of electrolyte overpotential
to  terminal  voltage   consists  of
\begin{enumerate*}[label=\emph{\alph*})]
    \item diffusion   overpotential
    \item time-dependent   ohmic  losses   that  originates   from  differential concentration  gradients  (that  is   indirectly  dependent  upon
concentration).
\end{enumerate*}
Once       electrolyte        concentration       at        each       time-step
is         available,~\cref{eq:electrolytepdwithce}          proposed         by
Prada~\etal~\cite{Prada2012}  may  be  used for  the  electrolyte  overpotential
computation.  Hence, the  accurate  determination of  spatio-temporal values  of
electrolyte concentration merits focus.

There exists a  subtle detail in the  use of~\cref{eq:electrolytepdwithce} which
is discussed here upfront before proceeding  ahead to the refined context of the
author's work. The intrinsic conductivity of electrolyte, $\kappa$ is a function
of the ionic concentration  (refer~\cref{subsec:basicspmsimsetup}). If the ionic
concentration at  the corresponding current  collectors are used  for evaluating
$\kappa_\text{neg}$  and  $\kappa_\text{pos}$, this  would  lead  to a  lopsided
computation of the overpotential in electrolyte. Furthermore, under this scheme,
the computation of electrolyte conductivity shall be rendered ambiguous since it
is unclear which  separator interface shall be chosen for  the separator's ionic
concentration. Although this  has not been discussed clearly  in literature, the
author  of this  thesis chose  to use  the mean  concentration within  each cell
region, defined as
\begin{equation}
    \mean{c}_{\text{e},j}(t) = \frac{1}{l_j}\int_0^{l_j} c_{\text{e}_j}(z,t)\, dz = \frac{Q_{\text{e,}j}(t)}{\varepsilon_j l_j}
\end{equation}
though other  measures of central  tendency might  be equally valid.  Hence, the
results of this section have the associated variability in them depending on how
the electrolyte concentration computations are  used in evaluating the intrinsic
conductivity of electrolyte.

As  the  ionic  concentration  has  both  a  direct  and  indirect  contribution
in~\cref{eq:electrolytepdwithce}, its spatio-temporal  computation is a critical
aspect. As discussed  in~\cref{sec:quadraticapprox}, the quadratic approximation
is a widely used spatio-temporal model for electrolyte concentration which makes
the best  use of available physical  constraints. As established in  the results
of~\cref{subsec:quadraticsimresultsanalysis}, while  the spatial  performance of
the quadratic approximation approach is acceptable, its time-domain performance,
particularly at  the crucial  locations of the  current collector  interfaces is
mediocre at best.

The  \emph{scope}  of  the  author's   work  is  to  obtain  suitable  alternate
expressions  for  improving  the   computation  of  \textbf{time  evolution}  of
the  electrolyte  concentration  whilst retaining  the  quadratic  approximation
approach   for   describing  its   spatial   profile.   Such  an   approach   is
motivated   by    the   keen    observation   that   the    baseline   quadratic
approximation  model  has  a  natural  `pause'  in  its  model  description.  To
clarify,~\crefrange{eq:cecontinuitynegsep}{eq:Qepbyintegration}  form a  tightly
coupled  square  system  \ie{}  a  set   of  seven  linear  equations  in  seven
unknowns.   In   this   system,   the   time   evolution   of   $Q_{\text{e,}j}$
are   described  through   a  separate   system  of   first  order   \glspl{ode}
given by~\crefrange{eq:negliionmolesquadratic}{eq:posliionmolesquadratic}.  In a
practical  implementation,  these  \glspl{ode}  are solved  independently  in  a
decoupled  manner \ie{}  by  using  the coefficients  obtained  from the  linear
system of~\crefrange{eq:cecontinuitynegsep}{eq:Qepbyintegration} in the previous
time-step. The  author's hypothesis is that  by taking advantage of  the natural
break  in the  operational  sequence which  involves  two separate  computations
between  two nearly  independent  subsystems,  it must  be  possible to  replace
the  underperforming  time-evolution  \glspl{ode} from  the  baseline  quadratic
approximation with a superior alternate model.

\subsection{Selection of Methodology --- Background and Rationale}\label{subsec:sysidbackground}

% This section presents  the methodology adopted in obtaining  an improved model
% for the rate of evolution of overall  moles per unit area of \ch{Li^+} ions in
% each of the three regions of the cell.

This  section presents  the  background and  thought  process in  systematically
arriving  at  the  choice of  the  methodology  that  was  adopted for  the  new
time-evolution model of the electrolyte concentration dynamics.

Based  upon the  experience  gained  in dealing  with  the literature  presented
in~\cref{sec:electrolyteinclusion}, it is  the author's view that,  owing to the
complex behaviour of electrolyte, a  naive top-down approach \ie{} including all
the physics upfront followed by a  systematic simplification, might only yield a
model that is  mathematically intractable for adoption in  an embedded \gls{bms}
environment.  The baseline  quadratic  approximation method  has  proven that  a
bottom-up approach \ie{} pre-assuming a simplified structure for the final model
and adapting its coefficients to  physical constraints yields a viable candidate
for describing electrolyte dynamics and  for later inclusion in the conventional
\gls{spm}.

Upon  a   closer  examination   of  the  rubrics   of  the   baseline  quadratic
approximation  model, it  comes  to  light that  the  natural `pause'  discussed
towards  the  end  of~\cref{subsec:scopenewelectrolyte}  permeates  to  a  level
more  than  merely  having  to  operate  sequentially  on  two  pseudo-decoupled
subsystems  --- it  goes to  the extent  of rendering  the operating  philosophy
of  fitting  physical  equations  semi-void.   To  clarify  this  statement  and
to  substantiate   the  claim,  while  there   is  no  doubt  that   the  linear
algebraic   equations  of~\crefrange{eq:cecontinuitynegsep}{eq:Qepbyintegration}
do    incorporate    physical    principles   from    the    \gls{dfn}    model,
the     same     does     not      hold     true     for     the     \glspl{ode}
of~\crefrange{eq:negliionmolesquadratic}{eq:posliionmolesquadratic}.   In  fact,
all   the   boundary   conditions   from   the   \gls{dfn}   model   have   been
exhausted   by  this   stage  (refer~\cref{subsec:quadraticsimresultsanalysis}).
Although~\crefrange{eq:negliionmolesgen}{eq:posliionmolesgen}     are    derived
from     the      \gls{dfn}     model,      the     coefficients      of     the
diffusivities   in    the   \gls{rhs}   of    the   next   set    of   equations
\ie{}~\crefrange{eq:negliionmolesquadratic}{eq:posliionmolesquadratic},   merely
involve  substitutions  of the  spatial  derivatives  of the  assumed  quadratic
expression.

Herein lies the weakness of the  baseline quadratic approach. Unlike the spatial
algebraic equations, which are tightly bound by the continuity and flux boundary
conditions at the  separator interfaces, there is no equality  constraint on the
spatial  derivative, which  is free  to grow  or shrink  without any  explicitly
imposed  bounds.  The  onus  of  being accurate  is  therefore  on  the  spatial
derivative evaluation which in-turn depends  on the correctness of the quadratic
functions~(\crefrange{eq:cenquadreduced}{eq:cepquadreduced})  themselves. It  is
not feasible to quantify the magnitude of error introduced in the time-evolution
of concentration  given a small-signal  perturbation in the coefficients  of the
quadratic spatial  computation \ie{} the  implicit coupling between them  is not
transparent. Since the quadratic approximation  itself is not perfect \ie{} does
not capture the  spatial gradient \emph{exactly} as the \gls{p2d}  model as seen
in~\cref{fig:spatialionicconc1C}, the internal coupling of coefficients leads to
errors in time-evolution computation.

The author's  approach is to  therefore break this detrimental  coupling between
spatial  derivative of  concentration  and its  temporal evolution  counterpart.
Inspired by the fact that the  quadratic approximation model had almost achieved
the desired goals with
\begin{enumerate}[label=\emph{\alph*})]
    \item a bottoms-up approach, \ie{} assuming some model structure a~priori, and
    \item not bound by any physical considerations due to the exhaustion of governing equations
\end{enumerate}
led  the author  to  broach a  suitable modelling  concept  that exhibits  these
common  traits, yet  of a  completely different  nature and  hitherto unexplored
in  physics-based battery  modelling  in general  and  electrolyte modelling  in
particular --- \emph{black-box system identification}.


