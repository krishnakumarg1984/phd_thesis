% -*- root: ../main.tex -*-
%!TEX root = ../main.tex
% this file is called up by main.tex
% content in this file will be fed into the main document
% vim:textwidth=80 fo=cqt

Black-box system identification techniques are composed of the following ---
\begin{enumerate*}[label=\emph{\alph*})]
     \item non-parametric methods, and
     \item parametric methods.
 \end{enumerate*}

\subsection{Non-parametric methods}
Non-parametric methods do not seek  a pre-assumed mathematical structure for the
system. They  aim to  directly estimate  the very  kernel of  what characterises
every system \viz{}  the Markov parameters in the time-domain  and the \gls{frf}
in the frequency domain, thereby requiring \emph{infinite} number of data points
for  their representation.  Major non-parametric  system identification  methods
include:
\begin{itemize}[topsep=0pt]
    \item Identification in time domain
        \begin{itemize}

            \item Direct  estimation of the system's  Markov parameters through
                statistical correlation of its response to an unit-pulse input.

        \end{itemize}
    \item Identification in frequency domain,  \ie{} of the \gls{frf}
        \begin{enumerate}

            \item   Direct   estimation    through   input-output   statistical
                cross-correlation.

            \item  \gls{etfe} using \glspl{dft} of input and output sequences.

            \item  Smoothed periodogram estimates using Welch's method.

            \item  Blackman-Tukey  Estimate   using  standard  filter
                windows  in digital  signal processing  (such as  Hamming, Hanning,
                Bartlett, Boxcar etc.).

        \end{enumerate}
\end{itemize}

\subsection{Parametric methods}\label{subsec:parametric}
Parametric  methods aim  to fit  specific input-output  data to  some family  of
well-known  mathematical constructs  that are  well-known and  widely applicable
to  a  large  variety  of  inputs.   It  is  important  to  recognise  that,  in
contrast to white-box system identification, the salient coefficients/properties
of  these  mathematical structures  \emph{do  not,}  in  any way  correspond  to
physical properties of  the system under consideration.  Major parametric system
identification methods use:
\begin{itemize}
    \item Transfer-function based frequency domain model structures
        \begin{enumerate}
            \item \gls{oe} model
            \item \gls{arx} model
            \item \gls{armax} model
            \item Box-Jenkins (BJ) model
        \end{enumerate}
    \item State-space time-domain model structures
        \begin{enumerate}
            \item Ho-Kalman realisation
            \item \gls{era} realisation
            \item Deterministic and stochastic \emph{subspace} structures
        \end{enumerate}
\end{itemize}

\subsection{Investigation of suitable system identification methodology}
% \subsubsection*{Ill-suitability of non-parametric methods}
Among   the   various   system    identification   techniques   mentioned,   the
non-parametric   methods    have   some    serious   drawbacks.    As   outlined
in~\cref{subsec:sysidbackground}, the author is inspired  by the trait of having
a pre-assumed model structure that  brought the baseline quadratic approximation
closer to a successful realisation. The non-parametric methods do not conform to
this philosophy. Furthermore, the requirement of infinite number of data samples
in order to fully quantify the system dampens its feasibility for implementation
in  resource  constrained  environments.  The  author is  of  the  opinion  that
resorting  to truncation  of  the  characteristic sequence  shall  only yield  a
sub-optimal solution. Hence non-parametric methods are ruled out for applying to
the task at hand.

While  parametric state-space  identification is  a feasible  alternative, these
methodologies are tedious and error-prone.  For instance, applying the Ho-Kalman
algorithm requires construction of large-sized Block-Hankel matrices followed by
a  \gls{svd} operation.  The  \gls{era}  uses the  identical  set of  operations
of  the  Ho-Kalman   procedure,  except  that  certain  blocks   in  the  Hankel
matrices  are chosen  at  random  for deletion  for  obtaining better  estimates
in  low  \gls{snr} environments  and  for  capturing slowly  decaying  phenomena
with  long time  constants. The  subspace methods  are mathematically  involved,
requiring  a  profound  understanding  of  concepts  from  linear  algebra  such
as  projections  to orthogonal  subspaces.  The  system under  consideration  is
presented in~\cref{sec:introtoplant} and after linearity considerations, refined
in~\cref{subsec:linearityanalysis}. It is composed of two independent \gls{siso}
subsystems.  However,   the  inflection  point  in   the  complexity-performance
trade-off  in state-space  identification  is achieved  only  when dealing  with
\gls{mimo} systems that suggest strong  cross-coupling among its internal states
or  at least  some degree  of  coupling among  the various  inputs and  outputs.
Furthermore, the impulse  responses of the system(s) under  consideration do not
have long tails since they are characterised by relatively short time constants.
Owing to these  reasons, it was decided that state  space identification methods
shall not be adopted here.

Owing to a  cornucopia of well-established technical  know-how readily available
in the systems  engineers toolkit, transfer function based  model structures are
naturally amenable  for control-oriented applications. However,  there exists an
apparent discrepancy to  its usage for this case, that  must be addressed first.
Transfer  function methods  are  a  frequency domain  technique  and hence,  the
resulting model  descriptions have  mathematical structures  radically different
from the time-domain model equations  of the conventional \gls{spm} within which
they are  to be embedded. This  conundrum is resolved by  closely inspecting the
model's scope  and its tractability for  conversion to time domain  as explained
next.

It  is  worth remembering  that,  as  per~\cref{subsec:freqdomainroms}, for  the
reduced order  modelling of the  \emph{entire cell}, all frequency  domain model
groups were considered  as out of scope  of this thesis specifically  due to the
overhead  of conversion  from frequency  to time  domain for  implementation and
other associated difficulties. The blanket exclusion nature of this statement is
to be revisited considering the specific scope  of the problem at hand. The body
of literature  on frequency  domain \glspl{rom} discuss  obtaining physics-based
\emph{transcendental}   transfer   functions  for   \emph{all}   electrochemical
quantities  of the  coupled  \gls{pdae} system  of~\cref{tbl:dfneqns} through  a
top-down approach.  However, the frequency domain  system identification methods
are concerned with  obtaining standard \emph{rational} transfer  functions for a
much narrower  scope \viz{}  the time-evolution  subsystem, through  a bottom-up
approach.  Such  rational transfer  functions  are  to obtained  for  \gls{siso}
systems for which an approximation-free  effortless conversion already exists in
classical control theory and is  presented in~\cref{sec:actualsysid}. In view of
their  overwhelming  simplicity  and  familiarity to  control  engineers,  after
considering these  arguments that circumvent  their only apparent  impediment to
adoption, \emph{transfer  function} based  system identification was  chosen for
tackling the problem at hand. The  steps leading to the identification procedure
is presented next.


% might possibly have to promote into its own detailed subsection

