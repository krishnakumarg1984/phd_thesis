%!TEX root = ../main.tex
% vim:nospell

% useful packages for writing any general  draft document with a small subset of
% packages  only for  book/thesis  style  docs and  another  subset of  packages
% applicable only for luatex
% Most notably, microtype is missing here since it needs to be loaded after babel (which is to be loaded after fontspec)

\usepackage{amsmath}
\usepackage{amsfonts}
\usepackage{amssymb}
\usepackage{anyfontsize} % typically for thesis use; fancy chapter font size and shading
% \usepackage{appendix}    % add appendices

\usepackage{babel}       % with british, we get OUP hyphenation material for free
\usepackage{biblatex}    % seems to have strong dependence on csquotes?
\usepackage{booktabs}

\usepackage{caption}     % for improved layout of figure captions with extra margin, smaller font than text
\usepackage{checkend}

% \usepackage{diffcoeff}   % looks pretty useful for any math-oriented document. Leaving it here

% \usepackage{engord}      % an alternative is to use the fmtcount package
\usepackage{enumitem}

\usepackage{fancyhdr}    % Define custom header (before hyperref)
% \usepackage{fixme}
\usepackage{flafter}     % is a latex builtin; https://tex.stackexchange.com/questions/261542/what-kind-of-a-package-is-flafter
\usepackage{footmisc}    % typically for thesis use;

\usepackage{geometry}
\usepackage{gitinfo2}
\usepackage{graphicx}    % important to load before fontspec

\usepackage{labelschanged}
\usepackage{lualatex-math}

\usepackage{makecell}
\usepackage{mathtools}
\usepackage{mathfixs}
\usepackage{multicol}
\usepackage{multirow}

\usepackage[section]{placeins} % Defines a \FloatBarrier command

\usepackage{setspace} % Define line spacing

\usepackage{siunitx}
\usepackage{subcaption}

\usepackage{titlesec}  % typically for thesis use;
\usepackage{titletoc}  % typically for thesis use;
\usepackage{threeparttable}
\usepackage{tocbibind}      % typically for thesis use; correct page numbers for bib in TOC; nottoc suppresses an entry for TOC itself

% \usepackage{ulem}

\usepackage{varwidth}

% \usepackage{witharrows}

% \usepackage{xfrac}
\usepackage{xpatch} % example use; helpful to replace parens with brackets for backreferencing with biblatex


% ---------- unused packages (but potentially useful) ----------
% \usepackage[backend=biber, style=ieee, sortlocale=en_GB, maxbibnames=50, url=true, doi=true, eprint=true ]{biblatex}
% \usepackage{blindtext}
% \usepackage{chkfloat}
% \usepackage{cite} % incompatible with biblatex
% \usepackage{cmdtrack}
% \usepackage{colortbl} % colortbl cannot be used if xcolor is used
% \usepackage{etoolbox} % not really required if using glossaries package, since this package then gets loaded automatically
% \usepackage{fnpct}
% \usepackage{footnote}
% \usepackage{layouts} % helpful for computing textwidth, textheight etc
% \usepackage{lettrine}
% \usepackage{luabibentry}
% \usepackage{makebox}
% \usepackage[activate={true,nocompatibility},final,tracking=true,factor=1100,stretch=10,shrink=10]{microtype}
% \usepackage{nccmath}
% \usepackage{needspace}
% \usepackage{nolbreaks}
% \usepackage[all,warning]{onlyamsmath} % if using Tikz, please include the calc and babel libraries (known incompatibilities)
% \usepackage{blindtext}
% \usepackage{soulutf8}
% \usepackage{subfiles}
% \usepackage{tablefootnote}
% \usepackage{tabularx}
% \usepackage[table]{xcolor} % loaded by pdfx package (if used) cannot explicitly load colortbl package either before or after
% \usepackage{url}
