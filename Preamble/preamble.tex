% -*- root: ../main.tex -*-
%!TEX root = ../main.tex
% -*- root: ../main.tex -*-
%!TEX root = ../main.tex
% vim:nospell
% \usepackage{shellesc}
% \AtEndDocument{\ShellEscape{mv thesis.pdf output_pdf/}}

\usepackage[modifier={by-nc-nd}]{doclicense}
\usepackage[frac,vfrac,multskip]{mathfixs}
\usepackage[normal]{engord}

\usepackage[textsize=scriptsize]{todonotes}
% \usepackage{luatodonotes}
% \usepackage{excludeonly}

\makeatletter
\renewcommand{\todo}[2][]{\tikzexternaldisable\@todo[#1]{#2}\tikzexternalenable}
\makeatother

% \makeatletter
% \renewcommand{\missingfigure}[2][]{\tikzexternaldisable\@missingfigure[#1]{#2}\tikzexternalenable}
% \makeatother

\usepackage[english]{isodate}
\usepackage[all,warning]{onlyamsmath}

\usepackage{relsize}

\RequirePackage[l2tabu, orthodox]{nag}

\newcommand{\etal}{\textit{et al}.}
\newcommand{\ie}{\textit{i}.\textit{e}.,}
\newcommand{\eg}{\textit{e}.\textit{g}.}
\newcommand{\viz}{\textit{viz}. }

\usepackage{labelschanged}

\newcommand{\jinnegpos}{\ensuremath{j  ∈  \left\{\text{neg, pos}\right\}}}
\newcommand{\jinnegposordered}{\ensuremath{j  \in  \left(\text{neg, pos}\right)}}
\newcommand{\jinpossepneg}{\ensuremath{j  ∈  \left\{\text{pos, sep, neg}\right\}}}
\newcommand{\sj}{\ensuremath{{\text{s}_j}}}
\newcommand{\pj}{\ensuremath{{\text{p}_j}}}
\newcommand{\pneg}{\ensuremath{{\text{p}_\text{neg}}}}
\newcommand{\ppos}{\ensuremath{{\text{p}_\text{pos}}}}
\newcommand{\nj}{\ensuremath{{\text{n}_j}}}
\newcommand{\npos}{\ensuremath{{\text{n}_\text{pos}}}}
\newcommand{\nneg}{\ensuremath{{\text{n}_\text{neg}}}}
\newcommand{\sjsurf}{\ensuremath{{\text{s,surf}_j}}}
\newcommand{\spossurf}{\ensuremath{{\text{s,surf}_\text{pos}}}}
\newcommand{\snegsurf}{\ensuremath{{\text{s,surf}_\text{neg}}}}
\newcommand{\sjmax}{\ensuremath{{\text{s,max}_j}}}
\newcommand{\snegmin}{\ensuremath{{\text{s,min}_\text{neg}}}}
\newcommand{\snegmax}{\ensuremath{{\text{s,max}_\text{neg}}}}
\newcommand{\sposmax}{\ensuremath{{\text{s,max}_\text{pos}}}}
\newcommand{\sjavg}{\ensuremath{{\text{s,avg}_j}}}
\newcommand{\spos}{\ensuremath{\text{s,pos}}}
\newcommand{\epos}{\ensuremath{\text{e,pos}}}
\newcommand{\sneg}{\ensuremath{\text{s,neg}}}
\newcommand{\eneg}{\ensuremath{\text{e,neg}}}
\newcommand{\jr}{\ensuremath{{\text{r}_j}}}
\newcommand{\posr}{\ensuremath{{\text{r}_\text{pos}}}}
\newcommand{\negr}{\ensuremath{{\text{r}_\text{neg}}}}
\newcommand{\maxj}{\ensuremath{{100\%_j}}}
\newcommand{\maxpos}{\ensuremath{{100\%_\text{pos}}}}
\newcommand{\maxneg}{\ensuremath{{100\%_\text{neg}}}}
\newcommand{\minpos}{\ensuremath{{0\%_\text{pos}}}}
\newcommand{\minneg}{\ensuremath{{0\%_\text{neg}}}}
\newcommand{\minj}{\ensuremath{{0\%_j}}}
\newcommand*\mean[1]{\overline{#1}}


% \usepackage[version=4]{mhchem}
\usepackage{chemformula}

\DeclarePairedDelimiter\abs{\lvert}{\rvert}
\usepackage{diffcoeff}
\diffset[p-delims = . |, p-nudge = 0]

\renewcommand{\footnoterule}{%
    \kern -3pt
    \hrule width 0.25\textwidth height 0.5pt
    \kern 2pt
}

\hyphenation{acad-e-my acad-e-mies af-ter-thought anom-aly anom-alies
    an-ti-deriv-a-tive an-tin-o-my an-tin-o-mies apoth-e-o-ses
    apoth-e-o-sis ap-pen-dix ar-che-typ-al as-sign-a-ble as-sist-ant-ship
    as-ymp-tot-ic asyn-chro-nous at-trib-uted at-trib-ut-able bank-rupt
    bank-rupt-cy bi-dif-fer-en-tial blue-print busier busiest
    cat-a-stroph-ic cat-a-stroph-i-cally con-gress cross-hatched data-base
    de-fin-i-tive de-riv-a-tive dis-trib-ute dri-ver dri-vers eco-nom-ics
    econ-o-mist elit-ist equi-vari-ant ex-quis-ite ex-tra-or-di-nary
    flow-chart for-mi-da-ble forth-right friv-o-lous ge-o-des-ic
    ge-o-det-ic geo-met-ric griev-ance griev-ous griev-ous-ly
    hexa-dec-i-mal ho-lo-no-my ho-mo-thetic ideals idio-syn-crasy
    in-fin-ite-ly in-fin-i-tes-i-mal ir-rev-o-ca-ble key-stroke
    lam-en-ta-ble light-weight mal-a-prop-ism man-u-script mar-gin-al
    meta-bol-ic me-tab-o-lism meta-lan-guage me-trop-o-lis
    met-ro-pol-i-tan mi-nut-est mol-e-cule mono-chrome mono-pole
    mo-nop-oly mono-spline mo-not-o-nous mul-ti-fac-eted mul-ti-plic-able
    non-euclid-ean non-iso-mor-phic non-smooth par-a-digm par-a-bol-ic
    pa-rab-o-loid pa-ram-e-trize para-mount pen-ta-gon phe-nom-e-non
    post-script pre-am-ble pro-ce-dur-al pro-hib-i-tive pro-hib-i-tive-ly
    pseu-do-dif-fer-en-tial pseu-do-fi-nite pseu-do-nym qua-drat-ic
    quad-ra-ture qua-si-smooth qua-si-sta-tion-ary qua-si-tri-an-gu-lar
    quin-tes-sence quin-tes-sen-tial re-arrange-ment rec-tan-gle
    ret-ri-bu-tion retro-fit retro-fit-ted right-eous right-eous-ness
    ro-bot ro-bot-ics sched-ul-ing se-mes-ter semi-def-i-nite
    semi-ho-mo-thet-ic set-up se-vere-ly side-step sov-er-eign spe-cious
    spher-oid spher-oid-al star-tling star-tling-ly sta-tis-tics
    sto-chas-tic straight-est strange-ness strat-a-gem strong-hold
    sum-ma-ble symp-to-matic syn-chro-nous topo-graph-i-cal tra-vers-a-ble
    tra-ver-sal tra-ver-sals treach-ery turn-around un-at-tached
    un-err-ing-ly white-space wide-spread wing-spread wretch-ed
    wretch-ed-ly Eng-lish Euler-ian Feb-ru-ary Gauss-ian
    Hamil-ton-ian Her-mit-ian Jan-u-ary Japan-ese Kor-te-weg
Le-gendre Mar-kov-ian Noe-ther-ian No-vem-ber Rie-mann-ian Sep-tem-ber}

\usepackage{xcolor}
\definecolor{mintedbg}{rgb}{0.95,0.95,0.95}
\definecolor{imperialbrick}{RGB}{165,25,0}
\definecolor{imperialnavy}{RGB}{0,33,71}
\definecolor{imperiallightgray}{RGB}{235,238,238}
\definecolor{imperialcoolgray}{RGB}{157,157,157}
\definecolor{intermediategray}{RGB}{196,196,196}
\definecolor{lightintergray}{RGB}{215,217,217}

\newcounter{filePrg}
\usepackage[draft=true,newfloat=true]{minted}
% \usepackage[newfloat=true]{minted}
\usepackage[most,minted]{tcolorbox}

\usepackage[makeroom]{cancel}
\renewcommand{\CancelColor}{\color{imperialbrick}}

\usepackage{tabstackengine}
\stackMath

\ExplSyntaxOn

\NewDocumentCommand \vect { s o m }
{
    \IfBooleanTF {#1}
    { \vectaux*{#3} }
    { \IfValueTF {#2} { \vectaux[#2]{#3} } { \vectaux{#3} } }
    ^T
}

\DeclarePairedDelimiterX \vectaux [1] {\lbrack} {\rbrack}
{ \, \dbacc_vect:n { #1 } \, }

\cs_new_protected:Npn \dbacc_vect:n #1
{
    \seq_set_split:Nnn \l_tmpa_seq { , } { #1 }
    \seq_use:Nn \l_tmpa_seq { \enspace }
}
\ExplSyntaxOff

\setmonofont[Scale=MatchLowercase]{Latin Modern Mono}

% \setmonofont[%
%             Extension = .otf,
%             UprightFont = *-Text,
%             BoldFont = *-Bold,
%             ItalicFont = *-TextItalic,
%             BoldItalicFont = *-BoldItalic,
%             Scale = MatchLowercase,
%             FakeStretch = 0.9,
%             ]{IBMPlexMono} % currently unavailable in Overleaf


% \usepackage{autobreak}
\newtcbinputlisting[use counter=filePrg,number within=chapter,list inside=mypyg]{\matlabcodelisting}[3][]{%
    listing engine=minted,
    minted language=matlab,
    % minted style=algol_nu, % xcode,emacs, perldoc, pastie, borland, vs, vim, tango
    listing file={#2},
    title=\small{\textbf{Listing \thetcbcounter}\quad {#1}},
    % fonttitle=\bfseries,
    listing only,
    list entry={\protect\numberline{\thetcbcounter} #1},
    enhanced jigsaw,
    breakable,
    % drop fuzzy shadow,
    minted options={
        fontsize=\scriptsize,
        breaklines,
        autogobble,
        linenos,
        numbersep=3mm,
        mathescape,
        baselinestretch=1,
        breakanywhere=true
    },
    % colback=offwhite,
    colback=white,
    colframe=imperialnavy,
    % colframe=black,
    % coltitle=white,
    % boxrule=0.2mm,
    left=5mm,
    overlay={\begin{tcbclipinterior}
            \fill[imperiallightgray] (frame.south west) rectangle ([xshift=5mm]frame.north west);
        \end{tcbclipinterior}
    },
    label=#3
}

%http://tex.stackexchange.com/questions/217489/numbering-tcolorbox-toc
\makeatletter % no indent for entries
\renewcommand{\l@tcolorbox}{\@dottedtocline{1}{0pt}{2.3em}}
\makeatother

% ******************************************************************************
% ****************************** Custom Margin *********************************
% Add `custommargin' in the document class options to use this section
% Set {innerside margin / outerside margin / topmargin / bottom margin}  and
% other page dimensions
\ifCLASSINFOcustommargin
  %\RequirePackage[left=37mm,right=30mm,top=35mm,bottom=30mm]{geometry}
  \RequirePackage[left=32mm,right=22mm,top=12mm,bottom=10mm,includeheadfoot,heightrounded]{geometry}

%\setlength\marginparwidth{2.3cm} %Die wird später zum Rechnen gebraucht, wird aber durch die Angabe im geometry package nicht automatisch richtig gesetzt.
  \setFancyHdr % To apply fancy header after geometry package is loaded
\fi
%\overfullrule=5pt

% Add spaces between paragraphs
% \setlength{\parskip}{1.0em}
% \setlength{\parskip}{0.9em}
% \setlength\parskip{1\baselineskip}
\setlength\parskip{0.75\baselineskip plus0.1\baselineskip  minus0.1\baselineskip}

% To remove the excess top spacing for enumeration, list and description
\usepackage{enumitem}
\setlist[enumerate,itemize,description]{topsep=0em}

%: ----------------------------------------------------------------------
%:                  TITLE PAGE: name, degree,..
% ----------------------------------------------------------------------
% below is to generate the title page with crest and author name


% ********************** TOC depth and numbering depth *************************
% levels are: 0 - chapter, 1 - section, 2 - subsection, 3 - subsection
\setcounter{secnumdepth}{3} % organisational level that receives a numbers
\setcounter{tocdepth}{3}    % print table of contents for level 3
%

%
% ******************************************************************************
% ******************************** Custom Packages *****************************
% ******************************************************************************
% ************************* Algorithms and Pseudocode **************************
\usepackage{algorithm}% http://ctan.org/pkg/algorithms
\usepackage{algpseudocode}% http://ctan.org/pkg/algorithmicx
% ************************* Math packages **************************
%\usepackage{upgreek}
\usepackage[amsmath,thmmarks]{ntheorem}
\newtheorem{theorem}{Theorem}
% ********************Captions and Hyperreferencing / URL **********************
\usepackage{graphics} % for improved inclusion of graphics
%\RequirePackage{wrapfig} % to include figure with text wrapping around it
\usepackage[margin=10pt,font=small,labelfont={bf},labelsep=quad,textformat=period]{caption} % for improved layout of figure captions with extra margin, smaller font than text
\usepackage{chapterfolder}
\usepackage{nameref}
\usepackage[all]{hypcap} % fix hyperref links to jump directly to Table or Figure
% ********************** New Chapter layout *************************
\RequirePackage{titlesec}
\renewcommand{\chaptername}{} % uncomment to print only "1" not "Chapter 1"
% Special layout for chapter numbers
\titleformat{\chapter}[display]
{\bfseries\sffamily\Huge}
{\hfill\fontsize{140}{50}\selectfont\color{lightgray}\rmfamily\textbf{\thechapter}}% label
{-0ex}
%{\filleft moves all to the right side
{\filleft\fontsize{50}{50}}
[\vspace{-0ex}]
% *************************** Graphics and figures *****************************
\usepackage{placeins} %Defines a \FloatBarrier command
\usepackage[countmax]{subfloat}
% \usepackage{subfig}
\usepackage{import}
%:-------------------------- packages for fancy things -----------------------

\setlength{\columnsep}{20pt} % space between columns; default 10pt quite narrow

%\RequirePackage[usenames, dvipsnames]{color}


%:-------------------------- BibLatex ---------------------------

\usepackage{csquotes}
% ********************************** Tables ************************************
\usepackage{booktabs}
\usepackage{multicol} % for pages with multiple text columns, e.g. References
\usepackage{multirow}
\usepackage{tabularx}
\usepackage{longtable}
\usepackage{hhline}
%\renewcommand{\arraystretch}{1.2}
\usepackage{xcolor,colortbl}
%dashed line
\usepackage{array}
\usepackage{ragged2e}
%\usepackage{arydshln}
%\setlength\dashlinedash{0.2pt}
%\setlength\dashlinegap{1.5pt}
% use P instead of p for RaggedRight tabel columns e.g. begin{tabular}{P{2cm}|P{4cm}|P{3cm}|P{3cm}}
\newcolumntype{P}[1]{>{\RaggedRight\hspace{0pt}}p{#1}}
%\setlength\arrayrulewidth{0.3pt}
% turn of those nasty overfull and underfull hboxes

% *********************************** SI Units *********************************

\usepackage[separate-uncertainty = true,multi-part-units=single]{siunitx}
\sisetup{
  locale = US ,
  per-mode = symbol,
  binary-units = true
}

% ********************** bibtex/biblatex *************************
%\usepackage{showframe}
\ifCLASSINFOcustombibstyle
\ifCLASSINFObiblatex
\usepackage[
    backend=biber,
    style=ieee,
    sortlocale=en_US,
    natbib=true,
    maxbibnames=50,
    url=true,
    doi=true,
    eprint=false
]{biblatex}

%\DeclareFieldFormat*{url}{}
%\DeclareFieldFormat[misc]{url}{\mkbibacro{URL}\addcolon\space\url{#1}}
%\DeclareFieldFormat*{urldate}{}
%\DeclareFieldFormat[misc]{urldate}{\mkbibparens{\bibstring{urlseen}\space#1}}

\AtEveryBibitem{%
  \ifentrytype{misc}{%
  }{%
    \ifentrytype{patent}{%
    }{%
      \clearfield{url}%
      \clearfield{urldate}%
      \clearfield{urlyear}%
    }%
  }%
}

\else
\usepackage[sort, numbers]{natbib}
\fi
\fi
\ifCLASSINFObiblatex
     % ls -F 9_backmatter/*.bib | tr '\n' '\0' | xargs -0 -n 1 basename | sed 's/.\{4\}$//'
    \addbibresource{9_backmatter/thesis_refs.bib}
    \addbibresource{9_backmatter/references.bib}
\DeclareSourcemap{
    \maps[datatype=bibtex]{
      \map{
           \step[fieldsource=doi, match={\regexp{\{\\textunderscore.?\}}}, replace={_}]
           \step[fieldsource=doi, match={\regexp{\{\\textless.?\}}}, replace={&lt;}]
           \step[fieldsource=doi, match={\regexp{\{\\textgreater.?\}}}, replace={&gt;}]
           %\step[fieldsource=doi, match={\regexp{\{\>.?\}}}, replace={&gt;}]
      }
      %\map{
      %     \step[fieldsource=doi, match={\regexp{\{\\textless.?\}}}, replace={<}]
      %     %\step[fieldsource=doi, match={\regexp{\{\\textgreater.*\}}}, replace={>}]
      %}
      %\map{
      %     \step[fieldsource=doi, match={\regexp{\{\\textgreater *\}}}, replace={>}]
      %     %\step[fieldsource=doi, match={\regexp{\{\\textgreater.*\}}}, replace={>}]
      %}
    }
}
\fi



% ******************************************************************************
% ************************* User Defined Commands ******************************
% ******************************************************************************

% *********** To change the name of Table of Contents / LOF and LOT ************
\addto\captionsenglish{
%\renewcommand{\contentsname}{My Table of Contents}
%\renewcommand{\listfigurename}{My List of Figures}
%\renewcommand{\listtablename}{My List of Tables}
}
% ************************ Formatting / Footnote *******************************

% turn of those nasty overfull and underfull hboxes

%\hbadness=10000
%\hfuzz=50pt

\tolerance=1414
\hbadness=1414
\emergencystretch=1.5em
\hfuzz=0.5pt
%\widowpenalty=10000
\vfuzz=\hfuzz
% Ragged bottom avoids extra whitespaces between paragraphs
% But the buttom line is not euqalized anymore!
\raggedbottom

% TeX default is 50
\hyphenpenalty=750
% The TeX default is 1000
%\hbadness=1350
% IEEE does not use extra spacing after punctuation
\frenchspacing

\binoppenalty=1000 % default 700
\relpenalty=800     % default 500

\interfootnotelinepenalty=10000

% Don't break enumeration (etc.) across pages in an ugly manner
\clubpenalty=10000
\widowpenalty=10000

%\linepenalty=1000
%\looseness=-1

%\usepackage[defaultlines=4,all]{nowidow}

%\usepackage[perpage]{footmisc}

