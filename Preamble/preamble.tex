%!TEX root = ../main.tex
% vim:nospell

\RequirePackage{shellesc} % unified shell-escape interface
\RequirePackage[debrief]{silence}
\RequirePackage[l2tabu, orthodox]{nag}

\RequirePackage{xstring}
\StrBetween{\luatexbanner}{\detokenize{n }}{\detokenize{.0 (}}[\luatexversionused] % obtain luatex version used

\providecommand{\pdfxopts}{a-3b,pdf17} % may use options such as a-1a. a-1b
\immediate\write18{rm \jobname.xmpdata} % comment out for non-*nix systems
\begin{filecontents*}{\jobname.xmpdata}
    \Author{Krishnakumar Gopalakrishnan}
\CopyrightURL{https://creativecommons.org/licenses/by-nc-nd/4.0/}
\Copyright{\textcopyright  2018 by Krishnakumar Gopalakrishnan. The copyright  of  this thesis  rests  with the  author  and is  made available  under a  Creative Commons  Attribution Non-Commercial  No Derivatives licence. Researchers are free to copy,  distribute or transmit the thesis on the condition  that they  attribute  it, that  they  do not  use  it for  commercial purposes and that they  do not alter, transform or build upon  it. For any reuse or redistribution,  researchers must make clear  to others the licence  terms of this work.}
\Creator{LuaTeX \luatexversionused\space and pdfx.sty with options \pdfxopts}
\Subject{Lithium ion batteries \sep Mathematical models \sep Electric vehicles \sep Plug-in electric vehicles \sep Electric vehicle industry \sep Electric vehicles--Batteries \sep Battery management systems \sep Battery charging stations (Electric vehicles) \sep Battery industry \sep electrochemistry \sep Electric batteries \sep Numerical calculations \sep Numerical analysis \sep Chemical processes--Mathematical models \sep Simulation methods \sep Computer simulation \sep Computer modeling \sep battery management systems }
\Keywords{lithium-ion batteries \sep electric vehicles \sep mathematical modelling \sep lithium-ion cell \sep electrochemical cell \sep electrolyte model \sep SPM \sep single particle model \sep layer optimisation \sep pouch cell design \sep layer optimisation \sep pseudo-2d model \sep fast charging \sep mathematical models \sep computer simulation }
\pdfxSetCMYKcolorProfileDir{icc_profiles/coated_FOGRA39L_argl.icc}
\pdfxSetRGBcolorProfileDir{icc_profiles/sRGB_IEC61966-2-1_black_scaled.icc}
\Producer{LuaTeX \luatexversionused\space and pdfx.sty with options \pdfxopts}
\PublicationType{PhD Thesis}
\Publisher{Imperial College London}
\Title{Computational modelling of lithium ion batteries for electric vehicle applications: analysis, design and implementation}


\end{filecontents*}

\PassOptionsToPackage{\pdfxopts}{pdfx}
\PassOptionsToPackage{table}{xcolor}
\PassOptionsToPackage{luatex}{pdflscape}
\PassOptionsToPackage{pdfa}{hyperref}

% Options to these packages are passed to them whenever they get loaded
\PassOptionsToPackage{title, titletoc}{appendix}
\PassOptionsToPackage{makeroom}{cancel}
\PassOptionsToPackage{backend=biber, style=numeric-comp, sorting=none, citestyle=numeric-comp, maxbibnames=50, url=true, doi=true, eprint=false, backref=true, backrefstyle=three}{biblatex}
\PassOptionsToPackage{margin=10pt, font=small, labelfont={bf}, labelsep=quad}{caption}
\PassOptionsToPackage{normal}{engord}
\PassOptionsToPackage{inline}{enumitem}
\PassOptionsToPackage{draft}{fixme}
\PassOptionsToPackage{bottom}{footmisc}
\PassOptionsToPackage{luatex, paper=a4paper, hmarginratio=1:1, vmarginratio=1:1, scale=0.75}{geometry}
\PassOptionsToPackage{local, mark}{gitinfo2}
\PassOptionsToPackage{frac, vfrac, multskip}{mathfixs}
\PassOptionsToPackage{section}{placeins}
\PassOptionsToPackage{separate-uncertainty = true, multi-part-units=single, detect-all}{siunitx}
\PassOptionsToPackage{nottoc}{tocbibind}
\PassOptionsToPackage{normalem}{ulem}
\PassOptionsToPackage{no-math, quiet}{fontspec}
\PassOptionsToPackage{warnings-off={mathtools-colon}}{unicode-math}
\PassOptionsToPackage{british}{babel}
\PassOptionsToPackage{final, activate={true, nocompatibility}, factor=1100, stretch=10, shrink=10, babel=true}{microtype}
\PassOptionsToPackage{british}{selnolig}
\PassOptionsToPackage{newfloat=true}{minted}
\PassOptionsToPackage{minted, most}{tcolorbox}
\PassOptionsToPackage{all}{hypcap}
\PassOptionsToPackage{nomain, acronym, xindy}{glossaries-extra}
\PassOptionsToPackage{nameinlink}{cleveref}
\PassOptionsToPackage{hyphenation, lastparline, nosingleletter}{impnattypo}
\PassOptionsToPackage{defaultlines=2, all}{nowidow}

\documentclass[12pt,a4paper,oneside,openright]{book} % using the book document class

%%%%%%%%%% list of packages to be loaded %%%%%%%%%%

% uncommment any commented packages before final submission
\usepackage{afterpage}
\usepackage{algpseudocode}  % http://ctan.org/pkg/algorithmicx
\usepackage{bigints}
\usepackage{cancel}
\usepackage{pdfpages}
\usepackage{tabstackengine} % stackmath macro uses this package

%!TEX root = ../main.tex
% vim:nospell

% useful packages for writing any general  draft document with a small subset of
% packages  only for  book/thesis  style  docs and  another  subset of  packages
% applicable only for luatex
% Most notably, microtype is missing here since it needs to be loaded after babel (which is to be loaded after fontspec)

\usepackage{amsmath}
\usepackage{amsfonts}
\usepackage{amssymb}
\usepackage{anyfontsize} % typically for thesis use; fancy chapter font size and shading
\usepackage{appendix}    % add appendices

\usepackage{babel}       % with british, we get OUP hyphenation material for free
\usepackage{biblatex}    % seems to have strong dependence on csquotes?
\usepackage{booktabs}

\usepackage{caption}     % for improved layout of figure captions with extra margin, smaller font than text
\usepackage{checkend}

% \usepackage[datesep=/,useregional]{datetime2}
\usepackage[datesep=/,style=ddmmyyyy]{datetime2}
% \DTMsetdatestyle{en-GB-numeric}
\usepackage{diffcoeff}   % looks pretty useful for any math-oriented document. Leaving it here

\usepackage{engord}      % an alternative is to use the fmtcount package
\usepackage{enumitem}

\usepackage{fancyhdr}    % Define custom header (before hyperref)
\usepackage{fixme}
\usepackage{flafter}
\usepackage{footmisc}    % typically for thesis use;

\usepackage{geometry}
\usepackage{gitinfo2}
\usepackage{graphicx}    % important to load before fontspec

\usepackage{labelschanged}
\usepackage{lualatex-math}

\usepackage{makecell}
\usepackage{mathtools}
\usepackage{mathfixs}
\usepackage{multicol}
\usepackage{multirow}

\usepackage[section]{placeins} % Defines a \FloatBarrier command

% \usepackage{rotating} % defines a sidewaystable environment
% \usepackage{pdflscape}
\usepackage{lscape}
\usepackage{longtable}
\usepackage{setspace} % Define line spacing

\usepackage{siunitx}
\usepackage{subcaption}

\usepackage{titlesec}  % typically for thesis use;
\usepackage{titletoc}  % typically for thesis use;
\usepackage{threeparttable}
\usepackage{tocbibind} % typically for thesis use; correct page numbers for bib in TOC, nottoc suppresses an entry for TOC itself

\usepackage{ulem}

\usepackage{varwidth}

\usepackage{witharrows}

\usepackage{xfrac}
\usepackage{xpatch} % example use; helpful to replace parens with brackets for backreferencing with biblatex

% ---------- unused packages (but potentially useful) ----------
% \usepackage[backend=biber, style=ieee, sortlocale=en_GB, maxbibnames=50, url=true, doi=true, eprint=true ]{biblatex}
% \usepackage{blindtext}
% \usepackage{chkfloat}
% \usepackage{cite} % incompatible with biblatex
% \usepackage{cmdtrack}
% \usepackage{colortbl} % colortbl cannot be used if xcolor is used
% \usepackage{etoolbox} % not really required if using glossaries package, since this package then gets loaded automatically
% \usepackage{fnpct}
% \usepackage{footnote}
% \usepackage{layouts} % helpful for computing textwidth, textheight etc
% \usepackage{lettrine}
% \usepackage{luabibentry}
% \usepackage{makebox}
% \usepackage[activate={true,nocompatibility},final,tracking=true,factor=1100,stretch=10,shrink=10]{microtype}
% \usepackage{nccmath}
% \usepackage{needspace}
% \usepackage{nolbreaks}
% \usepackage[all,warning]{onlyamsmath} % if using Tikz, please include the calc and babel libraries (known incompatibilities)
% \usepackage{blindtext}
% \usepackage{soulutf8}
% \usepackage{subfiles}
% \usepackage{tablefootnote}
% \usepackage{tabularx}
% \usepackage[table]{xcolor} % loaded by pdfx package (if used) cannot explicitly load colortbl package either before or after
% \usepackage{url}

 % a bit specific to luatex; also includes biblatex

\usepackage{unicode-math} % fontspec after graphicx and babel; https://tex.stackexchange.com/questions/188222/problem-with-babel-and-fontspec; no-math option (math-handling is left to unicode-math); silent to suppress all warnings (even in log file)
%!TEX root = ../main.tex
% vim:nospell

\setmainfont{LibertinusSerif}[%
Extension         = .otf,
Path              = ./fonts/,
UprightFont       = *-Regular,
BoldFont          = *-Bold,
ItalicFont        = *-Italic,
BoldItalicFont    = *-BoldItalic,
Numbers           = {Proportional, Lining},
Ligatures         = {TeX, Common, Required},
SmallCapsFeatures = {Letters = SmallCaps},
% Fractions         = On,
Kerning           = {Uppercase, On},
% FontFace       = {sb}{n}{*-Semibold},
% FontFace       = {sb}{it}{*-SemiboldItalic},
]%

\setsansfont{LibertinusSans}[%
Extension         = .otf,
Path              = ./fonts/,
UprightFont       = *-Regular,
BoldFont          = *-Bold,
ItalicFont        = *-Italic,
BoldItalicFont    = *-BoldItalic,
Numbers           = {Proportional, Lining},
Ligatures         = {TeX, Common, Required},
SmallCapsFeatures = {Letters = SmallCaps},
% Fractions         = On,
Kerning           = {Uppercase, On},
]%

% \setmainfont[Numbers={Proportional},Ligatures={TeX, Common%, Historic, Contextual, Rare, Discretionary
% }]{Libertinus Serif}
% \setsansfont{Libertinus Sans}

\setmonofont{Latin Modern Mono}

% % https://tex.stackexchange.com/questions/103379/minionpro-semibold-medium
% \DeclareRobustCommand\sbseries{\fontseries{sb}\selectfont}
% \DeclareTextFontCommand{\textsb}{\sbseries}

\defaultfontfeatures{ Scale = MatchLowercase }
\defaultfontfeatures[\rmfamily]{ Scale = 1}

% \setmathfont[bold-style = ISO]{LibertinusMath-Regular.otf} % https://github.com/libertinus-fonts/libertinus/issues/20
\setmathfont{LibertinusMath-Regular.otf}[%
Path       = ./fonts/,
bold-style = ISO, % https://github.com/libertinus-fonts/libertinus/issues/20
]%

\setmathfont{libertinusmath-bold.otf}[%
Path       = ./fonts/,
bold-style = ISO, % https://github.com/libertinus-fonts/libertinus/issues/20
version    = bold,
]%

% \setmathfont[math-style=upright,range={`e,`i}]{Latin Modern Math}
% \setmathfont[range={\mathunder,\triangleq},Scale=MatchUppercase]{STIX2Math.otf}
\setmathfont[range = {\mathunder,\triangleq,\underbrace},Scale = MatchUppercase]{Latin Modern Math}

 % selection of unicode text and math fonts

\usepackage{ragged2e}  %  Should be loaded after the body font and size have been established  eg, after font packages have been loaded
\usepackage{microtype} % if using option babel=true, babel must be loaded before microtype; microtype must be loaded after font selection (after fontspec)

\usepackage{selnolig}  % luatex package;  should go after loading babel

\usepackage{float} % Loading order is important here https://tex.stackexchange.com/questions/435529/correct-loading-order-of-package-newfloat-along-with-hyperref-and-algorithmic-pa/435597#435597
% \usepackage[draft=true,newfloat=true]{minted}
% \usepackage[minted]{tcolorbox}
\usepackage{minted}
\usepackage{tcolorbox}
\usepackage{csquotes}    % The fvextra package is loaded by minted, so you should load minted before csquotes; has a strong dependency with biblatex
\usepackage{chemformula} % uses tikz arrows

%!TEX root = ../main.tex
% vim: nospell

\usepackage{pdfx}
% https://tex.stackexchange.com/questions/449470/pdfx-package-incompatibility-with-bookmark-package-when-using-luatex-engine/449508#449508
\ifluatex
    \pdftrue
\fi

\usepackage{hyperxmp} % comment out if using the pdfx package
% \usepackage{hyperref} % comment out if using the pdfx package

\hypersetup{%
    anchorcolor        = black,
    baseurl            = {http://spiral.imperial.ac.uk/},
    breaklinks         = true,
    citecolor          = [RGB]{0,68,136}, % https://personal.sron.nl/https://personal.sron.nl/pault/#fig:scheme_highcontrastpault/#fig:scheme_highcontrast
    colorlinks         = true,
    final              = true,
    linkcolor          = [RGB]{0,68,136}, % https://personal.sron.nl/https://personal.sron.nl/pault/#fig:scheme_highcontrastpault/#fig:scheme_highcontrast
    linktocpage        = true,
    pdfauthor          = {Krishnakumar Gopalakrishnan},
    pdfauthortitle     = {PhD student},
    pdfapart           = {3},
    pdfborderstyle     = {/S/U/W 1},
    pdfcaptionwriter   = {Krishnakumar Gopalakrishnan},
    pdfcenterwindow    = true,
    pdfcontactaddress  = {Mechanical Engineering Dept, Exhibition Road, City and Guild Building, Imperial College London, South Kensington},
    pdfcontactcity     = {London, United Kingdom},
    pdfcontactcountry  = {United Kingdom},
    pdfcontactemail    = {krishnak at vt dot edu},
    pdfcontactpostcode = {SW7 2AZ},
    pdfcontactregion   = {Greater London},
    pdfcontacturl      = {https://krishnakumarg.gitlab.io, https://www.linkedin.com/in/krishnakumargopalakrishnan},
    pdfcopyright       = {\textcopyright{} 2019 by Krishnakumar Gopalakrishnan. The    copyright    of    this     thesis    rests    with    the    author. Unless   otherwise    indicated,   its    contents   are    licensed   under a   Creative Commons   Attribution-NonCommercial-NoDerivatives  4.0   International  (CC BY-NC-ND 4.0) Licence. Under this licence, you may  copy and redistribute the material in any medium or format on the condition that: you credit the author, do not use it for commercial purposes  and do not distribute  modified versions of the  work. When reusing or sharing this work, ensure you  make the licence terms clear to others by naming  the licence and linking  to the licence text.  Please seek permission from the copyright  holder for uses of  this work that are not  included in this licence or permitted under UK Copyright Law.},
    pdfcreator         = {LaTeX with hyperref},
    pdfdisplaydoctitle = true,
    pdffitwindow       = true,
    pdfkeywords        = {lithium-ion batteries, electric vehicles, mathematical modelling, lithium-ion cell, electrochemical cell, electrolyte model, SPM, single particle model, layer optimisation, pouch cell design, layer optimisation, pseudo-2d model, fast charging, mathematical models, computer simulation},
    pdflang            = {en-GB-oxendict},
    pdflicenseurl      = {https://creativecommons.org/licenses/by-nc-nd/4.0/},
    pdfmetalang        = {en-GB-oxendict},
    pdfproducer        = {LuaTeX-\luatexversionused},
    pdfsource          = {},
    pdfstartview       = {Fit},
    pdfsubject         = {Lithium ion batteries, Mathematical models, Electric vehicles, Plug-in electric vehicles, Electric vehicle industry, Electric vehicles--Batteries, Battery management systems, Battery charging stations (Electric vehicles), Battery industry, electrochemistry, Electric batteries, Numerical calculations, Numerical analysis, Chemical processes--Mathematical models, Simulation methods, Computer simulation, Computer modeling, battery management systems},
    pdftitle           = {Computational modelling of lithium ion batteries for electric vehicle applications: analysis, design and implementation},
    pdftoolbar         = true,
    plainpages         = false,
    pdfencoding        = auto,
    unicode            = true,
    urlcolor           = [RGB]{0,68,136}, % https://personal.sron.nl/https://personal.sron.nl/pault/#fig:scheme_highcontrastpault/#fig:scheme_highcontrast
}%


% ---------- packages to be loaded after hyperref ----------%

\usepackage{nameref}
\usepackage{algorithm} % http://ctan.org/pkg/algorithms
\usepackage{hypcap} % to be loaded after hyperref. fix hyperref links to jump directly to Table or Figure

\usepackage{pdftexcmds} % not required if using pdfx package

\usepackage{glossaries-extra} % should be loaded after hyperref, but before cleveref % consider 'symbols' option

\usepackage{hypdestopt} % seems to have problems with pdfx package?

\usepackage{bookmark} % improves bookmarks handling. % More features and faster updated bookmarks.
\usepackage{cleveref}

% \usepackage{showframe}
% \usepackage[noframe]{showframe}
\newcommand{\blackurl}{\hypersetup{urlcolor=black}}%
\newcommand{\regularurl}{\hypersetup{urlcolor=[RGB]{0,68,136}}}%

  % hyperref, nameref, algoriothm, hypcap, bookmark, glossaries, cleveref, showframe

\usepackage{impnattypo}
\usepackage{nowidow}

%---------- end of package loading ----------%


%---------- begin custom commands ----------%
%!TEX root = ../main.tex
% vim:nospell ft=tex

\definecolor{mintedbg}{rgb}{0.95,0.95,0.95}
\definecolor{imperialraspberry}{RGB}{145,0,72}

\definecolor{imperialbrick}{RGB}{165,25,0}
\definecolor{sepiadvipsnames}{RGB}{99,29,11}
\definecolor{imperialnavy}{RGB}{0,33,71}
% \definecolor{brickreddvipsnames}{RGB}{173,51,38}
% \definecolor{mahoganydvipsnames}{RGB}{161,53,40}

\definecolor{imperialblue}{RGB}{0,62,116}
\definecolor{cbrewerdarkblue}{RGB}{49,130,189}
\definecolor{viridistendarkblue}{RGB}{56,88,140}
\definecolor{viridistwentyblueseven}{RGB}{49,103,142}
\definecolor{viridistwentybluesix}{RGB}{54,91,141}
\definecolor{viridistwentybluefive}{RGB}{60,78,138}
\definecolor{viridistenlighterblue}{RGB}{45,111,142}
\definecolor{imperiallightblue}{RGB}{0,110,175}
\definecolor{imperialnewblue}{RGB}{0,86,146}

\definecolor{imperialdarkgreen}{RGB}{2,137,59}
\definecolor{imperialprocessblue}{RGB}{0,133,202}

\definecolor{imperiallightgray}{RGB}{235,238,238}
\definecolor{cbrewerlightgray}{RGB}{240,240,240}
\definecolor{cbrewerintergray}{RGB}{189,189,189}
\definecolor{imperialcoolgray}{RGB}{157,157,157}

\definecolor{intermediategray}{RGB}{196,196,196}
\definecolor{cbrewerdarkgray}{RGB}{99,99,99}
\definecolor{lightintergray}{RGB}{215,217,217}

%!TEX root = ../main.tex
% vim:nospell ft=tex

\newcommand{\effdelta}{\ensuremath{{\text{eff}_\delta}}}
\newcommand{\effj}{\ensuremath{{\text{eff}_j}}}
\newcommand{\efflambda}{\ensuremath{{\text{eff}_\lambda}}}
\newcommand{\effmu}{\ensuremath{{\text{eff}_\mu}}}
\newcommand{\effn}{\ensuremath{{\text{eff}_n}}}
\newcommand{\effp}{\ensuremath{{\text{eff}_p}}}
\newcommand{\effs}{\ensuremath{{\text{eff}_s}}}
\newcommand{\elambda}{\ensuremath{{\text{e}_\lambda}}}
\newcommand{\eneg}{\ensuremath{\text{e,neg}}}
\newcommand{\ensub}{\ensuremath{\text{e,n}}}
\newcommand{\epos}{\ensuremath{\text{e,pos}}}
\newcommand{\epsub}{\ensuremath{\text{e,p}}}
\newcommand{\essub}{\ensuremath{\text{e,s}}}

\newcommand{\jinnegposordered}{\ensuremath{j  \in  \left(\text{neg, pos}\right)}}
\newcommand{\jinnegpos}{\ensuremath{j  ∈  \left\{\text{neg, pos}\right\}}}
\newcommand{\jinnegseppos}{\ensuremath{j  ∈  \left\{\text{neg, sep, pos}\right\}}}
\newcommand{\jinnsp}{\ensuremath{j  ∈  \left\{\text{n,s,p}\right\}}}
\newcommand{\jinpossepneg}{\ensuremath{j  ∈  \left\{\text{pos, sep, neg}\right\}}}
\newcommand{\jr}{\ensuremath{{\text{r}_j}}}
\newcommand{\lambdainnegpos}{\ensuremath{\lambda  ∈  \left\{\text{neg, pos}\right\}}}
\newcommand{\lambdainnegseppos}{\ensuremath{\lambda  \in  \{\text{neg, sep, pos}\}}}
\newcommand{\lambdar}{\ensuremath{{\text{r}_\lambda}}}
\newcommand{\maxj}{\ensuremath{{100\%_j}}}
\newcommand{\maxneg}{\ensuremath{{100\%_\text{neg}}}}
\newcommand{\maxpos}{\ensuremath{{100\%_\text{pos}}}}
\newcommand{\minj}{\ensuremath{{0\%_j}}}
\newcommand{\minneg}{\ensuremath{{0\%_\text{neg}}}}
\newcommand{\minpos}{\ensuremath{{0\%_\text{pos}}}}
\newcommand{\muinnegseppos}{\ensuremath{\mu  \in  \{\text{neg, sep, pos}\}}}
\newcommand{\negr}{\ensuremath{{\text{r}_\text{neg}}}}
\newcommand{\nj}{\ensuremath{{\text{n}_j}}}
\newcommand{\nneg}{\ensuremath{{\text{n}_\text{neg}}}}
\newcommand{\npos}{\ensuremath{{\text{n}_\text{pos}}}}
\newcommand{\pj}{\ensuremath{{\text{p}_j}}}
\newcommand{\plambda}{\ensuremath{{\text{p}_\lambda}}}
\newcommand{\pneg}{\ensuremath{{\text{p}_\text{neg}}}}
\newcommand{\posr}{\ensuremath{{\text{r}_\text{pos}}}}
\newcommand{\ppos}{\ensuremath{{\text{p}_\text{pos}}}}
\newcommand{\refflambda}{\ensuremath{{\text{r,eff}_\lambda}}}
\newcommand{\sefflambda}{\ensuremath{{\text{s,eff}_\lambda}}}
\newcommand{\sjavg}{\ensuremath{{\text{s,avg}_j}}}
\newcommand{\sjmax}{\ensuremath{{\text{s,max}_j}}}
\newcommand{\sjsurf}{\ensuremath{{\text{s,surf}_j}}}
\newcommand{\sj}{\ensuremath{{\text{s}_j}}}
\newcommand{\slambdamax}{\ensuremath{{\text{s,max}_\lambda}}}
\newcommand{\slambdasurf}{\ensuremath{{\text{s,surf}_\lambda}}}
\newcommand{\slambda}{\ensuremath{{\text{s}_\lambda}}}
\newcommand{\snegmax}{\ensuremath{{\text{s,max}_\text{neg}}}}
\newcommand{\snegmin}{\ensuremath{{\text{s,min}_\text{neg}}}}
\newcommand{\snegsurf}{\ensuremath{{\text{s,surf}_\text{neg}}}}
\newcommand{\sneg}{\ensuremath{\text{s,neg}}}
\newcommand{\snsub}{\ensuremath{\text{s,n}}}
\newcommand{\sposmax}{\ensuremath{{\text{s,max}_\text{pos}}}}
\newcommand{\spossurf}{\ensuremath{{\text{s,surf}_\text{pos}}}}
\newcommand{\spos}{\ensuremath{\text{s,pos}}}
\newcommand{\spsub}{\ensuremath{\text{s,p}}}
\newcommand{\tplus}{\ensuremath{{t^0_\text{+}}}}

%!TEX root = ../main.tex
% vim:nospell ft=tex

% For unbroken lines in algorithmicx/algpseudocode when typesetting display math

% other alternative: https://tex.stackexchange.com/questions/110431/problems-with-vertical-lines-in-algorithmicx?noredirect=1&lq=1
% https://tex.stackexchange.com/questions/301462/why-are-vertical-rules-dashed-sometimes-with-algorithmic-package
%%%%%%%%%% https://tex.stackexchange.com/questions/350399/indentation-scope-lines-broken-in-algpseudocode%%%%%%%%%
\newcommand*{\algrule}[1][\algorithmicindent]{\hspace*{.2em}{\color{cbrewerdarkgray}\vrule\vrule
width 0pt height \baselineskip depth .1618\baselineskip\hspace*{\dimexpr#1-.5em}}}

\makeatletter
\newcount\ALG@printindent@tempcnta
\def\ALG@printindent{%
    \ifnum \theALG@nested>0% is there anything to print
        \ifx\ALG@text\ALG@x@notext% is this an end group without any text?
            % do nothing
    \else
        \unskip
        % draw a rule for each indent level
        \ALG@printindent@tempcnta=1
        \loop
        \algrule[\csname ALG@ind@\the\ALG@printindent@tempcnta\endcsname]%
        \advance \ALG@printindent@tempcnta 1
        \ifnum \ALG@printindent@tempcnta<\numexpr\theALG@nested+1\relax% can't do <=, so add one to RHS and use < instead
            \repeat
        \fi
    \fi
}%
% needs etoolbox, but this should have been already loaded with glossaries
\patchcmd{\ALG@doentity}{\noindent\hskip\ALG@tlm}{\ALG@printindent}{}{\errmessage{failed to patch}}
\makeatother

\AtBeginEnvironment{algorithmic}{\lineskip0pt}
%%%%%%%%%% https://tex.stackexchange.com/questions/350399/indentation-scope-lines-broken-in-algpseudocode%%%%%%%%%


\algnewcommand\algorithmicinput{\textbf{Initialise:}}
\algnewcommand\Initialise{\item[\algorithmicinput]}

\algnewcommand\algorithmicdata{\textbf{User data:}}
\algnewcommand\Userdata{\item[\algorithmicdata]}

\algnewcommand\algorithmicfulllinecomment{\qquad\quad  \scriptsize \textit{Note:}}
\algnewcommand\FullComment{\item[\algorithmicfulllinecomment]}

\makeatletter
\algrenewcommand\ALG@beginalgorithmic{\footnotesize}
\algrenewcommand\algorithmiccomment[2][\footnotesize]{{#1\hfill\(\triangleright\) #2}}
\makeatother

\algblockdefx[NAME]{ISR}{END}%
[2][Unknown]{\textbf{begin} \textproc{Interrupt Service Routine} #1(#2)}%
{\textbf{return} \Comment[\footnotesize]{resume suspended line in \textsc{Main()}}}

\algblockdefx[NAME]{OutputEqn}{EndOutputEqn}%
[2][\textbf{x}]{\textbf{subroutine} \textproc{ComputeCellVoltage}(#2)}%
{\textbf{return} $V_\text{cell}$ \Comment[\footnotesize]{resume suspended line in \textproc{Simulate\gls{spm}}}}

\newsavebox{\algboxA}
\newsavebox{\algboxB}

\makeatletter
\@addtoreset{algorithm}{chapter}% algorithm counter resets every chapter
\makeatother
\renewcommand{\thealgorithm}{\thechapter.\arabic{algorithm}}% Algorithm # is <chapter>.<algorithm>

\providecommand\algorithmname{algorithm}
\captionsetup[ruled]{font=small,labelfont={bf},labelsep=quad}

\newcommand{\tempcaption}{}% stores the caption
\newcommand{\templabel}{}% stores the label

\newenvironment{customalgo}[3][0.7\textwidth]
{%
    \begin{minipage}{#1}
        \begin{algorithm}[H]
            \centering
            \gdef\tempcaption{#2}% store the caption so we can use it later
            \gdef\templabel{#3}% store the label so we can use it later
            \begin{algorithmic}[1]
            }%
            {%
            \end{algorithmic}
            \caption{\tempcaption}% use the stored caption
            \label{\templabel}
        \end{algorithm}
    \end{minipage}
    % \smallskip
}%

% extent of line-spacing in algorithms
\let\Algorithm\algorithm
\renewcommand\algorithm[1][]{\Algorithm[#1]\setstretch{1.2390625}}

% % https://tex.stackexchange.com/questions/64674/coloring-lines-in-an-algorithm
% \makeatletter
% \newcommand{\algcolor}[2]{%
%     \hskip-\ALG@thistlm\colorbox{#1}{\parbox{\dimexpr\linewidth-2\fboxsep}{\hskip\ALG@thistlm\relax #2}}%
% }
% \newcommand{\algemph}[1]{\algcolor{cbrewerintergray}{#1}}
% \makeatother


%  https://tex.stackexchange.com/questions/64674/coloring-lines-in-an-algorithm
\makeatletter
% code borrowed from Andrew Stacey; See
% https://tex.stackexchange.com/a/50054/3954
\tikzset{%
    remember picture with id/.style={%
        remember picture,
        overlay,
        save picture id=#1,
    },
    save picture id/.code={%
        \edef\pgf@temp{#1}%
        \immediate\write\pgfutil@auxout{%
        \noexpand\savepointas{\pgf@temp}{\pgfpictureid}}%
    },
    if picture id/.code args={#1#2#3}{%
        \@ifundefined{save@pt@#1}{%
            \pgfkeysalso{#3}%
            }{
            \pgfkeysalso{#2}%
        }
    }
}

\def\savepointas#1#2{%
    \expandafter\gdef\csname save@pt@#1\endcsname{#2}%
}

\def\tmk@labeldef#1,#2\@nil{%
    \def\tmk@label{#1}%
    \def\tmk@def{#2}%
}

\tikzdeclarecoordinatesystem{pic}{%
    \pgfutil@in@,{#1}%
    \ifpgfutil@in@%
        \tmk@labeldef#1\@nil
    \else
        \tmk@labeldef#1,(0pt,0pt)\@nil
    \fi
    \@ifundefined{save@pt@\tmk@label}{%
        \tikz@scan@one@point\pgfutil@firstofone\tmk@def
        }{%
        \pgfsys@getposition{\csname save@pt@\tmk@label\endcsname}\save@orig@pic%
        \pgfsys@getposition{\pgfpictureid}\save@this@pic%
        \pgf@process{\pgfpointorigin\save@this@pic}%
        \pgf@xa=\pgf@x
        \pgf@ya=\pgf@y
        \pgf@process{\pgfpointorigin\save@orig@pic}%
        \advance\pgf@x by -\pgf@xa
        \advance\pgf@y by -\pgf@ya
    }%
}

\makeatother
% end of Andrew's code

% main command to draw the colored background
\newcounter{mymark}
\newcommand\ColorLine{%
    \stepcounter{mymark}%
    \tikz[remember picture with id=mark-\themymark,overlay] {;}%
    \begin{tikzpicture}[remember picture,overlay]%
        \filldraw[cbrewerintergray]%
            let \p1=(pic cs:mark-\themymark),
            \p2=(current page.east)  in
            ([xshift=-0.1em,yshift=-1.0ex]0,\y1)  rectangle ++([xshift=-2.525cm]\x2,\baselineskip);
    \end{tikzpicture}%
}%



 % for typesetting of algorithms
%!TEX root = ../main.tex
% vim:nospell

% \addbibresource{backmatter/thesis_refs.bib}
\ShellEscape{biber --tool backmatter/thesis_refs.bib} % deduplicate bibtex entries
\addbibresource{backmatter/thesis_refs_bibertool.bib}
\addbibresource{backmatter/thesis_refs_joint.bib}

% https://brianbuccola.com/how-to-cite-in-latex-without-the-citation-appearing-in-the-bibliography/
\DeclareBibliographyCategory{ignore}
\addtocategory{ignore}{Gopalakrishnan2018joint}

\renewcommand*{\bibfont}{\small} % make Bibliography left aligned, not justified

\DeclareSourcemap{
    \maps[datatype=bibtex]{
        \map{
            \step[fieldsource=doi,final]
            \step[fieldset=url,null]
        }
    }
}

\DefineBibliographyStrings{english}{%
    backrefpage = {cited on page},% originally "cited on page"
    backrefpages = {cited on pages},% originally "cited on pages"
}

\xpatchbibmacro{pageref}{parens}{brackets}{}{}   % helpful to replace parens with brackets for backreferencing with biblatex % needs xpatch

% \renewbibmacro*{pageref}{\iflistundef{pageref}{}{\printtext[brackets]{\printlist[​pageref][-\value{listtotal}]{pageref}}}}

\DeclareSourcemap{
    \maps[datatype=bibtex]{
        \map[overwrite]{
            \step[fieldsource=month, match=\regexp{\Ajan\Z}, replace=1]
            \step[fieldsource=month, match=\regexp{\Afeb\Z}, replace=2]
            \step[fieldsource=month, match=\regexp{\Amar\Z}, replace=3]
            \step[fieldsource=month, match=\regexp{\Aapr\Z}, replace=4]
            \step[fieldsource=month, match=\regexp{\Amay\Z}, replace=5]
            \step[fieldsource=month, match=\regexp{\Ajun\Z}, replace=6]
            \step[fieldsource=month, match=\regexp{\Ajul\Z}, replace=7]
            \step[fieldsource=month, match=\regexp{\Aaug\Z}, replace=8]
            \step[fieldsource=month, match=\regexp{\Asep\Z}, replace=9]
            \step[fieldsource=month, match=\regexp{\Aoct\Z}, replace=10]
            \step[fieldsource=month, match=\regexp{\Anov\Z}, replace=11]
            \step[fieldsource=month, match=\regexp{\Adec\Z}, replace=12]
        }
    }
}


% https://tex.stackexchange.com/questions/113039/trying-to-suppress-urls-with-biblatex-using-a-simple-persons-method
% \AtEveryCitekey{\clearfield{url}}

% https://tex.stackexchange.com/questions/46787/is-there-a-way-to-prevent-urls-from-appearing-in-biblatex-citations
\AtEveryCitekey{%
    \clearfield{url}%
    \clearfield{urlyear}%
    \clearfield{doi}%
}%
\renewbibmacro*{in:}{}


% https://tex.stackexchange.com/questions/24979/citing-authors-full-name-in-biblatex
\DeclareCiteCommand*{\citeauthor}
{\defcounter{maxnames}{99}%
    \defcounter{minnames}{99}%
    \defcounter{uniquename}{2}%
    \boolfalse{citetracker}%
    \boolfalse{pagetracker}%
\usebibmacro{prenote}}
{\ifciteindex{\indexnames{labelname}}{}%
    \printnames{labelname}}
    {\multicitedelim}
    {\usebibmacro{postnote}}


% \renewcommand*{\mkbibnamefirst}[1]{\edef\firstname{#1}\expandafter\first{\firstname}}
\renewcommand*{\mkbibnamegiven}[1]{\edef\firstname{#1}\expandafter\first{\firstname}}

\def\bibnamedelima{ }%
\def\bibnamedelimb{ }%

\makeatletter
\def\@empty{}
\def\first#1{\expandafter\@first#1 \@nil}
\def\@first#1 #2\@nil{#1\addspace%
  \if\relax\detokenize{#2}\relax\else\@initials#2\@nil\fi}
\def\initials#1{\expandafter\@initials#1 \@nil}
\def\@initials#1 #2\@nil{%
  \initial{#1}%
  \def\NextName{#2}%
  \ifx\@empty\NextName\relax%
  \else\bibinitdelim \@initials#2\@nil\fi}
\def\initial#1{\expandafter\@initial#1\@nil}
\def\@initial#1#2\@nil{#1\bibinitperiod}
\makeatother


% \makeatletter
% \DeclareCiteCommand{\longfullcite}
% {\usebibmacro{prenote}}
% {\usedriver
%     {\c@maxnames\blx@maxbibnames\relax
%     \DeclareNameAlias{sortname}{default}}
% {\thefield{entrytype}}}
% {\multicitedelim}
% {\usebibmacro{postnote}}
% \makeatother

% \newcommand{\longfullcite}{%
%     \AtNextCite{\defcounter{maxnames}{99}}%
% \fullcite}

\newcommand{\printpublication}[1]{\AtNextCite{\defcounter{maxnames}{99}}\fullcite{#1}}

\DeclareBibliographyAlias{software}{online}

% https://tex.stackexchange.com/questions/468623/indicating-joint-first-authorship-through-special-markup-in-biblatex-biber/468634#468634
\renewcommand*{\mkbibnamefamily}[1]{%
    \ifitemannotation{jointfirst}
        {\uline{\textbf{#1}}}
    {#1}}

\renewcommand*{\mkbibnamegiven}[1]{%
    \ifitemannotation{jointfirst}
        {\uline{\textbf{#1}}}
    {#1}}
     % 'bib' file goes in here
%!TEX root = ../main.tex
% vim:nospell ft=tex

\DeclareGraphicsExtensions{.pdf, .png, .jpg, .jpeg} % GIF doesn't work??

%!TEX root = ../main.tex
% vim:nospell ft=tex

\renewcommand{\CancelColor}{\color{imperialbrick}}
     % uses a predefined color
%!TEX root = ../main.tex
% vim:nospell ft=tex

\crefname{listing}{\MakeLowercase{\listingname}}{\MakeLowercase{\listingname s}}
\crefname{appchap}{appendix}{appendices}
\crefname{filePrg}{listing}{listings}
\Crefname{filePrg}{Listing}{Listings}

\newcommand{\crefrangeconjunction}{--}
\crefrangeformat{equation}{eqs.~(#3#1#4)--(#5#2#6)}


%!TEX root = ../main.tex
% vim:nospell ft=tex

% \diffset[p-delims = . |, p-nudge = 0]

\diffdef { p }
{
    op-symbol = \partial ,
    left-delim = \left .,
    right-delim = \right | ,
    subscr-nudge = 0 mu
}

%!TEX root = ../main.tex
% vim:nospell ft=tex

\setlist[enumerate,itemize,description]{topsep=0em}

\newcounter{descriptcount}
\newlist{enumdescriptnum}{description}{1}
\setlist[enumdescriptnum] {%
    before={\setcounter{descriptcount}{0}%
    \renewcommand*\thedescriptcount{\alph{descriptcount}.}}
    ,font=\footnotesize{\bfseries\stepcounter{descriptcount}\thedescriptcount~}
}


\newcommand{\customenum}[2]{
\item[$ \symbf{#1}\  (\times #2)$]
}


%!TEX root = ../main.tex
% vim:nospell ft=tex

\newcommand{\setFancyHdr}{
    \pagestyle{fancy}
    \renewcommand{\chaptermark}[1]{\markboth{\MakeUppercase{\thechapter. ##1 }}{}}
    \renewcommand{\sectionmark}[1]{\markright{\thesection\ ##1}}
    \fancyhf{}
    \fancyhead[R]{\bfseries\rightmark}
    \fancyfoot[C]{\thepage}
    \fancypagestyle{plain}{
        \fancyhead{}
        \renewcommand{\headrulewidth}{0pt}
    }
}

\setlength{\headheight}{14.5pt}
\setFancyHdr

\renewcommand{\headrulewidth}{\iffloatpage{0pt}{0.4pt}}


%!TEX root = ../main.tex
% vim:nospell ft=tex

\fxsetup{theme=color, marginface=\singlespacing \scriptsize}

\definecolor{fxnote}{RGB}{165,25,0} % imperialbrick (duplicating the color-line here, since `fxnote' is the specific color-name hard-coded by the fxnote package)

%!TEX root = ../main.tex
% vim:nospell ft=tex

% https://tex.stackexchange.com/questions/32819/draw-box-with-colored-background-and-linebreaks-which-adjusts-to-the-text-width
\newcommand\MyCBox[1]{%
      \colorbox{cbrewerfourclssor}{\begin{varwidth}{\dimexpr\linewidth-2\fboxsep}#1\end{varwidth}}}

\renewcommand{\gitMarkFormat}{\color{imperialraspberry} \small}
\renewcommand{\gitMark}{Author: {Krishnakumar Gopalakrishnan}, Imperial College London\, \textbullet{}\, \MyCBox{Confidential (Under Embargo)} \\ Typeset by \StrBehind{\luatexbanner}{\detokenize{This is}}{}  engine using the \LaTeX{} macro format}
\renewcommand{\gitMarkPref}{PhD Thesis}


%!TEX root = ../main.tex
% vim:nospell ft=tex

% \renewcommand*{\glstextformat}[1]{\textsf{#1}}
\renewcommand*{\glstextformat}[1]{\textcolor{black}{#1}} % link coloring to match normal text, ie black
\preto\chapter{\glsresetall} % expand acronyms every chapter https://tex.stackexchange.com/questions/435617/glossaries-expand-acronyms-for-first-time-use-within-each-chapter/435680#435680

% \glsdisablehyper
% \newglossary[slg]{symbolslist}{syi}{syg}{Symbols}

% \newglossarystyle{custom_acronyms}
% {
%     \setglossarystyle{long3colheader}%
%     \renewcommand*{\glossaryheader}{}%
%     \renewcommand{\glossentry}[2]{%
%         \textbf{\glsentryitem{##1}\glstarget{##1}{\glossentryname{##1}}}
%         & \glossentrydesc{##1}
%         & {\hspace*{\fill} ##2}
%     \tabularnewline}%
% }

% \renewcommand{\glossarypreamble}{\footnotesize}
% \renewcommand{\glossarypreamble}{\small}
\setglossarypreamble[acronym]{\small}
\setglossarypreamble[symbols]{\normalsize}
% \renewcommand\glstreepredesc{\qquad}

% https://tex.stackexchange.com/questions/118182/selectively-turn-off-hyperref-links-for-citations
\newcommand*{\nolink}[1]{%
      {\protect\NoHyper#1\protect\endNoHyper}%
  }

\glssetcategoryattribute{acronym}{nohyperfirst}{true} % no hyperlink on first use for entries with category=acronym % https://tex.stackexchange.com/questions/434160/line-break-long-glossaries-entry-when-using-hyperref-and-latex-dvips-ps2pdf
\setabbreviationstyle[acronym]{long-short} % applicable only for glossaries-extra.sty

\GlsXtrLoadResources[
src={frontmatter/acronyms},%
selection={all},
type=acronym,% put these entries in the 'acronym' glossary
save-locations=false% don't save locations
]

% assign titles to group labels:
\glsxtrsetgrouptitle{latin}{List of Latin Symbols}
\glsxtrsetgrouptitle{greek}{List of Greek Symbols}
\glsxtrsetgrouptitle{generic}{Other Generic Nomenclature}

\GlsXtrLoadResources[
src={frontmatter/latin-symbols},
sort={letter-lowerupper},
type=symbols,
selection={all},
category={same as entry},% set the category to the entry type
group={latin},% assign group label
set-widest,% needed for 'alttree' styles
save-locations=false
]

\GlsXtrLoadResources[
src={frontmatter/greek-symbols},% entries in 'greek-symbols.bib'
type=symbols,% put these entries in the 'symbols' glossary
selection={all},
category={same as entry},% set the category to the entry type
group={greek},% assign group label
set-widest,% needed for 'alttree' styles
save-locations=false% don't save locations
]

\GlsXtrLoadResources[
src={frontmatter/generic-symbols},
sort={en-GB},
type=symbols,
selection={all},
category={same as entry},% set the category to the entry type
group={generic},% assign group label
set-widest,% needed for 'alttree' styles
save-locations=false
]

\renewcommand{\glsnamefont}[1]{\textbf{#1}} % to bold-face acronym header with style=super
\glssetcategoryattribute{symbol}{glossdescfont}{normalsize} % https://tex.stackexchange.com/questions/303888/different-formatting-for-acronyms-and-glossary-entries


%!TEX root = ../main.tex
% vim:nospell ft=tex

% https://tex.stackexchange.com/questions/304449/combine-minted-and-tcolorbox-for-code-from-file-inputminted/304691#304691
% https://tex.stackexchange.com/questions/304449/combine-minted-and-tcolorbox-for-code-from-file-inputminted/304975#304975
\newcounter{filePrg}

\newtcbinputlisting[use counter=filePrg,number within=chapter,list inside=mypyg]{\matlabcodelisting}[3][]{%
    listing engine=minted,
    minted language=matlab,
    % minted style=algol_nu, % xcode,emacs, perldoc, pastie, borland, vs, vim, tango
    listing file={#2},
    title=\small{\textbf{Listing \thetcbcounter}\quad {#1}},
    % fonttitle=\bfseries,
    listing only,
    list entry={\protect\numberline{\thetcbcounter} #1},
    enhanced jigsaw,
    breakable,
    % drop fuzzy shadow,
    minted options={
        fontsize=\scriptsize,
        breaklines,
        autogobble,
        linenos,
        numbersep=3mm,
        mathescape,
        baselinestretch=1,
        breakanywhere=true
    },
    % colback=offwhite,
    colback=white,
    colframe=imperialnavy,
    % colframe=black,
    % coltitle=white,
    % boxrule=0.2mm,
    left=5mm,
    overlay={\begin{tcbclipinterior}
            \fill[imperiallightgray] (frame.south west) rectangle ([xshift=5mm]frame.north west);
        \end{tcbclipinterior}
    },
    label=#3
}

\pretocmd{\chapter}{\addtocontents{mypyg}{\addvspace{10pt}}}{}{}

\makeatletter % no indent for entries
\renewcommand{\l@tcolorbox}{\@dottedtocline{1}{0pt}{2.3em}}
\makeatother


\SetupFloatingEnvironment{listing}{name=Code snippet} % needs new float package

%!TEX root = ../main.tex
% vim:nospell ft=tex

\WarningFilter{latex}{Marginpar on page}

%!TEX root = ../main.tex
% vim:nospell ft=tex

\sisetup{
    locale = UK ,
    per-mode = reciprocal-positive-first,
    binary-units = true
}
\sisetup{range-phrase=--}
\sisetup{range-units=single}

\DeclareSIUnit \amphour { Ah }
\DeclareSIUnit \watthour { Wh }

%!TEX root = ../main.tex
% vim:nospell ft=tex

% \titleformat*{\section}{\large\bfseries}
\titleformat{\section}{\normalfont\fontsize{16}{21}\bfseries}{\thesection}{1em}{}


% \titlespacing*{<command>}{<left>}{<before-sep>}{<after-sep>}
% before =  1.4453125 canonical for chapter 2 litt review
% \titlespacing*{\section} {0pt}{1.4453125ex plus 0.7225ex minus 0.7225ex}{0ex}
\titlespacing*{\section}{0pt}{1ex plus 0.5ex minus 0.5ex}{0.22ex plus 0.11ex minus 0.11ex}

\titlespacing*{\subsection}{0pt}{1ex plus 0.5ex minus 0.5ex}{0.22ex plus 0.11ex minus 0.11ex}

\titlespacing*{\subsubsection}{0pt}{0.5ex plus 0.25ex minus 0.25ex}{0.11ex plus 0.055ex minus 0.055ex}


%!TEX root = ../main.tex
% vim:nospell ft=tex

\setlength{\ULdepth}{4.0pt}
\renewcommand{\ULthickness}{0.75pt}


%!TEX root = ../main.tex
% vim:nospell ft=tex

\WithArrowsOptions{displaystyle,tikz={font={\scriptsize}}}


%!TEX root = ../main.tex
% vim:nospell ft=tex


\newcolumntype{P}[1]{>{\RaggedRight\hspace{0pt}}p{#1}}
% \newcolumntype{R}[1]{>{\raggedleft\let\newline\\\arraybackslash\hspace{0pt}}m{#1}}
\newcolumntype{C}[1]{>{\centering\arraybackslash}p{#1}}

\makeatletter

\newcommand*{\@rowstyle}{}

\newcommand*{\rowstyle}[1]{% sets the style of the next row
    \gdef\@rowstyle{#1}%
    \@rowstyle\ignorespaces%
}

\newcolumntype{=}{% resets the row style
    >{\gdef\@rowstyle{}}%
}

\newcolumntype{+}{% adds the current row style to the next column
    >{\@rowstyle}%
}

\makeatother


%!TEX root = ../main.tex
% vim:nospell ft=tex

\usetikzlibrary{calc}
\usetikzlibrary{decorations.pathreplacing}

\newcommand{\tikzmark}[1]{\tikz[overlay,remember picture] \node (#1) {};}

% Tweak these as necessary
\newcommand*{\BraceAmplitude}{0.5em}%
\newcommand*{\BraceAspect}{0.5}%
\newcommand*{\VerticalOffset}{3.0ex}%
\newcommand*{\HorizontalOffset}{0.0em}%


\NewDocumentCommand{\InsertUnderBrace}{%
    O{} % #1 = draw options
    O{} % #2 = optional brace options
    m   % #3 = left tikzmark
    m   % #4 = right tikzmark
    m   % #5 = text to place underbrace
    }{%
    \begin{tikzpicture}[overlay,remember picture]
        \draw [decoration={brace, amplitude=\BraceAmplitude, aspect=\BraceAspect, #2}, decorate, thick, draw=blue, text=black, #1]
            ($(#4)+(\HorizontalOffset,-\VerticalOffset)$) --
            ($(#3)+(-\HorizontalOffset,-\VerticalOffset)$)
            node [below=\VerticalOffset, midway] {#5};
    \end{tikzpicture}%
}%

%!TEX root = ../main.tex
% vim:nospell ft=tex

% these custom commands  are general purpose definitions that are  suitable in a
% typical scientific  document. Some  of them  are pure  latex while  others use
% external packages

%---------- for text and other typographical elements ----------%
\newcommand{\eg}{\textit{e}.\textit{g}.}
\newcommand{\etal}{\textit{et~al}.}
\newcommand{\ie}{\textit{i}.\textit{e}.,}
\newcommand{\viz}{\textit{viz}. }

\def\Vhrulefill{\leavevmode\leaders\hrule height 0.7ex depth \dimexpr0.4pt-0.7ex\hfill\kern0pt}
\newcommand*{\xdash}[1][3em]{\rule[0.5ex]{#1}{0.55pt}}

\setlength\parskip{0.75\baselineskip plus0.1\baselineskip  minus0.1\baselineskip}

% https://tex.stackexchange.com/questions/23487/how-can-i-get-roman-numerals-in-text
\makeatletter
\newcommand*{\romanletter}[1]{\expandafter\@slowromancap\romannumeral #1@}
\makeatother


%---------- inline/display math macros ----------%
\newcommand*\mean[1]{\overline{#1}}

\DeclarePairedDelimiter\abs{\lvert}{\rvert}
\DeclarePairedDelimiter\ceil{\lceil}{\rceil}
\DeclarePairedDelimiter\floor{\lfloor}{\rfloor}

\let\mathbbalt\mathbb  % unicode-math changes mathbb to mathbbalt by default % https://tex.stackexchange.com/questions/360607/unicode-math-but-ordinary-blackboard-bold/360637#360637

% ---------- 'increasing spacing between matrix rows' -------------------- %
% https://tex.stackexchange.com/questions/14071/how-can-i-increase-the-line-spacing-in-a-matrix

% a redefinition of an internal amsmath LaTeX macro for customizing line spacing
% in  specific matrices  arbitrarily  as  desired: After  putting  this in  your
% preamble, you can write \begin{pmatrix}[1.5] vary  the value as you like, with
% pmatrix, vmatrix, bmatrix  and alike, or use it without  the optional argument
% as usually.

\makeatletter
\renewcommand*\env@matrix[1][\arraystretch]{%
    \edef\arraystretch{#1}%
    \hskip -\arraycolsep
    \let\@ifnextchar\new@ifnextchar
\array{*\c@MaxMatrixCols c}}
\makeatother
% ---------- end of 'increasing spacing between matrix rows' -------------------- %

% ---------- 'section/chapter/title/footnote handling' ---------- %

% improved handling of sectioning commands with titlesec
\setcounter{secnumdepth}{3} % organisational level that receives a numbers
\setcounter{tocdepth}{3}    % print table of contents for level 3

% needs the anyfontsize package
\renewcommand{\chaptername}{} % uncomment to print only "1" not "Chapter 1"
\titleformat{\chapter}[display]
{\bfseries\sffamily\Huge}
{\hfill\fontsize{140}{50}\selectfont\color{lightgray}\rmfamily\textbf{\thechapter}}% label
{-0ex}
{\filleft\fontsize{50}{50}}
[\vspace{-0ex}]

% \setlength{\columnsep}{20pt} % space between columns in two column mode; default 10pt quite narrow

\renewcommand{\footnoterule}{%
    \kern -3pt
    \hrule width 0.25\textwidth height 0.5pt
    \kern 2pt
}

\renewcommand{\thefootnote}{\fnsymbol{footnote}}

% ---------- end of 'section/chapter/title/footnote handling' ---------- %


% \newenvironment{mycenter}[1][\topsep]
% {\setlength{\topsep}{#1}\par\kern\topsep\centering}% \begin{mycenter}[<len>]
% {\par\kern\topsep}% \end{mycenter}

% not sure why this is being used
\makeatletter
\global\let\tikz@ensure@dollar@catcode=\relax
\makeatother

% To continue roman numbering at the end of the book (for bibliography, appendices etc.)

% \makeatletter
% \newcounter{savepagenumber}
% \renewcommand\mainmatter{%
%     \cleardoublepage
%     \setcounter{savepagenumber}{\value{page}}
%     \@mainmattertrue
%     \pagenumbering{arabic}%
% }
% \renewcommand\backmatter{%
%     \if@openright
%         \cleardoublepage
%     \else
%         \clearpage
%     \fi
%     \pagenumbering{roman}%
%     \setcounter{page}{\value{savepagenumber}}%
%     \@mainmatterfalse
% }
% \makeatother

%!TEX root = ../main.tex
% vim:nospell ft=tex

\makeatletter
\newcommand{\mathleft}{\@fleqntrue\@mathmargin0pt}
\newcommand{\mathcenter}{\@fleqnfalse}
\makeatother

\makeatletter\let\percentchar\@percentchar\makeatother
\directlua{
    % define a function that prints 2-letter ordinal strings
    function myord ( n )     % n: some positive number
    m = n \percentchar 100 % m = n modulo 100
    if m>3 and m<21 then tex.sprint ( "th" )
    elseif m \percentchar 10 == 1 then tex.sprint ( "st" )
    elseif m \percentchar 10 == 2 then tex.sprint ( "nd" )
    elseif m \percentchar 10 == 3 then tex.sprint ( "rd" )
    else tex.sprint ( "th" )
    end
    end
}
\newcommand\myord[1]{\directlua{myord(#1)}} % LaTeX "wrapper macro"
\newcommand{\ordfrac}[2]{\nicefrac{#1}{#2\textsuperscript{\myord{#2}}}}


%!TEX root = ../main.tex
% vim:nospell ft=tex

% https://tex.stackexchange.com/questions/75215/automating-the-height-of-a-drop-cap-initial/75218#75218
% using tikz instead of lettrine package
\makeatletter

% \RequirePackage{tikz}

\newlength\CLett% Nuova dimensione

\newcommand*\capolettera[2]{% #1 lettera da ingrandire #2 testo in maiuscoletto
    \par\noindent
    \setbox8\hbox{\textsc{#2}}%
    \setbox\z@\hbox{%
        \resizebox{!}{\dimexpr\baselineskip+\ht8\relax}{%
        \huge\color{sepiadvipsnames}#1}%
    }%
    \CLett=\wd\z@\hangindent\CLett\hangafter-2\relax%
\raisebox{-\baselineskip}[0pt][0pt]{\llap{\box\z@\kern1pt}}{\box8}}

\makeatother
% other choices for lettrine are
% https://tex.stackexchange.com/questions/769/how-can-i-create-documents-in-latex-using-a-calligraphic-first-letter-for-chapte/10260#10260
% https://tex.stackexchange.com/questions/250474/how-to-use-fancy-dropcaps-with-pdflatex
% https://tex.stackexchange.com/questions/38108/how-to-increase-the-size-of-first-character-in-a-chapter-drop-caps/38111#38111
% https://tex.stackexchange.com/questions/145490/how-to-get-libertine-initials-to-work-with-lettrine

%!TEX root = ../main.tex
% vim:nospell ft=tex

\renewcommand{\topfraction}{.85}
\renewcommand{\bottomfraction}{.7}
\renewcommand{\textfraction}{.15}
\renewcommand{\floatpagefraction}{.66}
\renewcommand{\dbltopfraction}{.66}
\renewcommand{\dblfloatpagefraction}{.66}
\setcounter{topnumber}{9}
\setcounter{bottomnumber}{9}
\setcounter{totalnumber}{20}
\setcounter{dbltopnumber}{9}

% https://tex.stackexchange.com/questions/50830/do-i-have-to-care-about-bad-boxes/50850#50850
\tolerance=1414
\hbadness=1414
\emergencystretch=1.5em
\hfuzz=0.5pt
\vfuzz=\hfuzz
\raggedbottom
\hyphenpenalty=750
\frenchspacing
\binoppenalty=1000 % default 700
\relpenalty=800     % default 500
\interfootnotelinepenalty=10000
% \clubpenalty=10000
% \widowpenalty=10000

\overfullrule=2cm



%---------- end custom commands ----------%

%%%%%%%%%% glossaries-related stuff %%%%%%%%%%%%%%%%%%%%
\makeglossaries % must be before starting to define any entries for the glossary. And, all new type of glossary (such as for symbols and any other custom glossaries) must be defined before this.
\glssetcategoryattribute{acronym}{nohyperfirst}{true} % no hyperlink on first use for entries with category=acronym % https://tex.stackexchange.com/questions/434160/line-break-long-glossaries-entry-when-using-hyperref-and-latex-dvips-ps2pdf
\setabbreviationstyle[acronym]{long-short} % applicable only for glossaries-extra.sty
% -*- root: ../../main.tex -*-
%!TEX root = ../../main.tex
% vim: nospell  tw=80

% Glossary entries are defined with the command \nomenclature{1}{2}
% 1 = Entry name, e.g. abbreviation; 2 = Explanation

% You can place  all explanations in this  separate file or declare  them in the
% middle of the text. Either way they will be collected in the glossary.

% required to print nomenclature name to page header
%\markboth{\MakeUppercase{\nomname}}{\MakeUppercase{\nomname}}


% ----------------------- contents from here ------------------------
% newacronym is actually the base glossaries.sty usage. For glossaries-extra, we need to use newabbreviation

\newacronym[firstplural={States of Charge (SOCs)}]{soc}{SOC}{State of Charge}
\newacronym[firstplural={Akaike's Information Criteria (AICs)}]{aic}{AIC}{Akaike's Information Criterion}

% \newacronym{dft}{DFT}{Density Functional Theory}
\newacronym{adc}{ADC}{Analog to Digital Converter}
\newacronym{aer}{AER}{All-Electric Range}
\newacronym{armax}{ARMAX}{Auto Regressive with Moving Average and Exogenous Inputs}
\newacronym{arx}{ARX}{Auto Regressive with Exogenous Inputs}
\newacronym{bev}{BEV}{Battery Electric Vehicle}
\newacronym{bms}{BMS}{Battery Management System}
\newacronym{bold}{BOLD}{Battery Optimal Layer Design}
\newacronym{cas}{CAS}{Computer Algebra System}
\newacronym{cccv}{CCCV}{Constant Current Constant Voltage}
\newacronym{cpu}{CPU}{Central Processing Unit}
\newacronym{cp}{CP}{Constant Power}
\newacronym{dac}{DAC}{Digital to Analog Converter}
\newacronym{dae}{DAE}{Differential Algebraic Equation}
\newacronym{dfn}{DFN}{Doyle-Fuller-Newman}
\newacronym{dft}{DFT}{Discrete Fourier Transform}
\newacronym{dmc}{DMC}{Dimethyl Carbonate}
\newacronym{dra}{DRA}{Discrete-Time Realisation Algorithm}
\newacronym{ecm}{ECM}{Equivalent Circuit Model}
\newacronym{ec}{EC}{Ethylene Carbonate}
\newacronym{eis}{EIS}{Electrochemical Impedance Spectroscopy}
\newacronym{emc}{EMC}{Ethyl Methyl Carbonate}
\newacronym{era}{ERA}{Eigensystem Realisation Algorithm}
\newacronym{etfe}{ETFE}{Empirical Transfer Function Estimate}
\newacronym{fdm}{FDM}{Finite Difference Method}
\newacronym{fft}{FFT}{Fast Fourier Transform}
\newacronym{frf}{FRF}{Frequency Response Function}
\newacronym{fv}{FV}{Finite Volume}
\newacronym{gitt}{GITT}{Galvanostatic Intermittent Titration Technique}
\newacronym{gpm}{GPM}{Galerkin Projection Method}
\newacronym{ice}{ICE}{Internal Combustion Engine}
\newacronym{ima}{IMA}{Integral Method Approximation}
\newacronym{isr}{ISR}{Interrupt Service Routine}
\newacronym{ivp}{IVP}{Initial Value Problem}
\newacronym{iv}{IV}{Instrumental Variable}
\newacronym{lco}{LCO}{Lithium Cobalt Oxide}
\newacronym{lfp}{LFP}{Lithium Iron Phosphate}
\newacronym{lhs}{LHS}{Left-Hand Side}
\newacronym{lti}{LTI}{Linear Time-Invariant}
\newacronym{mae}{MAE}{Mean Absolute Error}
\newacronym{mdl}{MDL}{Rissanen's Minimum Description Length}
\newacronym{mggp}{MGGP}{Multi-Gene Genetic Programming}
\newacronym{mimo}{MIMO}{Multi Input Multi Output}
\newacronym{mpm}{MPM}{Markov Parameter Matrix}
\newacronym{ocp}{OCP}{Open Circuit Potential}
\newacronym{ocv}{OCV}{Open Circuit Voltage}
\newacronym{ode}{ODE}{Ordinary Differential Equation}
\newacronym{oe}{OE}{Output Error}
\newacronym{ols}{OLS}{Ordinary Least Squares}
\newacronym{p2d}{P2D}{Pseudo Two-Dimensional}
\newacronym{pbm}{PBM}{Physics-Based Model}
\newacronym{pdae}{PDAE}{Partial Differential Algebraic System}
\newacronym{pde}{PDE}{Partial Differential Equation}
\newacronym{phev}{PHEV}{Plug-in Hybrid Electric Vehicle}
\newacronym{pmsm}{PMSM}{Permanent Magnet Synchronous Motor}
\newacronym{pp2d}{PP2D}{Polynomial Profile P2D}
\newacronym{prbs}{PRBS}{Pseudorandom Binary Signal}
\newacronym{pwl}{PWL}{Piecewise Linear}
\newacronym{qss}{QSS}{Quasi-Steady State}
\newacronym{ram}{RAM}{Random Access Memory}
\newacronym{rbs}{RBS}{Random Binary Signal}
\newacronym{rgs}{RGS}{Random Gaussian Signal}
\newacronym{rhs}{RHS}{Right-Hand Side}
\newacronym{rms}{RMS}{Root Mean Square}
\newacronym{rom}{ROM}{Reduced Order Model}
\newacronym{sei}{SEI}{Solid-Electrolyte Interphase}
\newacronym{simo}{SIMO}{Single Input Multiple Output}
\newacronym{siso}{SISO}{Single Input Single Output}
\newacronym{snr}{SNR}{Signal to Noise Ratio}
\newacronym{soap}{SOAP}{State of Available Power}
\newacronym{sov}{SoV}{Separation of Variables}
\newacronym{spdt}{SPDT}{Single Pole Double Throw}
\newacronym{spm}{SPM}{Single Particle Model}
\newacronym{ssa}{SSA}{Singular Spectrum Analysis}
\newacronym{svd}{SVD}{Singular Value Decomposition}
\newacronym{udds}{UDDS}{Urban Dynamometer Driving Schedule}
\newacronym{vop}{VOP}{Variation of Parameters}
\newacronym{xeV}{xEV}{Battery Electric/Plug-in Hybrid Electric Vehicle}
\newacronym{zoh}{ZOH}{Zero Order Hold}

% Do not forget to describe basic definitions like C-rate etc.
%%%%%%%%%%%%%%%%%%%%%%%%%%%%%%%%%%%%%%%%%%%%%%%%%%%%%%%%%%%%%%%%%%%%%%%%%%%%%%%%

% \newglossaryentry{uppercase}{
%     name={Uppercase},
%     text={uppercase},
%     description={Appears uppercase in the glossary and lowercase in the text}
% }
% \newglossaryentry{x}{
% name=\ensuremath{\{x\}},
% description={Variable x.},
% sort=x, type=symbolslist
% }


