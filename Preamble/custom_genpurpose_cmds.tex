%!TEX root = ../main.tex
% vim:nospell ft=tex

% these custom commands  are general purpose definitions that are  suitable in a
% typical scientific  document. Some  of them  are pure  latex while  others use
% external packages

%---------- for text and other typographical elements ----------%
\newcommand{\eg}{\textit{e}.\textit{g}.}
\newcommand{\etal}{\textit{et~al}.}
\newcommand{\ie}{\textit{i}.\textit{e}.,}
\newcommand{\viz}{\textit{viz}. }

\def\Vhrulefill{\leavevmode\leaders\hrule height 0.7ex depth \dimexpr0.4pt-0.7ex\hfill\kern0pt}
\newcommand*{\xdash}[1][3em]{\rule[0.5ex]{#1}{0.55pt}}

\setlength\parskip{0.75\baselineskip plus0.1\baselineskip  minus0.1\baselineskip}

% https://tex.stackexchange.com/questions/23487/how-can-i-get-roman-numerals-in-text
\makeatletter
\newcommand*{\romanletter}[1]{\expandafter\@slowromancap\romannumeral #1@}
\makeatother


%---------- inline/display math macros ----------%
\newcommand*\mean[1]{\overline{#1}}

\DeclarePairedDelimiter\abs{\lvert}{\rvert}
\DeclarePairedDelimiter\ceil{\lceil}{\rceil}
\DeclarePairedDelimiter\floor{\lfloor}{\rfloor}

\let\mathbbalt\mathbb  % unicode-math changes mathbb to mathbbalt by default % https://tex.stackexchange.com/questions/360607/unicode-math-but-ordinary-blackboard-bold/360637#360637

% ---------- 'increasing spacing between matrix rows' -------------------- %
% https://tex.stackexchange.com/questions/14071/how-can-i-increase-the-line-spacing-in-a-matrix

% a redefinition of an internal amsmath LaTeX macro for customizing line spacing
% in  specific matrices  arbitrarily  as  desired: After  putting  this in  your
% preamble, you can write \begin{pmatrix}[1.5] vary  the value as you like, with
% pmatrix, vmatrix, bmatrix  and alike, or use it without  the optional argument
% as usually.

\makeatletter
\renewcommand*\env@matrix[1][\arraystretch]{%
    \edef\arraystretch{#1}%
    \hskip -\arraycolsep
    \let\@ifnextchar\new@ifnextchar
\array{*\c@MaxMatrixCols c}}
\makeatother
% ---------- end of 'increasing spacing between matrix rows' -------------------- %

% typeset a horizontal vector in in-line math
\stackMath
\ExplSyntaxOn
\NewDocumentCommand \vect { s o m }
{
    \IfBooleanTF {#1}
    { \vectaux*{#3} }
    { \IfValueTF {#2} { \vectaux[#2]{#3} } { \vectaux{#3} } }
    ^T
}
\DeclarePairedDelimiterX \vectaux [1] {\lbrack} {\rbrack}
{ \, \dbacc_vect:n { #1 } \, }
\cs_new_protected:Npn \dbacc_vect:n #1
{
    \seq_set_split:Nnn \l_tmpa_seq { , } { #1 }
    \seq_use:Nn \l_tmpa_seq { \enspace }
}
\ExplSyntaxOff



% ---------- 'section/chapter/title/footnote handling' ---------- %

% improved handling of sectioning commands with titlesec
\setcounter{secnumdepth}{3} % organisational level that receives a numbers
\setcounter{tocdepth}{3}    % print table of contents for level 3

% needs the anyfontsize package
\renewcommand{\chaptername}{} % uncomment to print only "1" not "Chapter 1"
\titleformat{\chapter}[display]
{\bfseries\sffamily\Huge}
{\hfill\fontsize{140}{50}\selectfont\color{lightgray}\rmfamily\textbf{\thechapter}}% label
{-0ex}
{\filleft\fontsize{50}{50}}
[\vspace{-0ex}]

% \setlength{\columnsep}{20pt} % space between columns in two column mode; default 10pt quite narrow

\renewcommand{\footnoterule}{%
    \kern -3pt
    \hrule width 0.25\textwidth height 0.5pt
    \kern 2pt
}

\renewcommand{\thefootnote}{\fnsymbol{footnote}}

% ---------- end of 'section/chapter/title/footnote handling' ---------- %


% \newenvironment{mycenter}[1][\topsep]
% {\setlength{\topsep}{#1}\par\kern\topsep\centering}% \begin{mycenter}[<len>]
% {\par\kern\topsep}% \end{mycenter}

% not sure why this is being used
\makeatletter
\global\let\tikz@ensure@dollar@catcode=\relax
\makeatother

% To continue roman numbering at the end of the book (for bibliography, appendices etc.)

% \makeatletter
% \newcounter{savepagenumber}
% \renewcommand\mainmatter{%
%     \cleardoublepage
%     \setcounter{savepagenumber}{\value{page}}
%     \@mainmattertrue
%     \pagenumbering{arabic}%
% }
% \renewcommand\backmatter{%
%     \if@openright
%         \cleardoublepage
%     \else
%         \clearpage
%     \fi
%     \pagenumbering{roman}%
%     \setcounter{page}{\value{savepagenumber}}%
%     \@mainmatterfalse
% }
% \makeatother
