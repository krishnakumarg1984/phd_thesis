%!TEX root = ../main.tex
% vim:nospell ft=tex

% these custom commands  are general purpose definitions that are  suitable in a
% typical scientific  document. Some  of them  are pure  latex while  others use
% external packages

%---------- for text and other typographical elements ----------%
\newcommand{\eg}{\textit{e}.\textit{g}.}
\newcommand{\etal}{\textit{et~al}.}
\newcommand{\ie}{\textit{i}.\textit{e}.,}
\newcommand{\viz}{\textit{viz}. }

\def\Vhrulefill{\leavevmode\leaders\hrule height 0.7ex depth \dimexpr0.4pt-0.7ex\hfill\kern0pt}

\setlength\parskip{0.75\baselineskip plus0.1\baselineskip  minus0.1\baselineskip}

\makeatletter
\newcommand*{\rom}[1]{\expandafter\@slowromancap\romannumeral #1@}
\makeatother

%---------- inline/display math macros ----------%
\newcommand*\mean[1]{\overline{#1}}

\DeclarePairedDelimiter\abs{\lvert}{\rvert}
\DeclarePairedDelimiter\ceil{\lceil}{\rceil}
\DeclarePairedDelimiter\floor{\lfloor}{\rfloor}

\let\mathbbalt\mathbb  % unicode-math changes mathbb to mathbbalt by default % https://tex.stackexchange.com/questions/360607/unicode-math-but-ordinary-blackboard-bold/360637#360637

% ---------- 'increasing spacing between matrix rows' -------------------- %
% https://tex.stackexchange.com/questions/14071/how-can-i-increase-the-line-spacing-in-a-matrix

% a redefinition of an internal amsmath LaTeX macro for customizing line spacing
% in  specific matrices  arbitrarily  as  desired: After  putting  this in  your
% preamble, you can write \begin{pmatrix}[1.5] vary  the value as you like, with
% pmatrix, vmatrix, bmatrix  and alike, or use it without  the optional argument
% as usually.

\makeatletter
\renewcommand*\env@matrix[1][\arraystretch]{%
    \edef\arraystretch{#1}%
    \hskip -\arraycolsep
    \let\@ifnextchar\new@ifnextchar
\array{*\c@MaxMatrixCols c}}
\makeatother
% ---------- end of 'increasing spacing between matrix rows' -------------------- %

% typeset a horizontal vector in in-line math
\stackMath
\ExplSyntaxOn
\NewDocumentCommand \vect { s o m }
{
    \IfBooleanTF {#1}
    { \vectaux*{#3} }
    { \IfValueTF {#2} { \vectaux[#2]{#3} } { \vectaux{#3} } }
    ^T
}
\DeclarePairedDelimiterX \vectaux [1] {\lbrack} {\rbrack}
{ \, \dbacc_vect:n { #1 } \, }
\cs_new_protected:Npn \dbacc_vect:n #1
{
    \seq_set_split:Nnn \l_tmpa_seq { , } { #1 }
    \seq_use:Nn \l_tmpa_seq { \enspace }
}
\ExplSyntaxOff



% ---------- 'section/chapter/title/footnote handling' ---------- %

% improved handling of sectioning commands with titlesec
\setcounter{secnumdepth}{3} % organisational level that receives a numbers
\setcounter{tocdepth}{3}    % print table of contents for level 3

% needs the anyfontsize package
\renewcommand{\chaptername}{} % uncomment to print only "1" not "Chapter 1"
\titleformat{\chapter}[display]
{\bfseries\sffamily\Huge}
{\hfill\fontsize{140}{50}\selectfont\color{lightgray}\rmfamily\textbf{\thechapter}}% label
{-0ex}
{\filleft\fontsize{50}{50}}
[\vspace{-0ex}]

% \setlength{\columnsep}{20pt} % space between columns in two column mode; default 10pt quite narrow

\renewcommand{\footnoterule}{%
    \kern -3pt
    \hrule width 0.25\textwidth height 0.5pt
    \kern 2pt
}

\renewcommand{\thefootnote}{\fnsymbol{footnote}}

% ---------- end of 'section/chapter/title/footnote handling' ---------- %


% \newenvironment{mycenter}[1][\topsep]
% {\setlength{\topsep}{#1}\par\kern\topsep\centering}% \begin{mycenter}[<len>]
% {\par\kern\topsep}% \end{mycenter}

% not sure why this is being used
\makeatletter
\global\let\tikz@ensure@dollar@catcode=\relax
\makeatother

\WarningFilter{latex}{Marginpar on page}

\DeclareSIUnit \amphour { Ah }
\DeclareSIUnit \watthour { Wh }
\setkeys{Gin}{width=0.75\textwidth} % default width of graphics

\renewcommand*{\bibfont}{\small} % make Bibliography left aligned, not justified

\DeclareSourcemap{
    \maps[datatype=bibtex]{
        \map{
            \step[fieldsource=doi,final]
            \step[fieldset=url,null]
        }
    }
}

\DefineBibliographyStrings{english}{%
    backrefpage = {cited on page},% originally "cited on page"
    backrefpages = {cited on pages},% originally "cited on pages"
}

\xpatchbibmacro{pageref}{parens}{brackets}{}{}   % helpful to replace parens with brackets for backreferencing with biblatex % needs xpatch

% \renewbibmacro*{pageref}{\iflistundef{pageref}{}{\printtext[brackets]{\printlist[​pageref][-\value{listtotal}]{pageref}}}}

% \titleformat*{\section}{\large\bfseries}
\titleformat{\section}{\normalfont\fontsize{16}{21}\bfseries}{\thesection}{1em}{}

\DeclareSourcemap{
    \maps[datatype=bibtex]{
        \map[overwrite]{
            \step[fieldsource=month, match=\regexp{\Ajan\Z}, replace=1]
            \step[fieldsource=month, match=\regexp{\Afeb\Z}, replace=2]
            \step[fieldsource=month, match=\regexp{\Amar\Z}, replace=3]
            \step[fieldsource=month, match=\regexp{\Aapr\Z}, replace=4]
            \step[fieldsource=month, match=\regexp{\Amay\Z}, replace=5]
            \step[fieldsource=month, match=\regexp{\Ajun\Z}, replace=6]
            \step[fieldsource=month, match=\regexp{\Ajul\Z}, replace=7]
            \step[fieldsource=month, match=\regexp{\Aaug\Z}, replace=8]
            \step[fieldsource=month, match=\regexp{\Asep\Z}, replace=9]
            \step[fieldsource=month, match=\regexp{\Aoct\Z}, replace=10]
            \step[fieldsource=month, match=\regexp{\Anov\Z}, replace=11]
            \step[fieldsource=month, match=\regexp{\Adec\Z}, replace=12]
        }
    }
}

\newcommand*{\xdash}[1][3em]{\rule[0.5ex]{#1}{0.55pt}}

\renewcommand{\ULdepth}{4.0pt}
\renewcommand{\ULthickness}{0.75pt}

\WithArrowsOptions{displaystyle,tikz={font={\scriptsize}}}

\usetikzlibrary{calc}
\usetikzlibrary{decorations.pathreplacing}

\newcommand{\tikzmark}[1]{\tikz[overlay,remember picture] \node (#1) {};}

% Tweak these as necessary
\newcommand*{\BraceAmplitude}{0.5em}%
\newcommand*{\BraceAspect}{0.5}%
\newcommand*{\VerticalOffset}{3.0ex}%
\newcommand*{\HorizontalOffset}{0.0em}%


\NewDocumentCommand{\InsertUnderBrace}{%
    O{} % #1 = draw options
    O{} % #2 = optional brace options
    m   % #3 = left tikzmark
    m   % #4 = right tikzmark
    m   % #5 = text to place underbrace
    }{%
    \begin{tikzpicture}[overlay,remember picture]
        \draw [decoration={brace, amplitude=\BraceAmplitude, aspect=\BraceAspect, #2}, decorate, thick, draw=blue, text=black, #1]
            ($(#4)+(\HorizontalOffset,-\VerticalOffset)$) --
            ($(#3)+(-\HorizontalOffset,-\VerticalOffset)$)
            node [below=\VerticalOffset, midway] {#5};
    \end{tikzpicture}%
}%

\makeatletter
\newcommand{\mathleft}{\@fleqntrue\@mathmargin0pt}
\newcommand{\mathcenter}{\@fleqnfalse}
\makeatother

% \newcommand{\ordfrac}[2]{\nicefrac{#1}{\textsuperscript{\engordnumber{#2}}}}

\makeatletter\let\percentchar\@percentchar\makeatother
\directlua{
    % define a function that prints 2-letter ordinal strings
    function myord ( n )     % n: some positive number
    m = n \percentchar 100 % m = n modulo 100
    if m>3 and m<21 then tex.sprint ( "th" )
    elseif m \percentchar 10 == 1 then tex.sprint ( "st" )
    elseif m \percentchar 10 == 2 then tex.sprint ( "nd" )
    elseif m \percentchar 10 == 3 then tex.sprint ( "rd" )
    else tex.sprint ( "th" )
    end
    end
}
\newcommand\myord[1]{\directlua{myord(#1)}} % LaTeX "wrapper macro"

% \newcommand{\myfracA}[2]{\nicefrac{#1}{#2}\textsuperscript{\myord{#2}}}
\newcommand{\ordfrac}[2]{\nicefrac{#1}{#2\textsuperscript{\myord{#2}}}}

% \glsdisablehyper
% \newglossary[slg]{symbolslist}{syi}{syg}{Symbols}

% \newglossarystyle{custom_acronyms}
% {
%     \setglossarystyle{long3colheader}%
%     \renewcommand*{\glossaryheader}{}%
%     \renewcommand{\glossentry}[2]{%
%         \textbf{\glsentryitem{##1}\glstarget{##1}{\glossentryname{##1}}}
%         & \glossentrydesc{##1}
%         & {\hspace*{\fill} ##2}
%     \tabularnewline}%
% }

% \renewcommand{\glossarypreamble}{\footnotesize}
\renewcommand{\glossarypreamble}{\small}

% \setcounter{minitocdepth}{3}

% https://tex.stackexchange.com/questions/75215/automating-the-height-of-a-drop-cap-initial/75218#75218
% using tikz instead of lettrine package
\makeatletter

% \RequirePackage{tikz}

\newlength\CLett% Nuova dimensione

\newcommand*\capolettera[2]{% #1 lettera da ingrandire #2 testo in maiuscoletto
    \par\noindent
    \setbox8\hbox{\textsc{#2}}%
    \setbox\z@\hbox{%
        \resizebox{!}{\dimexpr\baselineskip+\ht8\relax}{%
        \huge\color{sepiadvipsnames}#1}%
    }%
    \CLett=\wd\z@\hangindent\CLett\hangafter-2\relax%
\raisebox{-\baselineskip}[0pt][0pt]{\llap{\box\z@\kern1pt}}{\box8}}

\makeatother
% other choices for lettrine are
% https://tex.stackexchange.com/questions/769/how-can-i-create-documents-in-latex-using-a-calligraphic-first-letter-for-chapte/10260#10260
% https://tex.stackexchange.com/questions/250474/how-to-use-fancy-dropcaps-with-pdflatex
% https://tex.stackexchange.com/questions/38108/how-to-increase-the-size-of-first-character-in-a-chapter-drop-caps/38111#38111
% https://tex.stackexchange.com/questions/145490/how-to-get-libertine-initials-to-work-with-lettrine

% \newcolumntype{d}[1]{D{.}{.}{#1}}

\makeatletter
\newlength{\qrr@dimen@}
\expandafter\pretocmd\csname tabular*\endcsname{\setlength{\qrr@dimen@}{#1}}{}{}
\newcommand*{\Rowcolor}[2][\tabcolsep]{%
    \ifx\relax#1\relax\else
        \kern-\the\dimexpr#1\relax
    \fi
    \makebox[0pt][l]{%
        \fboxsep=0pt
        \colorbox{#2}{%
            \strut\kern\qrr@dimen@
        }%
    }%
    \ifx\relax#1\relax\else
        \kern\the\dimexpr#1\relax
    \fi
    \ignorespaces
}
\makeatother

\renewcommand{\topfraction}{.85}
\renewcommand{\bottomfraction}{.7}
\renewcommand{\textfraction}{.15}
\renewcommand{\floatpagefraction}{.66}
\renewcommand{\dbltopfraction}{.66}
\renewcommand{\dblfloatpagefraction}{.66}
\setcounter{topnumber}{9}
\setcounter{bottomnumber}{9}
\setcounter{totalnumber}{20}
\setcounter{dbltopnumber}{9}

% https://tex.stackexchange.com/questions/23487/how-can-i-get-roman-numerals-in-text
\makeatletter
\newcommand*{\romanletter}[1]{\expandafter\@slowromancap\romannumeral #1@}
\makeatother

% % https://tex.stackexchange.com/questions/34225/different-font-sizes-for-different-rows-in-table/34226
% \makeatletter
% \g@addto@macro{\endtabular}{\rowfont{}}% Clear row font
% \makeatother
% \newcommand{\rowfonttype}{}% Current row font
% \newcommand{\rowfont}[1]{% Set current row font
%     \gdef\rowfonttype{#1}#1%
% }
% \newcolumntype{L}{>{\rowfonttype}l}


% To continue roman numbering at the end of the book (for bibliography, appendices etc.)

% \makeatletter
% \newcounter{savepagenumber}
% \renewcommand\mainmatter{%
%     \cleardoublepage
%     \setcounter{savepagenumber}{\value{page}}
%     \@mainmattertrue
%     \pagenumbering{arabic}%
% }
% \renewcommand\backmatter{%
%     \if@openright
%         \cleardoublepage
%     \else
%         \clearpage
%     \fi
%     \pagenumbering{roman}%
%     \setcounter{page}{\value{savepagenumber}}%
%     \@mainmatterfalse
% }
% \makeatother

% https://tex.stackexchange.com/questions/1072/which-graphics-formats-can-be-included-in-documents-processed-by-latex-or-pdflat
% prepend pdf before png
% \ifpdf
%     \makeatletter
%     \let\orig@Gin@extensions\Gin@extensions
%     \def\Gin@extensions{.pdf,\orig@Gin@extensions} %prepend .pdf before .png
%     \makeatother
% \fi
