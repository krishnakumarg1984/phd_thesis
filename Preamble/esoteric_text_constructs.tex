%!TEX root = ../main.tex
% vim:nospell ft=tex

% https://tex.stackexchange.com/questions/75215/automating-the-height-of-a-drop-cap-initial/75218#75218
% using tikz instead of lettrine package
\makeatletter

% \RequirePackage{tikz}

\newlength\CLett% Nuova dimensione

\newcommand*\capolettera[2]{% #1 lettera da ingrandire #2 testo in maiuscoletto
    \par\noindent
    \setbox8\hbox{\textsc{#2}}%
    \setbox\z@\hbox{%
        \resizebox{!}{\dimexpr\baselineskip+\ht8\relax}{%
        \huge\color{sepiadvipsnames}#1}%
    }%
    \CLett=\wd\z@\hangindent\CLett\hangafter-2\relax%
\raisebox{-\baselineskip}[0pt][0pt]{\llap{\box\z@\kern1pt}}{\box8}}

\makeatother
% other choices for lettrine are
% https://tex.stackexchange.com/questions/769/how-can-i-create-documents-in-latex-using-a-calligraphic-first-letter-for-chapte/10260#10260
% https://tex.stackexchange.com/questions/250474/how-to-use-fancy-dropcaps-with-pdflatex
% https://tex.stackexchange.com/questions/38108/how-to-increase-the-size-of-first-character-in-a-chapter-drop-caps/38111#38111
% https://tex.stackexchange.com/questions/145490/how-to-get-libertine-initials-to-work-with-lettrine

\newcommand{\fakesection}[1]{%
    \par\refstepcounter{section}% Increase section counter
    \sectionmark{#1}% Add section mark (header)
    \addcontentsline{toc}{section}{\protect\numberline{\thesection}#1}% Add section to ToC
    % Add more content here, if needed.
}
