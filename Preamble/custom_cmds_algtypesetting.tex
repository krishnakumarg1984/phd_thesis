%!TEX root = ../main.tex
% vim:nospell ft=tex

% For unbroken lines in algorithmicx/algpseudocode when typesetting display math

% other alternative: https://tex.stackexchange.com/questions/110431/problems-with-vertical-lines-in-algorithmicx?noredirect=1&lq=1
% https://tex.stackexchange.com/questions/301462/why-are-vertical-rules-dashed-sometimes-with-algorithmic-package
%%%%%%%%%% https://tex.stackexchange.com/questions/350399/indentation-scope-lines-broken-in-algpseudocode%%%%%%%%%
\newcommand*{\algrule}[1][\algorithmicindent]{\hspace*{.2em}{\color{cbrewerdarkgray}\vrule\vrule
width 0pt height \baselineskip depth .1618\baselineskip\hspace*{\dimexpr#1-.5em}}}

\makeatletter
\newcount\ALG@printindent@tempcnta
\def\ALG@printindent{%
    \ifnum \theALG@nested>0% is there anything to print
        \ifx\ALG@text\ALG@x@notext% is this an end group without any text?
            % do nothing
    \else
        \unskip
        % draw a rule for each indent level
        \ALG@printindent@tempcnta=1
        \loop
        \algrule[\csname ALG@ind@\the\ALG@printindent@tempcnta\endcsname]%
        \advance \ALG@printindent@tempcnta 1
        \ifnum \ALG@printindent@tempcnta<\numexpr\theALG@nested+1\relax% can't do <=, so add one to RHS and use < instead
            \repeat
        \fi
    \fi
}%
% needs etoolbox, but this should have been already loaded with glossaries
\patchcmd{\ALG@doentity}{\noindent\hskip\ALG@tlm}{\ALG@printindent}{}{\errmessage{failed to patch}}
\makeatother

\AtBeginEnvironment{algorithmic}{\lineskip0pt}
%%%%%%%%%% https://tex.stackexchange.com/questions/350399/indentation-scope-lines-broken-in-algpseudocode%%%%%%%%%


\algnewcommand\algorithmicinput{\textbf{Initialise:}}
\algnewcommand\Initialise{\item[\algorithmicinput]}

\algnewcommand\algorithmicdata{\textbf{User data:}}
\algnewcommand\Userdata{\item[\algorithmicdata]}

\algnewcommand\algorithmicfulllinecomment{\qquad\quad  \scriptsize \textit{Note:}}
\algnewcommand\FullComment{\item[\algorithmicfulllinecomment]}

\makeatletter
\algrenewcommand\ALG@beginalgorithmic{\footnotesize}
\algrenewcommand\algorithmiccomment[2][\footnotesize]{{#1\hfill\(\triangleright\) #2}}
\makeatother

\algblockdefx[NAME]{ISR}{END}%
[2][Unknown]{\textbf{begin} \textproc{Interrupt Service Routine} #1(#2)}%
{\textbf{return} \Comment[\footnotesize]{resume suspended line in \textsc{Main()}}}

\algblockdefx[NAME]{OutputEqn}{EndOutputEqn}%
[2][\textbf{x}]{\textbf{subroutine} \textproc{ComputeCellVoltage}(#2)}%
{\textbf{return} $V_\text{cell}$ \Comment[\footnotesize]{resume suspended line in \textproc{Simulate\gls{spm}}}}

\newsavebox{\algboxA}
\newsavebox{\algboxB}

\makeatletter
\@addtoreset{algorithm}{chapter}% algorithm counter resets every chapter
\makeatother
\renewcommand{\thealgorithm}{\thechapter.\arabic{algorithm}}% Algorithm # is <chapter>.<algorithm>

\providecommand\algorithmname{algorithm}
\captionsetup[ruled]{font=small,labelfont={bf},labelsep=quad}

\newcommand{\tempcaption}{}% stores the caption
\newcommand{\templabel}{}% stores the label

\newenvironment{customalgo}[3][0.7\textwidth]
{%
    \begin{minipage}{#1}
        \begin{algorithm}[H]
            \centering
            \gdef\tempcaption{#2}% store the caption so we can use it later
            \gdef\templabel{#3}% store the label so we can use it later
            \begin{algorithmic}[1]
            }%
            {%
            \end{algorithmic}
            \caption{\tempcaption}% use the stored caption
            \label{\templabel}
        \end{algorithm}
    \end{minipage}
    % \smallskip
}%

% extent of line-spacing in algorithms
\let\Algorithm\algorithm
\renewcommand\algorithm[1][]{\Algorithm[#1]\setstretch{1.2390625}}

% % https://tex.stackexchange.com/questions/64674/coloring-lines-in-an-algorithm
% \makeatletter
% \newcommand{\algcolor}[2]{%
%     \hskip-\ALG@thistlm\colorbox{#1}{\parbox{\dimexpr\linewidth-2\fboxsep}{\hskip\ALG@thistlm\relax #2}}%
% }
% \newcommand{\algemph}[1]{\algcolor{cbrewerintergray}{#1}}
% \makeatother


%  https://tex.stackexchange.com/questions/64674/coloring-lines-in-an-algorithm
\makeatletter
% code borrowed from Andrew Stacey; See
% https://tex.stackexchange.com/a/50054/3954
\tikzset{%
    remember picture with id/.style={%
        remember picture,
        overlay,
        save picture id=#1,
    },
    save picture id/.code={%
        \edef\pgf@temp{#1}%
        \immediate\write\pgfutil@auxout{%
        \noexpand\savepointas{\pgf@temp}{\pgfpictureid}}%
    },
    if picture id/.code args={#1#2#3}{%
        \@ifundefined{save@pt@#1}{%
            \pgfkeysalso{#3}%
            }{
            \pgfkeysalso{#2}%
        }
    }
}

\def\savepointas#1#2{%
    \expandafter\gdef\csname save@pt@#1\endcsname{#2}%
}

\def\tmk@labeldef#1,#2\@nil{%
    \def\tmk@label{#1}%
    \def\tmk@def{#2}%
}

\tikzdeclarecoordinatesystem{pic}{%
    \pgfutil@in@,{#1}%
    \ifpgfutil@in@%
        \tmk@labeldef#1\@nil
    \else
        \tmk@labeldef#1,(0pt,0pt)\@nil
    \fi
    \@ifundefined{save@pt@\tmk@label}{%
        \tikz@scan@one@point\pgfutil@firstofone\tmk@def
        }{%
        \pgfsys@getposition{\csname save@pt@\tmk@label\endcsname}\save@orig@pic%
        \pgfsys@getposition{\pgfpictureid}\save@this@pic%
        \pgf@process{\pgfpointorigin\save@this@pic}%
        \pgf@xa=\pgf@x
        \pgf@ya=\pgf@y
        \pgf@process{\pgfpointorigin\save@orig@pic}%
        \advance\pgf@x by -\pgf@xa
        \advance\pgf@y by -\pgf@ya
    }%
}

\makeatother
% end of Andrew's code

% main command to draw the colored background
\newcounter{mymark}
\newcommand\ColorLine{%
    \stepcounter{mymark}%
    \tikz[remember picture with id=mark-\themymark,overlay] {;}%
    \begin{tikzpicture}[remember picture,overlay]%
        \filldraw[cbrewerintergray]%
            let \p1=(pic cs:mark-\themymark),
            \p2=(current page.east)  in
            ([xshift=-0.1em,yshift=-1.0ex]0,\y1)  rectangle ++([xshift=-2.525cm]\x2,\baselineskip);
    \end{tikzpicture}%
}%



