%!TEX root = ../main.tex
% vim:nospell ft=tex

\makeatletter
\newcommand{\mathleft}{\@fleqntrue\@mathmargin0pt}
\newcommand{\mathcenter}{\@fleqnfalse}
\makeatother

\makeatletter\let\percentchar\@percentchar\makeatother
\directlua{
    % define a function that prints 2-letter ordinal strings
    function myord ( n )     % n: some positive number
    m = n \percentchar 100 % m = n modulo 100
    if m>3 and m<21 then tex.sprint ( "th" )
    elseif m \percentchar 10 == 1 then tex.sprint ( "st" )
    elseif m \percentchar 10 == 2 then tex.sprint ( "nd" )
    elseif m \percentchar 10 == 3 then tex.sprint ( "rd" )
    else tex.sprint ( "th" )
    end
    end
}
\newcommand\myord[1]{\directlua{myord(#1)}} % LaTeX "wrapper macro"
\newcommand{\ordfrac}[2]{\nicefrac{#1}{#2\textsuperscript{\myord{#2}}}}

% typeset a horizontal vector in inline math
\stackMath
\ExplSyntaxOn
\NewDocumentCommand \vect { s o m }
{
    \IfBooleanTF {#1}
    { \vectaux*{#3} }
    { \IfValueTF {#2} { \vectaux[#2]{#3} } { \vectaux{#3} } }
    ^T
}
\DeclarePairedDelimiterX \vectaux [1] {\lbrack} {\rbrack}
{ \, \dbacc_vect:n { #1 } \, }
\cs_new_protected:Npn \dbacc_vect:n #1
{
    \seq_set_split:Nnn \l_tmpa_seq { , } { #1 }
    \seq_use:Nn \l_tmpa_seq { \enspace }
}
\ExplSyntaxOff


