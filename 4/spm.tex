% -*- root: ../main.tex -*-
%!TEX root = ../main.tex
% this file is called up by main.tex
% content in this file will be fed into the main document
% vim:textwidth=80 fo=cqt

% As  discussed in  \cref{sec:classificationscheme},\todo{may need  to cross-ref
% the relevant subsection}


Reducing  the   number  of   computational  dimensions   in  a   physical  model
helps  in  formulating  their  low order  approximations,  thereby  facilitating
fast   computations.   The  \gls{spm},   originally   used   in  modelling   the
Metal-Hydride    chemistry~\cite{Haran1998}   and    later   on    adapted   for
Li-ion   cells~\cite{Ning2004},  represents   the  canonical   apogee  of   such
dimension-reduction strategies.


During the initial  years following its inception, the formulation  of the basic
\gls{spm} has  been discussed  extensively within  application-specific contexts
such  as  \gls{soc}  estimation~\cite{Santhanagopalan2006a,Santhanagopalan2008},
parameter   estimation~\cite{Santhanagopalan2007},   life   cycle   and   ageing
predictions~\cite{Santhanagopalan2008a,Safari2009}.   There   have   also   been
detailed    stand-alone    presentations    of    various    facets    of    the
basic   \gls{spm},   such   as    its   inherent   assumptions   and   governing
equations~\cite{Santhanagopalan2006,Chaturvedi2010}.    The   basic    \gls{spm}
suffers    from    certain    major     limitations    which    are    discussed
in~\cref{subsec:basicspmlimitations}. Since the turn  of the decade, researchers
have attempted to tackle many of these  issues and such efforts are discussed in
\cref{sec:electrolyteinclusion}.


A  survey of  the  most recent  literature  in all  \gls{spm}  family of  models
reveals  a  diminishing rate  of  advancement  in quantifiable  improvements  to
the  underlying  plant  model.  This   nearly-static  trend  can  be  attributed
to  the  general consensus  within  the  research  community that  these  models
may  be  too  simplistic  and  not  of  suitable  accuracy  to  warrant  further
studies.   Other  than   a  small   minority   of  papers   that  propose   core
modelling  improvements  to  tackle  their modelling  inaccuracies  or  add  new
enhancements  such as  mechanical-stress physics~\cite{Li2017a,Li2018b},  latest
work  in   this  family   of  models  predominantly   pertains  to   the  topics
of    state   estimation~\cite{Chaochun2018,Lin2017,Tran2017,Moura2017,Zou2016},
optimal    charging~\cite{Perez2015},    cycling    performance~\cite{Maia2017},
conversion           to            equivalent           circuits~\cite{Li2017b},
parametrisation~\cite{Li2018,Rajabloo2017,Bizeray2017,Namor2017}, pack-balancing
studies~\cite{Docimo2018}  and   observer  design  for   joint  state--parameter
estimation~\cite{Ascencio2016}. The \gls{spm} approach  was also extended to the
case  of  composite  electrodes,  leading  to a  state  estimator  design  after
basic  observablity  analysis~\cite{Bartlett2015}.  Owing to  their  simplicity,
this  author  believes that  \gls{spm}s  hold  the  highest potential  to  bring
a  physics  based  model  to  embedded  \gls{bms}s.  With  this  goal  in  view,
this  thesis seeks  to  resurrect  interest in  this  field  by addressing  this
paucity in fundamental  model improvements. A proposed enhancement  to the basic
\gls{spm}  is the  main  contribution of  this chapter  and  shall be  discussed
in~\cref{sec:newelectrolytemodel}.


In  order  to establish  a  context  for discussing  the  author's  work, it  is
imperative to provide  a holistic presentation of the  basic \gls{spm} modelling
art. The conventional \gls{spm} is the simplest of all time domain physics-based
models and  the rest  of this  section provides an  expository treatment  of its
rubrics.


\subsection{Model Development --- Geometry}\label{subsec:basicspmgeometry}

\begin{figure}[h]
    \centering
    \includegraphics[width=\textwidth]{placeholder_images/example-image-golden.pdf}
    \caption[Schematic illustration depicting geometrical origins of the \gls{spm}]
    {Schematic illustration depicting the geometrical origins of the \gls{spm}. The \gls{spm} is obtained through a degenerate spatial discretisation of one electrochemical layer within a typical Li-ion pouch cell. The axial direction along the cell's thickness is denoted by $x(t)$, whilst the pseudo-dimension along the radial depth of each electrode particle is denoted by $r(t)$. In the basic \gls{spm}, the active material of each porous electrode is represented by one representative spherical particle, thus entirely eliminating the spatial dimension along the axial direction.}
    \label{fig:sandwichtospm}
\end{figure}


\Cref{fig:sandwichtospm}  shows the  arrangement  of  one electrochemical  layer
within a  typical Li-ion pouch cell.  A description of the  working principle of
the  cell was  presented in~\cref{ch:chapter1}  and  is not  repeated here.  The
\gls{spm}, as the name suggests,  models the electrochemical phenomena along the
thickness  $l_j$, \jinnegpos{}  of  each porous  electrode  by a  representative
spherical particle.  Thus the two  distinct solid-phase porous materials  of the
cell, \ie{} the  negative and positive electrodes, are idealised  as two spheres
of radii $r_\text{neg}$ and $r_\text{pos}$ respectively.


In  this  reformulated  arrangement,  the  spatial  dimension  along  the  axial
thickness  of  each  electrode  degenerates   to  a  single  point.  Hence,  the
concentration of Lithium within each electrode $c_{\text{s}_j}$, \jinnegpos{} is
only a  function of the radial  position $r_j$, \jinnegpos{} along  the depth of
their  representative  spherical  particle,  and time,  $t$.  The  surface  area
of  each representative  sphere  is  scaled appropriately,  such  that they  are
equivalent  to the  active area  of  the corresponding  porous electrodes.  Thus
the  \gls{spm} accounts  for  the  reduced volume-fraction  arising  due to  the
microporous structure of  the solid-phase. However, the storage  capacity of the
representative  particles  match  that  of  the  corresponding  electrodes.  The
overarching  assumption  of  the  \gls{spm} modelling  philosophy  is  that  the
electrochemical performance of these representative electrodes are sufficient to
model the behaviour of the cell at its terminals. The \gls{spm} thus employs the
coarsest possible spatial  discretisation of the cell's thickness  with the goal
of minimising computational burden.


\subsection{Model Development --- Scope and Assumptions}

Having established the geometrical representation of the model, it is imperative
to establish its  aims and scope. This section discusses  the subset of physical
phenomena that can captured by the model and enumerates the inherent assumptions
in  model derivation.The  validity of  these  assumptions and  their effects  is
discussed  in~\cref{subsec:basicspmlimitations}.  As  a  broad  outline  of  the
\gls{spm}s scope,  the model  attributes the cell  polarisation to  two dominant
physics,  \viz{} reaction  kinetics and  solid-phase transport  phenomena, \ie{}
diffusion dynamics.


The  \gls{spm}  assumes that  charge  transfer  happens throughout  the  surface
of  each  representative  spherical  particle where  intercalation  occurs.  The
electronic  conductivity of  the solid-phase  is assumed  to be  high enough  to
ignore the  spatial distribution of  charge, \ie{} the local  volumetric current
density is assumed  to be uniform along the thickness  of each porous electrode.
This assumption is  motivated by the early calculations performed  by Newman and
Tobias~\cite{Newman1962} in their stand-alone  analysis of current distributions
in porous electrodes, wherein a volume-averaged molar flux is deemed sufficient.
This uniform current density assumption implies that all of the particles in the
electrode active material are in parallel.


In the  \gls{spm}, solid-phase  diffusion dynamics are  solved by  assuming this
averaged electrochemical  reaction rate.  In the simulation  study by  Smith and
Wang~\cite{Smith2006b},  it  is  reported  that, soon  after  the  beginning  of
discharge, solid-phase concentration and ionic flux become nearly independent of
spatial position, and  Lithium diffusion in solid particles may  be driven by an
averaged molar flux at the surface.


Based  on the  discussion thus  far, it  is clear  that the  \gls{spm} does  not
attempt  to model  all physical  processes within  the cell.  The model  assumes
instantaneous  charge transport  from one  electrode  to the  other through  the
solution phase.  This implies that  electrolytic diffusion is  sufficiently fast
(relative to  diffusion in the solid  phase). Thus, mass transport  phenomena in
the  electrolyte  are  not  considered.


During the  operation of the  cell, the  \gls{spm} assumes that  the electrolyte
concentration  $c_\text{e}$ remains  constant at  its equilibrium  initial value
$c_{\text{e},0}$ throughout  the cell thickness. Neglecting  local concentration
gradients  in the  solution phase,  together  with ignoring  its mass  transport
phenomena implies that  the current in the electrolyte does  not vary over space
and time. Hence,  in the conventional \gls{spm} there is  no contribution of the
solution  phase  to internal  overpotentials  and  electroyte dynamics  have  no
influence on the cell's terminal voltage.


Finally,  the  \gls{spm}  ignores  any   variations  in  material  porosity  and
ionic-flow tortuosity  along the axial  direction of the cell.  This facilitates
the  usage of  a constant  effective diffusion  coefficient for  the electrolyte
phase. Furthermore,  the solid-phase diffusion  and kinetic parameters  are held
constant. Thermal effects are assumed to be negligible and no degradation
effects are attempted to be modelled.


These simplifying  assumptions are  made so  as to enable  the formulation  of a
physics-based model without incurring a  heavy computational cost. The impact of
these  assumptions shall  be  examined in~\cref{subsec:basicspmlimitations}  and
later  sections presents  research that  strives  to straddle  the fine  balance
between model sophistication and computational complexity.


\subsection{Model Development --- Chemistry}

This section  provides a  brief overview of  the essential  chemistry principles
that helps to provide a background context for the governing equations presented
in~\cref{subsec:basicspmgoverningeqns}.


In  a Li-ion  cell,  the  positive electrode  consists  of  porous particles  of
Lithium--Transition Metal Oxide (MO) compounds. The negative electrode typically
employs  some  variant  of  microporous  graphite.  The  porous  nature  of  the
electrodes  provide interstitial  sites  which act  as  intercalation spots  for
Lithium shuttling  between the two  electrodes. The electrolyte,  whose dynamics
are ignored  in the \gls{spm},  helps in the  conduction of \ch{Li^+}  ions. The
separator membrane allows the passage of  these ions between the two electrodes,
but prevents internal short-circuit  by inhibiting electronic conduction through
it. The current collectors facilitate  passage of electrons generated during the
charge transfer reaction  at particle surface to the external  circuit. With the
help of~\cref{fig:chargetransferprocess}  the steps involved in  this process is
detailed next.

\begin{figure}[h]
    \centering
    \includegraphics[width=0.5\textwidth]{placeholder_images/example-image-golden.pdf}
    \caption
    {Simplified representation of charge-transfer process and illustration of
    basic working mechanism of a Li-ion cell}
    \label{fig:chargetransferprocess}
\end{figure}

At fully charged  condition, majority of Lithium in the  system is packed within
the negative electrode microstructure. During discharge, \ch{Li^0} atoms diffuse
out  of  deep  interstitial  sites  towards the  surface  of  the  particles  in
the  negative electrode.  At  the surface  (electrode-electrolyte interface),  a
charge-transfer process takes place according to Butler-Volmer kinetics, leading
to  the formation  of \ch{Li^+}  ions and  electrons. The  electrons are  passed
to  the external  circuit  through  \ch{Cu} current  collectors  onto which  the
conductive matrix  composed of  the negative electrode  material and  binders is
coated. The  \ch{Li^+} ions travel  through the electrolyte phase,  crossing the
separator membrane  to the positive  electrode where they encounter  an electron
influx from the external circuit. A  charge transfer reaction takes place at the
surface of the positive electrode particles, leading to the formation of neutral
\ch{Li^0} atoms that diffuse into the positive electrode microstructure.

During  the   charging  process,  the   reverse  phenomena  occur.   Lithium  is
de-intercalated  from  the  positive  electrode and  a  similar  charge-transfer
happens  at the  surface,  leading  to the  formation  of  \ch{Li^+} ions  which
reach  the  negative  electrode  by   passing  through  the  separator.  At  the
surface  of  the  negative  electrode particles,  these  ions  absorb  electrons
from  the  external circuit,  leading  to  the  formation of  neutral  \ch{Li^0}
that   diffuses  into   interior   vacant  spaces   in   the  layered   graphite
electrode. The  charge-transfer mechanism  and sequence  of events  are depicted
in~\cref{fig:chargetransferprocess}.
~\Cref{eq:NegElectrodeRxn,eq:PosElectrodeRxn} summarise the reactions during the
charging and discharging process at the surfaces of both electrode materials.
\tikzexternaldisable
\begin{align}
    \ch{Li_{$x$} C &<=>[\tiny{discharge}][\tiny{charge}] C + $x$ Li^+ + $x$ e^-}\label{eq:NegElectrodeRxn}\\
    \ch{Li_{1-$x$} M O2 + $x$ Li^+  + $x$ e^- &<=>[\tiny{discharge}][\tiny{charge}] LiMO2}\label{eq:PosElectrodeRxn}
\end{align}
\tikzexternalenable
where \ch{M} represents a transition metal compound such as
\ch{Ni_{1/3}Co_{1/3}Mn_{1/3}} (NMC), \ch{Ni_{0.8}Co_{0.15}Al_{0.05}} (NCA)
amongst other choices~\cite{Reddy2011}. Assuming no loss  of cycleable Lithium
due to parasitic side reactions or through other mechanisms, the process is
fully reversible.


The  electric potential  at  each  electrode is  dependent  upon  the extent  of
its  lithiation. An  empirical  relationship of  each  electrode's potential  as
a  function  of  its  stoichiometry  can be  obtained,  and  is  dependent  upon
the  specific design  and  material  properties of  each  active material  under
consideration. Finally, the \gls{ocv} of the cell is obtained by subtracting the
negative electrode potential from its positive electrode counterpart.

\subsection{Model Development --- Governing Equations}\label{subsec:basicspmgoverningeqns}

As discussed  in~\cref{subsec:basicspmgeometry}, the \gls{spm}  primarily models
the phenomena of solid-phase diffusion and reaction kinetics.

Conservation of \ch{Li^0} in the electrodes can be obtained by treating that the
movement of neutral  atoms within the solid phase is  primarily due to diffusion
within particles.  This diffusion  phenomena is induced  due to  a concentration
gradient that  exists between the  surface and interior/core of  the solid-phase
particles. This  diffusion effect  can be studied  by applying  standard Fickian
dynamics given by

\begin{equation}
    \frac{\partial x}{\partial t}
\end{equation}

\subsection{Limitations and Drawbacks}\label{subsec:basicspmlimitations}

The modelling foundations of the \gls{spm} have been

