% -*- root: ../main.tex -*-
%!TEX root = ../main.tex
% this file is called up by main.tex
% content in this file will be fed into the main document
% vim:textwidth=80 fo=cqt

As  evidenced by  the  results of  the constant  current  charge, discharge  and
dynamic  simulation runs,  the basic  \gls{spm} suffers  from a  \emph{critical}
drawback. The lack of electrolyte dynamics in the conventional \gls{spm} results
in  poor voltage  accuracy  even at  moderate C-rates.  This  renders the  model
unsuitable for  observer design in  \gls{soc} estimation applications  since the
output voltage from the model maps  to a radically different \gls{soc} operating
point. A number of candidate solutions have been proposed in literature in order
to mitigate this drawback. Their salient aspects are briefly evaluated here.

Even  the earliest  works which  attempt  to include  electrolyte dynamics  into
the  conventional  \gls{spm} were  published  only  within the  present  decade.
Schmidt~\etal~\cite{Schmidt2010c} proposed  an infinite-sum  eigenfunction modal
expansion paradigm for solving for the electrolyte concentration. It was claimed
that by  accounting for contribution from  only the first two  terms, sufficient
accuracy may be achieved. Furthermore, an \gls{ode} was proposed for the rate of
evolution of  the first temporal  mode. The solved electrolyte  concentration is
then  substituted into  an  approximate analytical  solution  for the  \gls{dfn}
model's  charge conservation  \gls{pde}\fxnote{cross-reference  to newman  model
equation here}  to obtain the  electrolyte potential. However,  the presentation
lacks the  sufficient depth  of explanation  which hinders  reproducibility. For
instance,  the  origin  and  explanation  of  the  approximation  terms  in  the
electrolyte  potential solution  is omitted.  Derivations are  performed from  a
rigorous mathematical perspective without providing contextual reference to cell
parameters or  electrochemical quantities. Introducing numerical  examples would
have been a redeeming factor to help keeping the mathematical aspects tractable.
This method has not seen further uptake for \gls{spm} modelling.

Di Domenico~\etal{}~\cite{DiDomenico2010} presented a step-by-step derivation of
the approximate analytical solution to the electrolyte overpotential. The
potential drop in the electrolyte is given by
\begin{equation}\label{eq:electrolytepd}
    \phi_\epos - \phi_\eneg = -\frac{I}{2 A}\left(\frac{l_\text{neg}}{\kappa_\text{eff}} + 2 \frac{l_\text{sep}}{\kappa_\text{eff}} + \frac{l_\text{pos}}{\kappa_\text{eff}}\right)
\end{equation}
and     can     be     substituted    into     the     subtraction     operation
involving~\cref{eq:posoverpotential} and~\cref{eq:negoverpotential} in computing
the   overall  overpotential   of~\cref{eq:overpotentialdifference}  and   hence
the  terminal  voltage.  The  effective   conductivity  of  the  electrolyte  in
a   given   region   \jinpossepneg{},   within   the   cell,   is   defined   as
$\kappa_\effj(c_e) = \kappa(c_e)\, \varepsilon_j^{\text{brugg}_j}$. As discussed
in~\cref{subsec:basicspmsimsetup},  the   intrinsic  and  hence   the  effective
electrolyte conductivity is a function of the concentration of \ch{Li^+} ions in
the  electrolyte.  Di  Domenico~\etal{}  did  not  discuss  the  spatio-temporal
calculation  of  electrolyte  concentration.  It   is  likely  that  a  constant
electrolyte concentration  at its  initial equilibrium value  was used.  As will
be  shown in~\cref{sec:newelectrolytemodel}\fxnote{do  not  forget  to show  the
gradient  in  $c_e$},  significant  spatial gradients  in  the  electrolyte  are
established at  even low  to moderate  C-rates during  the cell's  operation. In
extreme  cases  such  as  sustained unidirectional  application  of  current  at
moderate  C-rates,  even  starvation  of ions,  particularly  near  the  current
collectors is a possibility. Furthermore, Di Domenico~\etal{} do not present any
results of applying dynamic current profiles.  Since the critical aspect of mass
transport  contribution  to electrolyte  overpotential  is  omitted, this  model
cannot be viewed as a \emph{sufficient} enhancement to the basic \gls{spm}.


Rahimian~\etal{}~\cite{KhaleghiRahimian2013} discuss  the usage of  a polynomial
approximation  for   electrolyte  concentration  and  potentials.   However,  no
restriction was imposed on the order  of the polynomials chosen to represent the
electrolyte concentration within  each porous electrode region.  In the standard
\gls{dfn} model, the  number of equations and  corresponding boundary conditions
describing electrolyte charge and mass transport within the cell is insufficient
to uniquely solve for all  unknown coefficients of the polynomial approximation.
The  challenges  posed  due  to  this equation  deficiency  shall  be  discussed
in~\cref{temp:eqndeficiency}. Although  the original  \gls{spm} did  not involve
solving  for  the  electrolyte  concentrations  or  potentials,  The  polynomial
approximation of the single


Rahimian~\etal{} adopted  a cubic  polynomial for approximating  the electrolyte
concentration within  the porous electrodes.  To overcome the issue  of equation
deficiency, they adapted a scheme wherein one additional spatial location in the
interior  of  each  electrode  was  used. The  coefficients  of  the  polynomial
approximation  were then  obtained  by iteratively  solving  a relatively  large
coupled  system of  algebraic equations,  embedding within  them the  additional
equations evaluated at  the interior point. An additional  complicating issue is
the specific positioning of this  additional interior point. An online numerical
optimisation  was performed  to obtain  the optimal  placement of  this interior
node. Although it serves as a proof of concept towards implementing higher order
polynomial approximations,  the author of this  thesis deems this method  as too
complex for online implementations.


% A common  characteristic of  these proposals  is that  they seek  to incorporate
% electrolyte  dynamics  of  varying  degrees  of  complexity  directly  into  the
% \gls{spm}. But I did something different.  The high accuracy of the \gls{soc} in
% the basic  \gls{spm} leads to  the author's  hypothesis that if  the electrolyte
% concentration  can  be  solved  as an  independent  subsystem  and  incorporated
% into  the  terminal  voltage,  this  can lead  to  an  improved  \gls{spm}  with
% electrolyte  dynamics. Such  a decoupled  electrolyte inclusion  into the  basic
% \gls{spm}\fxnote{fix this sentence}

