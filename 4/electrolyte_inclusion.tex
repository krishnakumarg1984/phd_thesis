Rahimian~\etal{}~\cite{KhaleghiRahimian2013} discuss  the usage of  a polynomial
approximation  for   electrolyte  concentration  and  potentials.   However,  no
restriction was imposed on the order  of the polynomials chosen to represent the
electrolyte concentration within  each porous electrode region.  In the standard
\gls{dfn} model, the  number of equations and  corresponding boundary conditions
describing electrolyte charge and mass transport within the cell is insufficient
to uniquely solve for all  unknown coefficients of the polynomial approximation.
The  challenges  posed  due  to  this equation  deficiency  shall  be  discussed
in~\cref{temp:eqndeficiency}. Although  the original  \gls{spm} did  not involve
solving  for  the  electrolyte  concentrations  or  potentials,  The  polynomial
approximation of the single


Rahimian~\etal{} adopted  a cubic  polynomial for approximating  the electrolyte
concentration within  the porous electrodes.  To overcome the issue  of equation
deficiency, they adapted a scheme wherein one additional spatial location in the
interior  of  each  electrode  was  used. The  coefficients  of  the  polynomial
approximation  were then  obtained  by iteratively  solving  a relatively  large
coupled  system of  algebraic equations,  embedding within  them the  additional
equations evaluated at  the interior point. An additional  complicating issue is
the specific positioning of this  additional interior point. An online numerical
optimisation  was performed  to obtain  the optimal  placement of  this interior
node. Although it serves as a proof of concept towards implementing higher order
polynomial approximations,  the author of this  thesis deems this method  as too
complex for online implementations.

