% -*- root: ../main.tex -*-
%!TEX root = ../main.tex
% this file is called up by main.tex
% content in this file will be fed into the main document
% vim:nospell

\begin{table}[!htbp]
    \small
    \caption[Simulation parameters of  an \glsfmtshort{lco} cell]{Complete set of  parameters for  simulating the
        \gls{p2d} and  \gls{spm} implementations  of an  \gls{lco} cell  (with \ch{LiCoO_2}--\ch{LiC_6} electrode   pair
        and   \ch{LiPF_6}   electrolyte).   The  highlighted   entries represent the  parameters exclusive  to \gls{p2d}
        model.\quad  \protect{$j \in \{\text{pos},\text{sep},\text{neg}\}$}}
    \label{tbl:lcoSimParamsSPMp2d}
    \vspace{-2.6229525pt}
    \begin{threeparttable}
        \centering
        \begin{varwidth}[t]{0.48\linewidth}
            \begin{tabular*}{\textwidth}{@{} l @{\extracolsep{\fill}} r @{}}
                \multicolumn{2}{c}{\textbf{System Conditions}} \\
                \toprule
                \multicolumn{1}{@{}l}{Parameter} \\
                \midrule

                Lower cutoff cell voltage, $V_\text{min}$ (\si{\volt}) & \tnote{a}2.50   \\
                Upper cutoff cell voltage, $V_\text{max}$ (\si{\volt}) & \tnote{b}4.30   \\
                Cell temperature, $T_\text{cell}$ (\si{\kelvin})       & \tnote{c}298.15 \\

                \bottomrule
            \end{tabular*}
        \end{varwidth}
        \hfill
        \begin{varwidth}[t]{0.48\linewidth}
            \begin{tabular*}{\textwidth}{@{} l @{\extracolsep{\fill}} r @{}}
                \multicolumn{2}{c}{\textbf{Other Constants}} \\
                \toprule
                \multicolumn{1}{@{}l}{Parameter} \\
                \midrule

                Faraday constant, $F$ (\si{\coulomb\per\meter})                                                        & 96487         \\
                Universal gas constant, $R$ (\si{\joule\per\mole\per\kelvin})                                          & 8.314         \\
                \textcolor{viridistwentybluesix}{\emph{Init.}}\ electrolyte conc., $c_\text{e,0}$ (\si{\mole\per\meter\cubed}) & \tnote{c}1000 \\

                \bottomrule
            \end{tabular*}
        \end{varwidth}

        \bigskip
        \vspace{-2.6229525pt}
        % \vfill

        \begin{tabular*}{\textwidth}{@{} =P{7.5cm} @{\extracolsep{\fill}} +l +c +r @{}}
            \multicolumn{4}{c}{\textbf{Thermodynamic, Kinetic, Geometric and Transport Parameters}} \\
            \toprule
            \multicolumn{1}{@{}l}{Parameter} & \multicolumn{1}{l}{Pos} & \multicolumn{1}{c}{Sep} & \multicolumn{1}{r@{}}{Neg}\\
            \midrule

            \rowstyle{\color{viridistwentybluesix}} Bruggeman coefficient, $\text{brugg}_j$                                                 & \tnote{c}\num{4}        & \tnote{c}\num{4}                               & \tnote{c}\num{4}        \\
            \rowstyle{\color{viridistwentybluesix}} Intrinsic electrolyte diffusivity, $D$ (\si{\meter\squared\per\second})               & \tnote{d}\num{3.22e-10} & \tnote{d}\num{3.22e-10}                        & \tnote{d}\num{3.22e-10} \\
            \rowstyle{\color{viridistwentybluesix}} Intrinsic electrolyte conductivity, $\kappa$ (\si{\siemens\per\meter})                &
            \multicolumn{3}{c}{\textcolor{viridistwentybluesix}{\Vhrulefill{} see table note \emph{e} \&~\cref{subsec:basicspmsimsetup} \Vhrulefill{}}}\\
            \rowstyle{\color{viridistwentybluesix}} \ch{Li^+} transference number, $t^0_\text{+}$                                           & \tnote{c}\num{0.363}    & \tnote{c}\num{0.363}                           & \tnote{c}\num{0.363}    \\
            \rowstyle{\color{viridistwentybluesix}} Intrinsic electronic conductivity, $\sigma_j$ (\si{\siemens\per\meter})                 & \tnote{c}\num{100.00}   & ---                                            & \tnote{c}\num{100.00}   \\
            Thickness, $l_j$ (\si{\meter})                                                          & \tnote{c}\num{88e-6}    & \textcolor{viridistwentybluesix}{\tnote{c}\num{25e-6}} & \tnote{f}\num{72e-6}    \\
            Electrolyte porosity, ${\varepsilon}_j$                                                 & \tnote{c}\num{0.385}    & \tnote{c}\num{0.724}                           & \tnote{c}\num{0.485}    \\
            Filler vol.\ fraction, ${\varepsilon}_{\text{fi}_j}$                                    & \tnote{c}\num{0.025}    & ---                                            & \tnote{c}\num{0.033}    \\
            Particle radius, $R_\pj$ (\si{\meter})                                                  & \tnote{c}\num{2e-6}     & ---                                            & \tnote{c}\num{2e-6}     \\
            Specific interfacial surface area, $a_\sj$ (\si{\meter\squared\per\meter\cubed})        & \tnote{g}\num{885e3}    & ---                                            & \tnote{g}\num{723.6e3}  \\
            Electrode diffusivity, $D_{\text{s}_j}$ (\si{\meter\squared\per\second})                & \tnote{c}\num{1e-14}    & ---                                            & \tnote{c}\num{3.9e-14}  \\
            Stoichiometry, 0\% SOC, ${\theta}_{\text{min}_j}$                                       & \tnote{h}\num{0.9917}   & ---                                            & \tnote{h}\num{0.0143}   \\
            Stoichiometry, 100\% SOC, ${\theta}_{\text{max}_j}$                                     & \tnote{c}\num{0.4955}   & ---                                            & \tnote{c}\num{0.8551}   \\
            Max concentration, ${c_\text{s,max}}_j$ (\si{\mole\per\meter\cubed})                    & \tnote{c}\num{51554}    & ---                                            & \tnote{c}\num{30555}    \\
            Reaction rate coefficient, $k_\jr$ (\si{\meter\tothe{2.5}\mole\tothe{-0.5}\per\second}) & \tnote{c}\num{2.33e-11} & ---                                            & \tnote{c}\num{5.03e-11} \\
            Overall active surface area, $A$ (\si{\meter\squared})                                  & \tnote{i}\num{2.053}    & ---                                            & \tnote{i}\num{2.053}    \\
            Open circuit potential, $U_j$ (\si{\volt})                                              & \tnote{k}see table note & ---                                            & \tnote{m}see table note \\
            \bottomrule
        \end{tabular*}

        \bigskip
        \vspace{-2.6229525pt}
        % \vfill
        %%%%%%%%%%%%%%%%%%%%%%%%%%%% SIMULATION PARAMS TABLE %%%%%%%%%%%%%%%%%%%%%%%%%%%%%
        \begin{tabular*}{\textwidth}{@{} =P{7.5cm}  +l@{\extracolsep{\fill}}+c +r @{}}
            \multicolumn{4}{c}{\textbf{Spatial Discretisation}} \\
            \toprule
            \multicolumn{1}{@{}l}{Parameter} & \multicolumn{1}{l}{Pos} & \multicolumn{1}{c}{Sep} & \multicolumn{1}{r@{}}{Neg}\\
            \midrule

            \rowstyle{\color{viridistwentybluesix}} Nodes, through-thickness (axial), $N_{\text{a}_j}$          & \num{15} & \num{15} & \num{15} \\
            \rowstyle{\color{viridistwentybluesix}} Nodes, within spherical particle (radial), $N_{\text{r}_j}$ & \num{10} & ---      & \num{10} \\

            \bottomrule
        \end{tabular*}

        \medskip
        \vspace{-2.6229525pt}
        % \vfill
        \begin{tablenotes}[para,flushleft]
            \begin{scriptsize}
            \item[a] Ref.~\cite{Northrop2011}
            \item[b] Set to $\approx $\SI{100}{\milli\volt} above the cell's \gls{ocp} at \SI{100}{\percent} cell \gls{soc}
            \item[c] Ref.~\cite{Subramanian2009}
            \item[d] Computed at $T_\text{cell} = \SI{298.15}{\kelvin}$ using coefficients from table \rom{2} in Ref.~\cite{Valoen2005}\\%\fxnote{cross-ref to actual equation}
            \item[e] Computed at $T_\text{cell} = \SI{298.15}{\kelvin}$ using coefficients from table \rom{3} in Ref.~\cite{Valoen2005} at $c_\text{e,0}= \SI{1000}{\mole\per\meter\cubed}$\\
            \item[f] Set up for capacity balance of electrodes such that $l_\text{neg} = 1.22 \times l_\text{pos}$ here.
            \item[g] Computed as per~\cref{eq:specificsurfarea}\\
            \item[h] Obtained as residual stoichiometries after a C/\num{500} simulated discharge from \SI{100}{\percent} cell \gls{soc} to \SI{2.7}{V}
            \item[i] Chosen so that current density for the electrochemical layer is \SI{29.23}{\ampere\per\meter\squared} for \SI{60}{\ampere} applied current
            \end{scriptsize}
            \vspace{1ex}
            \item[k] $ \mathcal{U(\theta_\text{pos})} = \textstyle \frac{-4.656 + 88.669\theta_\text{pos}^2 - 401.119\theta_\text{pos}^4 + 342.909\theta_\text{pos}^6 - 462.471\theta_\text{pos}^8 + 433.434\theta_\text{pos}^{10}}{-1 + 18.933\theta_\text{pos}^2 - 79.532\theta_\text{pos}^4 + 37.311\theta_\text{pos}^6 - 73.083\theta_\text{pos}^8 + 95.96\theta_\text{pos}^{10}}$ \\[0.25em]
            \begin{footnotesize}
            \item[m] $\mathcal{U(\theta_\text{neg})} = 0.7222 + 0.1387\theta_\text{neg} + 0.029\theta_\text{neg}^{0.5} - \frac{0.0172}{\theta_\text{neg}} + \frac{0.0019}{\theta_\text{neg}^{1.5}} + 0.2808 e^{(0.9 - 15\theta_\text{neg})} - 0.7984 e^{(0.4465\theta_\text{neg} - 0.4108)}$\vfill
            \end{footnotesize}
        \end{tablenotes}
    \end{threeparttable}
\end{table}
