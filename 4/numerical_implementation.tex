% -*- root: ../main.tex -*-
%!TEX root = ../main.tex
% this file is called up by main.tex
% content in this file will be fed into the main document
% vim:textwidth=80 fo=cqt

The  equations   presented  in~\cref{sec:spmmodeldevelopment}   are  well-known,
self-sufficient and  fully descriptive so  as to implement  the basic~\gls{spm}.
Although  numerical implementation  of  circuit-oriented cell  models have  been
considered~\cite{Plett2004,Plett2004a,Plett2004b,Plett2006}, there  has yet been
no treatment  of this critical  aspect in \gls{spm} modelling  literature. Since
this thesis has a strong focus  towards enabling the use of physics-based models
in an embedded environment, at least the numerical aspects of implementing these
equations needs to be discussed. The  finer details of real-time programming, in
particular the integration  of the cell model into the  pack and its interaction
with other elements and aspects of  a typical vehicular drivetrain controller is
beyond the scope  of this academic work. Nevertheless, the  discussion here aims
to lower the barrier to real-time implementation and is a unique contribution of
this work in the context of the cell modelling art.


The equations  in~\cref{sec:spmmodeldevelopment} are derived  in continuous-time
form. In particular, the  state equation given by~\cref{eq:threestatesmatrixvec}
describes  the  continuous   time  dynamic  evolution  of   quantities  such  as
the  bulk  concentration   and  mean  radial  flux  rate.   However,  a  typical
embedded  controller such  as that  used in  a vehicular  \gls{bms} operates  in
discrete-time~\cite{Andrea2010}.  This implies  that \emph{samples}  of voltage,
current  and temperature  measurements  are obtained  at  specific intervals  of
time,~$T_s$. The  computations of the  model equations and updating  of solution
variables  (such as  bulk  concentrations and  terminal  voltage) are  performed
between two successive data acquisition events from the sensors.


% Figure from Steve Southward or from plett. Ignore the computational delay
% analytical solution (but not really applicable in state-estimation tasks)
% decoupled
% basic discretisation
% System transition matrix (matrix exponential)
% frequency range
% EPA
% Power input (lack of measurements in current)
% Parameter estimation
% Sampling/quantisation/z-domain/fourier analysis/Pre-conditioning/ZOH
% Sampled systems are also essential in considering Kalman versions
% Can talk in depth about Fwd Euler, Tustin vs matrix exponential approach
% better to perform a vector-update (although certainly it is easy to do scalar
% update)
% Block diagram and stuff

% \begin{figure}[htb]
%     \begin{algorithmic}[1]

%         \Procedure{SUM}{ $\{x\}$}

%         \State $y\gets0$
%         \For{$i \gets 1 : N^{x}$} \Comment{Time series $\{x\}$ has length $N^{x}$}
%         \State $y\gets y+x(i)$ \Comment{Summing up.}
%         \EndFor

%         \State \textbf{return}  $y$
%         \EndProcedure
%     \end{algorithmic}
%     \caption[Implementation of a algorithm for calculating a sum.]{Implementation of a algorithm for calculating a sum.}
%     \label{fig:algorithm1}
% \end{figure}

% The \textproc{Sum} algorithm shows blah blah blah
% % inside algorithm,
% % cases environment, displayed equations, chapter wise algorithm numbering
% % referencing function names in small caps

% % https://tex.stackexchange.com/questions/113719/cleveref-fails-to-reference-algorithms

% % https://tex.stackexchange.com/questions/110412/numbering-in-algorithmicx
% % https://tex.stackexchange.com/questions/65993/algorithm-numbering

% % https://tex.stackexchange.com/questions/203713/how-can-i-typeset-function-names-as-they-appear-in-algorithmic-environments
% % https://tex.stackexchange.com/questions/100346/typesetting-listofalgorithms-like-listoffigures-and-listoftables-using-titletoc
% % https://tex.stackexchange.com/questions/30363/how-do-i-define-a-new-command-in-algorithmicx

% % https://tex.stackexchange.com/questions/67908/customizing-the-algorithmic-package-break-and-loop-labels

% % https://tex.stackexchange.com/questions/69449/avoid-putting-statements-on-the-same-line-with-algorithmicx

% % \usepackage{float}
% % \newfloat{algorithm}{t}{lop}
% % Add \floatname{algorithm}{Algorithm} to capitalise the float name
