% -*- root: ../main.tex -*-
%!TEX root = ../main.tex
% this file is called up by main.tex
% content in this file will be fed into the main document
% vim:textwidth=80 fo=cqt

Having performed a  comprehensive analysis of the state of  the art in \gls{spm}
modelling, this section presents the  author's unique contribution to the field.
Firstly, the  scope of the  contribution is identified. The  methodology adopted
and corresponding results are presented thereafter.

\subsection{Scope and motivation}\label{subsec:scopenewelectrolyte}

This subsection is intended as a capstone summary helping to briefly recount the
discussion so far  and to provide a  context for the author's work  in the wider
realm  of the  \gls{spm}  modelling art.  In  the same  vein  as the  discussion
in~\cref{sec:electrolyteinclusion}, the scope of the proposed enhancement to the
\gls{spm} concerns entirely  with improving the electrolyte subsystem  as it has
already been established in~\cref{subsec:simresultsbasicspm} that the simplified
representation of  the solid-phase subsystem  through a fourth  order polynomial
approximation method  for diffusion of  \ch{Li^0} in  the solid particle,  is of
sufficiently high accuracy.

Inspecting the  electrolyte domain,  the electrolyte  overpotential contribution
to  terminal  voltage   consists  of  a  diffusion   overpotential  in  addition
to   the  time-dependent   ohmic  losses   that  originates   from  differential
concentration  gradients  that  is   indirectly  dependent  upon  concentration.
Hence,  accurate  determination  of   spatio-temporal  concentration  takes  the
centre  stage.  For   the  computation  of  overpotential   in  the  electrolyte
phase,~\cref{eq:electrolytepdwithce}  proposed  by  Prada~\etal~\cite{Prada2012}
may be used.

There exists a  subtle detail in the  use of~\cref{eq:electrolytepdwithce} which
is discussed here upfront before proceeding  ahead to the refined context of the
author's  work.  The  intrinsic  conductivity  of  electrolyte,  $\kappa$  is  a
function of  the ionic concentration  (refer~\cref{subsec:basicspmsimsetup}). If
the ionic  concentration at  the corresponding current  collectors are  used for
$\kappa_\text{neg}$  and  $\kappa_text{pos}$,  this  would lead  to  a  lopsided
computation of the overpotential in electrolyte. Furthermore, under this scheme,
the computation of electrolyte conductivity shall be rendered ambiguous since it
is unclear which  separator interface shall be chosen for  the separator's ionic
concentration. Although this  has not been discussed clearly  in literature, the
author  of this  thesis chose  to use  the mean  concentration within  each cell
region, defined as
\begin{equation}
    \mean{c}_{\text{e},j}(t) = \frac{1}{l_j}\int_0^{l_j} c_{\text{e}_j}(z,t)\, dz = \frac{Q_{\text{e}_j}(t)}{\varepsilon_j l_j}
\end{equation}
although other measures  of central tendency might be equally  valid. Hence, the
results of this section have the associated variability in them depending on how
the electrolyte concentration computations are  used in evaluating the intrinsic
conductivity of electrolyte.

As  the  ionic  concentration  has  both  a  direct  and  indirect  contribution
in~\cref{eq:electrolytepdwithce}, its spatio-temporal  computation is a critical
aspect. As discussed  in~\cref{sec:quadraticapprox}, the quadratic approximation
is a widely used spatio-temporal model for electrolyte concentration which makes
the best  use of available physical  constraints. As established in  the results
of~\cref{subsec:quadraticsimresultsanalysis}, while  the spatial  performance of
the quadratic approximation approach is acceptable, its time-domain performance,
particularly at the crucial current collector locations is mediocre at best.

Thus,   the  \emph{scope}   of  the   author's  work   is  to   obtain  suitable
alternate  expressions for  improving  the  \textbf{time-domain} performance  of
the  electrolyte  concentration  computation   whilst  retaining  the  quadratic
approximation   approach   for  describing   its   spatial   profile.  Such   an
approach  is motivated  by  the  keen observation  that  the baseline  quadratic
approximation  model  has  a  natural  `pause'  in  its  model  description.  To
clarify,~\crefrange{eq:cecontinuitynegsep}{eq:Qepbyintegration}  form a  tightly
coupled set of  seven linear equations in seven unknowns.  The time evolution of
$Q_{\text{e}_j}$  are described  through  a system  of  first order  \glspl{ode}
given by~\crefrange{eq:negliionmolesquadratic}{eq:posliionmolesquadratic}.  In a
practical  implementation,  these  \glspl{ode}  are solved  independently  in  a
decoupled  manner,\ie{}  by using  the  coefficients  obtained from  the  linear
system of~\crefrange{eq:cecontinuitynegsep}{eq:Qepbyintegration} in the previous
time-step. The  author's hypothesis is that  by taking advantage of  the natural
break  in the  operational  sequence which  involves  two separate  computations
between  two independent  subsystems (for  all practical  purposes), it  must be
possible to  replace the  underperforming time-evolution  model of  the baseline
quadratic approximation with a superior alternate model.
