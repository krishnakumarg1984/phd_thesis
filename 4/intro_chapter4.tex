\todo{uncomment and edit the intro in the latex source file later on}

% Battery modellers face the classic  conundrum of conjuring physics-based battery
% models that remain  amenable for control applications.  Firstly, the contrasting
% nature of this modelling objective is presented. Secondly, prior attempts by the
% research community to  tackle this issue is briefly examined.  A suitable family
% of models  from the broad  category of reduced-order  models is identified  as a
% promising  candidate  for implementation  in  controls  applications. Next,  the
% drawbacks of this family of models is  discussed in detail. The state of the art
% implementation for tackling these drawbacks  is presented and their inadequacies
% are discussed.


\todo{Write the chapter. Finally come back to summarize this}


% The following efforts/trials were done (failures)
% \begin{itemize}
%     \item first attempt
%     \item second attempt
% \end{itemize}
% The following successes were achieved.
% \begin{itemize}
%     \item first attempt
%     \item second attempt
% \end{itemize}


% At the  end of this  chapter, we have a  control oriented reduced  order battery
% model amenable for use in real-time applications for SOC, SOH etc.\ estimations.


Control-Oriented  models can  be considered  synonymous with  the term  `Reduced
Order Models'. This is because the complexity of physics-based models inherently
necessitates the  use of a low  order model. In  this thesis, the two  terms are
used interchangeably.


