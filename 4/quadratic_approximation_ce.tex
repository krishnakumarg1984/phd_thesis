% -*- root: ../main.tex -*-
%!TEX root = ../main.tex
% this file is called up by main.tex
% content in this file will be fed into the main document
% vim:textwidth=80 fo=cqt

In this  section, the  quadratic approximation  of ionic  spatial concentration,
that underpins the electrolyte model  in many improved \gls{spm} formulations is
presented. An analysis of  the weakness of this model is  performed based on the
results  from applying  this model.  Mitigation of  this critical  drawback lead
to  this  author's  decoupled spatio-temporal  electrolyte  concentration  model
structure which is presented next in~\cref{sec:newelectrolytemodel}.

\begin{figure}[!htb]
    \captionsetup{singlelinecheck=off}
    \centering
    \includegraphics{placeholder_images/example-image-golden.pdf}
    \caption[Co-ordinate systems for quadratic approximation of
    electrolyte concentration]{Schematic diagram of the electrochemical sandwich
        consisting of
        \begin{enumerate*}[label=\itshape\alph*\upshape)]
            \item negative electrode,
            \item separator, and
            \item positive electrode
        \end{enumerate*} depicting the co-ordinate system used in deriving the
        quadratic approximation profile. The global spatial co-ordinate is $x
        \in \{0,l_\text{tot}\}$, where $l_\text{tot} = l_\text{neg} +
        l_\text{sep} + l_\text{pos}$. Local co-ordinate systems specific to each
        region are also defined. It should be noted that the positive
        electrode's local co-ordinate axis direction is reversed.}
    \label{fig:coordsquadapprox}
\end{figure}

The  schematic  in~\cref{fig:coordsquadapprox}  shows   the  definition  of  the
co-ordinate  systems  used  in  deriving the  polynomial  approximation  of  the
electrolyte concentration  profile. The globally defined  $x$ co-ordinate starts
at  the negative  current  collector  interface ($x=0$)  and  terminates at  the
positive  current  collector  interface  ($x =  l_\text{tot},\,  l_\text{tot}  =
l_\text{neg} +  l_\text{sep} +  l_\text{pos}$). Three local  co-ordinate systems
$z_\mu$  valid  only  within  their  respective regions  are  also  defined.  In
particular, it  must be  noted that  the direction  of the  local $z_\text{pos}$
co-ordinate axis is opposite to that of  the other two local co-ordinate axes as
well as the global co-ordinate axis. In subsequent usages, the suffix in $z_\mu$
is dropped and  the reader is advised  to infer the region of  validity from the
usage  context  which are  unambiguous  as  they  occur in  separate  equations.
Furthermore, the  notation of  the three regions  $\{\text{neg, sep,  pos}\}$ is
abbreviated  to $\{n,s,p\}$  respectively in  all mathematical  expressions. The
author  is convinced  that this  notation does  not detract  from following  the
derivations, but rather aids it by keeping the notations compact.

A  standard  quadratic expression  is  chosen  a  priori for  approximating  the
electrolyte concentration profile within each region.
\begin{alignat}{2}
    c_\ensub(z,t) & = a_2(t) z^2 + a_1(t) z + a_0(t) \qquad &  & 0 \le z \le l_\text{n}\label{eq:cenqquadstart}   \\
    c_\essub(z,t) & = a_5(t) z^2 + a_4(t) z + a_3(t) \qquad &  & 0 \le z \le l_\text{s}\label{eq:cesqquadstart}   \\
    c_\epsub(z,t) & = a_8(t) z^2 + a_7(t) z + a_6(t) \qquad &  & 0 \le z \le l_\text{p}\label{eq:cepqquadstart}
    \shortintertext{where     the    coefficient     vector    $\vec{a}(t)     =
    \vect{a_0(t),a_1(t),   \dots  ,a_8(t)}$   is  to   be  determined   at  each
    time-step\footnotemark.  Applying  boundary  conditions of  the  electrolyte
    diffusion equation from  the \gls{dfn} model~(refer~\cref{eq:dfnliquiddiff})
    to~\crefrange{eq:cenqquadstart}{eq:cepqquadstart},  it  is clear  that  $a_1
    =  0$ and  $a_7  = 0$.  Thus,~\crefrange{eq:cenqquadstart}{eq:cepqquadstart}
    become}
    c_\ensub      & = a_2 z^2 + a_0         \qquad          &  & 0 \le z \le l_\text{n}\label{eq:cenquadreduced} \\
    c_\essub      & = a_5 z^2 + a_4 z + a_3 \qquad          &  & 0 \le z \le l_\text{s}\label{eq:cesquadreduced} \\
    c_\epsub      & = a_8 z^2 + a_6         \qquad          &  & 0 \le z \le l_\text{p}\label{eq:cepquadreduced}
\end{alignat}
\footnotetext{In rest of  the equations, time-dependence of  the coefficients is
dropped from  the notation. It is  implicitly noted that they  are time-varying.
Similarly, spatio-temporal dependence of the electrolyte concentration functions
$c_\text{e,j}$ is omitted  in circumstances where such explicit  notation is not
crucial for understanding.}

% -*- root: ../../main.tex -*-
%!TEX root = ../../main.tex
% this file is called up by main.tex
% content in this file will be fed into the main document
% vim:nospell textwidth=180 foldlevelstart=3 foldlevel=3 conceallevel=0

\begin{table}[!htbp]
    \centering
    \caption[Electrolyte equations \& boundary conditions of \glsfmtshort{dfn} model in separator]{Electrolyte-specific governing equations and boundary conditions of the \glsfmtlong{dfn}~(\glsfmtshort{dfn}) model within the separator domain.}
    \label{tbl:dfnelectrolyteeqnsinsep}
    \begingroup
    \makeatletter\def\f@size{9.25}\check@mathfonts
    \addtolength{\jot}{0.875em}
    \begin{tabular*}{\textwidth}{@{} l c r l r @{}}
        \toprule
        \multicolumn{1}{c}{\small Region} & \small Governing equations & \multicolumn{2}{c}{\small Boundary conditions } & {} \\
        {} & {} & \multicolumn{2}{c}{\scriptsize $(l_\text{neg} \coloneqq l_\text{n},\, l_\text{sep} \coloneqq l_\text{s},\, l_\text{pos} \coloneq l_\text{p})$} \\
        \midrule
        \multicolumn{1}{l |}{{\rotatebox[origin=c]{90}{\makecell{\footnotesize Separator\\ \scriptsize $\delta \in \{\text{sep}\}$}}}} &
        $\begin{aligned}
            \vphantom{D_{\text{\tiny eff}_\text{n}}\!\! \! \!\, \diffp{c_\text{e}}{x}{\mathrlap{x = l^{-}_\text{n}}}} \varepsilon_\delta \diffp{c_\text{e}}{t} &=D_\effdelta  \diffp[2]{c_\text{e}}{x} \\[-0.75em]
            \vphantom{D_{\text{\tiny eff}_\text{s}}\!\! \! \!\, \diffp{c_\text{e}}{x}{\mathrlap{x=(l_{\text{n}} + l_\text{s})^{-}}}}\\[1.25em]
            \vphantom{\kappa_{\text{\tiny eff}_\text{n}}\!\! \! \!\, \diffp{c_\text{e}}{x}{\mathrlap{x = l^{-}_\text{n}}}\hspace{5mm} =\kappa_{\text{\tiny eff}_\text{s}}\!\!\!\!\,\diffp{c_\text{e}}{x}{\mathrlap{x = l^{+}_\text{n}}}} \frac{I}{A} &= \overline{\kappa}_\effdelta \left( \diffp[2]{\phi_\text{e}}{x} + \frac{2 R T}{F} (t^0_{+}-1)\diffp[2]{ \ln c_\text{e}}{x}\right) \\[-0.75em]
            \vphantom{\kappa_{\text{\tiny eff}_\text{s}}\!\! \! \!\, \diffp{c_\text{e}}{x}{\mathrlap{x=(l_{\text{n}} + l_\text{s})^{-}}}} \\
        \end{aligned}$ &
        $\begin{aligned}
    \vphantom{D_{\text{\tiny eff}_\text{n}}\!\! \! \!\, \diffp{c_\text{e}}{x}{\mathrlap{x = l^{-}_\text{n}}}} \qquad c_\text{e}\Bigr\rvert_{\mathrlap{x=l^{-}_\text{n}}}\hspace{5mm} &= c_\text{e}\Bigr\rvert_{\mathrlap{x=l^{+}_\text{n}}},\\[-0.75em]
     \vphantom{\kappa_{\text{\tiny eff}_\text{n}}\!\! \! \!\, \diffp{c_\text{e}}{x}{\mathrlap{x = l^{-}_\text{n}}}\hspace{5mm} =\kappa_{\text{\tiny eff}_\text{s}}\!\!\!\!\,\diffp{c_\text{e}}{x}{\mathrlap{x = l^{+}_\text{n}}}} c_\text{e}\Bigr\rvert_{\mathrlap{x=(l_{\text{n}} + l_\text{s})^{-}}}\hspace{5mm} &= c_\text{e}\Bigr\rvert_{\mathrlap{x=(l_{\text{n}} + l_\text{s})^{+}}},\\[1.25em]
 \vphantom{\kappa_{\text{\tiny eff}_\text{n}}\!\! \! \!\, \diffp{c_\text{e}}{x}{\mathrlap{x = l^{-}_\text{n}}}} \vphantom{\left( \diffp[2]{\phi_\text{e}}{x} + \frac{2 R T}{F} (t^0_{+}-1)\diffp[2]{ \ln c_\text{e}}{x}\right)} \phi_\text{e}\Bigr\rvert_{\mathrlap{x=l^{-}_\text{n}}}\hspace{5mm} &= \phi_\text{e}\Bigr\rvert_{\mathrlap{x=l^{+}_\text{n}}},\\[-0.75em]
 \vphantom{\kappa_{\text{\tiny eff}_\text{s}}\!\! \! \!\, \diffp{c_\text{e}}{x}{\mathrlap{x=(l_{\text{n}} + l_\text{s})^{-}}}} \phi_\text{e}\Bigr\rvert_{\mathrlap{x=(l_{\text{n}} + l_\text{s})^{-}}}\hspace{5mm} &= \phi_\text{e}\Bigr\rvert_{\mathrlap{x=(l_{\text{n}} + l_\text{s})^{-}}},\\
    \end{aligned}$ &
    $\begin{aligned}
        \quad D_{\text{\tiny eff}_\text{n}}\!\! \! \!\, \diffp{c_\text{e}}{x}{\mathrlap{x = l^{-}_\text{n}}}\hspace{5mm} &=D_{\text{\tiny eff}_\text{s}}\!\!\!\!\,\diffp{c_\text{e}}{x}{\mathrlap{x = l^{+}_\text{n}}}\\[-0.75em]
        D_{\text{\tiny eff}_\text{s}}\!\! \! \!\, \diffp{c_\text{e}}{x}{\mathrlap{x=(l_{\text{n}} + l_\text{s})^{-}}}\hspace{5mm} &=D_{\text{\tiny eff}_\text{p}}\!\!\!\!\,\diffp{c_\text{e}}{x}{\mathrlap{x=(l_{\text{n}} + l_\text{s})^{+}}}\\[1.25em]
        \vphantom{\left( \diffp[2]{\phi_\text{e}}{x} + \frac{2 R T}{F} (t^0_{+}-1)\diffp[2]{ \ln c_\text{e}}{x}\right)} \kappa_{\text{\tiny eff}_\text{n}}\!\! \! \!\, \diffp{c_\text{e}}{x}{\mathrlap{x = l^{-}_\text{n}}}\hspace{5mm} &=\kappa_{\text{\tiny eff}_\text{s}}\!\!\!\!\,\diffp{c_\text{e}}{x}{\mathrlap{x = l^{+}_\text{n}}}\\[-0.75em]
        \kappa_{\text{\tiny eff}_\text{s}}\!\! \! \!\, \diffp{c_\text{e}}{x}{\mathrlap{x=(l_{\text{n}} + l_\text{s})^{-}}}\hspace{5mm} &=\kappa_{\text{\tiny eff}_\text{p}}\!\!\!\!\,\diffp{c_\text{e}}{x}{\mathrlap{x=(l_{\text{n}} + l_\text{s})^{+}}}\\
    \end{aligned}$ &
    $\begin{aligned}
        \vphantom{D_{\text{\tiny eff}_\text{n}}\!\! \! \!\, \diffp{c_\text{e}}{x}{\mathrlap{x = l^{-}_\text{n}}}} \quad \refstepcounter{equation}(\theequation)\label{eq:liquiddiffnsep} \\[-0.75em]
        \vphantom{D_{\text{\tiny eff}_\text{s}}\!\! \! \!\, \diffp{c_\text{e}}{x}{\mathrlap{x=(l_{\text{n}} + l_\text{s})^{-}}}}\\[1.25em]
        \vphantom{\kappa_{\text{\tiny eff}_\text{n}}\!\! \! \!\, \diffp{c_\text{e}}{x}{\mathrlap{x = l^{-}_\text{n}}}} \vphantom{\left( \diffp[2]{\phi_\text{e}}{x} + \frac{2 R T}{F} (t^0_{+}-1)\diffp[2]{ \ln c_\text{e}}{x}\right)} \refstepcounter{equation}(\theequation) \label{eq:liquidpotentialsep}\\[-0.75em]
        \vphantom{\kappa_{\text{\tiny eff}_\text{s}}\!\! \! \!\, \diffp{c_\text{e}}{x}{\mathrlap{x=(l_{\text{n}} + l_\text{s})^{-}}}}
    \end{aligned}$
    \\
    \bottomrule
\end{tabular*}
\endgroup
\end{table}



\Cref{tbl:dfnelectrolyteeqnsinsep} lists  the equations and  boundary conditions
for  phenomena  describing  electrolyte  diffusion  and  charge  balance  within
the separator  domain. \Cref{eq:liquiddiffnsep} and~\cref{eq:liquidpotentialsep}
are  obtained  by  applying  the  corresponding  electrolyte  equations  of  the
\gls{dfn}  model  (see~\cref{eq:dfnliquiddiff}  and~\cref{eq:dfnliquidpotential}
respectively) to the separator region.

Applying  the  continuity  and  flux  boundary  conditions  of  the  electrolyte
diffusion  equation from~\cref{eq:liquiddiffnsep}  at both  separator
interfaces
% results in the following set of linear algebraic equations
\begin{alignat}{2}
    a_2 l^2_\text{n} + a_0                      & = \hphantom{-}a_3 \qquad                    &  & \text{\footnotesize (continuity at neg/sep interface)} \\
    a_5 l^2_\text{s} + a_4 l_\text{s} + a_3     & = \hphantom{-}a_8 l^2_\text{p} + a_6 \qquad &  & \text{\footnotesize (continuity at sep/pos interface)} \\
    2 a_2 l_\text{n} D_\effn                    & = \hphantom{-}a_4 \qquad                    &  & \text{\footnotesize (flux b.c.\ at neg/sep interface)}  \\
    \left(2 a_5 l_\text{s} + a_4\right) D_\effs & = -2 a_8 l_\text{p} D_\effp \qquad          &  & \text{\footnotesize (flux b.c.\ at sep/pos interface)}\label{eq:quadcefluxseppos}
\end{alignat}
Note that the negative sign in~\cref{eq:quadcefluxseppos} is due to the specific
choice of the co-ordinate system used  for the positive electrode region. Due to
this,  fluxes  at  the  separator/positive  electrode  interface  have  opposing
directions.

Let  $Q_\text{e,j}$  denote  the  number  of moles  of  \ch{Li^+}  ions  in  the
electrolyte per  unit cross-sectional  area within each  region $\jinnegseppos$.
This is  computed as  the product  of
\begin{enumerate*}[label=\emph{\alph*})]
    \item the porosity and
    \item spatial integral of the concentration function
\end{enumerate*}
\ie{}  $ Q_\text{e,j}  =  \varepsilon_j \int_0^{l_j}  c_{\text{e},j}(z) \,dz  $.
Applying this to \crefrange{eq:cenquadreduced}{eq:cepquadreduced}
\begin{align}
    Q_\text{e,n} &= \varepsilon_\text{n} \left( \frac{1}{3} a_2 l^3_\text{n} + a_0 l_\text{n}\right)\\
    Q_\text{e,s} &= \varepsilon_\text{s} \left( \frac{1}{3} a_5 l^3_\text{s} + \frac{1}{2} a_4 l^2_\text{s} + a_3 l_\text{s}\right)\\
    Q_\text{e,p} &= \varepsilon_\text{p} \left( \frac{1}{3} a_8 l^3_\text{p} + a_6 l_\text{p}\right)
\end{align}

At this stage,  $Q_{\text{e},j}(t)$ are unknown. Since  these are time-dependent
functions,  the  derivation  naturally  progresses  towards  seeking  a  set  of
\glspl{ode} that describe a relationship  for their time evolution. We transform
the  second   order  \glspl{ode}  of~\cref{eq:dfnliquiddiff}   (for  electrodes)
and~\cref{eq:liquiddiffnsep}   (for  separator)   to   their  respective   local
co-ordinates and integrate  once along the thickness of  each region. Performing
this sequence of steps for the negative electrode region,
\begin{varwidth}[t]{\textwidth}
    $\begin{WithArrows}
        \varepsilon_\text{n} \int_0^{l_\text{n}} \left(\diffp*{c_\ensub(z,t)}{t}\right)\, dz &= \int_0^{l_\text{n}} \left(\diffp{}{z}\left(D_\effn \diffp{c_\ensub}{z} \right) + (1 - t^0_\text{+}) a_\snsub j_\text{n}\right)\, dz \Arrow[tikz={text width=3.1cm}]{transposing integration \& differentiation operations in the \glsfmtshort{lhs}} \\
        \varepsilon_\text{n} \diffp*{\int_0^{l_\text{n}} c_\ensub(z,t)}{t}\, dz &= \int_0^{l_\text{n}} \left(\diffp{}{z}\left(D_\effn \diffp{c_\ensub}{z} \right) + (1 - t^0_\text{+}) a_\snsub j_\text{n}\right)\, dz \Arrow[tikz={text width=3.1cm}]{applying the integral to the \glsfmtshort{rhs}}\\
        \tikzmark{StartBraceA}\varepsilon_\text{n} \diffp*{\int_0^{l_\text{n}} c_\ensub(z,t)}{t}\, dz\tikzmark{EndBraceA} &= D_\effn \diffp{c_\ensub}{z}{\mathrlap{z=l_\text{n}}} + (1 - t^0_\text{+}) a_\snsub \int_0^{l_\text{n}} j_\text{n}\, dz \Arrow[tikz={text width=3.1cm}]{by definition, the LHS is $Q_\text{e,n}(t)$}\\
        Q_\text{e,n}(t) &= D_\effn \diffp{c_\ensub}{z}{\mathrlap{z=l_\text{n}}} + (1 - t^0_\text{+}) a_\snsub \int_0^{l_\text{n}} j_\text{n}\, dz \\
    \end{WithArrows}$
    % \\\vspace*{1em} Performing similar operations for the separator~(\cref{eq:liquiddiffnsep}) and
    % positive electrode~(\cref{eq:dfnliquiddiff}) yields \\
    % \centering
    % $\begin{aligned}
    %     a & = b
    % \end{aligned}$
\end{varwidth}
\hfill
\begin{varwidth}[t]{0.2\textwidth}
    \begin{flushright}
        $\begin{WithArrows}
            {}&\vphantom{\varepsilon_\text{n} \int_0^{l_\text{n}} \left(\diffp*{c_\ensub(z,t)}{t}\right)\, dz = \int_0^{l_\text{n}} \left(\diffp{}{z}\left(D_\effn \diffp{c_\ensub}{z} \right) + (1 - t^0_\text{+}) a_\snsub j_\text{n}\right)\, dz  }\\
            {}&\vphantom{\varepsilon_\text{n} \diffp*{\int_0^{l_\text{n}} c_\ensub(z,t)}{t}\, dz = \int_0^{l_\text{n}} \left(\diffp{}{z}\left(D_\effn \diffp{c_\ensub}{z} \right) + (1 - t^0_\text{+}) a_\snsub j_\text{n}\right)\, dz }\\
            {}&\vphantom{\varepsilon_\text{n} \diffp*{\int_0^{l_\text{n}} c_\ensub(z,t)}{t}\, dz = D_\effn \diffp{c_\ensub}{z}{\mathrlap{z=l_\text{n}}} + (1 - t^0_\text{+}) a_\snsub \int_0^{l_\text{n}} j_\text{n}\, dz }\\
            {} &\hfill
            \refstepcounter{equation}(\theequation)\label{eq:negliionmolestoreduce}\vphantom{Q_\text{e,n}(t) = D_\effn \diffp{c_\ensub}{z}{\mathrlap{z=l_\text{n}}} + (1 - t^0_\text{+}) a_\snsub \int_0^{l_\text{n}} j_\text{n}\, dz }\\
        \end{WithArrows}$
    \end{flushright}
\end{varwidth}

\InsertUnderBrace[draw=black][aspect=0.28]{StartBraceA}{EndBraceA}{} % https://tex.stackexchange.com/questions/68526/asymmetric-overbrace

% \varepsilon_\text{s} \int_0^{l_\text{s}} \left(\diffp*{c_\essub(z,t)}{t}\right)\, dz &= \int_0^{l_\text{s}} \left(\diffp{}{z}\left(D_\effn \diffp{c_\essub}{z} \right) + (1 - t^0_\text{+}) a_\snsub j_\text{s}\right)\, dz \\



% \blindtext
% \AddToShipoutPicture*{\ShowFramePicture}

