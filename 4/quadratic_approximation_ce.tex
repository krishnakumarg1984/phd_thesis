% -*- root: ../main.tex -*-
%!TEX root = ../main.tex
% this file is called up by main.tex
% content in this file will be fed into the main document
% vim:textwidth=80 fo=cqt

In this  section, the  quadratic approximation  of ionic  spatial concentration,
that underpins the electrolyte model  in many improved \gls{spm} formulations is
presented. An analysis of  the weakness of this model is  performed based on the
results  from applying  this model.  Mitigation of  this critical  drawback lead
to  this  author's  decoupled spatio-temporal  electrolyte  concentration  model
structure which is presented next in~\cref{sec:newelectrolytemodel}.

\begin{figure}[!htb]
    \captionsetup{singlelinecheck=off}
    \centering
    \includegraphics{placeholder_images/example-image-golden.pdf}
    \caption[Co-ordinate systems for quadratic approximation of
    electrolyte concentration]{Schematic diagram of the electrochemical sandwich
        consisting of
        \begin{enumerate*}[label=\itshape\alph*\upshape)]
            \item negative electrode,
            \item separator, and
            \item positive electrode
        \end{enumerate*} depicting the co-ordinate system used in deriving the
        quadratic approximation profile. The global spatial co-ordinate is $x
        \in \{0,l_\text{tot}\}$, where $l_\text{tot} = l_\text{neg} +
        l_\text{sep} + l_\text{pos}$. Local co-ordinate systems specific to each
        region are also defined. It should be noted that the positive
        electrode's local co-ordinate axis direction is reversed.}
    \label{fig:coordsquadapprox}
\end{figure}

The  schematic  in~\cref{fig:coordsquadapprox}  shows   the  definition  of  the
co-ordinate  systems  used  in  deriving the  polynomial  approximation  of  the
electrolyte concentration  profile. The globally defined  $x$ co-ordinate starts
at  the negative  current  collector  interface ($x=0$)  and  terminates at  the
positive  current  collector  interface  ($x =  l_\text{tot},\,  l_\text{tot}  =
l_\text{neg} +  l_\text{sep} +  l_\text{pos}$). Three local  co-ordinate systems
$z_\mu$  valid  only  within  their  respective regions  are  also  defined.  In
particular, it  must be  noted that  the direction  of the  local $z_\text{pos}$
co-ordinate axis is opposite to that of  the other two local co-ordinate axes as
well as the global co-ordinate axis. In subsequent usages, the suffix in $z_\mu$
is dropped and  the reader is advised  to infer the region of  validity from the
usage  context  which are  unambiguous  as  they  occur in  separate  equations.
Furthermore, the  notation of  the three regions  $\{\text{neg, sep,  pos}\}$ is
abbreviated  to $\{n,s,p\}$  respectively in  all mathematical  expressions. The
author  is convinced  that this  notation does  not detract  from following  the
derivations, but rather aids it by keeping the notations compact.

A  standard  quadratic expression  is  chosen  a  priori for  approximating  the
electrolyte concentration profile within each region.
\begin{alignat}{2}
    c_\ensub &= a_2(t) z^2 + a_1(t) z + a_0(t),\quad &&0 \le z \le l_\text{n}\label{eq:cenquadstart} \\
    c_\essub &= a_5(t) z^2 + a_4(t) z + a_3(t),\quad &&0 \le z \le l_\text{s}\label{eq:cesquadstart} \\
    c_\epsub &= a_8(t) z^2 + a_7(t) z + a_6(t),\quad &&0 \le z \le l_\text{p}\label{eq:cepquadstart}
\end{alignat}
where  the  coefficient  vector  $\vect{a_0(t),a_1(t),  \dots  ,a_8(t)}$  is  to
be   determined   at   each   time-step\footnote{For   brevity,   in   rest   of
the   equations,   the   time-dependence   is   dropped   from   the   notation.
However,  it   must  be   implicitly  understood   that  all   coefficients  are
indeed  time-varying.}.   Applying  boundary   conditions  of   the  electrolyte
diffusion  equation  of   the  \gls{dfn}  model~(refer  \cref{eq:dfnliquiddiff})
to~\crefrange{eq:cenquadstart}{eq:cepquadstart}, it is clear  that $a_1 = 0$ and
$a_7 = 0$. Thus, ~\crefrange{eq:cenquadstart}{eq:cepquadstart} become
\begin{alignat}{2}
    c_\ensub &= a_2(t) z^2 + a_0(t),\quad &&0 \le z \le l_\text{n}\label{eq:cenquadreduced} \\
    c_\essub &= a_5(t) z^2 + a_4(t) z + a_3(t),\quad &&0 \le z \le l_\text{s}\label{eq:cesquadreduced} \\
    c_\epsub &= a_8(t) z^2 + a_6(t),\quad &&0 \le z \le l_\text{p}\label{eq:cepquadreduced}
\end{alignat}

% -*- root: ../../main.tex -*-
%!TEX root = ../../main.tex
% this file is called up by main.tex
% content in this file will be fed into the main document
% vim:nospell textwidth=180 foldlevelstart=3 foldlevel=3 conceallevel=0

\begin{table}[!htbp]
    \centering
    \caption[Electrolyte equations \& boundary conditions of \glsfmtshort{dfn} model in separator]{Electrolyte-specific governing equations and boundary conditions of the \glsfmtlong{dfn}~(\glsfmtshort{dfn}) model within the separator domain.}
    \label{tbl:dfnelectrolyteeqnsinsep}
    \begingroup
    \makeatletter\def\f@size{9.25}\check@mathfonts
    \addtolength{\jot}{0.875em}
    \begin{tabular*}{\textwidth}{@{} l c r l r @{}}
        \toprule
        \multicolumn{1}{c}{\small Region} & \small Governing equations & \multicolumn{2}{c}{\small Boundary conditions } & {} \\
        {} & {} & \multicolumn{2}{c}{\scriptsize $(l_\text{neg} \coloneqq l_\text{n},\, l_\text{sep} \coloneqq l_\text{s},\, l_\text{pos} \coloneq l_\text{p})$} \\
        \midrule
        \multicolumn{1}{l |}{{\rotatebox[origin=c]{90}{\makecell{\footnotesize Separator\\ \scriptsize $\delta \in \{\text{sep}\}$}}}} &
        $\begin{aligned}
            \vphantom{D_{\text{\tiny eff}_\text{n}}\!\! \! \!\, \diffp{c_\text{e}}{x}{\mathrlap{x = l^{-}_\text{n}}}} \varepsilon_\delta \diffp{c_\text{e}}{t} &=D_\effdelta  \diffp[2]{c_\text{e}}{x} \\[-0.75em]
            \vphantom{D_{\text{\tiny eff}_\text{s}}\!\! \! \!\, \diffp{c_\text{e}}{x}{\mathrlap{x=(l_{\text{n}} + l_\text{s})^{-}}}}\\[1.25em]
            \vphantom{\kappa_{\text{\tiny eff}_\text{n}}\!\! \! \!\, \diffp{c_\text{e}}{x}{\mathrlap{x = l^{-}_\text{n}}}\hspace{5mm} =\kappa_{\text{\tiny eff}_\text{s}}\!\!\!\!\,\diffp{c_\text{e}}{x}{\mathrlap{x = l^{+}_\text{n}}}} \frac{I}{A} &= \overline{\kappa}_\effdelta \left( \diffp[2]{\phi_\text{e}}{x} + \frac{2 R T}{F} (t^0_{+}-1)\diffp[2]{ \ln c_\text{e}}{x}\right) \\[-0.75em]
            \vphantom{\kappa_{\text{\tiny eff}_\text{s}}\!\! \! \!\, \diffp{c_\text{e}}{x}{\mathrlap{x=(l_{\text{n}} + l_\text{s})^{-}}}} \\
        \end{aligned}$ &
        $\begin{aligned}
    \vphantom{D_{\text{\tiny eff}_\text{n}}\!\! \! \!\, \diffp{c_\text{e}}{x}{\mathrlap{x = l^{-}_\text{n}}}} \qquad c_\text{e}\Bigr\rvert_{\mathrlap{x=l^{-}_\text{n}}}\hspace{5mm} &= c_\text{e}\Bigr\rvert_{\mathrlap{x=l^{+}_\text{n}}},\\[-0.75em]
     \vphantom{\kappa_{\text{\tiny eff}_\text{n}}\!\! \! \!\, \diffp{c_\text{e}}{x}{\mathrlap{x = l^{-}_\text{n}}}\hspace{5mm} =\kappa_{\text{\tiny eff}_\text{s}}\!\!\!\!\,\diffp{c_\text{e}}{x}{\mathrlap{x = l^{+}_\text{n}}}} c_\text{e}\Bigr\rvert_{\mathrlap{x=(l_{\text{n}} + l_\text{s})^{-}}}\hspace{5mm} &= c_\text{e}\Bigr\rvert_{\mathrlap{x=(l_{\text{n}} + l_\text{s})^{+}}},\\[1.25em]
 \vphantom{\kappa_{\text{\tiny eff}_\text{n}}\!\! \! \!\, \diffp{c_\text{e}}{x}{\mathrlap{x = l^{-}_\text{n}}}} \vphantom{\left( \diffp[2]{\phi_\text{e}}{x} + \frac{2 R T}{F} (t^0_{+}-1)\diffp[2]{ \ln c_\text{e}}{x}\right)} \phi_\text{e}\Bigr\rvert_{\mathrlap{x=l^{-}_\text{n}}}\hspace{5mm} &= \phi_\text{e}\Bigr\rvert_{\mathrlap{x=l^{+}_\text{n}}},\\[-0.75em]
 \vphantom{\kappa_{\text{\tiny eff}_\text{s}}\!\! \! \!\, \diffp{c_\text{e}}{x}{\mathrlap{x=(l_{\text{n}} + l_\text{s})^{-}}}} \phi_\text{e}\Bigr\rvert_{\mathrlap{x=(l_{\text{n}} + l_\text{s})^{-}}}\hspace{5mm} &= \phi_\text{e}\Bigr\rvert_{\mathrlap{x=(l_{\text{n}} + l_\text{s})^{-}}},\\
    \end{aligned}$ &
    $\begin{aligned}
        \quad D_{\text{\tiny eff}_\text{n}}\!\! \! \!\, \diffp{c_\text{e}}{x}{\mathrlap{x = l^{-}_\text{n}}}\hspace{5mm} &=D_{\text{\tiny eff}_\text{s}}\!\!\!\!\,\diffp{c_\text{e}}{x}{\mathrlap{x = l^{+}_\text{n}}}\\[-0.75em]
        D_{\text{\tiny eff}_\text{s}}\!\! \! \!\, \diffp{c_\text{e}}{x}{\mathrlap{x=(l_{\text{n}} + l_\text{s})^{-}}}\hspace{5mm} &=D_{\text{\tiny eff}_\text{p}}\!\!\!\!\,\diffp{c_\text{e}}{x}{\mathrlap{x=(l_{\text{n}} + l_\text{s})^{+}}}\\[1.25em]
        \vphantom{\left( \diffp[2]{\phi_\text{e}}{x} + \frac{2 R T}{F} (t^0_{+}-1)\diffp[2]{ \ln c_\text{e}}{x}\right)} \kappa_{\text{\tiny eff}_\text{n}}\!\! \! \!\, \diffp{c_\text{e}}{x}{\mathrlap{x = l^{-}_\text{n}}}\hspace{5mm} &=\kappa_{\text{\tiny eff}_\text{s}}\!\!\!\!\,\diffp{c_\text{e}}{x}{\mathrlap{x = l^{+}_\text{n}}}\\[-0.75em]
        \kappa_{\text{\tiny eff}_\text{s}}\!\! \! \!\, \diffp{c_\text{e}}{x}{\mathrlap{x=(l_{\text{n}} + l_\text{s})^{-}}}\hspace{5mm} &=\kappa_{\text{\tiny eff}_\text{p}}\!\!\!\!\,\diffp{c_\text{e}}{x}{\mathrlap{x=(l_{\text{n}} + l_\text{s})^{+}}}\\
    \end{aligned}$ &
    $\begin{aligned}
        \vphantom{D_{\text{\tiny eff}_\text{n}}\!\! \! \!\, \diffp{c_\text{e}}{x}{\mathrlap{x = l^{-}_\text{n}}}} \quad \refstepcounter{equation}(\theequation)\label{eq:liquiddiffnsep} \\[-0.75em]
        \vphantom{D_{\text{\tiny eff}_\text{s}}\!\! \! \!\, \diffp{c_\text{e}}{x}{\mathrlap{x=(l_{\text{n}} + l_\text{s})^{-}}}}\\[1.25em]
        \vphantom{\kappa_{\text{\tiny eff}_\text{n}}\!\! \! \!\, \diffp{c_\text{e}}{x}{\mathrlap{x = l^{-}_\text{n}}}} \vphantom{\left( \diffp[2]{\phi_\text{e}}{x} + \frac{2 R T}{F} (t^0_{+}-1)\diffp[2]{ \ln c_\text{e}}{x}\right)} \refstepcounter{equation}(\theequation) \label{eq:liquidpotentialsep}\\[-0.75em]
        \vphantom{\kappa_{\text{\tiny eff}_\text{s}}\!\! \! \!\, \diffp{c_\text{e}}{x}{\mathrlap{x=(l_{\text{n}} + l_\text{s})^{-}}}}
    \end{aligned}$
    \\
    \bottomrule
\end{tabular*}
\endgroup
\end{table}



\Cref{tbl:dfnelectrolyteeqnsinsep}    lists   the    equations   and    boundary
conditions   for  phenomena   describing   electrolyte   diffusion  and   charge
balance  within  the   separator  domain.  Essentially,~\cref{eq:liquiddiffnsep}
and~\cref{eq:liquidpotentialsep}  are  obtained  by applying  the  corresponding
electrolyte  equations  of  the \gls{dfn}  model  (see  ~\cref{eq:dfnliquiddiff}
and~\cref{eq:dfnliquidpotential}) respectively to the separator region.
