

\begin{figure}
	\caption{}
	\label{fig:memory}
\end{figure}

Figure~\ref{cputime} shows a comparison of \gls{cpu} times for computing
the \gls{rom} at a single \gls{soc} and temperature for the classical and improved
\gls{dra}methods. Appendix \ref{sec:Specifications-of-Workstation} lists the specifications
of the workstation used for the computations. Owing to the high flop
count for \gls{svd} operation (eq.~\ref{eq:cpu_op_count}), the classical
\gls{dra}requires approximately 40 minutes for a Hankel block size of 8000.
Clearly, the overall \gls{cpu} time for classical \gls{dra}is almost exclusively
used in computing the \gls{svd}. The improved \gls{dra}method reduces the overall
computational time by two orders of magnitude, taking approximately
40 seconds for the same block-size. In this case, \gls{cpu} time is evenly
split between \gls{svd} and Markov parameters computations.

\begin{figure}
	\caption{}
	\label{cputime}
\end{figure}

From Figure~\ref{memory}, it is evident that if classical \gls{dra}
is employed, a standard laptop with a nominal 10 GB \gls{ram} limit (dedicated
for \gls{rom} workflow) cannot capture the full cell dynamics and hence
is restricted to 2500 Hankel blocks. This necessitates early truncation
of Markov parameters at 5000 seconds. From Figure~\ref{markov_cse_pos},
the truncation residue at 5000 seconds for the unit-pulse response
of solid surface concentration at positive current collector is $-0.0087 \text{ mol m}^{-\text{3}}$.
The truncation errors in the Markov Parameter Matrix directly translate
to errors in computed singular values, adversely affecting accuracy
of simulated cell variables. It must be noted that the accuracy of
simulation results reported here does not bear a causal relationship
to the particular \gls{dra}scheme employed. In-principle, when an upper
bound on computational usage has not been enforced, the numerical
operations of both the existing and proposed \gls{dra}schemes lead to similar
error magnitudes for the modelled quantities. Instead, the accuracy
comparison illustrated here primarily serves to demonstrate the practical
usefulness of the improved \gls{dra}scheme when implemented in a commonplace
computing environment.

\begin{figure}
	\caption{}
	\label{truncated}
\end{figure}

Figure~\ref{truncated} shows a comparison of singular values obtained
by the classical and improved \gls{svd} methods computed by imposing a \gls{ram}
limit of 10GB. Owing to early truncation of the Markov parameter matrix,
the dominant singular values computed by the conventional method differ
significantly from those computed by the improved \gls{svd} operating on
untruncated data. With the same memory constraints, the improved \gls{dra}
can handle up to 30000 Hankel blocks, allowing for capture of 60000
seconds of Markov parameter data.

For comparative analysis of modelling accuracy under this memory constraint,
a time-domain simulation of the ROMs obtained by classical and improved
\gls{dra}methods is performed. The input current profile corresponding
to UDDS drive cycle reported in Lee~et~al. \cite{LeeChemistruckPlett2012}
is used. Figure~\ref{time_domain_sim} depicts the time-evolution
of the solid surface concentration at the positive electrode/separator
boundary. For comparing the accuracy of the two ROMs, a COMSOL Multiphysics~\cite{Multiphysics2012}
simulation of the full-order pseudo-2D porous-electrode PDE model
is used as the reference. The \gls{rom} employing classical \gls{dra}diverges
over time, and after 1500 seconds results in an error of 1120$\text{ mol m}^{-\text{3}}$$.$
The \gls{rom} incorporating the new \gls{dra}workflow accurately tracks the COMSOL
simulation trend-line. Table~\ref{table:salientresults} provides
a summary of the key simulation results.

\begin{figure}
	\caption{}
	\label{time_domain_sim}
\end{figure}

%%%%%%%%%%%%%%%%%%%%%%%%%%%%%%%%%%%%%%%%%%%%%%%%%%%%%%%%%%%%%%%%%%%%%%
\section{Conclusions\label{sec:Conclusion}}
In this paper, Singular Value Decomposition of a large Block-Hankel
matrix is identified as a key bottleneck in the classical \gls{dra}for
reduced-order Li-ion cell modelling. An improved \gls{svd} scheme is presented,
which employs a combination of the Golyandina-Usevich and Lanczos
algorithms. The results discussed in Section \ref{sec:Results} demonstrate
the performance improvement achieved by the new method without trading-off
model fidelity. At a single operating point of \gls{soc} and temperature,
for a Hankel block size of 8000, \gls{rom} workflow incorporating the improved
\gls{dra}is approximately 100 times faster than that employing classical
DRA. Using the machine specifications in Appenndix  \ref{sec:Specifications-of-Workstation},
for 100 operating points (combinations of 10 \gls{soc} and temperature values),
computing the \gls{rom} requires only 6 hours using the improved \gls{dra}, whereas
the classical \gls{dra}consumes 666 hours (27 days). Furthermore, for the
same block-size, the improved \gls{dra}is demonstrated to be superior in
terms of memory efficiency, drastically reducing the memory requirement
from 112 GB down to 2 GB. Finally, the improved \gls{dra}demonstrates superior
modelling accuracy when implemented even in moderately equipped computing
environments such as laptops.

The proposed method leads to the possibility of modelling other physical
quantities in the cell geometry unhindered by computing limitations.
Furthermore, high sample-rate models to handle highly dynamic load
profiles can be deployed in future BMS applications. The scheme also
empowers the \gls{rom} framework to tackle cells with slower dynamics and
other chemistries with different rate-limiting mechanisms. The improved
\gls{dra}method opens up a wide range of possibilities and brings the goal
of physics-based battery model implementation in a high performance
real-time BMS a step closer to realization.

%%%%%%%%%%%%%%%%%%%%%%%%%%%%%%%%%%%%%%%%%%%%%%%%%%%%%%%%%%%%%%%%%%%%%%
\begin{acknowledgment}
    Financial support for the research reported in this paper has been
    obtained through the Imperial College President's PhD Scholarships
    scheme. The sponsor had no role whatsoever in collection, analysis
    and interpretation of data, or in writing of the manuscript. Furthermore,
    the funding body has no role/involvement in the decision to submit
    the article for publication. The authors wish to acknowledge the support
    of The Department of Mathematics, Imperial College London for usage
    of the departmental computing cluster. The \gls{cpu} times, memory usage
    and all other computational results reported in this paper were obtained
    by using a computing node from this facility.
\end{acknowledgment}

\begin{nomenclature}
	\entry{$c_e \scriptstyle(x,t)$}{Concentration of Li$^\text{+}$ ions in the electrolyte at each spatial location within the 1-D cell geometry $(\text{mol m}^{-\text{3}})$}
	\entry{$c_{s,e} \scriptstyle(z,t) $}{Concentration of Li at the surface of each solid particle within the normalized domain length of each electrode $(\text{mol m}^{-\text{3}})$}
	\entry{$\medmuskip=0mu \tilde{c}_{\scriptscriptstyle s,e_{pos}}^*\scriptstyle(0,t) $}{Surface concentration of Li in the solid particle adjacent to positive current collector, obtained after model linearisation and subsequent removal of the integrator pole. The algorithms discussed in this paper require that all model variables have poles located strictly within the open left-half complex plane. Since the solid diffusion transfer functions have poles at the origin, it is necessary to remove this integrator pole before deriving the model $(\text{mol m}^{-\text{3}})$}
	\entry{$L_{neg}$}{Thickness of the negative electrode $(\text{m})$}
	\entry{$L_{sep}$}{Thickness of the separator domain $(\text{m})$}
	\entry{$L_{pos}$}{Thickness of the positive electrode $(\text{m})$}
	\entry{$j \scriptstyle(z,t) $}{Li molar flux density at electrode-electrolye interface of each particle within the normalized electrode domain $(\text{mol m}^{-\text{2}}s^{-\text{1}})$}
	\entry{$\phi_e \scriptstyle(x,t) $}{Electrolyte potential at each spatial location within the 1-D cell geometry $(\text{V})$}
	\entry{$\phi_{s,e} \scriptstyle(z,t) $}{Solid-electrolyte potential difference at the inteerfacial boundary for each spatial location within the 1-D cell geometry $(\text{V})$}
\end{nomenclature}
%%%%%%%%%%%%%%%%%%%%%%%%%%%%%%%%%%%%%%%%%%%%%%%%%%%%%%%%%%%%%%%%%%%%%%
\bibliographystyle{asmems4}
\bibliography{\gls{svd}_paper_Bibliography}
\newpage
%%%%%%%%%%%%%%%%%%%%%%%%%%%%%%%%%%%%%%%%%%%%%%%%%%%%%%%%%%%%%%%%%%%%%%
\appendix %%% starting appendix
\section{Specifications of Workstation Used\label{sec:Specifications-of-Workstation}}

\newpage
\section*{Listing of Table Captions}

\begin{description}
	\item[Table 1]   Parameters for \gls{rom} Computation\\
	\item[Table 2]   Salient Results - Classical vs. Improved \gls{dra}\\
	\item[Table 3]   Specifications of workstation used
\end{description}
\newpage
\section*{Listing of Figure Captions}

\begin{description}
	\item[Fig. 1.]   Reduced-order modelling (ROM) workflow using classical \gls{dra}.\\ (The shaded blocks represent computational bottlenecks).\\
	\item[Fig. 2.]   Time evolution of Markov parameters of pole-removed transfer function corresponding to surface concentration of Li in the solid particle adjacent to positive current collector.\\
	\item[Fig. 3.]   Reduced Order Modelling (ROM) Workflow using improved \gls{dra}.\\
	\item[Fig. 4.]   Comparison of singular values computed by the conventional and improved \gls{svd} methods.\\
	\item[Fig. 5.]	 Memory usage of classical and improved \gls{dra}. Overall \gls{ram} usage as well as \gls{ram} used only for \gls{svd} computation is illustrated.\\
	\item[Fig. 6.] 	 Computation times for classical and improved \gls{dra}schemes.\\
	\item[Fig. 7.]	 Comparison of singular values computed by conventional and improved \gls{svd} methods under a practical \gls{ram} limit of 10 GB.\\
	\item[Fig. 8.]	 Time-domain simulation depicting solid surface concentrations at the boundary of positive electrode and separator.\\
\end{description}



\newpage
\singlespacing

\begin{table}[h]
	\singlespacing
	\centering
	\caption{Salient Results - Classical vs. Improved \gls{dra}}
	\label{table:salientresults}
	\setlength{\extrarowheight}{1pt}
	%\centering
	\begin{tabular}{ c c c c }
		\hline
		\gls{rom} & Quantity 		& Classical & Improved \\
		Condition		   &				  & \gls{dra}		 & \gls{dra}	  \\
		\hline
		\multirow{8}{1.22cm}{8000 Hankel Blocks}& Memory 		  & 111.80 GB	  & 2.14 GB  \\[-5pt]
		                                        & \footnotesize (overall)		&			 & 		 \\
		                                        & Memory 		  & 97.93 GB	  & 0.03 GB  \\[-5pt]
		                                        & \footnotesize(\gls{svd} step)   	&			 & 		 \\
		                                        & \gls{cpu} Time 		& 39.78 min   & 0.63 min  \\[-5pt]
		                                        & \footnotesize(overall)    	&			 & 		 \\
		                                        & \gls{cpu} Time 		& 39.30 min   & 0.14 min  \\[-5pt]
		                                        & \footnotesize(\gls{svd} step)   	&			 & 		 \\[2.5pt]
		\hline
		\multirow{3}{1.22cm}{10 GB Memory Limit}& Block Size		  & 2500	  & 30000  \\[5pt]
		                                        & Max. error & 1120\scriptsize $\text{ mol m}^{-\text{3}}$ &  13\scriptsize $\text{ mol m}^{-\text{3}}$ \\[-5pt]
		                                        & in $c_{{s,e}_{pos}}$\scriptsize $(1,t)$ &  &     \\[5pt]
		\hline
	\end{tabular}
\end{table}
\newpage
\begin{table}[h]
	\caption{Specifications of workstation used}
	\label{table:comp_spec}
	\centering
	\begin{tabular}{ l l }
		\hline
		Processor & Intel\textregistered\space  Xeon \textregistered\space E5-2637 v3 \\
		Used Cores & 1 \\
		\gls{cpu} Stepping & 2 \\
		Clock Frequency & 3.50 GHz \\
		Installed \gls{ram} & 500 GB \\
		\hline
	\end{tabular}
\end{table}


% achieve the latent  potential and wide applicability  of the
% physics-based reduced order Li-ion battery model.

