% -*- root: ../main.tex -*-
%!TEX root = ../main.tex
% this file is called up by main.tex
% content in this file will be fed into the main document
% vim:textwidth=80 fo=cqt

In this  section, the performance  of the  basic \gls{spm} is  discussed through
simulation  and  by comparison  against  a  standard \gls{dfn}  benchmark  model
incorporating the full \gls{p2d} dynamics.

\subsection{Cell Parameters}
% -*- root: ../main.tex -*-
%!TEX root = ../main.tex
% this file is called up by main.tex
% content in this file will be fed into the main document
% vim:nospell

\begin{table}[!htbp]
    \small
    \caption[Simulation parameters of  an \protect{\gls{lco}} cell]{Complete set
        of  parameters for  simulating the  \protect{\gls{p2d}} and  \protect{\gls{spm}}
        implementations  of an  \protect{\gls{lco}} cell  (with \ch{LiCoO_2}--\ch{LiC_6}
        electrode   pair  and   \ch{LiPF_6}   electrolyte).   The  highlighted   entries
        represent the  parameters exclusive  to \gls{p2d} model.\quad  \protect{$j \in
    \{\text{pos},\text{sep},\text{neg}\}$}}
    \label{tbl:lcoSimParamsSPMp2d}
    \vspace{-2.6229525pt}
    \begin{threeparttable}
        \centering
        \begin{varwidth}[t]{0.48\linewidth}
            \begin{tabular*}{\textwidth}{@{} l @{\extracolsep{\fill}} r @{}}
                \multicolumn{2}{c}{\textbf{System Conditions}} \\
                \toprule
                \multicolumn{1}{@{}l}{Parameter} \\
                \midrule

                Lower cutoff cell voltage, $V_\text{min}$ (\si{\volt}) & \tnote{a}2.50   \\
                Upper cutoff cell voltage, $V_\text{max}$ (\si{\volt}) & \tnote{b}4.30   \\
                Cell temperature, $T_\text{cell}$ (\si{\kelvin})       & \tnote{c}298.15 \\

                \bottomrule
            \end{tabular*}
        \end{varwidth}
        \hfill
        \begin{varwidth}[t]{0.48\linewidth}
            \begin{tabular*}{\textwidth}{@{} l @{\extracolsep{\fill}} r @{}}
                \multicolumn{2}{c}{\textbf{Other Constants}} \\
                \toprule
                \multicolumn{1}{@{}l}{Parameter} \\
                \midrule

                Faraday constant, $F$ (\si{\coulomb\per\meter})                                                        & 96487         \\
                Universal gas constant, $R$ (\si{\joule\per\mole\per\kelvin})                                          & 8.314         \\
                \textcolor{imperialblue}{\emph{Init.}}\ electrolyte conc., $c_\text{e,0}$ (\si{\mole\per\meter\cubed}) & \tnote{c}1000 \\

                \bottomrule
            \end{tabular*}
        \end{varwidth}

        \bigskip
        \vspace{-2.6229525pt}
        % \vfill

        \begin{tabular*}{\textwidth}{@{} =P{7.5cm} @{\extracolsep{\fill}} +l +c +r @{}}
            \multicolumn{4}{c}{\textbf{Thermodynamic, Kinetic, Geometric and Transport Parameters}} \\
            \toprule
            \multicolumn{1}{@{}l}{Parameter} & \multicolumn{1}{l}{Pos} & \multicolumn{1}{c}{Sep} & \multicolumn{1}{r@{}}{Neg}\\
            \midrule

            \rowstyle{\color{imperialblue}} Bruggeman coefficient, $\text{brugg}_j$                                                 & \tnote{c}\num{4}        & \tnote{c}\num{4}                               & \tnote{c}\num{4}        \\
            \rowstyle{\color{imperialblue}} Intrinsic electrolyte diffusivity, $D$ (\si{\meter\squared\per\second})               & \tnote{d}\num{3.22e-10} & \tnote{d}\num{3.22e-10}                        & \tnote{d}\num{3.22e-10} \\
            \rowstyle{\color{imperialblue}} Intrinsic electrolyte conductivity, $\kappa$ (\si{\siemens\per\meter})                &
            \multicolumn{3}{c}{\textcolor{imperialblue}{\Vhrulefill{} see table note \emph{e} \&~\cref{subsec:basicspmsimsetup} \Vhrulefill{}}}\\
            \rowstyle{\color{imperialblue}} \ch{Li^+} transference number, $t^0_\text{+}$                                           & \tnote{c}\num{0.363}    & \tnote{c}\num{0.363}                           & \tnote{c}\num{0.363}    \\
            \rowstyle{\color{imperialblue}} Intrinsic electronic conductivity, $\sigma_j$ (\si{\siemens\per\meter})                 & \tnote{c}\num{100.00}   & ---                                            & \tnote{c}\num{100.00}   \\
            Thickness, $l_j$ (\si{\meter})                                                          & \tnote{c}\num{88e-6}    & \textcolor{imperialblue}{\tnote{c}\num{25e-6}} & \tnote{f}\num{72e-6}    \\
            Electrolyte porosity, ${\varepsilon}_j$                                                 & \tnote{c}\num{0.385}    & \tnote{c}\num{0.724}                           & \tnote{c}\num{0.485}    \\
            Filler vol.\ fraction, ${\varepsilon}_{\text{fi}_j}$                                    & \tnote{c}\num{0.025}    & ---                                            & \tnote{c}\num{0.033}    \\
            Particle radius, $R_\pj$ (\si{\meter})                                                  & \tnote{c}\num{2e-6}     & ---                                            & \tnote{c}\num{2e-6}     \\
            Specific interfacial surface area, $a_\sj$ (\si{\meter\squared\per\meter\cubed})        & \tnote{g}\num{885e3}    & ---                                            & \tnote{g}\num{723.6e3}  \\
            Electrode diffusivity, $D_{\text{s}_j}$ (\si{\meter\squared\per\second})                & \tnote{c}\num{1e-14}    & ---                                            & \tnote{c}\num{3.9e-14}  \\
            Stoichiometry, 0\% SOC, ${\theta}_{\text{min}_j}$                                       & \tnote{h}\num{0.9917}   & ---                                            & \tnote{h}\num{0.0143}   \\
            Stoichiometry, 100\% SOC, ${\theta}_{\text{max}_j}$                                     & \tnote{c}\num{0.4955}   & ---                                            & \tnote{c}\num{0.8551}   \\
            Max concentration, ${c_\text{s,max}}_j$ (\si{\mole\per\meter\cubed})                    & \tnote{c}\num{51554}    & ---                                            & \tnote{c}\num{30555}    \\
            Reaction rate coefficient, $k_\jr$ (\si{\meter\tothe{2.5}\mole\tothe{-0.5}\per\second}) & \tnote{c}\num{2.33e-11} & ---                                            & \tnote{c}\num{5.03e-11} \\
            Overall active surface area, $A$ (\si{\meter\squared})                                  & \tnote{i}\num{2.053}    & ---                                            & \tnote{i}\num{2.053}    \\
            Open circuit potential, $U_j$ (\si{\volt})                                              & \tnote{k}see table note & ---                                            & \tnote{m}see table note \\
            \bottomrule
        \end{tabular*}

        \bigskip
        \vspace{-2.6229525pt}
        % \vfill
        %%%%%%%%%%%%%%%%%%%%%%%%%%%% SIMULATION PARAMS TABLE %%%%%%%%%%%%%%%%%%%%%%%%%%%%%
        \begin{tabular*}{\textwidth}{@{} =P{7.5cm}  +l@{\extracolsep{\fill}}+c +r @{}}
            \multicolumn{4}{c}{\textbf{Spatial Discretisation}} \\
            \toprule
            \multicolumn{1}{@{}l}{Parameter} & \multicolumn{1}{l}{Pos} & \multicolumn{1}{c}{Sep} & \multicolumn{1}{r@{}}{Neg}\\
            \midrule

            \rowstyle{\color{imperialblue}} Nodes, through-thickness (axial), $N_{\text{a}_j}$          & \num{15} & \num{15} & \num{15} \\
            \rowstyle{\color{imperialblue}} Nodes, within spherical particle (radial), $N_{\text{r}_j}$ & \num{10} & ---      & \num{10} \\

            \bottomrule
        \end{tabular*}

        \medskip
        \vspace{-2.6229525pt}
        % \vfill
        \begin{tablenotes}[para,flushleft]
            \begin{scriptsize}
            \item[a] Ref.~\cite{Northrop2011}
            \item[b] Set to $\approx $\SI{100}{\milli\volt} above the cell's \gls{ocp} at \SI{100}{\percent} cell \gls{soc}
            \item[c] Ref.~\cite{Subramanian2009}
            \item[d] Computed at $T_\text{cell} = \SI{298.15}{\kelvin}$ using coefficients from table \rom{2} in Ref.~\cite{Valoen2005}\\%\fxnote{cross-ref to actual equation}
            \item[e] Computed at $T_\text{cell} = \SI{298.15}{\kelvin}$ using coefficients from table \rom{3} in Ref.~\cite{Valoen2005} at $c_\text{e,0}= \SI{1000}{\mole\per\meter\cubed}$\\
            \item[f] Set up for capacity balance of electrodes such that $l_\text{neg} = 1.22 \times l_\text{pos}$ here.
            \item[g] Computed as per~\cref{eq:specificsurfarea}\\
            \item[h] Obtained as residual stoichiometries after a C/\num{500} simulated discharge from \SI{100}{\percent} cell \gls{soc} to \SI{2.7}{V}
            \item[i] Chosen so that current density for the electrochemical layer is \SI{29.23}{\ampere\per\meter\squared} for \SI{60}{\ampere} applied current
            \end{scriptsize}
            \vspace{1ex}
            \item[k] $ \mathcal{U(\theta_\text{pos})} = \textstyle \frac{-4.656 + 88.669\theta_\text{pos}^2 - 401.119\theta_\text{pos}^4 + 342.909\theta_\text{pos}^6 - 462.471\theta_\text{pos}^8 + 433.434\theta_\text{pos}^{10}}{-1 + 18.933\theta_\text{pos}^2 - 79.532\theta_\text{pos}^4 + 37.311\theta_\text{pos}^6 - 73.083\theta_\text{pos}^8 + 95.96\theta_\text{pos}^{10}}$ \\[0.25em]
            \begin{footnotesize}
            \item[m] $\mathcal{U(\theta_\text{neg})} = 0.7222 + 0.1387\theta_\text{neg} + 0.029\theta_\text{neg}^{0.5} - \frac{0.0172}{\theta_\text{neg}} + \frac{0.0019}{\theta_\text{neg}^{1.5}} + 0.2808 e^{(0.9 - 15\theta_\text{neg})} - 0.7984 e^{(0.4465\theta_\text{neg} - 0.4108)}$\vfill
            \end{footnotesize}
        \end{tablenotes}
    \end{threeparttable}
\end{table}

% Electrolyte diffusivity, $D_j$ \si{(m^2.s^{-1})}                         & \multicolumn{3}{c} {\Vhrulefill{} refer to equation in \tnote{g} \Vhrulefill{}} \\
% \item[g] $ D_j = 10^{-4} \times 10^{-4.43 - \frac{54}{T_\text{cell} - 229 - 5\times10^{-3} c_\text{e}(x,t)} - 0.22\times10^{-3} c_\text{e}(x,t)}, \quad \jinpossepneg $\\

% $\scriptstyle 10^{-4} c_\text{e} \left(-10.5 + \num{0.668e-3} c_\text{e} + \num{0.494e-6} c_\text{e}^2 + (0.074 - \num{1.78e-5}) c_\text{e} - \num{8.86e-10} c_\text{e}^2\right)T_\text{cell} + \left(\num{-6.96e-5} + \num{2.8e-8} c_\text{e})T_\text{cell}^2\right)^2$
% \tnote{e}\num{26.24e-3} & \tnote{e}\num{328.15e-3}                       & \tnote{e}\num{66.08e-3} \\


\Cref{tbl:lcoSimParamsSPMp2d} lists  the simulation  parameters of  an \gls{lco}
cell  whose positive  and negative  electrodes are  \ch{LiCoO_2} and  \ch{LiC_6}
respectively.  The  electrolyte in  this  system  consists of  \ch{LiPF_6}  salt
in  a  solution of  \gls{ec}/\gls{dmc}/\gls{emc}  in  a  1:1:1 ratio.  The  vast
majority  of electrochemical  parameters, \viz{}  the geometric,  thermodynamic,
kinetic  and   transport  properties  of   the  cell  have  been   sourced  from
Subramanian~\etal{}~\cite{Subramanian2009}.

The simulation parameters that are exclusively applicable to the \gls{p2d} model
are shown with a shaded highlighting in~\cref{tbl:lcoSimParamsSPMp2d}. It can be
seen  that only  a subset  of the  isothermal \gls{dfn}  model's parameters  are
required for  the the  \gls{spm}. In particular,  the~\gls{spm} has  seven fewer
parameters  per  electrode. Furthermore,  all  six  physical parameters  of  the
separator  material required  in the  \gls{dfn} model  are ignored  in \gls{spm}
computations. Neglecting Arrhenius type temperature dependence of parameters and
their  corresponding activation  energies, the  basic \gls{spm}  facilitates the
ability to afford physics-based  modelling capabilities with \emph{twenty} fewer
parameters  than  the equivalent  isothermal  \gls{p2d}  model. With  the  naive
assumption of equal  parametrisation effort per physical  property, this implies
an  exact \SI{50}{\percent}  reduction in  parametrisation requirements  for the
basic \gls{spm} compared its \gls{dfn} counterpart.


The volume  fraction of the electrode and separator
materials are computed as $1 - \varepsilon_j - \varepsilon_{\text{fi}_j}$, where
$\varepsilon_j$  is the  electrolyte-phase volume  fraction within  each of  the
three regions of the electrode.

% \rowcolor{imperiallightgray} Material vol.\ fraction, $\varepsilon_\sj$                         & \tnote{d}\num{0.590}    & \tnote{d}\num{0.276}     & \tnote{d}\num{0.482}                                 \\

In the interest of reproducibility of  results, the design decisions made by the
author on  a few important aspects  of setting up the  simulations are discussed
next.

\subsection{Simulation Setup}

Smaller parametrisation requirements  is only half the story  towards speed The.
cells lower cutoff voltage is chosen to be \SI{2.5}{V}                         .
