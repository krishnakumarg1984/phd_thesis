% -*- root: ../main.tex -*-
%!TEX root = ../main.tex
% this file is called up by main.tex
% content in this file will be fed into the main document

%level followed %by section, subsection


% ----------------------- paths to graphics ------------------------

% change according to folder and file names
\graphicspath{{4/figures/}}
% ----------------------- contents from here ------------------------

Battery modellers face the classic conundrum of conjuring physics-based battery models that remain amenable for control
applications. Firstly, the contrasting nature of this modelling objective is presented. Secondly, prior attempts by the
research community to tackle this issue is briefly examined. A suitable family of models from the broad category of
reduced-order models is identified as a promising candidate for implementation in controls applications. Next, the
drawbacks of this family of models is discussed in detail. The state of the art implementation for tackling these
drawbacks is presented and their inadequacies are discussed.

\todo{\relsize{-1} write the chapter first. Come back to summarize this.}
The following efforts/trials were done (failures)
\begin{itemize}
    \item first attempt
    \item second attempt
\end{itemize}
The following successes were achieved.
\begin{itemize}
    \item first attempt
    \item second attempt
\end{itemize}

At the end of this chapter, we have a control oriented reduced order battery model amenable for use in real-time
applications for SOC, SOH etc.\ estimations.

Control-Oriented models can be considered synonymous with the term `Reduced Order Models'. This is because the
complexity of physics-based models inherently necessitates the use of a low order model. In this thesis, the two terms
are used interchangeably.

\section{Control-Oriented Time-domain Models}

Physics-based Control-Oriented Models broadly fall into the following categories.

\begin{itemize}
    \item Models obtained through model order reduction of physics based models
    \item Models motivated by physical principles,but formulated directly as reduced order models
    \item A hybrid combination of the above
\end{itemize}

Furthermore, it becomes important to distinguish between models that are derived directly in the time domain versus
those that are derived first in the frequency domain and later converted to time domain. Since the
topic of interest is in online application of battery models, models that are entirely implemented
in the frequency-domain is out of scope of this thesis.

Smith et al \cite{Smith2007}

% \begin{figure}[htb]
% \begin{algorithmic}[1]

% \Procedure{SUM}{ $\{x\}$}

% \State $y\gets0$
% \For{$i \gets 1 : N^{x}$} \Comment{Time series $\{x\}$ has length $N^{x}$}
%    \State $y\gets y+x(i)$ \Comment{Summing up.}
% \EndFor

% \State \textbf{return}  $y$
% \EndProcedure
% \end{algorithmic}
% \caption[Implementation of a algorithm for calculating a sum.]{Implementation of a algorithm for calculating a sum.}
% \label{fig:algorithm1}
% \end{figure}




% ---------------------------------------------------------------------------
%: ----------------------- end of thesis sub-document ------------------------
% ---------------------------------------------------------------------------

