% -*- root: ../main.tex -*-
%!TEX root = ../main.tex
% this file is called up by main.tex
% content in this file will be fed into the main document

%level followed %by section, subsection


% ----------------------- paths to graphics ------------------------

% change according to folder and file names
\graphicspath{{4/figures/}}
% ----------------------- contents from here ------------------------

Battery modellers face the classic conundrum of conjuring physics-based battery models that remain amenable for control
applications. Firstly, the contrasting nature of this modelling objective is presented. Secondly, prior attempts by the
research community to tackle this issue is briefly examined. A suitable family of models from the broad category of
reduced-order models is identified as a promising candidate for implementation in controls applications. Next, the
drawbacks of this family of models is discussed in detail. The state of the art implementation for tackling these
drawbacks is presented and their inadequacies are discussed.

\todo{write the chapter first. Come back to summarize this.} The following efforts/trials were done (failures)
\begin{itemize}
    \item first attempt
    \item second attempt
\end{itemize}
The following successes were achieved.
\begin{itemize}
    \item first attempt
    \item second attempt
\end{itemize}

At the end of this chapter, we have a control oriented reduced order battery model amenable for use in real-time
applications for SOC, SOH etc.\ estimations.

Control-Oriented models can be considered synonymous with the term `Reduced Order Models'. This is because the
complexity of physics-based models inherently necessitates the use of a low order model. In this thesis, the two terms
are used interchangeably.

\section{Introduction to Control-Oriented Models}

Jokar~\etal~\cite{Jokar2016} provide a comprehensive review of the various categories of reduced order physics based
battery modelling. However, a notable omission in~\cite{Jokar2016} is that it does not distinguish between models based
on time-vs-frequency domains. Fan~\etal{}~\cite{Fan2015} performed a similar review of reduced order battery models, but
only provide a brief comparative overview of models derived and implemented in these dual domains.
Unlike~\cite{Jokar2016}, this review did not aim to provide a classification of various reduced order models, but
instead emphasises on a broad survey of relevant methodologies and tools towards obtaining such models. Hence,
neither~\cite{Jokar2016} nor~\cite{Fan2015} provide specific emphasis on the rubrics and implication of the choice of
modelling in either of these dual domains. Although in principle, the transformation between them follows standard
mathematical practices,~\todo{citation(s) needed?} availability of models for final implementation in the time domain
aids immediate uptake by industry for adoption in online \gls{bms}es. Treatment of reduced order models from this aspect
is so germane to the central hypothesis of this chapter,\todo{can a simple time-domain model do the job?} that the
author of this thesis feels compelled to undertake a simpler classification exercise within the context of
implementation. In this discussion, the various modelling methodologies and the resulting models are viewed as a
continuum and hence this thesis discusses them with a unified perspective. Furthermore, there is also a need to
highlight the salient works among the more recent advances and extensions to then-prevailing models published
since~\cite{Jokar2016} and~\cite{Fan2015}. Hence, the specialised review of reduced order modelling literature covered
in this section intends to supplement ---  not supplant --- the breadth of modelling art covered between them. In
particular, care has been taken to minimize repetition of background art analysed in these aforementioned review
articles, ensuring that only a subset of prior research that is pertinent to illustrate the new classification scheme
introduced here shall be discussed.

Physics-based Control-Oriented Models can be classified as belonging to one of the following categories.\todo{just
enumerate my classification in-line here} It is important to distinguish between models that are derived directly in the
time domain versus those that are derived first in the frequency domain and later converted to time domain.\todo{write
this better}

In principle, a modelling method that yields a time-domain mathematical description of physical phenomena that is lower
in computational complexity by an arbitrary magnitude than the original \gls{dfn} model can be considered for further
investigation. In the absence of a quantitative definition of what can be considered as a true reduced order model, the
number of candidate family of models to consider is large and overwhelming. In practice, the constraints of realtime
implementation limits the choice of candidate modelling families. For instance, models relying primarily on classical
finite difference~\cite{Smith2006} and Galerkin Finite Element~\cite{Dao2012} methods for transformation and order
reduction of one or more field variables of the \gls{dfn} model are immediately excluded, owing to the impracticability
of implementing them in a resource-constrained environment.

Owing to the low entry-barrier for adoption in a real-time controller that logs samples of measurement data at specific
time intervals, this thesis focusses on models that are cast for implementation in the time domain. Such a decision
implies the exclusion of those models that are derived and implemented entirely in the frequency-domain. For the
interested reader wishing to peruse the sizeable literature in this field, the discussion here briefly alludes to a few
popular frequency domain modelling techniques.

The transfer-function oriented Padé approximation method for low order physics-based battery modelling pioneered by
Forman~\etal{}~\cite{Forman2011a} has gained widespread adoption in the areas of cell design~\cite{Marcicki2013}, charge
trajectory optimisation~\cite{Bashash2010}, controller design~\cite{Perez2015} and state
estimation~\cite{Marcicki2013,Moura2012,Bartlett2015}. Although online identification of a subset of aging parameters
using a Padé model and a recursive least squares algorithm was presented in Prasad and Rahn~\cite{Prasad2013}, a
detailed treatment was not given on the specific implementation details such as the transformation of the Padé reduced
impedance to discrete-time difference equations. Padé models are typically limited to offline applications owing to the
aggressive trade-offs required in the approximation order to maintain accuracy. Traditionally, models truncated to very
low Padé order exhibit poor fidelity and perform no better than classical equivalent circuit models, although recent
research attempts have focussed to mitigate this drawback~\cite{Yuan2017a,Yuan2017}.

Smith~\etal{}~\cite{Smith2007} pioneered the possibility of a semi-hybrid modelling approach by obtaining independent
closed form expressions for all electrochemical field variables in the frequency domain except electrolyte concentration
and potential, which were separately solved with the classical finite difference discretization method.  This is the
earliest published instance to the author's knowledge wherein the dynamics of the full order model were retained in the
frequency domain, enabled through the innovative use of transcendental transfer functions without being forced to resort
to truncation techniques such as Padé approximation. A composite impedance model was obtained which was then converted
by residue grouping and truncation techniques to a 12th order state space model thereby introducing this hybrid
modelling workflow. This model was capable of predicting the battery terminal voltage within 1\% of a full-order
\gls{dfn} model. The frequency domain impedance-based model was derived for the frequency range of interest from
\SIrange{0}{10}{\hertz} through Model order reduction of the \gls{dfn} model. The singular drawback of this model is its
complexity for an online implementation. Furthermore, extensive parametrisation efforts are required \todo{at the end,
count and substitute the canonical number of parameters here} to render the model useful for practical applications. The
difficulties associated with such extensive parametric requirements coupled with inherent uncertainties in such obtained
values act as a deterrent to stakeholders outside academia to adopt this model for online \gls{bms} implementations.
Nevertheless, this model was the first of its kind to provide a physics-based battery model in the classical state-space
formulation
\begin{align}
    \dot{x} &= Ax + Bu \\\notag
    y &= Cx + Du
\end{align}

\todo{might have to re-write as being common with Plett and offer my comments to both as an informed researcher}Although
the matrices $A, B, C$ and $D$ do not directly involve physical parameters of the \gls{dfn} model, computation of their
numeric values through the model reduction procedure has a direct relationship with them. The presence of 12 states
dilutes the effectiveness of state estimation algorithms. In the classical isothermal implementation of this model, with
only output voltage being available to measure, the observablity of the model degrades. Although observablity analysis
was proven in a noise-free context, the presence of process noise (via unmodelled phenomena and parameter uncertainty)
and sensor noise makes this model unattractive for implementation in a vehicular \gls{bms} application.

The identification of individual parameters of the \gls{dfn} model remains a key area in battery modelling that remains
only partially explored, and is key to implementation of the advanced models and control algorithms.\todo{does this fit
better in the introduction chapter?}. The state of the art in this area, the challenges involved and current efforts in
this direction are explored in~\cref{ch:chapter6}. Although sensitivity analysis of the \gls{dfn} parameters has been
performed in literature, \todo{citation here} the extent to which parameter uncertainties influence the numerical values
in the $A, B, C$ and $D$ matrices has not yet been attempted. In continuation of this research aspect, the order of
magnitude shift in eigen/singular values of the relevant system matrix also need to be quantified to enable an informed
choice about stability of such models for realtime implementations.

A key drawback of Smith~\etal{}~\cite{Smith2007} is the requirement for linearisation at a specific \gls{soc}.
In~\cite{Smith2007}, the operating point was chosen to be 50\% for the derivation of the state-space model. This
counteracts to a high degree, the usability of the model for state estimation tasks, wherein the \gls{soc} is itself the
unknown quantity to be estimated. Furthermore, the need to interpolate between matrices from a look-up table
pre-computed at different states of charge/temperature combinations is an ad-hoc approach to quantifying the model
across the entire operating range of the battery. This renders the robustness of cross-over between models during
state-estimation questionable, and demands the need for smoothing filters and other ad-hoc apparatus to obtain
acceptable \gls{soc} estimations during online operation. The requirement of linearisation also renders the model usable
only for a small range of \gls{soc}s.

Several attempts have been undertaken to improve and extend the ideas pioneered in Smith~\etal{}~\cite{Smith2007}.
Lee~\etal{}~addressed a critical missing aspect of~\cite{Smith2007}, \viz the derivation of transcendental transfer
functions for electrolyte concentration and potential. These transfer functions were obtained by using a Sturm-Louiville
approach and retaining the top 5 modes of the eigenfunction expansion, as detailed in~\cite{Lee2012} . To the author's
best knowledge this is the first published work wherein all electrochemical field variables of the \gls{dfn} model were
considered for inclusion in a deterministic model order reduction procedure whilst remaining in the frequency
domain. Furthermore, Lee~\etal devised the \gls{dra}, a novel algorithm to systematically transform all transcendental
transfer functions to the time-domain to obtain a standard state-space model~\cite{Lee2012a}.

\todo{write plett's group work in the next couple of paragraphs explaining the DRA. DRA was a way forward in bringing all the internal quantities closer to time-domain implementation rather than resorting to a lumped impedance parameter}

Physics-inspired equivalent circuit models~\cite{Merla2018,Prasad2014,Zhang2017} are a class of hybrid models that have
rapidly gained prominence since the publication of\cite{Jokar2016} and ~\cite{Fan2015}. Such models are first derived in
the frequency domain and later converted to an equivalent circuit representation. Prasad and Rahn~\cite{Prasad2014}
extended their Padé order reduced model first presented in~\cite{Prasad2013} to convert the impedance model into
standard equivalent circuits. approximation or \gls{eis} measurements, and later on converted to equivalent circuit
models. In these family of models, the classical Randles equivalent circuit model structure is retained. However, the
values of the electrical circuit components such as series resistance and equivalent capacitance  are obtained through
various mechanisms \todo{such as \ldots}. The biggest advantage of such models is that they serve as drop-in
replacements to traditional equivalent circuit models whilst still retaining their origins in physical principles rather
than empirical curve-fitting. Merla~\etal{}~\cite{Merla2018} introduced an equivalent circuit model that can be
parameterised through a systematic decoupling of the kinetics and diffusion at both electrodes and in the electrolyte.
Although these interacting phenomena can be complex to resolve over all length and time-scales, acceptable trade-offs in
accuracy was demonstrated to be achievable from a system-level simulation perspective. A drawback of such models is that
there exists an ad-hoc attribution of key model parameters such as the diffusion coefficients among the two electrodes
through non-verifiable assumptions. Furthermore, in~\cite{Merla2018}, notable discrepancies exist in the values of
parameters such as electrolyte conductivity (obtained through calculations from \gls{eis} measurements) to that
typically reported in literature. \Cref{ch:chapter6} discusses the author's efforts towards obtaining a similar
physics-informed equivalent circuit model based on the results from the improved DRA reported in
Gopalakrishnan\etal{}~\cite{Gopalakrishnan2017} and detailed in~\cref{ch:chapter3}. This physics-inspired equivalent
circuit model seeks to explore the present gap in impedance modelling by forming an equivalent composite impedance value
as laid out in~\cite{Smith2007}. The preliminary work towards this hitherto unexplored direction is briefly presented in
\cref{ch:chapter6}. This physics-informed impedance model can be extended to fit parameter values of components in an
equivalent circuit model. The scope of this attempt differs from~\cite{Merla2018} in that, the author's efforts are
focussed on suitability for online implementation for control applications as opposed to the objective of degradation
diagnosis in Merla~\etal~\cite{Merla2018}. It should be emphasised that in these category of models, there exists no
direct link between the final value of the circuit components and the physical parameters. Nevertheless, the author
believes that these models have the potential to elicit interest from stakeholders owing to the need for minimal code
changes, thereby facilitating instant adaptability to existing BMS control architectures.

\todo{tikz picture here, showing a nice layout classifying various models}

% \begin{figure}[htb]
% \begin{algorithmic}[1]

% \Procedure{SUM}{ $\{x\}$}

% \State $y\gets0$
% \For{$i \gets 1 : N^{x}$} \Comment{Time series $\{x\}$ has length $N^{x}$}
%    \State $y\gets y+x(i)$ \Comment{Summing up.}
% \EndFor

% \State \textbf{return}  $y$
% \EndProcedure
% \end{algorithmic}
% \caption[Implementation of a algorithm for calculating a sum.]{Implementation of a algorithm for calculating a sum.}
% \label{fig:algorithm1}
% \end{figure}




% ---------------------------------------------------------------------------
%: ----------------------- end of thesis sub-document ------------------------
% ---------------------------------------------------------------------------

