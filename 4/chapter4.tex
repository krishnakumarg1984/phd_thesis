% -*- root: ../main.tex -*-
%!TEX root = ../main.tex
% this file is called up by main.tex
% content in this file will be fed into the main document

%level followed %by section, subsection


% ----------------------- paths to graphics ------------------------

% change according to folder and file names
\graphicspath{{4/figures/}}
% ----------------------- contents from here ------------------------

Battery modellers face the classic conundrum of conjuring physics-based battery models that remain
amenable for control applications. Firstly, the contrasting nature of this modelling objective is
presented. Secondly, prior attempts by the research community to tackle this issue is briefly
examined. A suitable family of models from the broad category of reduced-order models is identified
as a promising candidate for implementation in controls applications. Next, the drawbacks of this
family of models is discussed in detail. The state of the art implementation for tackling these
drawbacks is presented and their inadequacies are discussed.

\todo{write the chapter first. Come back to summarize this.}
The following efforts/trials were done (failures)
\begin{itemize}
    \item first attempt
    \item second attempt
\end{itemize}
The following successes were achieved.
\begin{itemize}
    \item first attempt
    \item second attempt
\end{itemize}

At the end of this chapter, we have a control oriented reduced order battery model amenable for use
in real-time applications for SOC, SOH etc.\ estimations.

Control-Oriented models can be considered synonymous with the term `Reduced Order Models'. This is
because the complexity of physics-based models inherently necessitates the use of a low order model.
In this thesis, the two terms are used interchangeably.

\section{Control-Oriented Time-domain Models}

Jokar~\etal~\cite{Jokar2016} provide a comprehensive review of the various categories of reduced order physics based
battery modelling. However, a notable omission in~\cite{Jokar2016} is the classification of models based upon model
derivation as well as final implementation in the time-vs-frequency domains. Although in principle the transformation
between these domains follows standard mathematical practices \todo{citation needed?}, availability of models for final
implementation in the time domain models aids immediate uptake by industry for online adoption in \gls{bms}. This idea
is so germane to the hypothesis of this chapter \todo{can we form a simple, time-domain model that will get the job
done?}, that the author of this thesis feels compelled to undertake a simpler classification exercise within the context
of adaptability. Furthermore, there is also a need to highlight the salient works among the more recent advances and
extensions to then-prevailing models published since~\cite{Jokar2016}. Hence, the brief review of literature
covered in this section intends to complement and not supplant~\cite{Jokar2016}. In particular, care has been taken to
minimize repetition of background art from that reported in Jokar~\etal{}.

In this work, the model order reduction Physics-based Control-Oriented Models can be classified as belonging to one of the
following categories.

\begin{itemize}
    \item Models obtained through model order reduction of physics based models
    \item Models motivated by physical principles,but formulated directly as reduced order models
    \item A hybrid combination of the above
\end{itemize}

In principle,a modelling method that yields a time-domain mathematical description of physical phenomena that is lower
in computational complexity by an arbitrary magnitude than the original \gls{dfn} model can be considered for further
investigation. In the absence of a quantitative definition of what can be considered as a true reduced order model, the
number of candidate family of models to consider is large and overwhelming. In practice, the constraints of realtime
implementation limits the choice of candidate modelling families. For instance, models employing the classical finite
difference and Galerkin Finite Element~\cite{Dao2012} methods for transformation and order reduction of one or more field variables of
the \gls{dfn} model are immediately excluded, owing to the impracticability of implementing them in a
resource-constrained environment.

Owing to the low entry-barrier for adoption in a real-time controller that logs samples of
measurement data at specific time intervals, this thesis focusses on models that are cast for
implementation in the time domain. Such a decision implies the exclusion of those models that are
derived and implemented entirely in the frequency-domain. For the interested reader wishing to
peruse the sizeable literature in this field, the discussion here briefly alludes to a few popular
frequency domain modelling techniques.

The Padé approximation method has been widely adopted to yield low order battery
models~\cite{Forman2011}\todo{add other relevant references here}. Based on a transfer function
approach, such models are well-suited for controller design. However, they are typically limited to
such offline applications owing to the trade-offs required in the Padé approximation order.
Traditionally, models truncated to very low order exhibit poor fidelity and perform no better than
classical equivalent circuit models, although recent research attempts have focussed to mitigate
this drawback~\cite{Yuan2017a,Yuan2017}.

Yet another family of models, that have gained prominence since the publication of~\cite{Jokar2016} are physics-inspired
equivalent circuit models. \todo{Cite a few important ones here including Yuri's latest work}. In these family of
models, the classical equivalent circuit Randles model structure is retained. However, the values of the electrical
circuit components such as series resistance and equivalent capacitance  are obtained through various mechanisms
\todo{such as \ldots}. The biggest advantage of such models is that they serve as drop-in replacements to traditional
equivalent circuit models whilst still having origins rooted in physical principles rather than empirical curve-fitting.
For example,
It should be emphasised that in these category of models, typically there is no direct
link between the final value of the circuit components and the physical modelling parameters.

It is important to distinguish between models that are derived directly in the
time domain versus those that are derived first in the frequency domain and later converted to time
domain.

Smith~\etal{}~\cite{Smith2007} introduced a 12th order state-space dynamic model capable of
predicting the terminal voltage within 1\% of a full-order \gls{dfn} model. This frequency domain
impedance-based model was derived for the frequency range of interest from \SIrange{0}{10}{\hertz}
through Model order reduction of the \gls{dfn} model. The singular drawback of this model is its
extreme complexity for an online implementation. Furthermore, extensive parametrisation efforts are
required \todo{at the end, count and substitute the canonical number of parameters here} to render
the model useful for practical applications. The difficulties associated with such extensive
parametric requirements coupled with inherent uncertainties in such obtained values act as a
deterrent to stakeholders outside academia to adopt this model for online \gls{bms} implementations.
Nevertheless, this model was the first of its kind to provide a physics-based battery model in the
classical state-space formulation
\begin{align}
    \dot{x} &= Ax + Bu \\\notag
    y &= Cx + Du
\end{align}

Although the matrices $A, B, C$ and $D$ do not directly involve physical parameters of the
\gls{dfn} model, computation of their numeric values through the model reduction procedure has a
direct relationship with them. The presence of 12 states dilutes the effectiveness of state
estimation algorithms. In the classical isothermal implementation of this model, with only output
voltage being available to measure, the observablity of the model degrades. Although observablity
analysis was proven in a noise-free context, the presence of process noise (via unmodelled phenomena
and parameter uncertainty) and sensor noise makes this model unattractive for implementation in a
vehicular \gls{bms} application.

The identification of individual parameters of the \gls{dfn} model remains a key area in battery
modelling that remains only partially explored, and is key to implementation of the advanced models
and control algorithms.\todo{does this fit better in the introduction chapter?}. The state of the
art in this area, the challenges involved and current efforts in this direction are explored
in~\cref{ch:chapter7}. Although sensitivity analysis of the \gls{dfn} parameters has been performed
in literature, \todo{citation here} the extent to which parameter uncertainties influence the
numerical values in the $A, B, C$ and $D$ matrices has not yet been attempted. In continuation of
this research aspect, the order of magnitude shift in eigen/singular values of the relevant system
matrix also need to be quantified to enable an informed choice about stability of such models for
realtime implementations.

A key drawback of Smith~\etal{}~\cite{Smith2007} is the requirement for linearisation at a specific
\gls{soc}. In~\cite{Smith2007}, the operating point was chosen to be 50\% for the derivation of the
state-space model. This counteracts to a high degree, the usability of the model for state
estimation tasks, wherein the \gls{soc} is itself the unknown quantity to be estimated. Furthermore,
the need to interpolate between matrices from a look-up table pre-computed at different states of
charge/temperature combinations is an ad-hoc approach to quantifying the model across the entire
operating range of the battery. This renders the robustness of cross-over between models during
state-estimation questionable, and demands the need for smoothing filters and other ad-hoc apparatus
to obtain acceptable \gls{soc} estimations during online operation. The requirement of linearisation
also renders the model usable only for a small range of \gls{soc}s.

Several attempts have been undertaken, with varying degrees of success, to improve and extend this
controls-oriented model from Smith~\etal{}~\cite{Smith2007}.

% \begin{figure}[htb]
% \begin{algorithmic}[1]

% \Procedure{SUM}{ $\{x\}$}

% \State $y\gets0$
% \For{$i \gets 1 : N^{x}$} \Comment{Time series $\{x\}$ has length $N^{x}$}
%    \State $y\gets y+x(i)$ \Comment{Summing up.}
% \EndFor

% \State \textbf{return}  $y$
% \EndProcedure
% \end{algorithmic}
% \caption[Implementation of a algorithm for calculating a sum.]{Implementation of a algorithm for calculating a sum.}
% \label{fig:algorithm1}
% \end{figure}




% ---------------------------------------------------------------------------
%: ----------------------- end of thesis sub-document ------------------------
% ---------------------------------------------------------------------------

