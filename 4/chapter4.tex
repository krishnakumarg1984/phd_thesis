% -*- root: ../main.tex -*-
%!TEX root = ../main.tex
% this file is called up by main.tex
% content in this file will be fed into the main document
% vim:textwidth=80 fo=cqt

%level followed %by section, subsection


% ----------------------- paths to graphics ------------------------

% change according to folder and file names
\graphicspath{{4/figures/}}
% ----------------------- contents from here ------------------------

Battery modellers face the classic  conundrum of conjuring physics-based battery
models that remain  amenable for control applications.  Firstly, the contrasting
nature of this modelling objective is presented. Secondly, prior attempts by the
research community to  tackle this issue is briefly examined.  A suitable family
of models  from the broad  category of reduced-order  models is identified  as a
promising  candidate  for implementation  in  controls  applications. Next,  the
drawbacks of this family of models is  discussed in detail. The state of the art
implementation for tackling these drawbacks  is presented and their inadequacies
are discussed.

\todo{write the chapter first. Come back to summarize this.}

The following efforts/trials were done (failures)
\begin{itemize}
    \item first attempt
    \item second attempt
\end{itemize}
The following successes were achieved.
\begin{itemize}
    \item first attempt
    \item second attempt
\end{itemize}

At the  end of this  chapter, we have a  control oriented reduced  order battery
model amenable for use in real-time applications for SOC, SOH etc.\ estimations.

Control-Oriented  models can  be considered  synonymous with  the term  `Reduced
Order Models'. This is because the complexity of physics-based models inherently
necessitates the  use of a low  order model. In  this thesis, the two  terms are
used interchangeably.

\section{A new classification scheme for control-oriented models}

Jokar~\etal~\cite{Jokar2016}  provide  a  comprehensive review  of  the  various
categories of  reduced order  physics based battery  models. However,  a notable
omission  in~\cite{Jokar2016}  is  that  it  does not  aim  to  classify  models
based  on  time-vs-frequency  domains.  Fan~\etal{}~\cite{Fan2015}  performed  a
similar  review of  reduced  order  battery models,  but  only  provide a  brief
comparative overview  of models derived  and implemented in these  dual domains.
Unlike~\cite{Jokar2016}, this review did not  aim to provide a classification of
various  reduced  order  models,  but  instead  emphasises  on  a  broad  survey
of  relevant  methodologies and  tools  towards  obtaining such  models.  Hence,
neither~\cite{Jokar2016}  nor~\cite{Fan2015} provide  specific  emphasis on  the
rubrics  and implication  of the  choice of  modelling in  either of  these dual
domains. Although in principle, the transformation between them follows standard
mathematical  practices,~\todo{citation(s) needed?}  availability of  models for
final implementation  in the time domain  aids immediate uptake by  industry for
adoption  in online  \gls{bms}s. Treatment  of  reduced order  models from  this
aspect  is  so germane  to  the  central  hypothesis of  this  chapter,\todo{can
a  simple  time-domain  model do  the  job?}  that  the  author of  this  thesis
feels  compelled  to undertake  a  simpler  classification exercise  within  the
context  of  suitability for  online  implementation.  In this  discussion,  the
various  modelling  methodologies and  the  resulting  models  are viewed  as  a
continuum  and hence  this thesis  discusses  them from  a unified  perspective.
Furthermore,  there  is  also  a  need to  highlight  the  salient  works  among
the  more recent  advances and  extensions to  then-prevailing models  published
since~\cite{Jokar2016}  and~\cite{Fan2015}.  Hence,  the specialised  review  of
reduced order modelling literature covered in this section intends to supplement
---  not supplant  --- the  breadth of  modelling art  covered between  them. In
particular,  care  has been  taken  to  minimize  repetition of  background  art
analysed in these aforementioned review articles, ensuring that only a subset of
prior research  that is  pertinent to illustrate  the new  classification scheme
introduced here shall be discussed.

Physics-based control-oriented Models  can be classified as belonging  to one of
the following categories.\todo{just enumerate my classification in-line here} It
is important to distinguish between models that are derived directly in the time
domain versus  those that are  derived first in  the frequency domain  and later
converted to time domain.\todo{Need to write this better}

In  principle,  a  modelling  method  that  yields  a  time-domain  mathematical
description of physical  phenomena that is lower in  computational complexity by
an arbitrary magnitude  than the original \gls{dfn} model can  be considered for
further investigation. In  the absence of a quantitative definition  of what can
be considered as a  true reduced order model, the number  of candidate family of
models  to  consider  is  overwhelmingly large\todo{is  it  appropriate  to  use
such  language?}. In  practice, the  constraints and  challenges imposed  by the
scope  of this  work, \viz{}  suitability for  real-time implementation,  limits
the  choice  of  candidate  modelling families.  For  instance,  models  relying
primarily on  classical finite  difference~\cite{Smith2006} and  Galerkin Finite
Element~\cite{Dao2012} methods for transformation and  order reduction of one or
more field variables  of the \gls{dfn} model are immediately  excluded, owing to
the impracticability of implementing  them in a resource-constrained environment
such as an embedded \gls{bms} controller.

Owing to  the low entry-barrier for  adoption in a real-time  controller logging
data samples at specific time intervals, this thesis focusses on models that are
cast in a form for final implementation  in the time domain. This choice implies
the exclusion of  those models that are derived and  implemented entirely in the
frequency domain.  For the interested reader  who wishes to peruse  the sizeable
literature in  this field, the  discussion here  briefly alludes to  the salient
frequency domain modelling techniques.

The     transfer     function     oriented    Padé     approximation     method
for    low    order    physics-based     battery    modelling    pioneered    by
Forman~\etal{}~\cite{Forman2011a}    has   gained    widespread   adoption    in
the    areas   of    cell   design~\cite{Marcicki2013},    charging   trajectory
optimisation~\cite{Bashash2010},  controller  design~\cite{Perez2015} and  state
estimation~\cite{Marcicki2013,Moura2012,Bartlett2015}.   Although   Prasad   and
Rahn~\cite{Prasad2013} present  an online identification  of a subset  of ageing
parameters using a Padé model and a recursive least squares algorithm, specific
implementation details such as the transformation of the Padé reduced impedance
to  discrete-time  difference  equations  was not  provided.  Padé  models  are
typically limited  to offline  applications owing  to the  aggressive trade-offs
required in the approximation order to maintain accuracy. Those models truncated
to  very low  Padé  order exhibit  poor  fidelity and  perform  no better  than
classical  equivalent circuit  models,  although recent  research attempts  have
focussed  to  mitigate  this  drawback~\cite{Yuan2017a,Yuan2017}.\todo{Should  I
discuss how?}

Smith~\etal{}~\cite{Smith2007}  pioneered   a  semi-hybrid   modelling  approach
to  reduced  order  modelling  and  obtained closed  form  expressions  for  all
electrochemical field variables  in the frequency domain  except for electrolyte
concentration and  potential (which were  solved separately using  the classical
finite difference discretization method). To the author's knowledge, this is the
earliest  published  instance  wherein  all  the  dynamics  of  the  full  order
model were  completely retained  in the frequency  domain. This  was facilitated
through the  use of transcendental transfer  functions that helped to  avoid the
accuracy  degradation  brought about  by  truncation  techniques such  as  Padé
approximation. A composite impedance model  was obtained that was then converted
to  a \engordnumber{12}  order state  space model  through residue  grouping and
truncation techniques.  Thus, in the viewpoint  of the author of  this thesis, a
complete workflow of a hybrid modelling  workflow has been presented. This model
was capable of  predicting the battery terminal  voltage within \SI{1}{\percent}
of a full-order \gls{dfn} model. This frequency domain impedance-based model was
derived for the  frequency range of interest  from \SIrange{0}{10}{\hertz}. This
modelling effort  was the first of  its kind to provide  a physics-based battery
model for implementation in the classical state-space formulation
\begin{equation}
    \begin{aligned}
        \dot{x} &= Ax + Bu \\
        y &= Cx + Du
    \end{aligned}
\end{equation}

that  is amenable  for  controller design  and  further system-level  simulation
studies \eg{} as a component in a (hybrid) electric vehicle drivetrain.

The requirement of a relatively large  number of state variables to describe the
system dynamics dilutes the effectiveness of state estimation algorithms. In the
classical  isothermal implementation  of this  model, with  only output  voltage
being  the  only measured  quantity,  the  observablity  of the  model  degrades
significantly. Although Smith~\etal{} performed  an observablity analysis of the
model in  a noise-free context,  the presence  of process noise  (via unmodelled
electrochemical phenomena and parameter uncertainties) coupled with sensor noise
makes this model  unattractive for state estimation tasks in  an online embedded
application.

Several attempts have been undertaken to  improve and extend the ideas pioneered
in  Smith~\etal{}~\cite{Smith2007}.  Lee~\etal{}~addressed  a  critical  missing
aspect  of~\cite{Smith2007}, \viz{}  the derivation  of transcendental  transfer
functions  for  both electrolyte  concentration  and  potential. These  transfer
functions were  obtained by  using a Sturm-Louiville  approach by  retaining the
first five modes of the  eigenfunction expansion, as detailed in~\cite{Lee2012}.
To the  author's best knowledge,  this is the  first published work  wherein all
electrochemical  field variables  of  the \gls{dfn}  model  were considered  for
inclusion in  a deterministic model  order reduction procedure  whilst remaining
entirely within the frequency domain.

Obtaining  closed form  expressions for  the electrolyte  variables achieved  in
Lee~\etal{} also has an important computational implication. With these capstone
derivations serving to completing the model description in the frequency domain,
all  electrochemical  variables of  the  \gls{dfn}  model  could now  be  solved
independently at any  desired spatial location, \eg{} at  the current collectors
and separator  interfaces. Until  this point,  the state of  the art  in reduced
order  modelling invariably  necessitated  the solution  of all  electrochemical
quantities at  multiple node locations along  the thickness of each  cell layer,
placing a computational burden. This is  a significant deterrent to the adoption
of such  models if the intended  purpose of the  model is to simply  predict the
cell's terminal voltage.  Furthermore, among methods that retain  the full order
Fickian diffusion  for describing solid  phase diffusion within  each electrode,
this further necessitates discretisation also along the radial direction thereby
increasing  the overall  number  of  nodes by  an  order  of magnitude  directly
impacting the computational efficiency of such models.

The  impact of  high node  densities on  the computational  requirements of  the
original  \gls{p2d}  model have  led  researchers  to adopt  various  mitigation
strategies to  tackle this issue. In  contrast to the pure  frequency domain and
the  semi-hybrid/hybrid approaches  discussed thus  far, these  attempts aim  to
provided a simpler computational mesh,  whilst retaining high fidelity and hence
can be  classified as  time domain methods.  It should be  noted that  high node
densities are  mainly required near the  surface of the spherical  particles for
the  pseudo  \engordnumber{2} dimension,  whilst  such  clustering of  nodes  is
desirable near  the separator and  current collector interfaces along  the axial
dimension.\todo{need to  define `axial'  at the  start of  the thesis}.  Thus, a
large number  of model  order reduction  strategies in the  time domain  seek to
adopt  non-uniform node  spacing  towards lowering  the aforesaid  computational
issues.

Bizeray~\etal{}~\cite{Bizeray2015}         performed         a         Chebyshev
discretisation~\cite{Trefethen2000}  of   the  entire   \gls{p2d}  model   on  a
global  scale,  \ie{}  along  both   the  axial  and  radial  directions.  Thus,
a  spectral  scheme,   beyond  the  algebraic  orders   of  accuracy  achievable
with  classical  Finite  Difference,  Finite Element  or  Finite  Volume  Scheme
could  be  applied.  The reduced  number  of  nodes  as  well as  their  desired
clustering  at  the  aforementioned  spatial  locations  ensures  a  significant
lowering  of computational  burdens of  simulating a  physics-based cell  model.
Gopalakrishnan~\etal{}~\cite{Gopalakrishnan2018} undertook a  hybrid approach by
retaining a standard finite volume discretisation in the axial domain equations,
whilst adopting the  Chebyshev discretisation only for the  critical solid phase
diffusion component in the radial direction. Albeit in time domain, the standard
\gls{p2d} equations, their boundary conditions and corresponding field variables
are  mathematically transformed  to the  Chebyshev space  within which  they are
solved.  The details  of this  transformation is  presented in  \cref{sec:hybrid
fv-spectral} in  \cref{ch:chapter2} for  the case  of the  solid-phase diffusion
equation.  Finally,   these  solved  quantities   are  converted  back   to  the
physical  space through  a corresponding  inverse transformation.  Although this
bi-directional transformation is algebraic in nature, the requirement of running
a spatially resolved model coupled with  such overheads render these category of
time domain models unsuitable for online implementation. In this context, the
contribution of Lee~\etal{}, \ie{} the ability to solve for any electrochemical
variable at an arbitrary spatial location by completely eliminating the need for
spatial discretisation assumes special significance.

Furthermore,  Lee~\etal{}   also  devised  the  \gls{dra},   a  novel  algorithm
to  systematically  transform  all  transcendental  transfer  functions  to  the
time-domain to  obtain the standard state-space  model given by~\cite{Lee2012a}.
The  author of  this thesis  considers  the formulation  of the  \gls{dra} as  a
breakthrough  contribution  that  has  helped  in  bringing  a  physics-informed
time-domain  model  a  step  closer  to  online  implementation  without  having
to  resort to  forming  a  lumped impedance  and  then  truncating it  suitably.
Lee~\etal{}~\cite{Lee2014}  then  extended  this  work  for  a  wider  range  of
operation  across  various  \gls{soc},  temperature and  C-rates.  Although  the
final  state  space  model  is  simple to  implement,  the  classical  \gls{dra}
scheme   proposed  by   Lee\etal{}   suffers   from  significant   computational
bottlenecks  in   forming  the  required   block  Hankel  matrices   during  the
model-derivation phase.  A memory-efficient version of  the \gls{dra} exploiting
the   skew-symmetric   structure  of   these   Hankel   matrices  was   proposed
in  Gopalakrishnan~\etal{}~\cite{Gopalakrishnan2017}  resulting  in  drastically
reduced memory and processor requirements.

In both the original  as well as the improved \gls{dra},  the computation of the
EigenFunction modal expansion of electrolyte concentration transfer function was
slow.  A less  severe disadvantage  with the  transcendental transfer  functions
associated with the electrolyte concentration was that their derivation entailed
mathematically cumbersome  symbolic manipulations  that dictated  the need  of a
capable \gls{cas} package.  Albeit from a standalone  viewpoint this requirement
does  not  seem to  be  critical,  the Ho-Kalman  algorithm  that  forms a  core
component  of the  \gls{dra}  schemes  is steeped  in  numerical linear  algebra
routines. Furthermore, for facilitating  state estimator and controller designs,
it is  convenient to implement  the resultant  state-space model in  a classical
numerical  computation environment  such as  \textsc{MATLAB}. Taking  these into
consideration,  Rodriguez~\etal{}~\cite{Rodriguez2017}  introduced a  simplified
computation of the  electrolyte concentration transfer function  by applying the
variation of  parameters (VOP) scheme.  With this final improvement,  the hybrid
scheme by Lee~\eta{}  can be considered feature-complete  with low computational
requirements during both model derivation and implementation phase.

\todo{revise the  below paragraph as  drawbacks in both  dra as well  as smith's
work}

A key  drawback of  both Smith~\etal{}~\cite{Smith2007}  is the  requirement for
linearisation at a specific  \gls{soc}. In~\cite{Smith2007}, the operating point
was  chosen  to be  50\%  for  the derivation  of  the  state-space model.  This
counteracts to  a high degree, the  usability of the model  for state estimation
tasks, wherein  the \gls{soc} is  itself the  unknown quantity to  be estimated.
Furthermore,  the need  to interpolate  between  matrices from  a look-up  table
pre-computed at different states of charge/temperature combinations is an ad-hoc
approach  to quantifying  the model  across the  entire operating  range of  the
battery.  This  renders  the  robustness of  cross-over  between  models  during
state-estimation questionable,  and demands the  need for smoothing  filters and
other ad-hoc apparatus to obtain  acceptable \gls{soc} estimations during online
operation. The requirement  of linearisation also renders the  model usable only
for a small range of \gls{soc}s.Lee~\etal{}~\cite{Lee2014} overcame this through
an ad-hoc approach (only masks the issue temporarily.....rewrite)

Physics-inspired equivalent circuit models~\cite{Merla2018,Prasad2014,Zhang2017}
are  a  class  of  hybrid  models that  have  rapidly  gained  prominence  since
the publication  of~\cite{Jokar2016} and~\cite{Fan2015}.  The derivation  of the
relevant model  equations is performed  in the frequency domain.  This frequency
domain representation  is then converted  to a form suitable  for implementation
as  an  equivalent circuit.  Prasad  and  Rahn~\cite{Prasad2014} extended  their
Padé order  reduced model first  presented in~\cite{Prasad2013} to  convert the
impedance model into standard equivalent circuits. A key point to be highlighted
is  that  these  family of  models  do  not  necessarily  strive to  retain  the
classical  Randles  structure~\cite{Randles1947}  for their  equivalent  circuit
model representation. However,  the values of the  electrical circuit components
such  as  series resistance  and  equivalent  capacitance are  obtained  through
various  mechanisms \todo{such  as  EIS measurements  and  \ldots}. The  biggest
advantage  of  such  models  is  that they  serve  as  drop-in  replacements  to
traditional equivalent  circuit models whilst  still retaining their  origins in
physical principles rather than on empirical curve-fitting.

Merla~\etal{}~\cite{Merla2018} introduced  an equivalent circuit model  that can
be  parameterised by  attempting a  systematic  decoupling of  the kinetics  and
diffusion at  both electrodes  and the  electrolyte. Although  these interacting
phenomena can be complex to resolve  over all length and time-scales, acceptable
trade-offs in  accuracy was  demonstrated to be  achievable from  a system-level
simulation perspective. A  drawback of such models is that  key model parameters
such  as  solid  and  electrolyte   diffusion  coefficients  are  attributed  to
the  two electrodes  through  ad-hoc,  non-verifiable assumptions.  Furthermore,
in~\cite{Merla2018},  notable discrepancies  exist in  the values  of parameters
such as  electrolyte conductivity (obtained through  calculations from \gls{eis}
measurements) to that typically reported in literature.

\Cref{ch:chapter6}  briefly presents  the author's  latest work  and preliminary
results  towards   obtaining  a  similar  physics-informed   equivalent  circuit
model.  This  modelling  effort  is  a   direct  application  of  the  gains  in
computing  efficiency   through  the  application  of   the  improved  \gls{dra}
reported   in   Gopalakrishnan~\etal{}~\cite{Gopalakrishnan2017}  the   detailed
coverage  of  which  is   performed  in~\cref{ch:chapter3}.  This  work  intends
to  address  the hitherto  unexplored  gap  in  impedance modelling,  \ie{}  the
absence  of  an equivalent  impedance  model  that  accounts for  the  impedance
contribution from all electrochemical quantities  from the \gls{p2d} model. This
physics-informed  impedance  model  can  be extended  to  fit  parameter  values
of  components  in an  equivalent  circuit  model,  \eg{} through  the  standard
Levenberg-Marquardt~\cite{Levenberg1944, Marquardt1963} non-linear least squares
fitting algorithm.  The scope of  this attempt differs  from~\cite{Merla2018} in
that, the author's efforts are focussed on suitability for online implementation
for control applications as opposed to the objective of degradation diagnosis in
Merla~\etal~\cite{Merla2018}.

Finally, it should  be emphasised that, for  physics-inspired equivalent circuit
models, akin to the pseudo-hybrid model  of Smith~\etal{} as well as the various
derivative models employing  the \gls{dra}, there exists no  direct link between
the final value  of the circuit components and the  original physical parameters
of the \gls{dfn} model. This limits the applicability of the model, sapping them
of  the  powerful prediction  powers  that  are  inherently the  most  favorable
characteristic of  physics based models.  Despite this,  owing to the  fact that
their  final  implementation  dictates  only minimal  code  changes  for  online
implementation,  the author  believes that  these models  do have  the potential
to  elicit interest  from  industry stakeholders,  thereby facilitating  instant
adaptability to existing BMS control architectures.

The  singular   drawback  of   \todo{everthing  thus   far}  is   the  extensive
parametrisation efforts are required \todo{at  the end, count and substitute the
canonical number  of parameters here} to  render the model useful  for practical
applications. The  difficulties associated  with such  extensive parametrisation
requirements coupled with inherent uncertainties  in such obtained values act as
a deterrent  to stakeholders  outside academia  to adopt  this model  for online
\gls{bms} implementations.Among the various family of models presented thus far,
the lack of  physical meaning of the final model  parameters limits the insights
offered by these  models in the following important ways.  Although the matrices
$A, B, C$ and  $D$ do not directly involve physical  parameters of the \gls{dfn}
model, computation of their numeric values through the model reduction procedure
has a direct relationship with them.

\todo{\eg{}directly impacts the cell designer \ldots}

\todo{Where  does this  paragraph  go?  End of  this  section  or chapter?}  The
identification of  individual parameters  of the \gls{dfn}  model remains  a key
area in  battery modelling that remains  only partially explored, and  is key to
implementation of the advanced models and control algorithms.\todo{does this fit
better  in the  introduction  chapter?}. The  state  of the  art  in this  area,
the  challenges involved  and current  efforts  in this  direction are  explored
in~\cref{ch:chapter6}. Although sensitivity analysis of the \gls{dfn} parameters
has  been performed  in literature,  \todo{citation  here} the  extent to  which
parameter uncertainties influence the numerical values  in the $A, B, C$ and $D$
matrices has  not yet been attempted.  In continuation of this  research aspect,
the order  of magnitude shift  in eigen/singular  values of the  relevant system
matrix also need  to be quantified to enable an  informed choice about stability
of such models for real-time implementations.


\todo{pending SPM and PP2D}

\todo{Tikz picture  here, showing a nice  layout of various models  according to
the new classification scheme}

% \begin{figure}[htb]
% \begin{algorithmic}[1]

% \Procedure{SUM}{ $\{x\}$}

% \State $y\gets0$
% \For{$i \gets 1 : N^{x}$} \Comment{Time series $\{x\}$ has length $N^{x}$}
%    \State $y\gets y+x(i)$ \Comment{Summing up.}
% \EndFor

% \State \textbf{return}  $y$
% \EndProcedure
% \end{algorithmic}
% \caption[Implementation of a algorithm for calculating a sum.]{Implementation of a algorithm for calculating a sum.}
% \label{fig:algorithm1}
% \end{figure}




% ---------------------------------------------------------------------------
%: ----------------------- end of thesis sub-document ------------------------
% ---------------------------------------------------------------------------

