% -*- root: ../main.tex -*-
%!TEX root = ../main.tex
% this file is called up by main.tex
% content in this file will be fed into the main document
% vim:textwidth=80 fo=cqt

\capolettera{B}{ased}  upon  the insights  gained  from  the extensive  analyses
performed in  \cref{ch:spmanalysis}, this  chapter presents the  thesis author's
attempts to  surpass the performance  of the  current pinnacle in  modelling art
through the development of a new  electrolyte model. In particular, this chapter
describes  the attempts  towards  arriving  at an  accurate  description of  the
spatio-temporal  evolution of  the electrolyte  concentration in  a lithium  ion
cell. The performance of the  quadratic approximation model which was introduced
in \cref{ch:spmanalysis} is analysed through the novel application of a symbolic
regression framework. This  helps to expose the issue of  equation deficiency in
the underlying \gls{dfn} model and questions  its suitability for the purpose of
reduced  order  modelling  of  the electrolyte  concentration  dynamics  with  a
pre-assumed equation structure. Although this framework did not ultimately yield
the desired outcomes, it did nevertheless facilitate a comprehensive analysis of
the  strengths  and  weaknesses  of  the  quadratic  approximation  model  which
has  not  been performed  in  existing  literature.  Next, the  author's  unique
contribution to the art of single particle modelling \viz~a novel time-evolution
model of  electrolyte concentration evolutions  through the technique  of system
identification  is  presented. The  results  of  the  new approach  is  compared
against  the baseline  quadratic approximation  model as  well as  the benchmark
\gls{p2d}  model. Finally,  the new  model  equations are  incorporated into  an
electrolyte-enhanced  \gls{spm}  and  the  improved  performance  of  the  newly
developed composite model is quantified.

