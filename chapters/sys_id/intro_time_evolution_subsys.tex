% -*- root: ../main.tex -*-
%!TEX root = ../main.tex
% this file is called up by main.tex
% content in this file will be fed into the main document
% vim:textwidth=80 fo=cqt

The                  first                 order~\glspl{ode}                  of
\crefrange{eq:negliionmolesquadratic}{eq:posliionmolesquadratic} in the baseline
quadratic  approximation  model  for   electrolyte  concentration  describe  the
time-evolution of the overall number of moles  of \ch{Li^+} in each of the three
regions of the cell $Q_{\text{e,}j}$.  Through system identification, the author
seeks  to obtain  the two  rational transfer  functions of  $Q_\text{e,neg}$ and
$Q_\text{e,pos}$  to the  applied current  $I$  in the  frequency domain \ie~
$\frac{Q_\text{e,n}(s)}{I(s)}$ and $\frac{Q_\text{e,p}(s)}{I(s)}$.  Based on the
\gls{dfn} model,  the total moles of  \ch{Li^+} per unit area  in the separator,
$Q_\text{e,s}$ is  not a function  of the exogenous applied  current. Therefore,
the baseline  quadratic approximation  \gls{ode} is  retained for  computing its
time-evolution.

