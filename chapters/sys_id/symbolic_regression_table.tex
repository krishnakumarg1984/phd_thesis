% -*- root: ../../main.tex -*-
%!TEX root = ../../main.tex
% this file is called up by main.tex
% content in this file will be fed into the main document
% vim:nospell textwidth=180 foldlevelstart=3 foldlevel=3 conceallevel=0

\begin{table}[!htbp]
    \centering
    \caption[Transient \& \glsfmtshort{qss} expressions for electrolyte
    concentration obtained by \glsfmtshort{mggp}]{Best fit expressions for the
        transient and \glsfmtlong{qss} approximaton functions for the
    electrolyte functions obtained by the \gls{mggp} approach.}
    \label{tbl:symbreg}
    \begingroup
    \addtolength{\jot}{0.25em}
    \begin{tabular}{@{} c c r @{}}
        \toprule
        \multicolumn{1}{l}{Transient Function} & \multicolumn{1}{c}{\glsfmtlong{qss} Function} & \multicolumn{1}{c}{Region} \\
        \midrule
        $\begin{aligned}
            c_{\text{e,n}_\text{trans}} &= a_1(t) z^6 \ln z^6 + a_0(t) \\
            c_{\text{e,s}_\text{trans}} &= a_4(t) z^2 + a_3(t) z + a_2(t) \\
            c_{\text{e,p}_\text{trans}} &= a_6(t) z^6 \ln z^6 + a_5(t) \\
        \end{aligned}$ &
        $\begin{aligned}
            c_{\text{e,n}_\text{QSS}} &= a_1(t) \sinh z^2 + a_0(t) \\
            c_{\text{e,s}_\text{QSS}} &= a_4(t) z^2 + a_3(t) z + a_2(t) \\
            c_{\text{e,p}_\text{QSS}} &= a_6(t) \sinh z^2 + a_5(t)
        \end{aligned}$ &
        $\begin{aligned}
            &0 \le z \le l_\text{n} \\
            &0 \le z \le l_\text{s} \\
            &0 \le z \le l_\text{p}
        \end{aligned}$
        \\
        \bottomrule
    \end{tabular}
    \endgroup
\end{table}

