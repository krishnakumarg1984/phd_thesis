% -*- root: ../main.tex -*-
%!TEX root = ../main.tex
% this file is called up by main.tex
% content in this file will be fed into the main document
% vim:textwidth=80 fo=cqt


In the  author's analysis of the  quadratic approximation model, the  origin and
nature of  its sub-optimal  performance can  be explained  as per  the following
rationale. The quadratic  approximation model uses a  bottom-up approach wherein
the final  simplified model structure  is pre-assumed  and then the  physics are
made  to  fit  within  this  framework. Given  that  the  top-down  approach  to
electrolyte  modelling \ie{}  accounting for  all physical  phenomena and  then
simplifying  them yields  mathematically intractable  and overly  complex models
(see \cref{sec:electrolyteinclusion} for  a detailed discussion),  this approach
seems to be a pragmatic alternative  to enhancing the \gls{spm} with electrolyte
dynamics. However, a detailed look at  this model from an alternate viewpoint is
necessary for further analysis.


\subsection{Symbolic regression using \glsfmtlong{mggp}}\label{subsec:symbolicreg}

The  question boils  down to  whether a  quadratic approximation  is indeed  the
\emph{best}  model structure  that  can  be assumed  a~priori  for the  spatial
profile  of  ionic  concentration  in  the  electrolyte.  This  author  embarked
on  a journey  to  find  suitable alternate  model  structures \ie{} a  single
family of  curves that can  capture both  the transient and  \gls{qss} behaviour
exhibited by  the ionic  concentration. The open-source  \textsc{MATLAB} toolbox
GPTIPS2~\cite{Searson2015}  uses the  state-of-the-art  \gls{mggp} approach  for
symbolic data  mining and is ideally  suited for such symbolic  regression tasks
(fitting a  mathematical equation structure,  and not merely  obtaining best-fit
coefficients to a  pre-assumed curve as in classical numerical  regression, to a
given data).

In employing the \gls{mggp} approach, it  is important to recognize that the key
criteria that restricts
\begin{enumerate*}[label=\emph{\alph*})]
    \item the choice of  gene-sequence depth, as well as
    \item the choice in number of parent mutations,
\end{enumerate*}
is  the   total  number  of   unknown  symbolic  coefficients  required   to  be
solved  in the  assumed  model  structure. There  are  a  total of  \emph{seven}
linear   equations  available   from  the   physics  of   the  \gls{dfn}   model
(see \crefrange{eq:cecontinuitynegsep}{eq:Qepbyintegration}). Hence, in order to
guarantee  a solution  the  assumed  family of  curves  cannot  consist of  more
than  seven coefficients.  Furthermore, for  a  unique solution,  the number  of
coefficients must be \emph{exactly} seven. Yet another restriction on the choice
of locus of feasible  curves arises due to the fact that  the behaviour of ionic
concentration  in the  negative and  positive electrode  regions are  similar in
complexity\footnote{\label{fn:complexity}The  concept  of complexity  of  curves
used  here is  not  based on  a  precise mathematical  definition  such as  that
employed by Neumann-Coto and Arenas~\cite{Neumann-Coto2017}, but is loosely used
to simply  convey an empirical sense  of their curvature. However,  the analysis
here applies  to the more  rigorous definition as well.},  and hence need  to be
mathematically described by an identical family of curves.

Upon a close inspection of the  spatial concentration profile from the \gls{p2d}
simulation  results shown  in \cref{fig:spatialionicconc1C}, it  is evident  the
electrolyte approximation  functions within the  electrode regions is  of higher
complexity\textsuperscript{\ref{fn:complexity}} than  the approximation function
suitable  for  use  in  the  separator  region. Based  on  the  results  of  the
quadratic  model, it  is  clear  that at  least  two  coefficients are  required
within the  electrode regions \ie{}  $n_{\text{c,elec}} \ge 2$. There  exists an
inhibiting  factor that  prevents  the  use of  a  lower  order function  within
the  separator. As  per  the  simulation results  of  the  \gls{p2d} model,  the
time-domain  change  of number  of  moles  per  square  meter in  the  separator
is  non-zero.  Since  the  time-derivate   of  a  linear  approximation  applied
to \cref{eq:sepliionmolesquadratic}  is  zero,  this straight-line  equation  is
immediately ruled out. Among the  non-polynomial mathematical curves tried (such
as trigonometric, hyperbolic, power series  among others), none could obtain the
relatively simple shape of the separator function without being forced to reduce
the  contribution from  one  of  the coefficients  to  below machine  precision.
This necessitates  the retention  of the  quadratic approximation  function used
thus  far  (with  no  missing  terms) \ie{}  $n_\text{c,sep}  =  3$.  Thus,  the
overall number  of coefficients in the  best possible approximation shall  be $2
n_\text{c,elec} + n_\text{c,sep} =  2\cdot2 + 3 = 7$, which  is the total number
of electrolyte-specific physical constraints available from the \gls{dfn} model.
Hence it  can be concluded  that the  quadratic approximation model  does indeed
make the \emph{best}  use of all of the available  physical equations. The final
question remains to  answer is, with these coefficient  limitations, whether the
quadratic  equation structure  indeed  the optimal  one.  This was  investigated
through the \gls{mggp} approach described next.

The GPTIPS2  toolbox uses a variety  of heuristic algorithms from  the theory of
\gls{mggp}  to hypothesise  a suitable  equation structure  for the  data to  be
fitted.  The dataset  consisted of  the  simulation results  from the  \gls{p2d}
model,  in particular  the  values  of electrolyte  concentration  at the  three
different cell  regions captured  at various snapshots  of time.  Both transient
and  \gls{qss} data  were  fed into  this  symbolic data  mining  process and  a
single all-encompassing  family of curves  capable of capturing  the electrolyte
concentration behaviour was  sought for. However, the constraints  in the number
of coefficients that  can be employed results  in a restriction of  the depth of
gene  mutations as  well  as the  number of  parent/seed  populations. The  best
equation set  (without strictly  enforcing the aforementioned  hard constraints,
yet  minimizing  the  distance  to  the constraint  vector)  that  the  symbolic
regression approach yielded was
\begin{alignat}{2}
    c_\ensub(z,t) & = a_2(t) \cosh z^2 + a_1(t) \sinh z + a_0(t) \qquad &  & 0 \le z \le l_\text{n} \label{eq:cnstrviolneg} \\
    c_\essub(z,t) & = a_5(t) z^2 + a_4(t) z + a_3(t) \qquad             &  & 0 \le z \le l_\text{s}                         \\
    c_\epsub(z,t) & = a_8(t) \cosh z^2 + a_7(t) \sinh z + a_6(t) \qquad &  & 0 \le z \le l_\text{p}\label{eq:cnstrviolpos}
\end{alignat}

Although \crefrange{eq:cnstrviolneg}{eq:cnstrviolpos}  fit   the  transient  and
\gls{qss}  profiles  well,  they  violate   the  constraint  on  the  number  of
coefficients  available resulting  in  an underdetermined  system of  equations.
Both  the least-squares  and  least-norm  solution of  this  system were  tried.
However, the  results were inferior to  that produced by the  baseline quadratic
approximation method.

Next, an  attempt was made to  obtain different mathematical structures  for the
transient phase  and \gls{qss}  phase both  of which  respect the  constraint on
number  of  coefficients  allowed.  The symbolic  regression  outputs  for  this
approach are shown in \cref{tbl:symbreg}.
% -*- root: ../../main.tex -*-
%!TEX root = ../../main.tex
% this file is called up by main.tex
% content in this file will be fed into the main document
% vim:nospell textwidth=180 foldlevelstart=3 foldlevel=3 conceallevel=0

\begin{table}[!htbp]
    \centering
    \caption[Transient \& \glsfmtshort{qss} expressions for electrolyte
    concentration obtained by \glsfmtshort{mggp}]{Best fit expressions for the
        transient and \glsfmtlong{qss} approximaton functions for the
    electrolyte functions obtained by the \gls{mggp} approach.}
    \label{tbl:symbreg}
    \begingroup
    \addtolength{\jot}{0.25em}
    \begin{tabular}{@{} c c r @{}}
        \toprule
        \multicolumn{1}{l}{Transient Function} & \multicolumn{1}{c}{\glsfmtlong{qss} Function} & \multicolumn{1}{c}{Region} \\
        \midrule
        $\begin{aligned}
            c_{\text{e,n}_\text{trans}} &= a_1(t) z^6 \ln z^6 + a_0(t) \\
            c_{\text{e,s}_\text{trans}} &= a_4(t) z^2 + a_3(t) z + a_2(t) \\
            c_{\text{e,p}_\text{trans}} &= a_6(t) z^6 \ln z^6 + a_5(t) \\
        \end{aligned}$ &
        $\begin{aligned}
            c_{\text{e,n}_\text{QSS}} &= a_1(t) \sinh z^2 + a_0(t) \\
            c_{\text{e,s}_\text{QSS}} &= a_4(t) z^2 + a_3(t) z + a_2(t) \\
            c_{\text{e,p}_\text{QSS}} &= a_6(t) \sinh z^2 + a_5(t)
        \end{aligned}$ &
        $\begin{aligned}
            &0 \le z \le l_\text{n} \\
            &0 \le z \le l_\text{s} \\
            &0 \le z \le l_\text{p}
        \end{aligned}$
        \\
        \bottomrule
    \end{tabular}
    \endgroup
\end{table}



Although  the equations  from \cref{tbl:symbreg}  produced  a markedly  improved
response during the transient phase,  the performance during the \gls{qss} phase
was merely at par to the baseline quadratic approximation model. This raised the
prospect of  employing a  blended approach, wherein  a model  changeover between
the  transient  and \gls{qss}  was  contemplated.  However,  since there  is  no
precise definition  of what constitutes  the transient phase of  the electrolyte
concentration response, this  approach required some ad-hoc  timing criteria for
correctly  transitioning  between  the  two \gls{mggp}  equation  sets.  Further
complications arise  during dynamic input conditions,  wherein the concentration
profiles are mostly in a state of  flux and could linger for longer durations at
the contiguous  boundary between  transient-like and  \gls{qss}-like behaviours.
In  the  interest  of  reproducibility  across  different  cell-chemistries  and
corresponding parameter sets, the proposed blended model transition approach was
not further pursued.

Overall,  the  long  and  arduous  process of  symbolic  regression  was  not  a
definitive  success in  this  case  mainly due  to  the  limitation of  equation
deficiency of physical constraints. Perhaps  if yet another physics-based model,
\ie{} an  alternative to the  widely prevalent \gls{dfn}  model, can be  used as
the  baseline,  a  few  more  physical governing  equations  could  possibly  be
made  available. This  can perhaps  result in  a less  restrictive gene-set  for
coefficient  determination  and  consequently  pave the  way  for  a  successful
implementation  through  this  hitherto  unexplored  route  of  \gls{mggp}-based
equation synthesis.

In conclusion,  the quadratic model for  electrolyte concentration approximation
makes the  best use of the  available physical equations. Given  the constraints
with  respect to  physical equations  discussed here,  it is  also deemed  to be
the  optimum  family  of  a~priori  chosen  curves  capable  of  modelling  the
\emph{spatial}  profile of  ionic concentration.  Notwithstanding these  merits,
the  \emph{temporal}  performance of  the  quadratic  approximation approach  is
sub-optimal as seen in \cref{fig:temporalcequadratic}. The author of this thesis
addresses this specific issue with a completely different approach that shall be
discussed next in \cref{sec:newelectrolytemodel}.

