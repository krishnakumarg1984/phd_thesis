% -*- root: ../../main.tex -*-
%!TEX root = ../../main.tex
% this file is called up by main.tex
% content in this file will be fed into the main document
% vim:textwidth=80 fo=cqt

Having performed a  comprehensive analysis of the state of  the art in \gls{spm}
modelling with electrolyte  dynamics, this section presents  the author's unique
contribution to the field. Firstly, the scope of the contribution is identified.
The methodology adopted and corresponding results are presented thereafter.

\subsection{Scope and motivation}\label{subsec:scopenewelectrolyte}

This subsection is intended as a capstone summary helping to briefly recount the
discussion so far  and to provide a  context for the author's work  in the wider
realm  of the  \gls{spm}  modelling art.  In  the same  vein  as the  discussion
in \cref{sec:electrolyteinclusion}, the scope of the proposed enhancement to the
\gls{spm}  concerns  entirely with  improving  the  electrolyte subsystem  since
it  has already  been established  in \cref{subsec:simresultsbasicspm} that  the
simplified representation  of the solid-phase  subsystem through a  fourth order
polynomial approximation method for diffusion of \ch{Li^0} in the solid particle
is of sufficiently high accuracy.

Inspecting the electrolyte domain, the contribution of electrolyte overpotential
to  terminal  voltage   consists  of
\begin{enumerate*}[label=\emph{\alph*})]
    \item diffusion   overpotential
    \item time-dependent   ohmic  losses   that  originates   from  differential concentration  gradients  (that  is   indirectly  dependent  upon
concentration).
\end{enumerate*}
Once       electrolyte        concentration       at        each       time-step
is         available, \cref{eq:electrolytepdwithce}          proposed         by
Prada~\etal~\cite{Prada2012}  may  be  used for  the  electrolyte  overpotential
computation.  Hence, the  accurate  determination of  spatio-temporal values  of
electrolyte concentration merits focus.

There exists a  subtle detail in the  use of \cref{eq:electrolytepdwithce} which
is discussed here upfront before proceeding  ahead to the refined context of the
author's work. The intrinsic conductivity of electrolyte~$\kappa$ is a function
of the ionic concentration  (refer \cref{subsec:basicspmsimsetup}). If the ionic
concentration at  the corresponding current  collectors are used  for evaluating
$\kappa_\text{neg}$  and  $\kappa_\text{pos}$, this  would  lead  to a  lopsided
computation of the overpotential in electrolyte. Furthermore, under this scheme,
the computation of electrolyte conductivity shall be rendered ambiguous since it
is unclear which  separator interface shall be chosen for  the separator's ionic
concentration. Although this  has not been discussed clearly  in literature, the
author  of this  thesis chose  to use  the mean  concentration within  each cell
region, defined as
\begin{equation}
    \mean{c}_{\text{e},j}(t) = \frac{1}{l_j}\int_0^{l_j} c_{\text{e}_j}(z,t)\, dz = \frac{Q_{\text{e,}j}(t)}{\varepsilon_j l_j}
\end{equation}
though other  measures of central  tendency might  be equally valid.  Hence, the
results of this section have the associated variability in them depending on how
the electrolyte concentration computations are  used in evaluating the intrinsic
conductivity of electrolyte.

As  the  ionic  concentration  has  both  a  direct  and  indirect  contribution
in \cref{eq:electrolytepdwithce}, its spatio-temporal  computation is a critical
aspect. As discussed  in \cref{sec:quadraticapprox}, the quadratic approximation
is a widely used spatio-temporal model for electrolyte concentration which makes
the best  use of available physical  constraints. As established in  the results
of \cref{subsec:quadraticsimresultsanalysis}, while  the spatial  performance of
the quadratic approximation approach is acceptable, its time-domain performance,
particularly at  the crucial  locations of the  current collector  interfaces is
mediocre at best.

The  \emph{scope}  of  the  author's   work  is  to  obtain  suitable  alternate
expressions  for  improving  the   computation  of  \textbf{time  evolution}  of
the  electrolyte  concentration  whilst retaining  the  quadratic  approximation
approach   for   describing  its   spatial   profile.   Such  an   approach   is
motivated   by    the   keen    observation   that   the    baseline   quadratic
approximation  model  has  a  natural  `pause'  in  its  model  description.  To
clarify, \crefrange{eq:cecontinuitynegsep}{eq:Qepbyintegration}  form a  tightly
coupled  square  system  \ie~a  set   of  seven  linear  equations  in  seven
unknowns.   In   this   system,   the   time   evolution   of~$Q_{\text{e,}j}$
are   described  through   a  separate   system  of   first  order   \glspl{ode}
given by \crefrange{eq:negliionmolesquadratic}{eq:posliionmolesquadratic}.  In a
practical  implementation,  these  \glspl{ode}  are solved  independently  in  a
decoupled  manner \ie~by  using  the coefficients  obtained  from the  linear
system of \crefrange{eq:cecontinuitynegsep}{eq:Qepbyintegration} in the previous
time-step. The  author's hypothesis is that  by taking advantage of  the natural
break  in the  operational  sequence which  involves  two separate  computations
between  two nearly  independent  subsystems,  it must  be  possible to  replace
the  under-performing  time-evolution  \glspl{ode} from  the  baseline  quadratic
approximation with a superior alternate model.

\subsection{Selection of Methodology --- Background and Rationale}\label{subsec:sysidbackground}

% This section presents  the methodology adopted in obtaining  an improved model
% for the rate of evolution of overall  moles per unit area of \ch{Li^+} ions in
% each of the three regions of the cell.

This  section presents  the  background and  thought  process in  systematically
arriving  at  the  choice of  the  methodology  that  was  adopted for  the  new
time-evolution model of the electrolyte concentration dynamics.

Based  upon the  experience  gained  in dealing  with  the literature  presented
in \cref{sec:electrolyteinclusion}, it is  the author's view that,  owing to the
complex behaviour of electrolyte, a  naive top-down approach \ie~including all
the physics upfront followed by a  systematic simplification, might only yield a
model that is  mathematically intractable for adoption in  an embedded \gls{bms}
environment.  The baseline  quadratic  approximation method  has  proven that  a
bottom-up approach \ie~pre-assuming a simplified structure for the final model
and adapting its coefficients to  physical constraints yields a viable candidate
for describing electrolyte dynamics and  for later inclusion in the conventional
\gls{spm}.

Upon  a   closer  examination   of  the  rubrics   of  the   baseline  quadratic
approximation  model, it  comes  to  light that  the  natural `pause'  discussed
towards  the  end  of \cref{subsec:scopenewelectrolyte}  permeates  to  a  level
more  than  merely  having  to  operate  sequentially  on  two  pseudo-decoupled
subsystems  --- it  goes to  the extent  of rendering  the operating  philosophy
of  fitting  physical  equations  semi-void.   To  clarify  this  statement  and
to  substantiate   the  claim,  while  there   is  no  doubt  that   the  linear
algebraic   equations  of \crefrange{eq:cecontinuitynegsep}{eq:Qepbyintegration}
do    incorporate    physical    principles   from    the    \gls{dfn}    model,
the     same     does     not      hold     true     for     the     \glspl{ode}
of \crefrange{eq:negliionmolesquadratic}{eq:posliionmolesquadratic}.   In  fact,
all   the   boundary   conditions   from   the   \gls{dfn}   model   have   been
exhausted   by  this   stage  (refer \cref{subsec:quadraticsimresultsanalysis}).
Although \crefrange{eq:negliionmolesgen}{eq:posliionmolesgen}     are    derived
from     the      \gls{dfn}     model,      the     coefficients      of     the
diffusivities   in    the   \gls{rhs}   of    the   next   set    of   equations
\ie~\crefrange{eq:negliionmolesquadratic}{eq:posliionmolesquadratic},   merely
involve  substitutions  of the  spatial  derivatives  of the  assumed  quadratic
expression.

Herein lies the weakness of the  baseline quadratic approach. Unlike the spatial
algebraic equations, which are tightly bound by the continuity and flux boundary
conditions at the  separator interfaces, there is no equality  constraint on the
spatial  derivative, which  is free  to grow  or shrink  without any  explicitly
imposed  bounds.  The  onus  of  being accurate  is  therefore  on  the  spatial
derivative evaluation which in-turn depends  on the correctness of the quadratic
functions~(\crefrange{eq:cenquadreduced}{eq:cepquadreduced})  themselves. It  is
not feasible to quantify the magnitude of error introduced in the time-evolution
of concentration  given a small-signal  perturbation in the coefficients  of the
quadratic spatial  computation \ie~the  implicit coupling between them  is not
transparent. Since the quadratic approximation  itself is not perfect \ie~does
not capture the  spatial gradient \emph{exactly} as the \gls{p2d}  model as seen
in \cref{fig:spatialionicconc1C}, the internal coupling of coefficients leads to
errors in time-evolution computation.

The author's  approach is to  therefore break this detrimental  coupling between
spatial  derivative of  concentration  and its  temporal evolution  counterpart.
Inspired by the fact that the  quadratic approximation model had almost achieved
the desired goals with
\begin{enumerate}[label=\emph{\alph*})]
    \item a bottoms-up approach \ie~assuming some model structure a~priori, and
    \item not bound by any physical considerations due to the exhaustion of governing equations
\end{enumerate}
led  the author  to  broach a  suitable modelling  concept  that exhibits  these
common  traits, yet  of a  completely different  nature and  hitherto unexplored
in  physics-based battery  modelling  in general  and  electrolyte modelling  in
particular --- \emph{black-box system identification}.

