 % -*- root: ../../main.tex -*-
% !TEX root = ../../main.tex
% vim:textwidth=80 fo=cqt

\graphicspath{{chapters/litt_review/figures/}}
% ----------------------- contents from here ------------------------

\clearpage
\chapter{Conclusions}\label{ch:conclusions}
\startcontents[chapters]
\printcontents[chapters]{}{1}{\setcounter{tocdepth}{1}}

\bigskip

\capolettera{I}{n} this  thesis, the  deployment of  physics-based computational
modelling  of lithium  ion  cells  for electric  vehicle  applications has  been
rigorously examined with a three-pronged strategy \viz{} through their analysis,
design and implementation. Salient conclusions  drawn from each of these aspects
are presented  in this  chapter. The \gls{p2d}  implementation of  the \gls{dfn}
model was used  as the backbone of  all modelling efforts in  this thesis. Based
upon the  invaluable experience gained during  the course of this  research, and
specifically from the  findings presented herein, key areas  of \glspl{pbm} that
can benefit  from further  study in  the future are  also identified.  These are
discussed in-line at the appropriate junctures embedded into the narrative.

\section{Using \glsfmtshortpl{pbm} as a Design Tool}

\Cref{ch:modelbaseddesign} presented  a computational framework to  optimise the
number  of electrochemical  layers to  be  used within  a  pouch cell  so as  to
maximise its usable energy while  meeting specific power demands. In particular,
this  helped to  construct  a model-based  deterministic  approach to  optimally
design cells  that can  be subjected  to fast-charging  without the  concerns of
plating. In  the context of  electric vehicle applications, using  this approach
has the  potential to alleviate  the two  immediate concerns of  consumers viz{}
range anxiety and long charging times.

From a  perspective of technical  advancements to the underlying  \gls{pbm}, the
standard form  of the \gls{p2d} model  has been suitably modified  to facilitate
a  direct  application of  input  power  densities.  This was  achieved  through
reformulation of the boundary conditions on the solid-phase potential \gls{pde}.
Although  this innate  power input  capability has  been claimed  as purportedly
developed,  its independent  derivation  and  accessible documentation  provided
herein shall help  other researchers to apply this in  a straight-forward manner
for vehicle drivecycles, acceleration tests and power-based charging protocols.

This investigation also  revealed that the fast charging  process determines the
optimal layer  configuration instead of  either acceleration runs  or drivecycle
requirements. This may  help to counter the current trend  in publications which
often rely  primarily upon a  drivecycle-based dynamic input current  profile to
evaluate various  aspects of cell  modelling such as predicting  degradation and
for advanced control and estimation  algorithms. While validation against such a
dynamic input profile  is certainly vital, for future  advancements dealing with
aspects  of  cell  design  or plating-related  degradation,  validation  against
fast-charging power profiles is absolutely essential.

To facilitate immediate adoption by relevant stakeholders, the concepts presents
in the aforementioned optimisation framework  has been realised in computer code
and presented  in the form  of an  open-source design toolbox,  \gls{bold}. This
toolbox  was  applied to  the  optimal  layer design  of  an  example cell  from
literature to  obtain two sets  of power-dependent optimal  layer configurations
for two drivetrains  --- a \gls{bev} and a \gls{phev}.  By suitably adapting the
numerical values  of parameters to  a real-world cell, model-based  cell designs
can be  obtained, which can potentially  help to eliminate the  current trend of
over-engineering of cells using conservative empirical designs.

However,  in the  interest of  simplicity  and to  lay the  foundation for  such
model-based cell designs, the scope of this work was intentionally narrowed down
to solely focus on changing the  layer configurations within cells. In doing so,
certain assumptions  were made that might  have to be revisited  and potentially
relaxed before its application to real-world cell designs.




By  reducing the  time  for final  prototyping
through replacement  of expensive  experimental iterations,  cost-effective cell
manufacturing paradigms can  be ushered, which can be passed  on to customers of
electric vehicles. This in-turn can  help to accelerate the mass-market adoption
of electrified transport.

This approach reduces the cost as well as time expenditure required by an
automotive OEM. This adds a new xEV to their product range, speeding up  for a
cleaner, safer and  decarbonised transport.



% through   . that can  help to unleash  the latent  potential of lithium ion batteries has been studied
