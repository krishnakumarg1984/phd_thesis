 % -*- root: ../../main.tex -*-
% !TEX root = ../../main.tex
% vim:textwidth=80 fo=cqt

\graphicspath{{chapters/litt_review/figures/}}
% ----------------------- contents from here ------------------------

\clearpage
\chapter{Conclusions}\label{ch:conclusions}
\startcontents[chapters]
\printcontents[chapters]{}{1}{\setcounter{tocdepth}{1}}

\bigskip

\capolettera{I}{n} this  thesis, the  deployment of  physics-based computational
modelling  of lithium  ion  cells  for electric  vehicle  applications has  been
rigorously examined with a three-pronged strategy \viz{} through their analysis,
design and implementation. Salient conclusions  drawn from each of these aspects
are presented  in this  chapter. The \gls{p2d}  implementation of  the \gls{dfn}
model was used  as the backbone of  all modelling efforts in  this thesis. Based
upon the  invaluable experience gained during  the course of this  research, and
specifically from the  findings presented herein, key areas  of \glspl{pbm} that
can benefit  from further  study in  the future are  also identified.  These are
discussed in-line at the appropriate junctures embedded into the narrative.

\section{Using \glsfmtshortpl{pbm} as a Design Tool}

\Cref{ch:modelbaseddesign} presented  a computational framework to  optimise the
number  of electrochemical  layers to  be  used within  a  pouch cell  so as  to
maximise its usable energy while  meeting specific power demands. In particular,
this chapter discussed a model-based  deterministic approach to optimally design
cells that can be subjected to fast-charging without the concerns of plating. In
the  context of  electric  vehicle  applications, using  this  approach has  the
potential  to alleviate  the two  immediate  concerns of  consumers viz{}  range
anxiety and long charging times.

To facilitate immediate adoption by relevant stakeholders, the concepts presents
in the aforementioned optimisation framework has been realised in computer code
and presented in the form of an open-source design

a design toolbox  \viz{}
\gls{bold} has been  eliminating  cell  over-engineering,  increasing  xEV  AER,  replacing  involved
experimental reiterations consuming a lot of time, reducing EV cost for both xEV
and cell manufacturers has been devised.

% through   . that can  help to unleash  the latent  potential of lithium ion batteries has been studied
