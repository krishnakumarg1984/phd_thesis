 % -*- root: ../../main.tex -*-
% !TEX root = ../../main.tex
% vim:textwidth=80 fo=cqt

\graphicspath{{chapters/litt_review/figures/}}
% ----------------------- contents from here ------------------------

\clearpage
\chapter{Conclusions}\label{ch:conclusions}
\startcontents[chapters]
\printcontents[chapters]{}{1}{\setcounter{tocdepth}{1}}

\bigskip

\capolettera{I}{n} this  thesis, the  deployment of  physics-based computational
modelling  of lithium  ion  cells  for electric  vehicle  applications has  been
rigorously examined with a three-pronged strategy \viz{} through their analysis,
design and implementation. Salient conclusions  drawn from each of these aspects
are  presented in  this chapter.  Based  upon the  invaluable experience  gained
during the course of this research, and specifically from the findings presented
herein, key  areas of  \glspl{pbm} that  can benefit from  further study  in the
future  are also  identified. These  are  discussed in-line  at the  appropriate
junctures embedded into the narrative.

\section{Using \glsfmtshortpl{pbm} as a Design Tool}




% through   . that can  help to unleash  the latent  potential of lithium ion batteries has been studied
