 % -*- root: ../../main.tex -*-
% !TEX root = ../../main.tex
% vim:textwidth=80 fo=cqt

% \newpage
% \thispagestyle{plain} % empty
% \mbox{}
\clearpage
\chapter{Conclusions}\label{ch:conclusions}
\ChapFrame
\renewcommand{\baselinestretch}{0.8}\normalsize
\vspace*{-10mm}
\startcontents[chapters]
\printcontents[chapters]{}{1}{\setcounter{tocdepth}{2}}
\setstretch{1.348361657291667} % golden-ratio stretch (1.2 x 1.348 = 1.618)

\vspace*{\fill}

\capolettera{I}{n} this  thesis, the  deployment of  physics-based computational
modelling  of lithium  ion  cells  for electric  vehicle  applications has  been
rigorously examined  with a three-pronged strategy  \viz~through their analysis,
design and implementation.  The \gls{p2d} implementation of  the \gls{dfn} model
was  used  as  the  backbone  of  all  modelling  efforts.  Salient  conclusions
drawn  from each  of these  aspects are  presented in  this chapter.  Based upon
the  invaluable  experience gained  during  the  course  of this  research,  and
particularly  from the  findings presented  herein, key  areas in  physics-based
modelling that can benefit from further study in the future are also identified.

% These are discussed in-line at the appropriate junctures embedded into the narrative.

\section{\glsfmtlongpl{pbm} as a Design Tool}

\subsection{Conclusions from the model-based design study}\label{sec:modelbasedconclusion}

\addlines[0.5]

\Cref{ch:modelbaseddesign} presented  a computational framework to  optimise the
number  of electrochemical  layers within  a pouch  cell so  as to  maximise its
usable energy while  meeting specific power demands. In  particular, this helped
to construct a model-based deterministic approach to optimally design cells that
can  be subjected  to  fast-charging without  the concerns  of  plating. In  the
context of electric vehicle applications,  using this approach has the potential
to alleviate  the two immediate concerns  of consumers \viz{} range  anxiety and
long charging times.

To  facilitate  immediate  adoption   by  relevant  stakeholders,  the  concepts
developed  in  the  optimisation  framework  of  \cref{ch:modelbaseddesign}  was
realised in  computer code and  presented in the  form of an  open-source design
toolbox, \gls{bold}. This toolbox was applied  to the optimal layer design of an
example cell from literature to obtain two sets of power-dependent optimal layer
configurations for two drivetrains --- a \gls{bev} and a \gls{phev}. By suitably
adapting the  numerical values of  parameters to a real-world  cell, model-based
cell  designs can  be  obtained, which  can potentially  help  to eliminate  the
current trend  of over-engineering of  cells using conservative  empirical layer
choices.

From a  perspective of technical  advancements to the underlying  \gls{pbm}, the
standard form  of the \gls{p2d} model  has been suitably modified  to facilitate
a  direct  application of  input  power  densities.  This was  achieved  through
reformulation of the boundary conditions on the solid-phase potential \gls{pde}.
Although  this innate  power input  capability has  been claimed  as purportedly
developed  in  extant  literature,  its independent  derivation  and  accessible
documentation  provided herein  shall help  other researchers  to apply  this in
a  straight-forward  manner  for  vehicle drivecycles,  acceleration  tests  and
power-based charging protocols.

This investigation also  revealed that the fast charging  process determines the
optimal layer  configuration instead of  either acceleration runs  or drivecycle
requirements. This may  help to counter the current trend  in publications which
often rely  primarily upon a  drivecycle-based dynamic input current  profile to
evaluate various  aspects of cell  modelling such as predicting  degradation and
for advanced control and estimation  algorithms. While validation against such a
dynamic input profile  is certainly vital, for future  advancements dealing with
aspects  of  cell  design  or plating-related  degradation,  validation  against
fast-charging power profiles is deemed to be absolutely essential.

The  study  also  provided  important  guidelines  about  the  role  of  thermal
environment on cell performance. It became  clear that at very low temperatures,
a high  number of layers was  required for satisfying a  specific charging power
level relative  to that  needed at  moderate thermal  environments such  as room
temperatures. A practical  takeaway from this conclusion is that  it suffices to
use a low number of layers in  vehicles to be sold in territories with perennial
moderate temperature conditions.

Finally, this design study revealed that the speed at which lithium intercalates
into the negative  electrode during charging limits the  charge-addition rate to
the cell. Lowering  the charging times of electric vehicles  requires the use of
higher charging  powers. However, this necessitates  a high number of  layers to
absorb the overpotentials  and to provide adequate number  of thermal conduction
pathways (owing  to the higher  number of  current collectors) to  dissipate the
additional heat  generated. Consequently, this  has a detrimental effect  on the
capacity and hence, the \gls{aer} of the vehicle.

\subsection{Future work informed by  the optimal layer design framework}

\addlines[0.5]
As     a     direct     inference     of     the     final     conclusion     in
\cref{sec:modelbasedconclusion}, assuming that  the electric grid infrastructure
is  adequately equipped  to cope  with  the surge  in future  power demands  for
charging  of  electric vehicles,  the  solid  state  diffusion at  the  negative
electrode becomes the bottleneck. It  is therefore important for future research
to  focus  on the  development  of  new  materials  for the  negative  electrode
possessing much higher  solid phase diffusion coefficients,  particularly at low
temperatures.

From a perspective of improving the framework itself, at the outset, it is clear
that the plating threshold assessment can be made more accurate by incorporating
the solid phase diffusion coefficient as  function of \gls{soc}. In the interest
of simplicity and  to lay the foundation for such  model-based cell designs, the
scope of this  work was intentionally narrowed down to  solely focus on changing
the layer  configurations within  cells. In doing  so, certain  assumptions were
made  that  might have  to  be  revisited  and  potentially relaxed  before  its
application to real-world cell designs.

For instance, adherence  to the specific type of cooling  phenomena used \ie~tab
cooling, is one of the stronger assumptions used in this work. In this work, the
benefits of  employing this  cooling mechanism  has been  enumerated, and  it is
desirable that future  pack designs adopt it. However, the  vast majority of the
current generation of battery packs  use surface-cooled designs. This means that
temperature for all layers within the pouch shall not be the same, which further
implies that it longer suffices to  simulate just a single layer. Therefore, the
framework  needs  to be  suitably  modified  to  handle multiple  layer  choices
concurrently.

Furthermore,  using surface  cooling shall  invalidate the  assumption of  small
thermal gradients along the planar axis of  the cell. This means that the lumped
thermal model  shall no longer  be accurate  to model the  temperature evolution
of  the cell.  Furthermore,  the differential  temperature  evolution along  the
cross-sectional direction  shall influence the transport  and kinetic properties
of the  cell. This electrochemical-thermal  coupling along the  planar direction
shall necessitate adding another spatial  dimension to the underlying \gls{pbm},
thereby rendering the  presently used \gls{p2d} model ineffective.  With the aim
of minimising  the optimising  run-times, future  research may  focus on  how to
suitably adapt the computational infrastructure that was proposed in this thesis
to account for these considerations.

For  real-world cell  designs, it  is prudent  to examine  the examine  the role
of  variable  porosities  to  achieve  the balancing  of  active  materials  per
layer.  Since  the computational  framework  developed  in  this thesis  uses  a
modular  approach, in  the  future  the constant  volume-fractions  used in  the
methodology  could simply  be replaced  by values  informed by  optimal porosity
computations. For this purpose, researchers could investigate the feasibility of
adapting  a suitable  scheme from  the available  pool of  literature that  have
focussed  on  using  model-based  porosity  optimisation~\cite{Xue2013,Xue2014,
Christensen2006}. Finally, experimental prototyping is  an important step in any
cell  design and  it  is  no exception  here  either. Therefore,  experimentally
applying the desired power levels to confirm the optimal layer choices predicted
by the  framework is an important  step to be undertaken  before its large-scale
deployment.


\section{Analysis of Salient Physics-based \glsfmtlongpl{rom}}

Chapters~\ref{ch:improveddra} and~\ref{ch:spmanalysis} of  this thesis primarily
focussed on performing an in-depth analysis of two distinct \glspl{rom} from two
distinct perspectives.  In \cref{ch:improveddra}, the hybrid  \gls{rom} obtained
by  using the  \gls{dra}  is analysed  to  investigate its  \emph{computational}
bottlenecks. In \cref{ch:spmanalysis}, an in-depth analysis of a niche selection
of  candidate models  from  the  family of  \glspl{spm}  is  presented from  the
perspective of \emph{modelling accuracy} \ie~their ability to faithfully compute
the  system-level quantities  of the  cell such  as its  \gls{soc} and  terminal
voltage. Finally, the early portions of \cref{ch:newelectrolytemodel} provides a
thorough analysis  of the bottlenecks  of the quadratic approximation  model for
computing the  spatial profile  of ionic concentration  in the  electrolyte. The
conclusions drawn  from all aforementioned  analyses are presented  next. Topics
that may  be of interest to  future researchers engaging with  these \glspl{rom}
are also proposed.

\subsection{Conclusions from analysis of the \glsfmtshort{dra}-based state-space \glsfmtshort{rom}}
\addlines[0.5]

In \cref{ch:improveddra}, \gls{svd} of a large Block-Hankel
matrix~(\approx\SI{27}{\giga\byte}) was identified as a computational bottleneck
in applying the classical \gls{dra} scheme for the hybrid \gls{rom} discussed
therein. It is concluded that this bottleneck arises due  to slow dynamics
of  solid phase diffusion  which leads  to the aforesaid large sized
Block-Hankel matrix.

% \addlines[1.5]
To mitigate this  bottleneck, an improved \gls{dra} scheme  was presented, whose
centrepiece is an iterative \gls{svd}  algorithm. This algorithm was obtained as
a  combination of  the Golyandina-Usevich  and Lanczos  algorithms discussed  in
\cref{ch:improveddra}.  The results  of applying  the improved  \gls{dra} scheme
demonstrate  a significant  performance improvement  achieved by  using the  new
method  without trading-off  model  fidelity.  At a  single  operating point  of
\gls{soc}  and temperature,  for  a  Hankel block  size  of~8000, the  \gls{rom}
workflow incorporating the improved  \gls{dra} is approximately 100~times faster
than that employing the classical  \gls{dra}. On a standard computer workstation
whose  specifications  are  given in  \cref{ch:improveddra},  for  100~operating
points  (combinations of  10~\gls{soc}  and temperature  values), obtaining  the
\gls{rom}  required  only 6~hours  using  the  improved \gls{dra},  whereas  the
classical \gls{dra}  consumed 666~hours~(\approx 27~days). Furthermore,  for the
same block-size, the improved \gls{dra} is  demonstrated to be superior in terms
of memory efficiency,  drastically reducing the memory  requirements from 112~GB
down to  2~GB. Finally, the  improved \gls{dra} demonstrates  improved modelling
accuracy even  in moderately  equipped computing  environments such  as standard
consumer-grade laptops.

\subsection{Future outlook for the \glsfmtshort{dra}-based hybrid state-space \glsfmtshort{rom}}

The improved  \gls{dra} method  proposed in  \cref{ch:improveddra} opens  up the
possibility of  applying this hybrid \gls{rom}  for physical quantities in other
locations within the  cell's geometry \eg~in the  middle of the electrode
region,  without being hindered by  computational limitations that would  have
otherwise  rendered  it intractable.  Furthermore, high  sample-rate \glspl{rom}
capable of handling highly dynamic  load profiles can be deployed in future
\gls{bms} applications.  The proposed scheme also  empowers the \gls{rom}
framework to tackle other cell chemistries with slower diffusion coefficients or
even  those with  completely different  rate-limiting mechanisms,  and therefore
prima~facie, appears to be promising.

% % The improved \gls{dra} method therefore
% opens up a wide range of possibilities % and prima~facie appears promising.

Despite   the  benefits facilitated  by   the streamlining  of  the reduced
order  modelling  workflow through  the  improved \gls{dra},  there  exists a
fundamental  deficiency  in this  hybrid  modelling approach  that impede  its
near-term  adoption in  state estimation  tasks. This aspect was already
discussed in \cref{ch:littreview} and is reiterated here. The entries  in  the
matrices  of  the  final  state  space  model  depend  on  the linearisation
point of \gls{soc} and temperature. This linearisation requirement adversely
affects the usability of the model for state estimation tasks, wherein the
\gls{soc} is in fact an unknown quantity and is to be estimated. This cyclic
dependency  between  the linearisation  point  and  the state-estimation  entity
renders this  model questionable for use  in a demanding application  such as in
embedded \glspl{bms} on-board electric vehicles. Therefore, the immediate future
step is to analyse  the stability of this internal feedback  loop. Once this has
been performed,  researchers may consider to  engage in the process  of adapting
the derivation of this hybrid  state-space \gls{rom} approach to higher C-rates,
in conjunction  with the numerical  benefits afforded by the  improved \gls{dra}
developed here.

\subsection{Conclusions from analysis of the \glsfmtshort{spm} family of models}

Based    upon   the    holistic   evaluation    of   various    \glspl{rom}   in
\cref{ch:littreview},  owing   to  its  simplicity,  the   \gls{spm}  family  of
models  was  identified  as  the  most promising  modelling  candidate  from  an
\emph{implementation}  perspective. \Cref{ch:spmanalysis}  provided an  in-depth
analysis of the strengths and drawbacks of this modelling family.

Results from  simulation of  the basic \gls{spm}  revealed that  two contrasting
aspects. Firstly, the  \gls{soc} computation of the \gls{spm}  was of acceptable
accuracy even at moderate C-rates thereby validating the fourth order polynomial
approximation approach  for capturing  solid phase diffusion  dynamics. However,
the basic  \gls{spm} suffers  from severe inaccuracies  in computing  the cell's
terminal voltage  at currents above  0.5C. This dichotomous  behaviour, revealed
through  this  analysis, has  not  been  explicitly  commented upon  in  present
literature.  In  particular,  despite  its overarching  simplicity,  it  can  be
concluded that it is this contrasting  aspect that renders the model unusable as
the  plant model  for  state  estimation tasks.  This  is  because, the  voltage
measured  from the  model  in a  feedback  estimator shall  map  to a  radically
distant \gls{soc} operating point requiring  excessively strong gain factors for
correcting it adversely affecting the stability of estimators.

% Although the  mitigation of mismatch  between the measured  quantity (terminal
% voltage) and  the estimated quantity (\gls{soc})  has not been studied  from a
% stability perspective, perhaps owing to its futility,

Several  research efforts  from literature  that  have attempted  to tackle  the
voltage inaccuracies  of the  basic \gls{spm}  through inclusion  of electrolyte
dynamics were also presented in  \cref{ch:spmanalysis}. A critical evaluation of
each  of  the  salient  efforts  revealed that  a  suitable  approach  that  can
successfully handle all possible operational scenarios  is yet to be devised. In
particular, the  existing modelling approaches for  electrolyte inclusion either
made  far-fetched assumptions  that  were easily  violated  or presented  nearly
intractable  mathematical  expressions  that  pushed  the  \gls{spm}  closer  in
computational complexity  to the  \gls{p2d} dynamics.  The author's analysis  of
pertinent literature  also  revealed  the  possibility of  re-using  certain
portions  of existing work \eg~the electrolyte  overpotential expression, whilst
also arguing the need  for a fresh  approach to model  other aspects of
electrolyte dynamics such as the ionic concentration profile within the cell's
thickness.

The  final  section  of  \cref{ch:spmanalysis}  presented  the  assumptions  and
governing equations of the quadratic approximation model for the spatial profile
of  ionic concentration. The  results obtained by simulating  this model reveal
that while its spatial  profile computation for a galvanostatic operation is of 
acceptable accuracy, its temporal  performance is mediocre. Particularly, the
time evolution of ionic concentration at the current collector interfaces
computed by the model fail to capture the intricacies in the temporal evolution
exhibited by  the \gls{p2d} model's profile.  Furthermore, the spatial profile
uses  the \gls{qss} approximation  which is violated during  the initial
transient. Hence, the  most impactful conclusion from the  foregoing analysis is
that the  basic quadratic approximation  model is unsuitable for  dynamic loads,
thereby ruling it out for deployment in vehicular applications.

The initial portions of \cref{ch:newelectrolytemodel} analysed the causal factor
contributing to the mediocre performance of the quadratic approximation model.
For this analysis, a genetic programming approach was employed to synthesise
suitable alternative reduced order equations for modelling the spatial profile
of the electrolyte concentration. The foregoing analysis helped to reveal that
given the constraints and interior-point boundary conditions of the underlying
\gls{dfn} model at the separator interfaces, the number of coefficients used in
the quadratic spatial profile does indeed make the best use of all available
equations. The genetic programming approach being able to synthesise suitable
mathematical structures that can potentially capture the spatial profile
accurately during both transient and steady state conditions. However, there is
a shortfall in the number of equations in order to uniquely solve for these
extra coefficients. The final conclusion that stems from this fact is that, by decoupling
the spatial and temporal profiles of the electrolyte concentration dynamics, the
performance of the present state of the art can indeed be improved upon.


\subsection{Proposed analysis routes for the \glsfmtshort{spm} modelling family}

Albeit extensive, the analyses of \glspl{spm}  performed in this thesis is by no
means complete. It is more appropriate to  say that this author chose to analyse
\glspl{spm} from  their implementation  perspective, particularly  their voltage
and \gls{soc}  accuracies in  an open-loop  implementation. However,  a holistic
evaluation covering  other aspects  are necessary before  their deployment  in a
vehicular \gls{bms} can be considered. For instance, the observability of models
must be  proven before  they can  be employed in  the feedback  path of  a state
estimator. Even  within the  broad stream of  observability analysis,  there are
several details that  need to be meticulously handled of  which some avenues for
future exploration are identified.

In the  simplification of the  state-vector of  the basic \gls{spm}  proposed by
Di~Domenico~\cite{DiDomenico2010}, the bulk concentration  in the two electrodes
was assumed to be equal  (see \cref{subsec:basicspmstatespace}). This allowed to
eliminate the bulk concentrations of one of the two electrodes, thereby reducing
the state  vector to~$\mathbb{R}^{3  \times 1}$. This  assumption hinges  on the
assumption  that  there is  no  loss  of  cycleable  lithium. However,  such  an
assumption  is valid  only for  a brand  new cell.  As the  cell ages,  owing to
various  phenomena such  as parasitic  side reactions  and decomposition  of the
solid conductive matrix,  this assumption shall no longer  hold true. Therefore,
analysis  of models  obtained by  augmenting the  basic or  electrolyte-enhanced
\glspl{spm}  with equations  describing degradation  shall become  critical. Not
only does this force to retain both bulk concentrations in the state vector, the
states from the  degradation rate-equations appear in it,  thereby enlarging it.
In  future  studies, the  effect  of  these  augmented  state variables  on  the
observability of the system needs to be quantified.

In this thesis,  only an \emph{isothermal} analysis of  various \glspl{spm} have
been performed. While this is an important initial step to analyse the strengths
and weaknesses of  the state of the art incumbent  \gls{spm} models by isolating
the  purely  electrochemical  aspects,  the  effect  of  temperature  cannot  be
underplayed. It  is well-known that the  thermal conditions of the  cell affects
its  electrochemical  and  vice-versa.  Therefore, future  works  shall  benefit
from  performing  comprehensive   analyses  of  coupled  electrochemical-thermal
\glspl{spm}. Considering the simplicity of \glspl{spm}, the lumped thermal model
might be a suitable candidate for it be paired with. Finally, in addition to its
influence  on the  accuracy of  open-loop  terminal voltage  and \gls{soc},  the
effect of  temperature on the  observability of the  holistic model needs  to be
evaluated. Since the cell temperature is available as an additional measurement,
it is likely to improve the model's observability. However, there is a trade-off
since this comes at  the cost of increased size of the state  vector. It is also
worthwhile  to explore  the  inflection point  of this  trade-off  to limit  the
model's fidelity at  a sufficient level whilst reaping the  maximum gains out of
the additional temperature measurement.

Based upon the conclusion that the performance of any reduced order spatial
profile model is hamstrung by the equation deficiencies in the underlying
\gls{dfn} model, it is worthwhile to reproduce the \gls{mggp} approach with a
different \gls{pbm}. For this future task, it is imperative to choose a model
that provides additional interior-point conditions \ie{} not just at the
separator, but perhaps a second order condition within an interior location at
each electrode. Once such a suitable \gls{pbm} has been identified, this thesis
author is confident that the genetic programming approach shall indeed
synthesise an accurate description of the electrolyte's spatio-temporal
dynamics.

\section{Implementation Aspects of Basic and Composite \glsfmtshort{spm}}
% \addlines[-2]

A driving impetus behind this thesis is to deliver \glspl{rom} amenable  to
embedded \emph{implementation}. Large portions of \cref{ch:spmanalysis} and
nearly the entirety of \cref{ch:newelectrolytemodel} were devoted to this study.
The conclusions that emerged from these implementation efforts are summarised
here. Future areas of research are also identified in-line.

In \cref{ch:spmanalysis}, the aspects of discrete-time implementation in a
microcontroller were discussed. In particular, the computation of the
discrete-time system and input matrices through  the  matrix exponential
approach is presented. A straightforward conclusion from the discrete-time
implementation is that they run orders of magnitude faster than their
continuous-time counterparts. This is owing to the fact that adaptive
time-stepping integrators are not required and can be replaced by a simple
matrix-vector multiplication scheme. This implementation detail provides
necessary starting point for battery researchers to start coding the \glspl{rom}
discussed herein. Future work in this direction shall benefit from including
various real-world issues such as accounting for sample delays as well as
implementing a progressive update scheme for the matrix entries to account for
degradation effects. Important aspects of numerical implementation arise during
the design of state estimators. Having a robust open-loop implementation of a
discrete-time plant model shall facilitate easy implementation of such feedback
algorithms.

\Cref{ch:newelectrolytemodel} discussed the implementation aspects of \gls{rom}
from a different perspective --- that of formulating a new electrolyte enhanced
composite \gls{spm} for embedded implementation. The temporal evolution of
electrolyte dynamics in the two electrode regions were characterised through
system identification techniques that have never been tried before in
literature. The identified model structures are presented in the final form as
discrete-time difference equations that can be readily incorporated into the
discrete-time equations developed in \cref{ch:spmanalysis} for the basic
\gls{spm}. The salient conclusion from this modelling effort is that, system
identification is indeed a viable candidate for the derivation of \glspl{rom} of
lithium ion batteries. The improved accuracy of electrolyte concentration
dynamics relative to the state of the art quadratic approximation model
corroborates the success of this strategy. Nevertheless, before widespread
adoption of this technique can be advocated, certain aspects of this approach
must be examined carefully. For instance, the assumptions of linearity and
time-invariance must be revisited for higher magnitudes of charge and discharge
currents. Thermal effects must also be incorporated into the overall system
identification approach. Machine learning and system identification are two
complementary approaches to the same task \ie~identification of internal
dynamics from terminal measurements. Future research could also explore this
possibility and perhaps a combined system identification/machine learning
strategy might metamorphose into the approach that powers future generation of
surrogate \glspl{rom} for electric vehicle applications.

\section{Closing Remarks}

The feasibility of bringing \glspl{pbm} of lithium batteries into electrified
transportation has been studied in this thesis from a multi-dimensional
perspective. Yet, there exists a common underpinning that connects these
seemingly different routes to the enhancement of our extant knowledge --- that
of parametrising the underlying \gls{pbm}. The number of parameters in any
\gls{pbm} is excessively large to the point of detracting potential adopters.
Even without considering degradation effects, obtaining the physical, geometric,
transport and kinetic parameters of a cell shall require access to specialised
laboratory equipment that are sometimes exclusive to large scale research
organisations. The studies discussed here shall assume real-world significance
only upon a successful demonstration of a cost-effective parametrisation
strategy. The author's contributions to an optimal `Design of Experiments~(DoE)'
numerical procedure for identifying the parameters of an \gls{spm} is detailed
in~\cite{Pozzi2018}. However, the broader task of parametrising a full-order
\gls{pbm} is well beyond the scope of this thesis or indeed any single
researcher, and needs the concerted effort of the community as a whole to tackle
it.

The next task in the journey to real-world implementation of the advancements
brought about by this thesis is to  experimentally validate them. However, this
is contingent upon obtaining a successful parametrisation strategy, and should
be embarked upon as soon as a robust procedure to identify the physical
parameters is made available. Nevertheless, this author hopes that through the
three-pronged approach to cell modelling presented in this thesis, the goal of
bringing physics-based battery models to electrified transportation has been
brought a step closer to realisation.

