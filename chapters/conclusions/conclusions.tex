 % -*- root: ../../main.tex -*-
% !TEX root = ../../main.tex
% vim:textwidth=80 fo=cqt

\graphicspath{{chapters/litt_review/figures/}}
% ----------------------- contents from here ------------------------

\clearpage
\chapter{Conclusions}\label{ch:conclusions}
\startcontents[chapters]
\printcontents[chapters]{}{1}{\setcounter{tocdepth}{2}}

\bigskip

\capolettera{I}{n} this  thesis, the  deployment of  physics-based computational
modelling  of lithium  ion  cells  for electric  vehicle  applications has  been
rigorously examined with a three-pronged strategy \viz{} through their analysis,
design and implementation. Salient conclusions  drawn from each of these aspects
are presented  in this  chapter. The \gls{p2d}  implementation of  the \gls{dfn}
model was used  as the backbone of  all modelling efforts in  this thesis. Based
upon the  invaluable experience gained during  the course of this  research, and
specifically from the  findings presented herein, key areas  of \glspl{pbm} that
can benefit from further study in the future are also identified.

% These are discussed in-line at the appropriate junctures embedded into the narrative.

\section{\glsfmtlongpl{pbm} as a Design Tool}

\subsection{Conclusions from model-based design study}\label{sec:modelbasedconclusion}

\Cref{ch:modelbaseddesign} presented  a computational framework to  optimise the
number  of electrochemical  layers to  be  used within  a  pouch cell  so as  to
maximise its usable energy while  meeting specific power demands. In particular,
this  helped to  construct  a model-based  deterministic  approach to  optimally
design cells  that can  be subjected  to fast-charging  without the  concerns of
plating. In  the context of  electric vehicle applications, using  this approach
has the  potential to alleviate  the two  immediate concerns of  consumers viz{}
range anxiety and long charging times.

To facilitate immediate adoption by relevant stakeholders, the concepts presents
in the aforementioned optimisation framework  has been realised in computer code
and presented  in the form  of an  open-source design toolbox,  \gls{bold}. This
toolbox  was  applied to  the  optimal  layer design  of  an  example cell  from
literature to  obtain two sets  of power-dependent optimal  layer configurations
for two drivetrains  --- a \gls{bev} and a \gls{phev}.  By suitably adapting the
numerical values  of parameters to  a real-world cell, model-based  cell designs
can be  obtained, which can potentially  help to eliminate the  current trend of
over-engineering of cells using conservative empirical designs.

From a  perspective of technical  advancements to the underlying  \gls{pbm}, the
standard form  of the \gls{p2d} model  has been suitably modified  to facilitate
a  direct  application of  input  power  densities.  This was  achieved  through
reformulation of the boundary conditions on the solid-phase potential \gls{pde}.
Although  this innate  power input  capability has  been claimed  as purportedly
developed,  its independent  derivation  and  accessible documentation  provided
herein shall help  other researchers to apply this in  a straight-forward manner
for vehicle drivecycles, acceleration tests and power-based charging protocols.

This investigation also  revealed that the fast charging  process determines the
optimal layer  configuration instead of  either acceleration runs  or drivecycle
requirements. This may  help to counter the current trend  in publications which
often rely  primarily upon a  drivecycle-based dynamic input current  profile to
evaluate various  aspects of cell  modelling such as predicting  degradation and
for advanced control and estimation  algorithms. While validation against such a
dynamic input profile  is certainly vital, for future  advancements dealing with
aspects  of  cell  design  or plating-related  degradation,  validation  against
fast-charging power profiles is absolutely essential.

The  study also  provided important  guidelines about  the role  of the  thermal
environment on cell performance. It became  clear that at very low temperatures,
a high  number of layers was  required for satisfying a  specific charging power
level relative  to that  needed at  moderate thermal  environments such  as room
temperatures. A practical  takeaway from this conclusion is that  it suffices to
use a low number of layers in  vehicles to be sold in territories with perennial
moderate  temperature  conditions, which  imply  a  higher \gls{aer}  for  these
vehicles.

Finally, this design study revealed that the speed at which lithium intercalates
into the negative  electrode during charging limits the  charge-addition rate to
the cell and hence to the pack. Lowering the charging times of electric vehicles
necessitates the  use of  higher charging powers.  However, this  necessitates a
high  number of  layers to  absorb the  overpotentials and  to provide  adequate
number of  thermal conduction pathways  (owing to  the higher number  of current
collectors) to dissipate the additional heat generated. Consequently, this has a
detrimental effect on the capacity and hence, the \gls{aer} of the vehicle.

\subsection{Future work informed by  the optimal layer design framework}

As     a     direct     inference     of     the     final     conclusion     in
\cref{sec:modelbasedconclusion}, assuming that  the electric grid infrastructure
is  adequately equipped  to cope  with  the surge  in future  power demands  for
charging  of  electric vehicles,  the  solid  state  diffusion at  the  negative
electrode becomes the bottleneck. It  is therefore important for future research
to  focus  on the  development  of  new  materials  for the  negative  electrode
possessing much higher  solid phase diffusion coefficients,  particularly at low
temperatures.

From a perspective of improving the framework itself, at the outset, it is clear
that the plating threshold assessment can be made more accurate by incorporating
the solid phase diffusion coefficient as  function of \gls{soc}. In the interest
of simplicity and  to lay the foundation for such  model-based cell designs, the
scope of this  work was intentionally narrowed down to  solely focus on changing
the layer  configurations within  cells. In doing  so, certain  assumptions were
made  that  might have  to  be  revisited  and  potentially relaxed  before  its
application to real-world cell designs.

For instance, adherence to the specific type of cooling phenomena used \ie{} tab
cooling, is one of  the stronger assumptions used in this  work. The benefits of
this cooling  type has  been enumerated,  and it is  desirable that  future pack
designs adopt it. However, the vast majority of battery packs use surface-cooled
designs. This means  that temperature for all layers within  the pouch shall not
be the same,  which further implies that  it longer suffices to  simulate just a
single layer. Therefore,  the framework needs to be suitably  modified to handle
multiple layer choices concurrently.

Furthermore,  using surface  cooling shall  invalidate the  assumption of  small
thermal gradients along the planar axis of  the cell. This means that the lumped
thermal model  shall no longer  be accurate  to model the  temperature evolution
of  the cell.  Furthermore,  the differential  temperature  evolution along  the
cross-sectional direction  shall influence the transport  and kinetic properties
of the  cell. This electrochemical-thermal  coupling along the  planar direction
shall necessitate adding another spatial  dimension to the underlying \gls{pbm},
thereby rendering the  presently used \gls{p2d} model ineffective.  With the aim
of minimising  the optimising  run-times, future  research may  focus on  how to
adapt the proposed computational infrastructure from this thesis to handle this.

For  real-world cell  designs, it  is prudent  to examine  the examine  the role
of  variable  porosities  to  achieve  the balancing  of  active  materials  per
layer.  Since  the computational  framework  developed  in  this thesis  uses  a
modular  approach, in  the future,  the  constant volume-fractions  used in  the
methodology  could simply  be replaced  by values  informed by  optimal porosity
computations. For this purpose, researchers could investigate the feasibility of
adapting  a suitable  scheme from  the available  pool of  literature that  have
focussed  on  using  model-based  porosity  optimisation~\cite{Xue2013,Xue2014a,
Christensen2006}. Finally, experimental prototyping is  an important step in any
cell  design and  it  is  no exception  here  either. Therefore,  experimentally
applying the desired power levels to confirm the optimal layer choices predicted
by the  framework is an important  step to be undertaken  before its large-scale
deployment.





\section{Analysis of Salient Physics-based \glsfmtlongpl{rom}}

\Cref{ch:improveddra}  and   \cref{ch:spmanalysis}  of  this   thesis  primarily
focussed on performing an in-depth analysis of two distinct \glspl{pbm} from two
distinct perspectives.  In \cref{ch:improveddra}, the hybrid  \gls{rom} obtained
by  using the  \gls{dra}  is analysed  to  investigate its  \emph{computational}
bottlenecks.  In \cref{ch:spmanalysis},  an  in-depth analysis  of the  simplest
time-domain physics-based \gls{rom} --- the  \gls{spm} --- is presented from the
perspective of \emph{modelling accuracy} \ie{} its ability to faithfully compute
the  system-level quantities  of the  cell such  as its  \gls{soc} and  terminal
voltage. The conclusions drawn from these analyses are presented here.

\subsection{Conclusions of analysing the \glsfmtshort{dra}-based state-space \glsfmtshort{rom}}

\section{Closing Remarks}

% closing remarks
% Parametrisation is a problem as well.
% solid-state diffusion is a problem informed by two chapters. Need advanced
% materials
% By  reducing the  time  for final  prototyping
% through replacement  of expensive  experimental iterations,  cost-effective cell
% manufacturing paradigms can  be ushered, which can be passed  on to customers of
% electric vehicles. This in-turn can  help to accelerate the mass-market adoption
% of electrified transport.

% This approach reduces the cost as well as time expenditure required by an
% automotive OEM. This adds a new xEV to their product range, speeding up  for a
% cleaner, safer and  decarbonised transport.


% % through   . that can  help to unleash  the latent  potential of lithium ion batteries has been studied
