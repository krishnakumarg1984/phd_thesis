% -*- root: ../../main.tex -*-
%!TEX root = ../../main.tex
% vim:textwidth=80 fo=cqt

% As discussed  in \cref{ch:intro},  there is  a growing  need to  replace replace
% fossil  fuel-based  vehicles  with   cleaner  alternatives  wherein  Lithium-ion
% batteries hold  much promise.  The next generation  of battery  powered vehicles
% shall  require  higher  energy  density  batteries  without  compromising  power
% density. Furthermore,  the costs of the  on-board battery pack shall  have to be
% considerably lowered.

% Researchers in the field have aimed to bring about improvements to the incumbent
% lithium ion battery technology by focussing on various aspects.

% Modifications are  attempted  through various  approaches i)  through
% fundamental material advances [  2 , 3 ], ii) new chemistries [  4 ], iii) novel
% cell designs and  manufacturing techniques [ 5 ], iv)  system design or reducing
% the costs of assembly [ 6 , 7  ], and v) improved controller design for advanced
% battery management systems [8].

\subsection{Introduction}

In sharp contrast to the cornucopia of published literature dealing with reduced
order  modelling  of cells  (see  \cref{sec:classificationscheme}),  there is  a
relative paucity of prior art that discuss model-based cell design. This section
aims to critically evaluate this sparse pool of knowledge.

At  the  outset, it  is  important  to clarify  this  thesis  author's views  on
model-based design in general and why this topic was chosen as an aspect of this
thesis.  In  the  recent  decades,  industries  pertaining  to  different  walks
of  life,  cutting  across  any  one  discipline,  have  begun  to  establish  a
model-led product/process development culture. For instance, a broad overview of
industrial  research  policy by  Thomke~\cite{Thomke1998}  towards  the turn  of
the  last millennium  provides  quantifiable evidence  that  a simulation  based
design approach  in the  automotive industry  has had a  positive impact  in the
crash-worthiness of resulting  vehicular designs. At a high  level, the benefits
of any model-based design are ---
\begin{enumerate*}[label=\itshape\alph*\upshape)]
    \item reducing the number of design iterations thereby speeding up the time to a production-ready prototype, and
    \item improved understanding of the variables influencing design gained by formalising empirical/ad-hoc knowledge.
\end{enumerate*}
Finally, it is a well-known fact that, computer-time is cheaper than human time.
Therefore, with  a simulation-oriented design  approach, it is often  cheaper to
explore Monte~Carlo-like design scenarios  using a computer model than iterating
over  a  series  of  rudimentary  prototypes  in  the  lab.  Finally,  once  the
results  predicted by  the model  has stabilised  enough to  satisfy the  design
specifications,  prototyping can  be  attempted, thereby  reducing  the time  to
market.

The   aforementioned    views   of    the   thesis    author   is    echoed   by
Becker~\etal~\cite{Becker2005}  who  present  a   persuasive  view  that,  since
it  becomes  essential  to  have   a  deeper  understanding  of  the  simulation
tools  to successfully  employ a  model-led  design, this  can trigger  sweeping
changes percolating  into the  very core  of the  problem-solving culture  in an
organisation. Although relatively  at its infancy when it comes  to cell design,
model-based  designs have  been applied  at the  pack-level in  the past  and is
therefore not a  newcomer to the battery  industry in general. In  the middle of
the last  decade, a clarion  call to the industry  to adopt simulation  tools in
battery  engineering was  issued  by Spotnitz~\cite{Spotnitz2005}.  In the  said
article,  the author  questioned  the anachronistic  industry  trend of  relying
heavily on `making and testing' rather than aiming to understand the fundamental
governing equations  and principles  of a  battery and  using this  know-how for
design. Spotnitz~\cite{Spotnitz2005}  further argues  that using  \glspl{pbm} of
batteries could provide  reliable understanding of their behaviour,  and that as
the  understanding of  the  community steadily  grows, it  could  bring about  a
significant speed-up of battery development.

The nature and  scope of `model-based design' as intended  by this thesis author
needs to  be clarified. Since  this thesis focuses exclusively  on physics-based
modelling, the term  `model' pertains to \glsfirst{pbm}s and  not \glspl{ecm} or
any  other type  of  battery  models (such  as  other empirical/ad-hoc  models).
Therefore, the  survey of literature  here does  not include any  design efforts
outside of this scope.

There is  evidence that  computational modelling has  been successfully  used to
facilitate the development of novel energy storage materials. The review article
by Islam and Fisher~\cite{Islam2014} provides an overview of the use of computer
simulation  into gaining  a deeper  insight  into the  working of  new types  of
cathode  materials. Meng  and Arroyo-de~Domp~\cite{Meng2009}  also surveyed  the
topic  of using  computational tools  for  ``design and  optimisation of  energy
storage materials. The computer models referred  to in these review articles are
not volume-averaged models  operating at the cell-level,  but detailed ab-initio
models  constructed using  techniques such  as Density  Functional Theory~(DFT).
While  it is  heartening  to  see such  comprehensive  studies of  computational
techniques applied to energy storage, there are two distinct reasons why the art
reviewed therein does not  align with the goals of this  thesis. Firstly, in the
works  reviewed in  the  aforementioned articles,  computer  simulation is  used
primarily  to facilitate  researchers' understanding  that can  help to  develop
the  next generation  of energy  storage components.  The computational  methods
presented  therein are  \emph{not}  employed as  design  tools. Secondly,  these
articles use computational modelling to  study structural properties at the meso
and nano scales. While these are of utmost importance to researchers involved in
synthesising prototypes of next generation of energy storage materials, they are
not applicable for production-ready cell-designs at scale. Since this thesis has
a  strong  focus on  providing  readily  applicable  solutions to  industry  for
incumbent lithium  ion chemistries, this thesis  author chose not to  pursue the
methods discussed in  the aforementioned works for the  design studies discussed
here.

A    holistic   computational    structure    screening    was   performed    by
Sendek~\etal~\cite{Sendek2017}  to  study  the  suitability  of  12831~candidate
materials for their suitability as  solid state electrolytes. The same rationale
presented  by  this thesis  author  \ie{}  rapid  prototyping  is cited  as  the
motivation behind  this study undertaken  by the authors  of the said  work. The
study  helped to  narrow down  the viable  family structures  to 21.  This study
serves  as  a  concrete  example  in this  literature  survey  wherein  computer
simulation is  directly used as  a design tool for  components of a  lithium ion
cell. The  scope of  this work falls  into the realm  of material  synthesis and
applied  physics ---  topics outside  the educational  background and  expertise
of  this  thesis  author.  Nevertheless,  the success  of  this  study  strongly
motivates  the case  for  performing  a computational  study  of similar  scale.
Instead of  screening out thousands  of candidate materials, the  work presented
in  \cref{ch:modelbaseddesign} computationally  screens out  thousands of  layer
configurations within a pouch cell and provides a simulation-based design tool.

Curiously,  the use  of volume-averaged  models  for design  simulations at  the
cell-scale  \glspl{pbm}  has not  yet  gained  sufficient traction  despite  the
prevalence of  the popular  \gls{p2d} implementation of  the \gls{dfn}  model in
academia  for applications  such as  degradation analysis  and state-estimation.
Ramadesigan~\etal~\cite{Ramadesigan2012} postulate that this could be attributed
to the  computational challenges  presented by  the complex  reaction, diffusion
and  kinetics of  lithium  ion  batteries occurring  over  different length  and
time  scales.  Nevertheless,  some  design  efforts  at  the  cell  scale  using
electrochemical  models have  been reported  in literature,  which are  examined
next.

\subsection{State of the art in cell-scale model-led design optimisations}

The pioneering  work by Newman~\cite{Newman1995}  was the  first of its  kind to
develop a scheme to optimise cell  design based on an electrochemical model. The
two parameters  optimised are electrode  thicknesses and porosities.  This study
makes the assumption that electrode kinetics  are fast relative to diffusion and
furthermore ignores local concentration  gradients. A reaction-zone model, which
considers  that reactions  occur in  a narrowly  confined area  was used  as the
underlying  model. The  separator  thickness was  assumed to  be  fixed and  the
specific energy  of the cell was  maximised. However, this work  does not impose
any constraints on the extent of specific power that can be drawn from the cell.
Nevertheless, this seminal effort provided the key guidance to other researchers
that  electrode thicknesses  and porosities  can be  considered as  the critical
design variables to be optimised for in a cell design.

Arora~\etal~\cite{Arora1999}  used an  electrochemical  model  to inform  design
decisions on cell parameters such as particle size, electrode thickness and mass
ratio. Furthermore, the model thus obtained was used to optimise the cell design
against the risk of Lithium plating. This thesis author considers this to be the
early vanguard in applying a successful model-based design for cells.

At the other  end of the spectrum, a  comprehensive multi-objective optimisation
for optimal design of batteries was  recently proposed by Changhong Liu and~ Lin
Liu~\cite{Liu2017b}.  This optimisation  involved  a comprehensive  multiphysics
model  of  the  cell where  a  number  of  design  variables such  as  electrode
thicknesses, porosities  and particle sizes  were considered. The  objectives of
optimization were  to maximize discharge  specific power and specific  energy as
well  as minimizing  capacity loss.  The  model was  solved by  using a  genetic
algorithm scheme. While this work is certainly  the first of its kind in using a
formal mathematical  optimisation framework for  cell design, in the  opinion of
this thesis author,  the complexity of the problem is  excessively high --- both
mathematically and  computationally. For  instance, with three  design variables
considered,  it  requires  a  careful interpretation  of  the  resulting  pareto
front  to  distinguish their  relative  importance.  While prima~facie  this  is
not  an  issue, the  question  of  resolving the  pareto  front  for cells  with
different parameter sets needs to  be addressed. The optimisation algorithm used
is  an  exotic genetic  algorithm  to  be run  on  a  parallel cluster.  From  a
near-term industrial application viewpoint, this scheme is less attractive since
real-world  constraints, such  as supply-chain  dictated raw-material  sourcing,
typically restrict the number of  degrees of freedom available for optimisation.
Nevertheless, this scheme is valuable  from a long-term research perspective and
merits attention by researchers interested in model-based design.

The  design study  by  Xue~\etal~\cite{Xue2013a} represents  a  rare example  of
a  model-led  design study  in  electrified  transportation.  In this  work,  an
electrochemical cell model  \viz{} the \gls{p2d} model  was successfully adapted
to  perform the  design optimisation  of a  \gls{phev}. Although  the underlying
model operates at the cell-scale, this study considers a pack-level optimisation
through  a  series-parallel  combination  of  cells.  Furthermore,  an  advanced
numerical  algorithm in  the  form  of a  hybrid  solver  \ie{} a  gradient-free
optimiser  in conjunction  with a  gradient-based optimiser  was used.  Albeit a
unique example in the context of vehicular application, this work falls slightly
outside the  scope of  this thesis wherein  the design is  confined at  the cell
level. The aforementioned work takes into account pack-level constraints such as
safety limits as well as energy and power levels. Furthermore, even the presence
of layers within each cell  were considered.


The  system-level  constraints   considered  in  Xue~\etal~\cite{Xue2013a}  were
instrumental   in  informing   the  choice   of  constraints   in  this   thesis
author's  design  study  presented in  \cref{ch:modelbaseddesign}.  However,  in
the  aforementioned article,  the  translation from  pack-level into  cell-level
quantities does not account for secondary  effects \ie{} the influence layers on
pack mass  (a quantity that is  optimised in the  said work) is not  studied. In
this  thesis  author's  simulations,  it  became clear  that  such  effects  are
significant enough that they cannot be neglected.

A critical issue in the approach by Xue~\etal~\cite{Xue2013a} is that the number
of design variables were excessively large which necessitated the use of complex
numerical algorithms.  The said work  assumes that the  cell design shall  be an
integral  part  of the  pack  optimisation.  However,  in this  thesis  author's
understanding,  it  appears  that  cells  are  designed  independently  by  cell
manufacturers, that merely  adhere to specifications~\cite{Maksimovic2012}. Pack
configuration decisions such as the  choice of number of series-parallel modules
in  the vehicular  pack are  undertaken at  a system  level in  conjunction with
the  rest  of  the  drivetrain specifications  by  vehicle  manufacturers.  This
natural  separation  of the  real-world  paradigm  helps  to decouple  the  cell
design  from  the  pack  design  thereby  drastically  reducing  the  number  of
degrees  of  freedom and  simplifying  the  optimisation  task.  This is  a  key
assumption in  this thesis author's design  optimisation study and is  listed in
\cref{subsec:layeroptassumptions}.



