% -*- root: ../../main.tex -*-
%!TEX root = ../../main.tex
% vim:textwidth=80 fo=cqt

% As discussed  in \cref{ch:intro},  there is  a growing  need to  replace replace
% fossil  fuel-based  vehicles  with   cleaner  alternatives  wherein  Lithium-ion
% batteries hold  much promise.  The next generation  of battery  powered vehicles
% shall  require  higher  energy  density  batteries  without  compromising  power
% density. Furthermore,  the costs of the  on-board battery pack shall  have to be
% considerably lowered.

% Researchers in the field have aimed to bring about improvements to the incumbent
% lithium ion battery technology by focussing on various aspects.

% Modifications are  attempted  through various  approaches i)  through
% fundamental material advances [  2 , 3 ], ii) new chemistries [  4 ], iii) novel
% cell designs and  manufacturing techniques [ 5 ], iv)  system design or reducing
% the costs of assembly [ 6 , 7  ], and v) improved controller design for advanced
% battery management systems [8].

In sharp contrast to the cornucopia of published literature dealing with reduced
order  modelling  of cells  (see  \cref{sec:classificationscheme}),  there is  a
relative paucity of prior art that discuss model-based cell design. This section
aims to critically evaluate this sparse pool of knowledge.

At  the  outset, it  is  important  to clarify  this  thesis  author's views  on
model-based design in general and why this topic was chosen as an aspect of this
thesis.  In  the  recent  decades,  industries  pertaining  to  different  walks
of  life,  cutting  across  any  one  discipline,  have  begun  to  establish  a
model-led product/process development culture. For instance, a broad overview of
industrial  research  policy by  Thomke~\cite{Thomke1998}  towards  the turn  of
the  last millennium  provides  quantifiable evidence  that  a simulation  based
design approach  in the  automotive industry  has had a  positive impact  in the
crash-worthiness of resulting  vehicular designs. At a high  level, the benefits
of any model-based design are ---
\begin{enumerate*}[label=\itshape\alph*\upshape)]
    \item reducing the number of design iterations thereby speeding up the time to a production-ready prototype, and
    \item improved understanding of the variables influencing design gained by formalising empirical/ad-hoc knowledge.
\end{enumerate*}
Finally, it is a well-known fact that, computer-time is cheaper than human time.
Therefore, with  a simulation-oriented design  approach, it is often  cheaper to
explore Monte~Carlo-like design scenarios  using a computer model than iterating
over  a  series  of  rudimentary  prototypes  in  the  lab.  Finally,  once  the
results  predicted by  the model  has stabilised  enough to  satisfy the  design
specifications,  prototyping can  be  attempted, thereby  reducing  the time  to
market.

The   aforementioned    views   of    the   thesis    author   is    echoed   by
Becker~\etal~\cite{Becker2005}  who  present  a   persuasive  view  that,  since
it  becomes  essential  to  have   a  deeper  understanding  of  the  simulation
tools  to successfully  employ a  model-led  design, this  can trigger  sweeping
changes percolating  into the  very core  of the  problem-solving culture  in an
organisation. Although relatively  at its infancy when it comes  to cell design,
model-based  designs have  been applied  at the  pack-level in  the past  and is
therefore not a  newcomer to the battery  industry in general. In  the middle of
the last  decade, a clarion  call to the industry  to adopt simulation  tools in
battery  engineering was  issued  by Spotnitz~\cite{Spotnitz2005}.  In the  said
article,  the author  questioned  the anachronistic  industry  trend of  relying
heavily on `making and testing' rather than aiming to understand the fundamental
governing equations  and principles  of a  battery and  using this  know-how for
design. Spotnitz~\cite{Spotnitz2005}  further argues  that using  \glspl{pbm} of
batteries could provide  reliable understanding of their behaviour,  and that as
the  understanding of  the  community steadily  grows, it  could  bring about  a
significant speed-up of battery development.

The nature and  scope of `model-based design' as intended  by this thesis author
needs to  be clarified. Since  this thesis focuses exclusively  on physics-based
modelling, the term  `model' pertains to \glsfirst{pbm}s and  not \glspl{ecm} or
any  other type  of  battery  models (such  as  other empirical/ad-hoc  models).
Therefore, the  survey of literature  here does  not include any  design efforts
outside of this scope.

There is  evidence that  computational modelling has  been successfully  used to
facilitate the development of novel energy storage materials. The review article
by Islam and Fisher~\cite{Islam2014} provides an overview of the use of computer
simulation  into  gaining  a  deeper  insight into  the  working  of  new  types
of  cathode  materials.  The  computer  models referred  to  in  this  work  are
not  volume-averaged models  at the  cell-level, but  detailed ab-initio  models
constructed using techniques  such as Density Functional  Theory~(DFT). While it
is heartening  to see  such a comprehensive  review of  computational techniques
applied to energy  storage, there are two distinct reasons  why the art reviewed
therein does  not align  with the goals  of this thesis.  Firstly, in  the works
reviewed in the  said article, computer simulation is used  primarily to aid the
understanding needed to develop next generation of energy storage components and
\emph{not} as a design tool. Secondly, this article uses computational modelling
to study  structural properties  at the  meso and nano  scales. While  these are
of  utmost importance  to  researchers involved  in  synthesising prototypes  of
next  generation  of energy  storage  materials,  they  are not  applicable  for
production-ready cell-designs at scale. Since this  thesis has a strong focus on
providing readily  applicable solutions  to industry  for incumbent  lithium ion
chemistries,  there is  no motivation  to pursue  the methods  discussed in  the
aforementioned work for the design aspects of this thesis.


