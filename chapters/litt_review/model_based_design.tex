% -*- root: ../../main.tex -*-
%!TEX root = ../../main.tex
% vim:textwidth=80 fo=cqt

% As discussed  in \cref{ch:intro},  there is  a growing  need to  replace replace
% fossil  fuel-based  vehicles  with   cleaner  alternatives  wherein  Lithium-ion
% batteries hold  much promise.  The next generation  of battery  powered vehicles
% shall  require  higher  energy  density  batteries  without  compromising  power
% density. Furthermore,  the costs of the  on-board battery pack shall  have to be
% considerably lowered.

% Researchers in the field have aimed to bring about improvements to the incumbent
% lithium ion battery technology by focussing on various aspects.

% Modifications are  attempted  through various  approaches i)  through
% fundamental material advances [  2 , 3 ], ii) new chemistries [  4 ], iii) novel
% cell designs and  manufacturing techniques [ 5 ], iv)  system design or reducing
% the costs of assembly [ 6 , 7  ], and v) improved controller design for advanced
% battery management systems [8].

\subsection{Introduction}

In  sharp  contrast to  the  cornucopia  of  published literature  dealing  with
reduced order modelling of cells (see \cref{sec:classificationscheme}), there is
currently  a relative  paucity  of  prior art  that  discusses model-based  cell
design. Albeit the relevant pool of  knowledge is presently sparse, this section
nevertheless aims to critically evaluate it.

At the outset, it is important to  clarify this thesis author's general views on
model-based design. It is  also important to state why this  topic was chosen as
an  aspect of  this  thesis. In  the recent  decades,  industries pertaining  to
different  walks of  life,  cutting across  any one  discipline,  have begun  to
establish a model-led product/process development culture. For instance, a broad
overview of  industrial research policy by  Thomke~\cite{Thomke1998} towards the
turn of  the last  millennium provides quantifiable  evidence that  a simulation
based design  approach in the automotive  industry has had a  positive impact in
the  crash-worthiness of  resulting  vehicular  designs. At  a  high level,  the
salient benefits of any model-based design include ---
\begin{enumerate*}[label=\itshape\alph*\upshape)]
    \item reducing the number of design iterations thereby speeding up the time to a production-ready prototype, and
    \item facilitating improved understanding of the variables influencing design that is gained by formalising empirical/ad-hoc knowledge through modelling.
\end{enumerate*}
Furthermore, it  is a well-known fact  that computer-time is cheaper  than human
time. Therefore, with a simulation-oriented design approach, it is often cheaper
to  explore  Monte~Carlo-like  design  scenarios using  a  computer  model  than
iterating over a series of rudimentary  prototypes in the lab. Finally, once the
results predicted  by the  model have  stabilised enough  to satisfy  the design
specifications, prototypes closely  matching the final design  objectives can be
realised, thereby reducing the overall lead time to the market.

The   aforementioned    views   of    the   thesis    author   is    echoed   by
Becker~\etal~\cite{Becker2005}  who present  a  persuasive view  that, since  it
shall be  mandatory to have  a deeper understanding  of the simulation  tools in
order  to successfully  employ a  model-led  design, this  can trigger  sweeping
changes percolating  into the  very core  of the  problem-solving culture  in an
organisation. Although relatively  at its infancy when it comes  to cell design,
model-based designs have been applied at  the battery pack-level in the past and
is therefore not a newcomer to the battery industry in general. In the middle of
the last  decade, a clarion  call to the industry  to adopt simulation  tools in
battery  engineering was  issued  by Spotnitz~\cite{Spotnitz2005}.  In the  said
article,  the author  questioned  the anachronistic  industry  trend of  relying
heavily on `making and testing' rather than aiming to understand the fundamental
governing equations  and principles  of a  battery and  using this  know-how for
design. Spotnitz~\cite{Spotnitz2005}  further argues  that using  \glspl{pbm} of
batteries could provide  reliable understanding of their behaviour,  and that as
the  understanding of  the  community steadily  grows, it  could  bring about  a
significant speed-up of battery development.

The nature and  scope of `model-based design' as intended  by this thesis author
needs to  be clarified. Since  this thesis focuses exclusively  on physics-based
modelling,  the  term  `model'  as  used  here  pertains  to  \glsfmtlongpl{pbm}
(\glsfmtshortpl{pbm}) and  not \glspl{ecm} or  any other type of  battery models
including other empirical/ad-hoc models such as surrogate models. Therefore, the
survey of  literature here  does not  include any  prior efforts  on model-based
design that lie outside of this scope.

Among  the published  set of  literature, there  is evidence  that computational
modelling  has  been   successfully  used  to  facilitate   the  development  of
novel   energy   storage   materials.   The  review   article   by   Islam   and
Fisher~\cite{Islam2014} provides an  overview of the use  of computer simulation
into  gaining  a  deeper insight  into  the  working  of  new types  of  cathode
materials. Meng  and Arroyo-de~Domp~\cite{Meng2009}  also surveyed the  topic of
using  computational tools  for the  design and  optimisation of  energy storage
materials. The  computer models  referred to  in these  review articles  are not
volume-averaged  models  operating at  the  cell-level,  but detailed  ab-initio
models  constructed using  techniques such  as Density  Functional Theory~(DFT).
While  it is  heartening  to  see such  comprehensive  studies of  computational
techniques being applied  to energy storage, there are two  distinct reasons why
the body of  research reviewed in the aforementioned articles  do not align with
the goals of  this thesis. Firstly, in the works  reviewed in the aforementioned
articles, computer simulation is used  primarily to enhance researchers' current
understanding of these materials that can help to develop the next generation of
energy  storage  components. The  computational  methods  presented therein  are
\emph{not}  directly employed  as  design tools.  Secondly,  these articles  use
computational  modelling to  study structural  properties at  the meso  and nano
scales.  While  these  are  of  utmost importance  to  researchers  involved  in
synthesising prototypes of next generation of energy storage materials, they are
less relevant for production-ready cell-designs  at scale. Since this thesis has
a  strong  focus on  providing  readily  applicable  solutions to  industry  for
incumbent lithium  ion chemistries,  it was  decided not  to pursue  the methods
discussed in the aforementioned works for the design studies discussed herein.

A      holistic      computational      screening     was      performed      by
Sendek~\etal~\cite{Sendek2017}  to  study  the  suitability  of  12831~candidate
materials for their  suitability as solid state  electrolytes in electrochemical
cells. These authors  cite the same rationale of this  thesis author \ie~rapid
prototyping, as the  motivation behind this model-based  design simulations. The
study  helped to  narrow the  initial candidate  pool down  to 21~viable  family
structures. This effort  serves as a concrete example in  this literature survey
wherein computer simulation is directly used  as a design tool for components of
a lithium  ion cell. The  scope of  this work falls  into the realm  of material
synthesis and applied physics ---  topics outside the educational background and
expertise  of this  thesis  author.  Nevertheless, the  success  of this  effort
strongly  motivates the  case for  performing a  computational study  of similar
scale, wherein  instead of screening  out thousands of candidate  materials, the
framework  presented in  \cref{ch:modelbaseddesign} computationally  screens out
thousands of layer configurations within a pouch cell.

Curiously,  the  use  of  volume-averaged   models  for  design  simulations  of
\glspl{pbm} at  the cell-scale has not  yet gained sufficient traction.  This is
despite the prevalence of the  popular \gls{p2d} implementation of the \gls{dfn}
model that  in applications such  as degradation analysis  and state-estimation.
Ramadesigan~\etal~\cite{Ramadesigan2012} postulate  that this slow  uptake could
be attributed to the computational challenges presented by the complex reaction,
diffusion and kinetics of lithium ion  cells occurring over different length and
time scales. A few design efforts using electrochemical models at the cell scale
have nevertheless been reported in literature, which are examined next.

\subsection{State of the art in cell-scale model-led design optimisations}

The pioneering  work by Newman~\cite{Newman1995}  was the  first of its  kind to
develop a  scheme to  optimise cell  design based  on an  electrochemical model.
The  two  parameters optimised  in  this  work  were electrode  thicknesses  and
porosities. This  study makes  the assumption that  electrode kinetics  are fast
relative to diffusion, and furthermore  ignores local concentration gradients. A
reaction-zone model, which considers that reactions occur in a narrowly confined
area, was used  as the underlying model. The separator  thickness was assumed to
be fixed and the  specific energy of the cell was  maximised. However, this work
does not  impose any  constraints on the  extent of specific  power that  can be
drawn from the cell. Nevertheless, this seminal effort provided the key guidance
to other researchers that electrode thicknesses and porosities can be considered
as the critical design variables to be  optimised for in a cell design. Building
upon the  foundation laid  by the  aforementioned study,  a few  other model-led
design efforts  have been  published. However,  in order  to provide  a focussed
review,  only a  subset  of the  prior  art  that is  deemed  pertinent for  the
model-based design  of \emph{pouch} cells for  \emph{automotive} applications is
considered here.

Arora~\etal~\cite{Arora1999}  used an  electrochemical  model  to inform  design
decisions on cell parameters such as particle size, electrode thickness and mass
ratio. Furthermore, the model thus obtained was used to optimise the cell design
against the  risk of Lithium  plating. Cells  are susceptible to  plating during
fast charging and hence a key aspect to consider in the cell design for electric
vehicles. This thesis author considers this work to be a vanguard in model-based
design  for cells  that is  potentially applicable  for vehicular  applications.
Nevertheless, the  aspect of  stacking layers  inside pouch  cells and  how this
influences the optimisation of the  aforementioned design variables has not been
studied.

Xue~\etal~\cite{Xue2013}  presented a  simulation based  design study  wherein a
cell design was  optimised using a gradient-based  algorithm. Specifically, this
work dealt with a numerical framework for providing automated design outputs for
maximising  the cell's  energy  density whilst  meeting  specific power  density
requirements. This  criterion appears to  be highly  relevant in the  context of
cell design for  electric vehicles and hence,  was adopted as the  basis for the
layer  optimisation framework  presented in  \cref{ch:modelbaseddesign}. In  the
aforementioned work, various power levels  were also tried. However, these power
requirements were computed indirectly rather  than a direct reformulation of the
underlying \gls{p2d} model to accept power density inputs.

% reformulated to account for applied power density. Particularly notable is the
% absence of relevant boundary conditions to the governing \glspl{pde}.


The  design study  by Xue~\etal~\cite{Xue2014}  represents a  rare example  of a
model-led design study performed in  the backdrop of electrified transportation.
In  this work,  an electrochemical  cell model  \viz~the  \gls{p2d} model  was
successfully  adapted  to  perform  the design  optimisation  of  a  \gls{phev}.
Although the underlying  model operates at the cell-scale,  this study considers
a  pack-level  optimisation  through  a series-parallel  combination  of  cells.
Furthermore, an advanced numerical algorithm in  the form of a hybrid solver was
used  which used  a  unique scheme  of employing  a  gradient-free optimiser  in
conjunction with  a gradient-based optimiser.  Albeit a standout example  in the
context of vehicular application, this work  falls slightly outside the scope of
this thesis  wherein the  design study is  strictly confined to  be at  the cell
level. The aforementioned work takes into account pack-level constraints such as
safety limits  as well as  energy and  power levels. Furthermore,  these authors
even accounted for important details such  as the presence of layers within each
cell in their design optimisation process.

The  system-level   constraints  considered  in   Xue~\etal~\cite{Xue2014}  were
instrumental  in  informing  the  choice  of constraints  in  the  design  study
presented in \cref{ch:modelbaseddesign}. However, in the aforementioned article,
the translation from pack-level into  cell-level quantities does not account for
secondary effects \ie~the influence of layers on pack mass (a quantity that is
optimised in the said work) is not studied. Simulations conducted as part of the
work  reported  in \cref{ch:modelbaseddesign}  revealed  that  such effects  are
non-negligible.

% Furthermore, although  not
% explicitly mentioned,  it is  assumed that, despite  layer counts  being changed
% during  optimisation,  the  electrode  thicknesses are  held  constant,  thereby
% violating  key principles  such as  active  material capacity  balancing of  the
% electrodes.

A critical issue in the approach  by Xue~\etal~\cite{Xue2014} is that the number
of design variables were excessively large which necessitated the use of complex
numerical  algorithms.  The  said  work  assumes  that  the  cell  design  shall
be  an  integral  part  of  the  pack  optimisation.  However,  in  this  thesis
author's  understanding,  it  appears  that in  real-world  designs,  cells  are
sourced from a  specialised manufacturer. Furthermore, these  cells are designed
independently to merely  adhere to certain specifications~\cite{Maksimovic2012}.
Pack configuration  decisions such  as the choice  of number  of series-parallel
modules in  the vehicular pack are  undertaken at a system  level in conjunction
with the  rest of the  drivetrain specifications by vehicle  manufacturers. This
natural separation of the real-world paradigm  helps to decouple the cell design
from  the pack  design thereby  drastically reducing  the number  of degrees  of
freedom and simplifying  the optimisation task. This is a  key assumption in the
design optimisation study presented in \cref{ch:modelbaseddesign}.

A  comprehensive multi-objective  optimisation for  optimal design  of batteries
was  recently  proposed  by  Changhong  Liu  and~Lin  Liu~\cite{Liu2017b}.  This
optimisation  involved a  multiphysics model  of the  cell wherein  a number  of
design variables  such as electrode  thicknesses, porosities and  particle sizes
were considered. The optimisation objectives were to maximise the specific power
and specific energy  during discharge as well as to  minimise capacity loss. The
model was solved by using a genetic  algorithm. While this work is certainly the
first of its  kind to use a formal mathematical  optimisation framework for cell
design, in the opinion  of this thesis author, the complexity  of the problem is
excessively high --- both mathematically and computationally. For instance, with
three  design variables  considered,  it requires  a  careful interpretation  of
the  resulting pareto  front  to distinguish  their  relative importance.  While
prima~facie this is not an issue, the question of resolving the pareto front for
cells  with different  parameter sets  needs to  be addressed.  The optimisation
algorithm  used  is  an  exotic  genetic  algorithm to  be  run  on  a  parallel
cluster.  From a  near-term  industrial application  viewpoint,  this scheme  is
less  attractive since  real-world  constraints, such  as supply-chain  dictated
raw-material  sourcing, typically  restrict  the number  of  degrees of  freedom
available  for  optimisation.  Nevertheless,  this scheme  is  valuable  from  a
long-term research perspective and merits attention by researchers interested in
model-based design.

From the  body of the  relevant literature surveyed here,  it is clear  that the
issue of optimally  stacking up layers within  a pouch cell, has  not been dealt
with in a  systematic manner. Pouch cells  are the most common type  of cells in
automotive  applications.  Therefore,  optimising  their design  shall  yield  a
beneficial improvement to the overall driving range and fast charging capability
of electric  vehicles while simultaneously  helping to increase the  lifetime of
their  battery  packs.  With  these goals  in  mind,  \cref{ch:modelbaseddesign}
presents  a mathematical  framework for  a model-based  optimal design  of pouch
cells.


