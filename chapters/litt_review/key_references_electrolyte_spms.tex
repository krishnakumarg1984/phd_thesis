 % -*- root: ../../main.tex -*-
%!TEX root = ../../main.tex

% \setlength\LTleft{0pt}
% \setlength\LTright{1cm}

{% begin box to localize effect of arraystretch change
% \setlength\tymin{.25\textwidth}
\setlength{\LTpre}{-5pt}
\singlespacing
\renewcommand{\arraystretch}{1.50}
\small
\centering
\begin{ltabulary}[c]{@{} l L L @{}}
    \caption
    {%
    Summary of salient literature on electrolyte enhanced \glsfmtshortpl{spm} 
    }\label{tbl:spmlassificationlittreviewsummary}\\
    \toprule
    \multicolumn{1}{c}{Source} & \multicolumn{1}{c}{Key Contributions} & \makecell{Limitations/ \\ Other Remarks} \\        
    \midrule
    \endfirsthead
    \multicolumn{3}{c}%
    {{\normalsize \bfseries \tablename\ \thetable{} --- \normalfont  continued from previous page}} \\
    \toprule
    \multicolumn{1}{c}{Source} & \multicolumn{1}{c}{Key Contributions} & \makecell{Limitations/ \\ Other Remarks} \\
    \midrule
    \endhead
    \midrule
    \multicolumn{3}{ r @{}}{{\normalsize  Continued on next page}} \\[-0.5ex]
    \bottomrule
    \endfoot

    \bottomrule
    \endlastfoot

    Schmidt~\etal~\cite{Schmidt2010c} & {Electrolyte concentration solved by an eigenfunction expansion for spatial profile and an \gls{ode} solution for temporal dynamics} & {Excessive theoretical emphasis without regular contextual references to cell modelling, which hinders reproducibility} \\
    Guo~\etal~\cite{Guo2011a} & {Non-linear resistance as a function of current and temperature to capture electrolyte overpotential} & {Empirical approach which is difficult to generalise across parameter sets since large corrections of the order of a few \si{\milli\volt} become essential; Non-linear optimisation may converge to a local minimum} \\
    Di~Domenico~\etal~\cite{DiDomenico2010} & {First to present an approximate analytical solution for electrolyte overpotential} & {Lacks discussion on spatio-temporal calculation of ionic concentration; presumably used constant initial value which is problematic with sustained unidirectional currents} \\
    Guduru~\etal~\cite{DiDomenico2010} & {Pioneered an analytical solution for the spatio-temporal evolution of electrolyte concentration using the \glsfirst{sov} method} & {The derived analytical solution is applicable only for galvanostatic discharge; assumption of near instantaneous establishment of \gls{qss} hinders extensions using \gls{pwl} approximations while trigonometric computations at each time-step impacts embedded applicability} \\
    Prada~\etal~\cite{Prada2012} & {First to incorporate polarisation due to ionic diffusion in the expression for electrolyte overpotential} & {The solution of spatio-temporal ionic concentration is not detailed. However, post-computation of this term, the equation in Prada~\etal~\cite{Prada2012} is widely used for the electrolyte overpotential contribution to cell terminal voltage} \\
    Rahimian~\etal~\cite{KhaleghiRahimian2013} & {Cubic polynomials for spatial approximation of electrolyte concentration and demonstrated satisfactory accuracy for currents up to 5C} & {Limited by the issue of equation deficiency in the \gls{p2d} model; proposed workaround involves computations at additional interior points which is determined by an expensive placement optimisation algorithm} \\
    Luo~\etal~\cite{Luo2013,Luo2013a} & {Derived a modified parabolic  approximation (with exponential scaling functions) for electrolyte spatial concentration profile} & {Computation of time constants of the exponential scaling functions is not explained; improvements over a standard quadratic approximation model is not elucidated} \\
    Tanim~\etal~\cite{Tanim2014} & {Derived transfer functions for ionic concentration distribution and electrolyte overpotential using \glsfmtlong{ima}} & {Coefficients of transfer functions are excessively long and mathematically intractable; Is based upon many high entropy expressions whose analytical derivations are omitted} \\
\end{ltabulary}
}% end box
