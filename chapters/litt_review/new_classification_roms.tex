% -*- root: ../../main.tex -*-
%!TEX root = ../../main.tex
% vim:textwidth=80 fo=cqt

\glsreset{rom}

Battery  modellers face  the  classic conundrum  of  conjuring \glspl{pbm}  that
remain  amenable  for control  applications.  The  prior  attempts made  by  the
research  community  to  tackle  this  challenge  is  examined  here.  The  term
`control-oriented model' can  be considered synonymous with  the term \gls{rom}.
This  is  due  to  the  fact  that  the  complexity  of  \glspl{pbm}  inherently
necessitates the  use of  some order  reduction strategy  for their  adoption in
control and  real-time applications. In this  thesis as well as  in the relevant
literature discussed here, these two terms have been used interchangeably.

Research   into  \glspl{rom}   is  motivated   by  the   pressing  need   for  a
real-time   model   with   accuracy   properties   of   full-order   \glspl{pbm}
but   possessing    the   computational    simplicity   of    \glspl{ecm}   (see
\crefrange{subsec:ecms}{subsec:pbms} for  an overview).  A number  of approaches
to  reduce  the  computational  complexity of  \glspl{pbm}  have  been  explored
in  literature.  Jokar~\etal~\cite{Jokar2016}  provide  a  comprehensive  review
of  the  various  categories  of  reduced  order  \glspl{pbm}  for  lithium  ion
batteries.  However,  the  aforesaid  work  does  not  aim  to  classify  models
based  on  time-vs-frequency  domains.  Fan~\etal{}~\cite{Fan2015}  conducted  a
review  of  reduced  order  modelling   methods,  but  only  provide  a  generic
overview of  deriving and implementing models  in these dual domains  without an
expository  analysis of  the  implications of  these  modelling choices.  Unlike
Jokar~\etal~\cite{Jokar2016}, the review by Fan~\etal{} did not aim to provide a
classification  of various  reduced order  models, but  instead emphasises  on a
broad survey of relevant methodologies  and tools towards \emph{obtaining} them.
Hence,  neither  of  these  works  provide  an  insight  into  the  rubrics  and
implications  of  the  choice  of  either  of  these  domains  to  underpin  the
\glspl{rom}. Although  in principle,  the transformation  between them  is often
a  straightforward  mathematical  exercise,  availability of  models  for  final
implementation in the time domain aids immediate uptake by industry for adoption
in  online \glspl{bms}.  The treatment  of \glspl{rom}  from this  aspect is  so
germane to the central hypothesis of this  thesis (a simple time domain model is
the  key to  large scale  deployment of  \glspl{pbm}), that  the author  of this
thesis feels  compelled to  undertake a simpler  classification exercise  of the
existing  modelling art,  within the  context  of their  suitability for  online
implementation.


In this discussion,  various modelling methodologies and  their resultant models
are viewed as  a single continuum. Consequently this thesis  discusses them from
such a  unified perspective without  microscopic separation of the  final models
from their progenitor mathematical methods. Furthermore, there is also a need to
highlight the salient works among the more recent advances and extensions to the
then  prevailing models  to obtain  an updated  view of  the modelling  art that
have  gained  traction  since the  publication  of  Jokar~\etal~\cite{Jokar2016}
and  Fan~\etal~\cite{Fan2015}. Hence,  the specialised  review of  reduced order
modelling  literature  covered  in  this  section  intends  to  supplement,  not
supplant,  the breadth  of  research  covered between  the  aforesaid works.  In
particular, care has been taken to minimise repetition of background art already
analysed in these aforementioned review articles, thereby striving to report the
subset of prior research that is  pertinent to illustrate the new classification
scheme introduced  here. The author  does not aim  to adhere to  a chronological
presentation of such  background works. Instead, salient  \gls{rom} families are
introduced  in  the  context  of  discussion  of  their  significance  within  a
particular mathematical modelling technique.


In the views of this thesis author, physics-based control-oriented models can be
classified as belonging to one of the following categories
% \begin{itemize}[topsep=0pt, partopsep=0pt, parsep=0pt, itemsep=0pt]
\begin{itemize}
    \item Frequency domain \glsfmtshort{rom}s
    \item Quasi-hybrid time/frequency domain \glsfmtshort{rom}s
    \item Hybrid \glsfmtshort{rom}s based on equivalent circuits
    \item Time-domain \glsfmtshort{rom}s
\end{itemize}
A common characteristic of all control-oriented models is that their ultimate
goal is to lower the computational burden of the pertinent physical quantities
during operation of the cell.  However upon a closer study, the contrasting
aspects that set them apart become apparent. Understanding the behavioural
differences that stem from their pedigree is the motive behind distinguishing between those models that are derived directly in
the time  domain versus  those that  are derived first  in the  frequency
domain, but  later  converted  to  time  domain.

The classical \mbox{modus operandi} in
frequency domain modelling is to transform the underlying physical equations
into the Laplace space (the complex S~plane) followed by a Padé approximation to reduce the number
of coefficients in the resulting transfer functions. This category of models is
briefly evaluated in \cref{subsec:freqdomainroms}. At the other end of the
spectrum are the time-domain \glspl{rom} which typically adopt the strategy of either
\begin{enumerate*}[label=\itshape\alph*\upshape)]
    \item simplifying the computational mesh used,
    \item capturing only the salient dynamics of the cell which leads to a smaller parameter set,
        and/or
    \item a combination of the two.
\end{enumerate*}
The state of the art in these models is studied in \cref{subsec:timedomainroms}.
An overlapping continuum of mathematical techniques is encountered in the literature discussing the wide assortment of \glspl{rom} spanning the intervening space between these
extrema. The classification
presented here reflects this author's opinion on their broad approach to model
reduction. For instance, equivalent circuits have only a few
parameters and are an attractive option for the final model
implementation. In the recent years, there has been an uptrend in the
efforts employing mathematical methods to generate physics-based equivalent
circuits, the salient of which are discussed in \cref{subsec:hybridroms}.
Finally, there exist Quasi-hybrid \glspl{rom} which do not necessarily seek to
formulate equivalent circuits, but rather seek to arrive at a time domain reduced order
state-space realisation by proceeding through a series of mathematical transformations
starting from the frequency domain. The order reduction here is achieved
through a reduction in the number of \emph{states} and not directly through a
reduction in number of parameters. These methods typically  use employ
time-domain and frequency-domain aspects of control theory, such as
Markov parameters and the Ho-Kalman algorithm. An evaluation of the salient
hybrid \glspl{rom} from literature is presented in \cref{subsec:hybridroms}.


In  principle, any  modelling  method  that yields  a  time domain  mathematical
description  of physical  phenomena that  is lower  in computational  complexity
by  some  arbitrary   magnitude  than  the  original  \gls{dfn}   model  can  be
considered  as  a  candidate  for  further  investigation.  In  the  absence  of
a  canonical  or  quantitative  definition  of  what  constitutes  a  \gls{rom},
the  number  of  candidate  family  of  models  to  consider  is  overwhelmingly
large.  In  practice,  the  constraint  imposed   by  the  scope  of  this  work
\ie~suitability  for real-time  implementation, limits  the choice  of candidate
modelling  families.  For  instance,   models  relying  primarily  on  classical
finite  difference~\cite{Smith2006}, Galerkin's  approximation~\cite{Dao2012} or
Galerkin's  projection~\cite{Fan2016,Fan2018}  methods  for  transformation  and
order  reduction of  one or  more  field variables  of the  \gls{dfn} model  are
excluded from  further study. This  is done in  view of the  impracticability of
implementing  such  models in  a  resource-constrained  environment such  as  an
embedded \gls{bms} controller.

\subsection{Frequency   domain   \glsfmtshort{rom}s}\label{subsec:freqdomainroms}

Owing  to the  low entry-barrier  for adoption  in a  real-time controller  that
typically logs data samples at  specific time intervals, this thesis prioritises
those models  that are cast in  a mathematical form directly  suitable for final
implementation in  the time domain. This  choice implies the exclusion  of those
models that  are derived and implemented  entirely in the frequency  domain. For
the sake of readers interested in  frequency domain methods, the discussion here
briefly introduces  salient literature  employing the Padé  approximation method
that serves as a backbone of a wide variety of frequency domain models.


The     transfer     function     oriented     Padé     approximation     method
for    low    order    physics-based     battery    modelling    pioneered    by
Forman~\etal{}~\cite{Forman2011a}    has   gained    widespread   adoption    in
the    areas     of    cell     design~\cite{Marcicki2013},    charge-trajectory
optimisation~\cite{Bashash2010},    controller    design~\cite{Perez2015}    and
state    estimation~\cite{Marcicki2013,Moura2012}.     Although    Prasad    and
Rahn~\cite{Prasad2013} present  an online identification  of a subset  of ageing
parameters  using  a Padé  model  and  the  recursive least  squares  algorithm,
specific implementation details  such as the transformation of  the Padé reduced
impedance to discrete-time  difference equations were not  provided. Padé models
are typically limited to offline applications owing to the aggressive trade-offs
required in  its approximation order  so as  to maintain high  accuracies. Those
models truncated  to very low  Padé order exhibit  poor fidelity and  perform no
better  than  classical  \glspl{ecm},  although recent  research  attempts  have
focused to mitigate this drawback~\cite{Yuan2017a,Yuan2017}.


\subsection{Quasi-hybrid time/frequency domain \glsfmtshort{rom}s}\label{subsec:quasiroms}

Smith~\etal{}~\cite{Smith2007} pioneered a semi-hybrid approach to reduced order
modelling and  obtained closed  form expressions  for all  electrochemical field
variables  in  the frequency  domain  except  for those  describing  electrolyte
concentration and  potential (which were  solved separately using  the classical
finite  difference  discretisation  method).  To the  author's  knowledge,  this
is  the  earliest published  instance  wherein  all  the  dynamics of  the  full
order  model  were  completely  retained  in  the  frequency  domain.  This  was
facilitated through  the use  of transcendental  transfer functions  that helped
to  avoid  the  accuracy  degradation brought  about  by  truncation  techniques
such  as  Padé  approximation.  In  the first  stage  of  the  model  derivation
detailed in  the said  article, a  composite impedance  model for  the frequency
range  of interest  from  \SIrange{0}{10}{\hertz} was  obtained.  This was  then
converted to a \engordnumber{12} order state  space model using the technique of
residue  grouping  and  truncation,  thereby demonstrating  the  first  instance
of  the   so-called  hybrid  modelling   workflow.  The  \gls{rom}   derived  in
Smith~\etal{}~\cite{Smith2007}  was capable  of predicting  the cell's  terminal
voltage within \SI{1}{\percent} of the full-order \gls{dfn} model.

The  modelling  effort by  Smith~\etal{}~\cite{Smith2007}  also  has the  unique
distinction of being  the first of its  kind to render a  \gls{pbm} suitable for
implementation in the classical \gls{lti} state-space formulation
\begin{equation}\label{eq:LTIstatespace}
    % \SwapAboveDisplaySkip
    \begin{aligned}
        \dot{\mathbf{x}} &= A\,\mathbf{x} + B\,\mathbf{u} \\
        \mathbf{y} &= C \, \mathbf{x} + D\, \mathbf{u},
    \end{aligned}
\end{equation}
\begin{flalign}
    % \SwapAboveDisplaySkip
    \text{where   }    \mathbf{x}   \in   \mathbb{R}^{n\times   1},\:    A   \in
    \mathbb{R}^{n\times   n},\:   B  \in   \mathbb{R}^{n\times   m},\:\mathbf{y}
    \in  \mathbb{R}^{p\times  1},\:  C   \in  \mathbb{R}^{p\times  n},\:  D  \in
    \mathbb{R}^{p\times m}\: \text{and }  \mathbf{u} \in \mathbb{R}^{m \times
    1}
    && \notag
\end{flalign}
that is amenable  for controller design and for  further system-level simulation
studies  \eg~as a  component  in  the energy  storage  subsystem  of a  (hybrid)
electric vehicle drivetrain.
% an  \gls{xeV} drivetrain.

The requirement  of a  relatively large  number of  state variables  (12~in this
case) for  describing the system's  dynamics dilutes the effectiveness  of state
estimation  algorithms.  In  the  classical isothermal  implementation  of  this
\gls{rom}, with  the cell's terminal  voltage being the only  measured quantity,
the  observablity of  the model  degrades significantly.  Although Smith~\etal{}
performed  an  observablity analysis  of  the  model  in a  noise-free  context,
the  presence of  process noise  (via unmodelled  electrochemical phenomena  and
parameter uncertainties)  coupled with corruption of  measurement values through
sensor  noise  in  a  harsh  electrical  environment  such  as  in  a  vehicle's
drivetrain, makes  this model  unattractive for state  estimation tasks  in such
embedded applications.


Several attempts have been undertaken to  improve and extend the ideas pioneered
in Smith~\etal{}  For instance,  Lee~\etal{}~\cite{Lee2012a,Lee2012}~addressed a
critical missing aspect \viz~the derivation of transcendental transfer functions
for  both  the  electrolyte  concentration and  its  potential.  These  transfer
functions were  obtained by  using a Sturm-Liouville  approach by  retaining the
first  five modes  of an  eigenfunction  expansion procedure  which is  detailed
in~\cite{Lee2012,Lee2012a}. To  the author's best  knowledge, this is  the first
published  work wherein  all electrochemical  field variables  of the  \gls{dfn}
model were  considered for  inclusion in a  deterministic model  order reduction
procedure whilst keeping the derivation entirely in the frequency domain.


Obtaining closed form expressions for  the electrochemical variables achieved in
Smith~\etal{} (for all quantities other than electrolyte transfer functions) and
Lee~\etal{} (all  quantities including electrolyte transfer  functions) also has
an important computational implication.  With these capstone derivations serving
to complete the  model description in the frequency  domain, all electrochemical
variables  of the  \gls{dfn}  model could  now be  solved  independently at  any
desired spatial location, in particular at certain crucial locations such as the
interface of each electrode with  the respective current collector or separator.
This groundbreaking idea sharply contrasted with the then prevalent state of the
art  in  reduced  order  modelling.  For  the  simplification  of  the  original
\gls{pdae} of  equations in \cref{tbl:dfneqns}, most  order reduction approaches
(excluding the \gls{spm} that shall  be discussed later) invariably required the
solution of all electrochemical quantities  at multiple node locations along the
thickness of  the cell, thereby adding  to the computational burden.  This was a
significant deterrent to  the adoption of such \glspl{rom},  particularly if the
intended purpose of  the model is to simply predict  the cell's terminal voltage
or serve as the plant model in \gls{soc} estimation applications.


In the  same publications~\cite{Lee2012a,Lee2012}, Lee~\etal{} also  devised the
\gls{dra},  a  novel  scheme  to  systematically  transform  all  transcendental
transfer functions to  the time domain so as to  obtain an \gls{lti} state-space
model  given  by  \cref{eq:LTIstatespace}.  The  \gls{dra}  method  retains  the
physical character of the original \gls{dfn}  equations until the very last step
wherein the matrices governing the  system's dynamics are generated. This yields
a one-dimensional  discrete-time \gls{rom}  of the cell  that is  entirely based
upon fundamental  physical principles.  The \gls{rom}  thus obtained  could then
be  used to  compute  the  time-evolution of  all  the internal  electrochemical
quantities of the \gls{dfn} model. As an illustrative application of the method,
Lee~\etal~\cite{Lee2012a} performed a simulation study of
\begin{enumerate*}[label=\itshape\alph*\upshape)]
    \item the reaction flux density,
    \item surface concentration of Li,
    \item ionic concentration of \ch{Li^+} in the electrolyte,
    \item potential in electrolyte, and
    \item potential in solid
\end{enumerate*}
in   the  anode   and  cathode   at   the  respective   domain  boundaries   and
demonstrated  their high  accuracies relative  to a  benchmark \gls{dfn}  model.
In  Lee~\etal~\cite{Lee2012,Lee2012a}, the  cell  voltage  was computed  through
linear  combinations of  these  time-domain variables  with suitable  non-linear
corrections. Yet another advantage of this model order reduction process is that
the method does not involve any  form of non-linear optimisation that is typical
of other order reduction schemes that attempt a top-down approach of simplifying
the \gls{pbm} equations. In  particular, the \gls{dra} scheme provides
a deterministic method for selection of the order of the simplified model, which
is a pioneering  contribution in the field of reduced  order modelling of Li-ion
cells.


The author  of this thesis  considers the formulation of  the \gls{dra} to  be a
breakthrough  contribution that  has  helped in  bringing physics-informed  time
domain models a step closer to online implementation without having to resort to
forming a lumped impedance and then truncating it suitably. This seminal work is
a first  of its  kind that  is amenable to  implementing real-time  controls for
an  entire  cell  without  relying  upon such  empirical  and  \mbox{ad hoc}  modelling
constructs. In a  subsequent paper by the same  lead author~\cite{Lee2014}, this
approach was  then extended to  a wider  range of operating  conditions spanning
various  choices  of  initial  \glspl{soc}, temperature  and  C-rates.  Although
the  final  state  space  model  thus  obtained  is  simple  to  implement,  the
classical \gls{dra} scheme suffers from significant computational bottlenecks in
forming the  required block-Hankel  matrices during the  model-derivation phase.
A  memory-efficient  version  of  the \gls{dra}  exploiting  the  skew-symmetric
structure  of  these  Hankel  matrices   was  proposed  by  this  thesis  author
\ie~Gopalakrishnan~\etal{}~\cite{Gopalakrishnan2017},  which drastically  lowers
the requirements for  computational memory and processing power.  The details of
this contribution shall be presented in \cref{ch:improveddra} of this thesis.


In both the original as well  as the improved \gls{dra}, the eigenfunction modal
expansion  of electrolyte  concentration  transfer  function is  computationally
intensive.  A  slightly  less  detrimental   disadvantage  with  the  series  of
transcendental transfer functions associated  with the electrolyte concentration
was   that  their   derivation  entailed   mathematically  cumbersome   symbolic
manipulations that  dictated the need  of a  capable \gls{cas}. Although  from a
standalone  viewpoint  this  requirement  does  not seem  to  be  critical,  the
\mbox{Ho-Kalman}  algorithm  that  forms  a  core  component  of  the  \gls{dra}
scheme  is  steeped  in  numerical linear  algebra  routines.  Furthermore,  for
facilitating  state  estimator  and  controller designs,  it  is  convenient  to
implement the resultant  state-space model in a  classical numerical computation
environment   such  as   \textsc{MATLAB}.  Taking   these  into   consideration,
Rodriguez~\etal{}~\cite{Rodriguez2017}  introduced a  simplified computation  of
the  electrolyte  concentration  transfer  function by  applying  the  \gls{vop}
scheme.  With  this  final  improvement,  the  hybrid  \gls{rom}  implementation
originally envisaged by Lee~\etal{} can  be considered feature-complete with low
computational  requirements  during  both model  derivation  and  implementation
phases.


A  key  drawback  of  the  transcendental  transfer  function  approach  is  the
requirement for  linearisation at  a specific \gls{soc}.  This implies  that the
entries in  the matrices of the  state space model depends  on the linearisation
point. In all published works  employing this approach, these transfer functions
were  obtained by  linearising the  \gls{p2d} equations  of the  \gls{dfn} model
(see \cref{tbl:dfneqns}), typically  at an operating point  of \SI{50}{\percent}
\gls{soc}.  The linearisation  requirement renders  the model  usable only  in a
narrow range of  \glspl{soc}. Furthermore, this adversely  affects the usability
of the  model for state  estimation tasks, wherein the  \gls{soc} is in  fact an
unknown quantity and is to be estimated.


In order to extend the model's range of validity, Lee~\etal{}~\cite{Lee2014} had
used a  simple model-blending approach  by interpolating between  several linear
models  pre-computed at  different  \gls{soc} and  temperature combinations.  To
guarantee robustness  during change-over, a  naive approach is to  incorporate a
large  number  of  break-points  in  the  look-up  table.  Since  the  model  is
intended  for  online  operation,  this would  entail  significant  requirements
of  both operating  memory  and non-volatile  storage.  An alternative  approach
is  to  implement  a  fairly  coarse  break-point  table  with  a  sophisticated
changeover  mechanism. However,  this  demands careful  tuning  of the  blending
parameters  and  gain values,  an  in-depth  treatment  of  which has  not  been
provided in Lee~\etal~\cite{Lee2014}.  Furthermore, employing these interpolated
matrices---whose  entries are  obtained  from pre-computed  matrices at  various
\glspl{soc} and temperature---for state-estimation creates a subtle cyclic loop.
The  stability of  this  internal feedback  loop thus  introduced  has not  been
analysed in  literature. This  renders the idea  of state-estimation  using such
run-time interpolated models questionable.


The author  of this thesis hypothesises  that any perceivable drawbacks  such as
non-smooth changes  in \gls{soc} estimates  arising from using  blended matrices
could  be potentially  mitigated by  using  smoothing filters  and other  \mbox{ad hoc}
mathematical apparatus. However,  there exists no published  work that discusses
these engineering aspects  or on how to actually implement  them in \glspl{bms}.
Coupled with  the absence  of a  theoretical analysis  of loop  stability, these
models are deemed  as not being suitable for immediate  adoption by industry, at
least until  these aforementioned gaps  have been addressed  satisfactorily. The
non-linear state variable model  presented by Guo~\etal{}~\cite{Guo2017} aims to
address this  issue through a \gls{rom}  in the frequency domain  by eliminating
the linearisation phase  from the workflow. However, the online  solution of its
field  variables  entails  a complex  prediction-refinement  procedure,  loosely
defined as  implicit and explicit  solution methods,  for each subsystem  of the
\gls{dfn} model. The  formulation of the final model is  not clearly illustrated
and in the views of this author, is not easily comprehensible. In the absence of
actual source  code, a numerical example  or pseudo-code of the  model reduction
workflow could  have immensely  helped with the  reproducibility of  the results
claimed in the aforesaid publication.

In summary, the  concept of hybrid \glspl{rom} is  certainly promising, although
more work is required to address the  present gaps, most prominently the need to
linearise their equations at certain operating points.


\subsection{Hybrid \glsfmtshort{rom}s based on equivalent circuits}\label{subsec:hybridroms}

Physics-inspired
\glspl{ecm}~\cite{Prasad2012,Prasad2014,Zhang2017,Cheng2017,Merla2018}
are a class of hybrid models that have rapidly gained prominence since the
publication of Jokar~\etal~\cite{Jokar2016} and Fan~\etal~\cite{Fan2015}. In
this case, the derivation of the relevant model equations is performed in
the frequency domain. This frequency domain representation is then converted
to a form suitable for implementation as an equivalent circuit. Prasad
and Rahn~\cite{Prasad2014} extended their Padé order reduced model, first
presented in~\cite{Prasad2013}, by converting their impedance model into
standard equivalent circuits. A key point to be highlighted is that these
family of models do not necessarily strive to retain the classical Randles
structure~\cite{Randles1947} for their equivalent circuit representation.
Instead, the values of the electrical circuit components such as series
resistance and equivalent capacitance are obtained through various mechanisms
such as \gls{eis} measurements under load. The biggest advantage of such
models is that they serve as drop-in replacements to traditional \glspl{ecm}
whilst still retaining their origins in physical principles rather than on
empirical curve-fitting.


A  common  characteristic  of all  hybrid  models  is  the  lack of  a  physical
meaning to  their model  parameters. This severely  limits the  insights offered
by  such  models  into  electrochemical  phenomena internal  to  the  cell.  The
biggest  attraction  of  using  \glspl{pbm} is  the  possibility  of  predicting
quantities such as the \gls{soap} or  phenomena such as cell degradation through
accurate computation  of the solid  phase surface concentration  and potentials.
Furthermore,  a model  capable  of  implying a  direct  and causal  relationship
between a  group of physical  parameters and internal overpotentials  at various
spatial locations within the cell serves as a powerful tool for in-situ lifetime
estimation  of batteries.  Although the  circuit components  of physics-informed
\glspl{ecm} and the state-space models discussed here trace their origins to the
original parameters  of the \gls{dfn}  model, the  link between the  final model
coefficients and their progenitor physical parameter sets is tenuous at best.


With  the  goal of  translating  physical  parameters  of  a cell  into  circuit
components, Zhang~\etal{}~\cite{Zhang2017} presented a lumped \gls{ecm} based on
Padé approximation and  model truncation. However, the sensitivity  of the final
model values owing to perturbations in  the original physical parameters was not
evaluated. Consequently, there  is a lack of clarity in  the relative importance
of physical parameters and their influence on circuit component values.


Merla~\etal{}~\cite{Merla2018} introduced an \gls{ecm}  that can be parametrised
by attempting  a systematic  decoupling of  the kinetics  and diffusion  at both
electrodes  and the  electrolyte. Although  these interacting  phenomena can  be
complex to  resolve over  all length and  time-scales, acceptable  trade-offs in
accuracy  was  demonstrated to  be  achievable  from a  system-level  simulation
perspective. A  drawback of this approach  is that key physical  parameters such
as  solid and  electrolyte  diffusion  coefficients are  attributed  to the  two
electrodes  through  \mbox{ad hoc},  non-verifiable assumptions.  Furthermore,  in  the
aforesaid article, notable discrepancies exist  in the values of parameters such
as  electrolyte  conductivity  (obtained  through  calculations  from  \gls{eis}
measurements) to that typically reported in literature.

It must be acknowledged that presently  there exists no modelling candidate that
provides  all the  desirable  characteristics  sought after  in  a \gls{rom}  to
unconditionally adopt it  for final implementation in the  time domain. However,
it is  strongly desirable  that the  majority of the  final model  values retain
their physical meaning, yielding system  engineers and cell designers alike with
a  direct  and  causal  relationship  between groups  of  parameters  and  their
influence on the cell's operational performance.  Since one of the goals of this
thesis is to  provide a readily usable \gls{rom} that  is immediately deployable
in an online implementation, the author  concludes that at present, the benefits
offered by physics-inspired hybrid \glspl{ecm}  do not decisively outweigh their
drawbacks.

\subsection{Time-domain  \glsfmtshort{rom}s}\label{subsec:timedomainroms}

The working rubric of all time domain \glspl{rom} typically consists of attempts
to  reformulate the  original  \gls{p2d}  model equations  towards  the goal  of
simplifying  them to  as much  extent  as possible.  In contrast  to the  hybrid
models, all tasks involved in both model derivation and final implementation are
carried out  entirely within the time  domain. While a subset  of prior research
has  focused  only  on  simplifying  certain  aspects  of  the  cell's  dynamics
\eg~diffusion  in  the two  electrodes,  other  published  works have  aimed  at
providing a  simplified description of  the time domain evolution  of \emph{all}
physical quantities of  the cell. An evaluation of the  salient literature based
upon both these approaches is performed here.


In  this discussion,  the  modelling approaches  that  entail computations  with
medium or large dense matrices~\cite{Li2016,Xu2016,Corno2015} or those involving
concepts such  as fractional  order derivatives~\cite{Sabatier2014,Sabatier2015,
Li2017, Mu2017, Wang2017}  shall not be discussed. In the  views of this author,
it appears that  the academic community has implicitly considered  them to be so
abstruse that there has not yet  been a comparative study pitting these families
of models against  the prevalent art. Comparing with the  typical published work
in  this field,  it  is not  clear  on how  such  models distinguish  themselves
uniquely within the broader landscape of reduced order battery modelling.


A few mathematical  techniques for \gls{pde} simplification in  the time domain,
such as Hilbert space representation and singular perturbation, were applied for
cell  modelling in  Manzie~\etal~\cite{Manzie2015}. However,  their presentation
lacks expository  visual information such as  plots of time domain  evolution of
the internal and terminal variables  for dynamic load profiles. Furthermore, the
authors  have  not provided  a  tabulated  set  of  physical parameters  of  the
cell being  simulated which  therefore impedes  reproducibility of  the results.
Consequently, these methods have not seen a healthy uptake either in academia or
in industry. The  author of this thesis  considers this presentation to  be of a
cursory nature  and therefore,  it shall  not be  discussed here.  The remainder
of  this  section discusses  several  popular  families  of time  domain  models
and  provides a  summary evaluation  of  their relative  merits and  weaknesses.
\Cref{ch:spmanalysis} presents a formal in-depth  analysis of all aspects of the
state of the art  in the field of single particle  modelling. The author reckons
that  the  \gls{spm}---reviewed briefly  later  in  this section---is  the  most
promising  candidate identified  among  these time-domain  models  to nurture  a
latent  potential  to  facilitate  faster  adoption  of  \glspl{pbm}  in  online
applications.

In the \gls{dfn} model, the evolution of lithium in the solid phase is described
by the classical  diffusion equation given by  Fick's first law~\cite{Fick1995}.
In order  to solve for  this concentration profile  in full-order models,  it is
required to discretise every spherical particle (represented by the placement of
a  node  in  the  axial  \ie~through-thickness  direction)  along  its  radial
direction (pseudo  dimension). This  additional discretisation along  the pseudo
dimension dramatically increases the overall  number of discretisation nodes and
adversely  affects  computational  efficiency.  The impact  of  such  high  node
densities  on the  computational requirements  of the  original \gls{p2d}  model
coupled  with the  fact  that diffusion  in  the solid  phase  is typically  the
rate-limiting  aspect  of  batteries  have  led  researchers  to  adopt  various
mitigation strategies  to tackle this issue.  In contrast to the  pure frequency
domain and the semi-hybrid/hybrid approaches  discussed thus far, these attempts
typically strive  to arrive at  a simpler  computational mesh, whilst  aiming to
retain high  fidelity. It should  be noted that  high node densities  are mainly
required near the  surface of the spherical particles for  the pseudo dimension.
Similarly, the clustering  of nodes is desirable near the  separator and current
collector interfaces along the axial dimension.
Thus, a sizeable number  of order reduction strategies in the time  domain seek
to adopt non-uniform node  spacing towards lowering the aforesaid computational
issues.

Computationally    efficient     pseudo-spectral    schemes     for    numerical
solution   of   \glspl{pde}   can   be  employed   by   placing   discretisation
nodes    at    orthogonal    collocation     points    obtained    by    solving
for     the     zeros    of     certain     class     of    polynomial     basis
functions~\cite{Ferguson1971,Trefethen2000,Boyd2001,Shizgal2015,Dutykh2016}. The
accuracy of such  schemes extend beyond the algebraic orders  of that achievable
with  classical Finite  Difference,  Finite Element  or  Finite Volume  Schemes.
Northrop~\etal{}~\cite{Northrop2011}  pioneered  their  application  in  battery
modelling  by  employing  Jacobi  polynomials  as  underlying  basis  functions.
Suthar~\etal{}~\cite{Suthar2014}  replaced  Jacobi  polynomials  with  Chebyshev
polynomials to  extend the  applicability of the  resulting \gls{rom}  to higher
C-rates. Bizeray~\etal{}~\cite{Bizeray2015} provide a  detailed treatment of the
usage  of Chebyshev  discretisation for  the full  \gls{p2d} model  on a  global
scale \ie~along both the axial and radial directions for all equations of the
\gls{dfn}  model. A  Lagrangian-like integral  method was  proposed by  Rahn and
Wang~\cite{Rahn2013}  to  deal  with  electrolyte  and  solid-state  diffusions.
However, this method works well only at low C-rates.


The  reduced  number   of  nodes  as  well  as  their   clustered  placement  at
the  desired  spatial  locations  facilitated by  these  discretisation  schemes
lowers  the computational  burdens  of simulating  a  physics-based cell  model.
In~\cite{Gopalakrishnan2018}, wherein  the author of  this thesis serves  as the
joint lead author,  a hybrid scheme is proposed which  retains a standard finite
volume  discretisation  in  the  axial  domain  whilst  adopting  the  Chebyshev
discretisation  only for  the critical  solid phase  diffusion component  in the
radial direction. The  details of this work  is presented in the  context of the
research methodology  on delivering  a model-based  pouch cell  design discussed
in  \cref{ch:modelbaseddesign}.   In  pseudo-spectral  methods,   the  \gls{p2d}
equations,  their  boundary conditions  and  corresponding  field variables  are
mathematically transformed to the Chebyshev  space within which they are solved.
The  details  of  this  transformation  is  presented  in  the  context  of  the
author's aforementioned work in \cref{sec:hybridfv-spectral} for the solid-phase
diffusion equation. Finally,  these solved quantities are converted  back to the
physical  space through  a corresponding  inverse transformation.  Although this
bi-directional  transformation is  purely algebraic  in nature,  the requirement
of  running a  spatially  resolved  model coupled  with  the  overheads of  such
variable  transformations render  these class  of models  unsuitable for  online
implementation.  The contribution  of Lee~\etal~\cite{Lee2012a,Lee2012}  \ie~the
ability to solve for any electrochemical variable at arbitrary spatial locations
by completely eliminating the need for spatial discretisation assumes particular
significance in this context.


In all  non-uniform discretisation schemes  discussed here, the  implications of
using  a non-adaptive  support mesh  obtained by  the placement  of nodes  whose
locations are optimised a~priori must  be considered carefully. For instance, in
the prolonged operation of the cell with a net unidirectional charge flow \eg~in
an electric vehicle  application, the reaction front drifts  from separator back
towards the current collectors. This is due  to the exhaustion of lithium at the
surface  of particles  near  the  separator interfaces.  In  this scenario,  the
solutions  produced by  these models  could be  worse than  simpler models  with
uniform mesh-density. Although  adaptive meshing strategies can  be employed for
desktop simulation  with minimal effort,  it remains to be  seen if this  can be
deployed successfully in a resource-constrained  environment such as an embedded
\gls{bms} controller, and hence is a candidate for future research.


The   computational   bottlenecks  arising   due   to   discretisation  in   the
radial   direction    have   motivated   researchers   to    explore   mesh-free
approaches   to    solve   for   the   solid    phase   concentration   profile.
Subramanian~\etal~\cite{Subramanian2004}  pioneered  the  concept  of  employing
polynomial approximations of the Fickian diffusion equation to solve for lithium
concentrations  in the  porous  electrodes. In  this  approach, the  solid-phase
surface  concentrations were  expressed  as correction  terms  applied to  their
average concentrations (which  was described using a  second degree polynomial).
In  a  follow-on  study~\cite{Subramanian2005},  the same  authors  presented  a
solution using higher  order polynomials and performed  a dimensionless analysis
of  their proposed  reformulation.  The details  of  the \engordnumber{4}  order
polynomial approximation  is presented  in the context  of this  thesis author's
comprehensive  analysis of  the  \gls{spm}  modelling art  and  is discussed  in
\cref{subsec:basicspmfurtherdimensionalityreduction}.  In  the  \engordnumber{2}
and  \engordnumber{4}  order  solutions,  the polynomial  equation  for  surface
concentration  was  accompanied  by  a corresponding  \gls{ode}  for  describing
the  temporal  evolution   of  average  concentration,  thereby   leading  to  a
system  of \glspl{dae}.  Furthermore, Subramanian~\etal{}~\cite{Subramanian2007}
convincingly demonstrated  the application  of polynomial approximation  for the
solid  phase  diffusion equation  in  the  numerical  simulation of  a  complete
\gls{dfn} cell model.


Using  polynomial  approximation  for  the  solid  phase  concentration  results
in  a  drastic reduction  in  the  number of  \glspl{dae}  needed  to solve  the
full  model since  now  discretisation  needs to  be  performed  only along  the
axial  direction. The  polynomial approximation  solution applied  to solve  for
the  surface  concentration  in  the  solid  phase can  hence  be  viewed  as  a
dimension reduction  approach, as it removes  the need to numerically  solve the
concentration  dependence  in the  radial  direction.  The textbook  by  Carslaw
and Jaeger~\cite{Carslaw1947}  provides detailed  derivations for  obtaining the
standard  analytical  solution  to  Fick's  law  of  diffusion  in  the  context
of  heat  conduction  in  solids.  Liu~\cite{Liu2006}  derived  this  analytical
solution for the  lithium intercalation process in the solid  phase, taking into
account  the  idiosyncrasies  of  porous electrodes.  However,  this  expression
involves an infinite sum expansion  of eigen modes. Guo and White~\cite{Guo2012}
formulated  an expression  for a  truncated  approximation of  this solution  to
arbitrary number of  terms. Furthermore, they demonstrated the  validity of this
approximation  by  comparing the  analytical  solution  truncated to  the  first
5~terms to that obtained from a classical finite element solution. However, this
truncated analytical  solution involves exponential and  trigonometric terms and
is  non-trivial to  implement on  \gls{bms}  chips, particularly  in those  that
lack  support for  floating  point  computations. Moreover,  there  has been  no
extensive study  comparing the analytical  solution to the  polynomial approach.
Consequently,  this approach  has not  yet gained  widespread popularity  in the
inherent elimination  of the radial dimension  that is so deeply  ingrained as a
core aspect of the cell-level order reduction approaches discussed here.

The computational  speed-up facilitated  by using polynomial  approximations for
the  solid  phase diffusion  has  motivated  other  researchers to  extend  this
approach  to  all  other  electrochemical  variables  of  the  \gls{dfn}  model.
Deng~\etal{}~\cite{Deng2018}  presented a  polynomial-centric evaluation  of the
full  \gls{p2d}  model, whose  notable  contribution  is  in providing  such  an
approximation for the molar flux density along the thickness of the cell. To the
best knowledge  of this  thesis author,  this is the  first published  work that
provides a  spatially dependent simplified  computation of the  interfacial flux
density. This represents a balanced choice  between the need to use the strongly
non-linear Butler-Volmer kinetics \cref{eq:butlervolmer} or  having to resort to
a lumped  representation of average  kinetic behaviour. Hence, this  approach is
particularly suited  to reduced order  modelling of cells with  medium electrode
thicknesses wherein the  lumped representation of flux density  is not generally
applicable.

One  serious drawback  in Deng~\etal{}~\cite{Deng2018}  is the  use of  a Finite
Difference approximation for computing the spatial gradients of the \gls{ocp} at
three electrode locations. This  adversely affects computational performance and
is not  suitable for online  implementation. Unless  a proven solution  for such
computational  challenges is  made available,  it is  worthwhile to  continue to
explore other avenues to identify the  most apropos first candidate for adoption
in real-time \gls{bms} environments.

Farag~\etal{}~\cite{Farag2017}  proposed   a  \gls{pwl}  approximation   of  all
governing equations of the electrochemical  model. Given that straight-line fits
to complex phenomena  are inherently too simplistic,  these authors acknowledged
that a naive implementation of their  approach shall therefore result in a crude
approximation of  the cell's dynamics.  Hence, an optimal  knot-placement scheme
was proposed and solved through a  genetic algorithm to compute the break-points
of the \gls{pwl} fit. Since  this computationally intensive step occurs offline,
it does not  adversely affect the real-time performance of  the model. The final
\gls{rom}  is implemented  using  standard state-space  matrices. However,  this
model  also exhibits  the primary  drawback of  all hybrid  modelling approaches
\ie~a complete lack of physical  interpretation of its parameters. As with any
other \gls{rom} involving \gls{soc}-based linearisation points, the stability of
the model  to uncertainties in  physical parameters is questionable.  A detailed
sensitivity analysis  of the knot  placement scheme's output to  such parametric
variation is  to be performed  in order to  establish confidence in  the model's
robustness, before such  \gls{pwl} approaches can gain  widespread acceptance in
online applications.


It  is evident  that all  physics-based  \glspl{rom} presented  thus far  entail
extensive  parametrisation  efforts to  render  them  suitable for  a  practical
application. The difficulties associated with such parametrisation, coupled with
inherent  uncertainties  in  the  obtained  parameter values  act  as  a  strong
deterrent  to stakeholders  outside  academia to  adopt  \glspl{pbm} for  online
implementation  in  a  \gls{bms}.  This  motivates the  need  for  even  further
simplified \glspl{pbm}. One such modelling  candidate is briefly introduced next
and is analysed in detail in \cref{ch:spmanalysis}.


Haran~\etal{}~\cite{Haran1998}  proposed a  highly simplified  representation of
porous  electrodes  for the  metal  hydride  cell  chemistry. In  the  aforesaid
article,  each   porous  electrode  was   represented  as  a   single  spherical
particle.  This concept  was  adopted  for lithium  ion  batteries  by Ning  and
Popov~\cite{Ning2004} and has since become  quite popular. Models employing this
lumped  representation  of  electrodes  are referred  to  as  \glsfmtlongpl{spm}
(\glspl{spm}).  These models  have  three advantages.  Since  they involve  only
a  subset  of  parameters  of  the   original  \gls{dfn}  model,  most  of  them
being  geometric quantities  that  can be  directly  measured without  extensive
chemical  or electrical  testing,  \glspl{spm} are  easier  to parametrise  than
other physics-based  \glspl{rom}. Furthermore,  they are  computationally cheap,
especially when coupled with the  polynomial approximation for solving the solid
diffusion  equation for  each electrode.  Finally, all  model parameters  in the
\gls{spm}  retain their  physical character,  aiding in  a direct  and intuitive
understanding of physical parameters on the cell's operation.

This concludes  the author's review of  literature of the various  reduced order
modelling  strategies. Based  on the  wealth  of information  gleaned from  this
study,  it was  possible to  make  an informed  choice to  pursue the  \gls{spm}
approach  for further  research.  In particular,  it came  to  light that  there
has  been  no systematic  analysis  of  the  state  of the  art  \gls{spm}-based
approach with  a view  to quantify their  performance boundaries.  This author's
contributions to  the time-domain  implementation-focus reduced  order modelling
field include
\begin{itemize}[noitemsep,topsep=0pt, before={\vspace*{-0.25\baselineskip}}]
    \item a thorough analysis of the existing \gls{spm} variants ranging from the rudimentary to the sophisticated
    \item enumeration of issues plaguing each \gls{spm} variant, including current state of the art
    \item further evaluation of literature on the current tactics to overcome the current challenges along with their associated drawbacks
    \item conducting a  wide range of hypotheses-driven trials in an attempt to enhance  the basic \gls{spm} (some of which did not yield the desired
        improvements), and
    \item arriving at  a  feasible  approach  capable  of  moulding  the
        \gls{spm}  framework into  a  readily implementable  solution in
        electric vehicle applications
\end{itemize}
This research was  performed through an iterative cycle of  analysis, design and
simulation-based verification  and shall be  elucidated in a  discourse spanning
two chapters \viz~\cref{ch:spmanalysis} and \cref{ch:newelectrolytemodel}.

