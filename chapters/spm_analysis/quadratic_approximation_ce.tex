% -*- root: ../../main.tex -*-
%!TEX root = ../../main.tex
% this file is called up by main.tex
% content in this file will be fed into the main document
% vim:textwidth=80 fo=cqt

In   this    section,   the    quadratic   approximation   of    ionic   spatial
concentration,   that  underpins   the  electrolyte   model  in   many  improved
\gls{spm}  formulations  is  presented.  The  steps  involved  in  deriving  the
quadratic   approximation   is  detailed   in \cref{subsec:quadraticmodelderiv}.
In \cref{subsec:quadraticsimresultsanalysis},  an analysis  of  the weakness  of
this  model is  performed  based  on the  results  from  applying the  quadratic
approximation scheme. Mitigation of this critical drawback lead to this author's
decoupled  spatio-temporal electrolyte  concentration model  structure which  is
presented next in \cref{sec:newelectrolytemodel}.

\subsection{Model derivation}\label{subsec:quadraticmodelderiv}

The  schematic  in  \cref{fig:coordsquadapprox}  shows  the  definition  of  the
co-ordinate  systems  used  in  deriving the  polynomial  approximation  of  the
electrolyte concentration  profile. The globally defined  $x$~co-ordinate starts
at  the  negative  current   collector  interface~(${x=0}$)  and  terminates  at
the  positive current  collector  interface ($x  = l_\text{tot},\,  l_\text{tot}
=  l_\text{neg}  +  l_\text{sep}   +  l_\text{pos}$).  Three  local  co-ordinate
systems~$z_\mu$   valid  only   within   their  respective   regions  are   also
defined.  In particular,  it  must be  noted  that the  direction  of the  local
$z_\text{pos}$~co-ordinate  axis is  opposite to  that  of the  other two  local
co-ordinate axes as  well as the global co-ordinate axis.  In subsequent usages,
the  suffix in~$z_\mu$  is dropped  and  the  reader is  advised  to infer  the
region  of  validity from  the  usage  context  which  are unambiguous  as  they
occur  in separate  equations. Furthermore,  the notation  of the  three regions~${\{\text{neg, sep, pos}\}}$ is abbreviated to ${\{n,s,p\}}$~respectively in all
mathematical expressions.  The author is  convinced that this notation  does not
detract  from following  the  derivations, but  rather aids  it  by keeping  the
notations compact.

\begin{figure}[!htbp]
    \captionsetup{singlelinecheck=off}
    \centering
    \includegraphics{co_ord_sys}
    \caption[Co-ordinate systems for quadratic approximation of
    electrolyte concentration]{Schematic diagram of the electrochemical sandwich
        consisting of
        \begin{enumerate*}[label=\itshape\alph*\upshape)]
            \item negative electrode,
            \item separator, and
            \item positive electrode
        \end{enumerate*} depicting the co-ordinate system used in deriving the
        quadratic approximation profile. The global spatial co-ordinate is~${x
            \in \{0,L\}}$, where~${L = l_\text{tot} = l_\text{neg} + l_\text{sep} +
        l_\text{pos}}$. Local co-ordinate systems~$z$ specific to each
        region are also defined. It should be noted that the positive
        electrode's local co-ordinate axis direction is reversed with respect to
        the global co-ordinate axis. Illustration reproduced from
    Deng~\etal~\cite{Deng2018}.}
    \label{fig:coordsquadapprox}
\end{figure}

A  standard  quadratic expression  is  chosen  a~priori for  approximating  the
electrolyte concentration profile within each region.
\begin{alignat}{2}
    c_\ensub(z,t) & = a_2(t) z^2 + a_1(t) z + a_0(t) \qquad &  & 0 \le z \le l_\text{n}\label{eq:cenqquadstart}   \\
    c_\essub(z,t) & = a_5(t) z^2 + a_4(t) z + a_3(t) \qquad &  & 0 \le z \le l_\text{s}\label{eq:cesqquadstart}   \\
    c_\epsub(z,t) & = a_8(t) z^2 + a_7(t) z + a_6(t) \qquad &  & 0 \le z \le l_\text{p}\label{eq:cepqquadstart}
    \shortintertext{where     the    coefficient     vector    $\vec{a}(t)     =
        \vect{a_0(t),a_1(t),   \dots  ,a_8(t)}$   is  to   be  determined   at  each
        time-step\footnotemark.  Applying  boundary  conditions of  the  electrolyte
        diffusion equation from  the \gls{dfn} model~(refer \cref{eq:dfnliquiddiff})
        to \crefrange{eq:cenqquadstart}{eq:cepqquadstart},  it  is clear  that  ${a_1 =  0}$ and~${a_7  = 0}$.  Thus, \crefrange{eq:cenqquadstart}{eq:cepqquadstart}
    become}
    c_\ensub      & = a_2 z^2 + a_0         \qquad          &  & 0 \le z \le l_\text{n}\label{eq:cenquadreduced} \\
    c_\essub      & = a_5 z^2 + a_4 z + a_3 \qquad          &  & 0 \le z \le l_\text{s}\label{eq:cesquadreduced} \\
    c_\epsub      & = a_8 z^2 + a_6         \qquad          &  & 0 \le z \le l_\text{p}\label{eq:cepquadreduced}
\end{alignat}
\footnotetext{In rest of  the equations, time-dependence of  the coefficients is
    dropped from  the notation. It is  implicitly noted that they  are time-varying.
    Similarly, spatio-temporal dependence of the electrolyte concentration functions~$c_\text{e,j}$ is omitted  in circumstances where such explicit  notation is not
crucial  for  understanding.} with  the  coefficient  vector being  modified  to~${\vec{a} = \vect{a_0,a_2, \dots ,a_6, a_8}}$.

% -*- root: ../../main.tex -*-
%!TEX root = ../../main.tex
% this file is called up by main.tex
% content in this file will be fed into the main document
% vim:nospell textwidth=180 foldlevelstart=3 foldlevel=3 conceallevel=0

\begin{table}[!htbp]
    \centering
    \caption[Electrolyte equations \& boundary conditions of \glsfmtshort{dfn} model in separator]{Electrolyte-specific governing equations and boundary conditions of the \glsfmtlong{dfn}~(\glsfmtshort{dfn}) model within the separator domain.}
    \label{tbl:dfnelectrolyteeqnsinsep}
    \begingroup
    \makeatletter\def\f@size{9.25}\check@mathfonts
    \addtolength{\jot}{0.875em}
    \begin{tabular*}{\textwidth}{@{} l c r l r @{}}
        \toprule
        \multicolumn{1}{c}{\small Region} & \small Governing equations & \multicolumn{2}{c}{\small Boundary conditions } & {} \\
        {} & {} & \multicolumn{2}{c}{\scriptsize $(l_\text{neg} \coloneqq l_\text{n},\, l_\text{sep} \coloneqq l_\text{s},\, l_\text{pos} \coloneq l_\text{p})$} \\
        \midrule
        \multicolumn{1}{l |}{{\rotatebox[origin=c]{90}{\makecell{\footnotesize Separator\\ \scriptsize $\delta \in \{\text{sep}\}$}}}} &
        $\begin{aligned}
            \vphantom{D_{\text{\tiny eff}_\text{n}}\!\! \! \!\, \diffp{c_\text{e}}{x}{\mathrlap{x = l^{-}_\text{n}}}} \varepsilon_\delta \diffp{c_\text{e}}{t} &=D_\effdelta  \diffp[2]{c_\text{e}}{x} \\[-0.75em]
            \vphantom{D_{\text{\tiny eff}_\text{s}}\!\! \! \!\, \diffp{c_\text{e}}{x}{\mathrlap{x=(l_{\text{n}} + l_\text{s})^{-}}}}\\[1.25em]
            \vphantom{\kappa_{\text{\tiny eff}_\text{n}}\!\! \! \!\, \diffp{c_\text{e}}{x}{\mathrlap{x = l^{-}_\text{n}}}\hspace{5mm} =\kappa_{\text{\tiny eff}_\text{s}}\!\!\!\!\,\diffp{c_\text{e}}{x}{\mathrlap{x = l^{+}_\text{n}}}} \frac{I}{A} &= \overline{\kappa}_\effdelta \left( \diffp[2]{\phi_\text{e}}{x} + \frac{2 R T}{F} (t^0_{+}-1)\diffp[2]{ \ln c_\text{e}}{x}\right) \\[-0.75em]
            \vphantom{\kappa_{\text{\tiny eff}_\text{s}}\!\! \! \!\, \diffp{c_\text{e}}{x}{\mathrlap{x=(l_{\text{n}} + l_\text{s})^{-}}}} \\
        \end{aligned}$ &
        $\begin{aligned}
    \vphantom{D_{\text{\tiny eff}_\text{n}}\!\! \! \!\, \diffp{c_\text{e}}{x}{\mathrlap{x = l^{-}_\text{n}}}} \qquad c_\text{e}\Bigr\rvert_{\mathrlap{x=l^{-}_\text{n}}}\hspace{5mm} &= c_\text{e}\Bigr\rvert_{\mathrlap{x=l^{+}_\text{n}}},\\[-0.75em]
     \vphantom{\kappa_{\text{\tiny eff}_\text{n}}\!\! \! \!\, \diffp{c_\text{e}}{x}{\mathrlap{x = l^{-}_\text{n}}}\hspace{5mm} =\kappa_{\text{\tiny eff}_\text{s}}\!\!\!\!\,\diffp{c_\text{e}}{x}{\mathrlap{x = l^{+}_\text{n}}}} c_\text{e}\Bigr\rvert_{\mathrlap{x=(l_{\text{n}} + l_\text{s})^{-}}}\hspace{5mm} &= c_\text{e}\Bigr\rvert_{\mathrlap{x=(l_{\text{n}} + l_\text{s})^{+}}},\\[1.25em]
 \vphantom{\kappa_{\text{\tiny eff}_\text{n}}\!\! \! \!\, \diffp{c_\text{e}}{x}{\mathrlap{x = l^{-}_\text{n}}}} \vphantom{\left( \diffp[2]{\phi_\text{e}}{x} + \frac{2 R T}{F} (t^0_{+}-1)\diffp[2]{ \ln c_\text{e}}{x}\right)} \phi_\text{e}\Bigr\rvert_{\mathrlap{x=l^{-}_\text{n}}}\hspace{5mm} &= \phi_\text{e}\Bigr\rvert_{\mathrlap{x=l^{+}_\text{n}}},\\[-0.75em]
 \vphantom{\kappa_{\text{\tiny eff}_\text{s}}\!\! \! \!\, \diffp{c_\text{e}}{x}{\mathrlap{x=(l_{\text{n}} + l_\text{s})^{-}}}} \phi_\text{e}\Bigr\rvert_{\mathrlap{x=(l_{\text{n}} + l_\text{s})^{-}}}\hspace{5mm} &= \phi_\text{e}\Bigr\rvert_{\mathrlap{x=(l_{\text{n}} + l_\text{s})^{-}}},\\
    \end{aligned}$ &
    $\begin{aligned}
        \quad D_{\text{\tiny eff}_\text{n}}\!\! \! \!\, \diffp{c_\text{e}}{x}{\mathrlap{x = l^{-}_\text{n}}}\hspace{5mm} &=D_{\text{\tiny eff}_\text{s}}\!\!\!\!\,\diffp{c_\text{e}}{x}{\mathrlap{x = l^{+}_\text{n}}}\\[-0.75em]
        D_{\text{\tiny eff}_\text{s}}\!\! \! \!\, \diffp{c_\text{e}}{x}{\mathrlap{x=(l_{\text{n}} + l_\text{s})^{-}}}\hspace{5mm} &=D_{\text{\tiny eff}_\text{p}}\!\!\!\!\,\diffp{c_\text{e}}{x}{\mathrlap{x=(l_{\text{n}} + l_\text{s})^{+}}}\\[1.25em]
        \vphantom{\left( \diffp[2]{\phi_\text{e}}{x} + \frac{2 R T}{F} (t^0_{+}-1)\diffp[2]{ \ln c_\text{e}}{x}\right)} \kappa_{\text{\tiny eff}_\text{n}}\!\! \! \!\, \diffp{c_\text{e}}{x}{\mathrlap{x = l^{-}_\text{n}}}\hspace{5mm} &=\kappa_{\text{\tiny eff}_\text{s}}\!\!\!\!\,\diffp{c_\text{e}}{x}{\mathrlap{x = l^{+}_\text{n}}}\\[-0.75em]
        \kappa_{\text{\tiny eff}_\text{s}}\!\! \! \!\, \diffp{c_\text{e}}{x}{\mathrlap{x=(l_{\text{n}} + l_\text{s})^{-}}}\hspace{5mm} &=\kappa_{\text{\tiny eff}_\text{p}}\!\!\!\!\,\diffp{c_\text{e}}{x}{\mathrlap{x=(l_{\text{n}} + l_\text{s})^{+}}}\\
    \end{aligned}$ &
    $\begin{aligned}
        \vphantom{D_{\text{\tiny eff}_\text{n}}\!\! \! \!\, \diffp{c_\text{e}}{x}{\mathrlap{x = l^{-}_\text{n}}}} \quad \refstepcounter{equation}(\theequation)\label{eq:liquiddiffnsep} \\[-0.75em]
        \vphantom{D_{\text{\tiny eff}_\text{s}}\!\! \! \!\, \diffp{c_\text{e}}{x}{\mathrlap{x=(l_{\text{n}} + l_\text{s})^{-}}}}\\[1.25em]
        \vphantom{\kappa_{\text{\tiny eff}_\text{n}}\!\! \! \!\, \diffp{c_\text{e}}{x}{\mathrlap{x = l^{-}_\text{n}}}} \vphantom{\left( \diffp[2]{\phi_\text{e}}{x} + \frac{2 R T}{F} (t^0_{+}-1)\diffp[2]{ \ln c_\text{e}}{x}\right)} \refstepcounter{equation}(\theequation) \label{eq:liquidpotentialsep}\\[-0.75em]
        \vphantom{\kappa_{\text{\tiny eff}_\text{s}}\!\! \! \!\, \diffp{c_\text{e}}{x}{\mathrlap{x=(l_{\text{n}} + l_\text{s})^{-}}}}
    \end{aligned}$
    \\
    \bottomrule
\end{tabular*}
\endgroup
\end{table}



\Cref{tbl:dfnelectrolyteeqnsinsep} lists  the equations and  boundary conditions
for  phenomena  describing  electrolyte  diffusion  and  charge  balance  within
the separator  domain. \Cref{eq:liquiddiffnsep} and \cref{eq:liquidpotentialsep}
are  obtained  by  applying  the  corresponding  electrolyte  equations  of  the
\gls{dfn}  model  (see \cref{eq:dfnliquiddiff}  and \cref{eq:dfnliquidpotential}
respectively) to the separator region.

Applying  the  continuity  and  flux  boundary  conditions  of  the  electrolyte
diffusion  equation from \cref{eq:liquiddiffnsep}  at both  separator
interfaces
\begin{alignat}{2}
    a_2 l^2_\text{n} + a_0                      & = \hphantom{-}a_3 \qquad                    &  & \text{\footnotesize (continuity at neg/sep interface)} \label{eq:cecontinuitynegsep} \\
    a_5 l^2_\text{s} + a_4 l_\text{s} + a_3     & = \hphantom{-}a_8 l^2_\text{p} + a_6 \qquad &  & \text{\footnotesize (continuity at sep/pos interface)}                               \\
    2 a_2 l_\text{n} D_\effn                    & = \hphantom{-}a_4 D_\effs \qquad            &  & \text{\footnotesize (flux b.c.\ at neg/sep interface)}                               \\
    \left(2 a_5 l_\text{s} + a_4\right) D_\effs & = -2 a_8 l_\text{p} D_\effp \qquad          &  & \text{\footnotesize (flux b.c.\ at sep/pos interface)}\label{eq:quadcefluxseppos}
\end{alignat}
Note that the negative sign in \cref{eq:quadcefluxseppos} is due to the specific
choice of the co-ordinate system used  for the positive electrode region. Due to
this,  fluxes  at  the  separator/positive  electrode  interface  have  opposing
directions.

Let  $Q_\text{e,j}$  denote  the  number  of moles  of  \ch{Li^+}  ions  in  the
electrolyte per  unit cross-sectional  area within each  region~$\jinnegseppos$.
This is  computed as  the product  of
\begin{enumerate*}[label=\emph{\alph*})]
    \item the porosity and
    \item spatial integral of the concentration function
\end{enumerate*}
\ie~${ Q_\text{e,j}  =  \varepsilon_j \int_0^{l_j}  c_{\text{e},j}(z) \,dz  }$.
Applying this to \crefrange{eq:cenquadreduced}{eq:cepquadreduced}
\begin{align}
    Q_\text{e,n} &= \varepsilon_\text{n} \left( \frac{1}{3} a_2 l^3_\text{n} + a_0 l_\text{n}\right)\label{eq:Qenbyintegration}\\
    Q_\text{e,s} &= \varepsilon_\text{s} \left( \frac{1}{3} a_5 l^3_\text{s} + \frac{1}{2} a_4 l^2_\text{s} + a_3 l_\text{s}\right)\\
    Q_\text{e,p} &= \varepsilon_\text{p} \left( \frac{1}{3} a_8 l^3_\text{p} + a_6 l_\text{p}\right) \label{eq:Qepbyintegration}
\end{align}

At this stage,  $Q_{\text{e},j}(t)$~are unknown. Since  these are time-dependent
functions,  the  derivation  naturally  progresses  towards  seeking  a  set  of
\glspl{ode} that describe a relationship  for their time evolution. We transform
the  second   order  \glspl{ode}  of \cref{eq:dfnliquiddiff}   (for  electrodes)
and \cref{eq:liquiddiffnsep}   (for  separator)   to   their  respective   local
co-ordinates and integrate  once along the thickness of  each region. Performing
this sequence of steps for the negative electrode region,
\mathleft
\begin{equation}
    \begin{WithArrows}[b]
        \varepsilon_\text{n} \int_0^{l_\text{n}} \left(\diffp*{c_\ensub(z,t)}{t}\right)\, dz &= \int_0^{l_\text{n}} \left(\diffp{}{z}\left(D_\effn \diffp{c_\ensub}{z} \right) + (1 - t^0_\text{+}) a_\snsub j_\text{n}\right)\, dz \Arrow[tikz={text width=3.4cm}]{transposing integration \& differentiation operations in the \glsfmtshort{lhs}} \\
        \varepsilon_\text{n} \diffp*{\int_0^{l_\text{n}} c_\ensub(z,t)}{t}\, dz &=
        \int_0^{l_\text{n}} \left(\diffp{}{z}\left(D_\effn \diffp{c_\ensub}{z}
        \right) + (1 - t^0_\text{+}) a_\snsub j_\text{n}\right)\, dz
        \Arrow[tikz={text width=3.4cm}]{apply time-derivative operator to the whole \glsfmtshort{lhs}}\\
        \diffp*{\left(\tikzmark{StartBraceA}\varepsilon_\text{n} \int_0^{l_\text{n}}
        c_\ensub(z,t)\, dz\tikzmark{EndBraceA}\right)}{t} &=  \int_0^{l_\text{n}}
        \left(\diffp{}{z}\left(D_\effn \diffp{c_\ensub}{z} \right) + (1 -
            t^0_\text{+}) a_\snsub j_\text{n}\right)\, dz \Arrow[tikz={text
        width=3.4cm}]{apply integral to the \glsfmtshort{rhs}}\\
        \diff*{Q_\text{e,n}(t)}{t} &= D_\effn \diffp{c_\ensub}{z}{\mathrlap{z=l_\text{n}}} + (1 - t^0_\text{+}) a_\snsub \int_0^{l_\text{n}} j_\text{n}\, dz
    \end{WithArrows}
    \label{eq:negliionmolestoreduce}
\end{equation}
\mathcenter
\InsertUnderBrace[draw=black][aspect=0.26]{StartBraceA}{EndBraceA}{} % https://tex.stackexchange.com/questions/68526/asymmetric-overbrace

% \blindtext
% \AddToShipoutPicture*{\ShowFramePicture}

Performing     the     identical     sequence     of     operations     starting
from~(\cref{eq:liquiddiffnsep}) for  the separator and~(\cref{eq:dfnliquiddiff})
for the positive electrode yields
\begin{align}
    \diff*{Q_\text{e,s}(t)}{t} &= D_\effs \diffp{c_\essub}{z}{\mathrlap{z=l_\text{s}}} \label{eq:sepliionmolestoreduce}\\
    \diff*{Q_\text{e,p}(t)}{t} &= D_\effp \diffp{c_\epsub}{z}{\mathrlap{z=l_\text{p}}} + (1 - t^0_\text{+}) a_\spsub \int_0^{l_\text{p}} j_\text{p}\, dz\label{eq:posliionmolestoreduce}
\end{align}

In    order    to   evaluate    the    integral    term   in    the    \gls{rhs}
of \cref{eq:negliionmolestoreduce}    and \cref{eq:posliionmolestoreduce},   the
solid  phase  charge  conservation  equation~(\cref{eq:solidchargeconserve})  is
integrated  along the  local  co-ordinate  axis of  the  negative electrode  and
positive electrode respectively.
\begin{align}
    \int_0^{l_\text{n}} j_\text{n}\, dz & =  \frac{I}{a_\snsub A F} \label{eq:negfluxintegral}\\
    \int_0^{l_\text{p}} j_\text{p}\, dz & =  \frac{-I}{a_\spsub A F}\label{eq:posfluxintegral}
\end{align}

Substituting       \crefrange{eq:negfluxintegral}{eq:posfluxintegral}       into
\crefrange{eq:negliionmolestoreduce}{eq:posliionmolestoreduce} respectively,
\begin{align}
    \allowdisplaybreaks
    \diff*{Q_\text{e,s}}{t} &= D_\effn \diffp{c_\ensub}{z}{\mathrlap{z=l_\text{n}}} - (1 - t^0_\text{+}) \cancel{a_\snsub} \frac{I}{\cancel{a_\snsub} A F} \\
    \diff*{Q_\text{e,p}}{t} &= D_\effp \diffp{c_\epsub}{z}{\mathrlap{z=l_\text{p}}} - (1 - t^0_\text{+}) \cancel{a_\spsub} \frac{-I}{\cancel{a_\spsub} A F}
\end{align}

which leads to the general expressions  for the cross-sectional molar density of
\ch{Li^+} ions in each of the three regions as
\begin{align}
    \diff*{Q_\text{e,n}}{t} & = D_\effn \diffp{c_\ensub}{z}{\mathrlap{z=l_\text{n}}} + (1 - t^0_\text{+}) \frac{I}{A F} \label{eq:negliionmolesgen} \\
    \diff*{Q_\text{e,s}}{t} & = D_\effs \diffp{c_\essub}{z}{\mathrlap{z=l_\text{s}}} \label{eq:sepliionmolesgen}                                             \\
    \diff*{Q_\text{e,p}}{t} & = D_\effp \diffp{c_\epsub}{z}{\mathrlap{z=l_\text{p}}} - (1 - t^0_\text{+}) \frac{I}{A F}\label{eq:posliionmolesgen}
    \intertext{Substituting the assumed quadratic expressions for electrolyte concentrations in
        each of the three regions \crefrange{eq:cenquadreduced}{eq:cepquadreduced}
    in the above system \ie~\crefrange{eq:negliionmolesgen}{eq:posliionmolesgen}}
    \diff*{Q_\text{e,n}}{t} & = 2 a_2 l_\text{n} D_\effn + (1 - t^0_\text{+}) \frac{I}{A F} \label{eq:negliionmolesquadratic}                                \\
    \diff*{Q_\text{e,s}}{t} & = 2 a_5 l_\text{s} D_\effs\label{eq:sepliionmolesquadratic}                                                                                                     \\
    \diff*{Q_\text{e,p}}{t} & = 2 a_8 l_\text{p} D_\effp - (1 - t^0_\text{+}) \frac{I}{A F} \label{eq:posliionmolesquadratic}
\end{align}
The  initial   ionic  concentration   in  the   electrolyte  are   identical  in
all  three  regions  of  the  cell,  assuming  equilibrium  starting  conditions
\ie~${c_{\text{e,0}_j}  = c_\text{e,0},  \jinnsp}$.  Hence the  initial number  of
moles of \ch{Li^+} per unit area in each of the three regions is given by
\begin{align}
    Q_\text{e,n}(0) & = \varepsilon_\text{n} c_\text{e,0} l_\text{n} \label{eq:Qeninit}\\
    Q_\text{e,s}(0) & = \varepsilon_\text{s} c_\text{e,0} l_\text{s}\\
    Q_\text{e,p}(0) & = \varepsilon_\text{p} c_\text{e,0} l_\text{p} \label{eq:Qepinit}\\
    \shortintertext{and the initial coefficient vector which satisfies the system
    equations is obtained as}
    \begin{bmatrix}
        a_0 \\
        a_2 \\
        a_3 \\
        a_4 \\
        a_5 \\
        a_6 \\
        a_8
        \end{bmatrix} & = \begin{bmatrix}
        c_\text{e,0} \\
        0 \\
        c_\text{e,0} \\
        0 \\
        0 \\
        c_\text{e,0} \\
        0
    \end{bmatrix} \label{eq:coeffinit}
\end{align}

The             system             of             three             \glspl{ode},
\eqref{eq:negliionmolesquadratic}--\eqref{eq:posliionmolesquadratic}    together
with  \crefrange{eq:Qeninit}{eq:Qepinit}  representing the  initial  conditions,
form   an    \gls{ivp}.   \Crefrange{eq:cecontinuitynegsep}{eq:Qepbyintegration}
represent a square system of seven linear algebraic equations with seven unknown
coefficients which must be solved at each time-step. These algebraic constraints
coupled with the aforementioned \gls{ivp} form a \gls{dae} system.

There are now two choices for proceeding  with solution of the system. The naive
approach  would be  to  solve  the \gls{dae}  using  advanced \gls{dae}  solvers
specially designed to handle \mbox{index-1}  semi-explicit systems such as DASSL
or  DASPK. For  start-stop  type  of input  currents  with discontinuities,  the
consistent initialisation of algebraic conditions and derivatives is numerically
challenging.  All   \gls{dae}  solvers  typically  use   adaptive  time-stepping
algorithms.  The  feasibility of  using  such  a  complex scheme  for  real-time
computation is questionable. On the other hand, the overall system can be viewed
as composed of two numerical subsystems ---
\begin{enumerate*}[label=\emph{\alph*})]
    \item an independent \gls{ode} system, and
    \item an independent algebraic system
\end{enumerate*}
with both these  systems running back to back in  succession using solution from
the previous time-step.

To  clarify  the sequence  of  operations,  in  order  to bootstrap  the  model,
it  is  required  to  compute  $Q_{\text{e},j}(t)$ in  all  three  regions.  The
\gls{ode} system is  integrated for one time-step by  retaining the coefficients
at  their initial  value.  The $Q_{\text{e},j}(t)$  thus  solved is  substituted
into  the  algebraic system  to  yield  the  updated  value of  the  coefficient
vector~$\vec{a}(t=t_k)$. This new value of the coefficient vector is substituted
back  into  the  \gls{ode}  system  and  the  process  continues.  Although  the
continuous simulation of  the overall \gls{dae} is not possible,  this scheme is
pragmatic from an engineering viewpoint as  the periodic pauses needed to update
the intertwined  sub-systems translate  naturally into  fixed time-steps  and is
well-suited for a \gls{bms} controller operating at a fixed sample rate. This is
also an effective workaround to mitigate the complexities of having to implement
and solve \glspl{dae} in real time.

% need to write algorithm
The   simulation   results   of   the   quadratic   approximation   scheme   and
the   analysis   of   its   strengths   and   weaknesses   is   presented   next
in \cref{subsec:quadraticsimresultsanalysis}.

\subsection{Numerical implementation, simulation results and analysis}\label{subsec:quadraticsimresultsanalysis}

\subsubsection*{Numerical implementation}
From an analysis point of view,  the quadratic approximation model for computing
the spatio-temporal evolution  of electrolyte concentration can  be simulated as
an  independent  subsystem,  and  hence  can be  implemented  numerically  as  a
standalone module as shown  in \cref{alg:quadraticce}. In practice, this modular
code is  embedded as  a subroutine  within the  main \gls{spm}  algorithm listed
in \cref{alg:disctimespm}.

% -*- root: ../main.tex -*-
%!TEX root = ../main.tex
% this file is called up by main.tex
% content in this file will be fed into the main document
% vim:nospell

\begin{algorithm}[!htbp]
    \caption{Quadratic approximation model for spatio-temporal electrolyte concentration}\label{alg:quadraticce}
    \begin{algorithmic}[1]
        \Require Load profile \Comment{\eg~a \texttt{csv} file of $t$ vs.\ C-rate}
        \Require Electrolyte model parameter set  \Comment{\eg~stored in a struct \texttt{ceparams}}
        \Userdata $ t_\text{f}$,  sample rate $T_s, c_\text{e,init}$
        \Function{QuadraticElectrolyteModel}{}
        \State Set $Q_{\text{e,init}_j}$ as per \crefrange{eq:Qeninit}{eq:Qepinit}
        \State $\vec{a}[1]$ \gets values from \cref{eq:coeffinit}
        \State $V_\text{cell}[1] \gets \textsc{ComputeCellVoltage}(\textbf{x}[1],I[1],\texttt{params})$ \Comment{from direct feedthrough}
        \For{$k \gets 2 : N_\text{max}$}
        \State $I[k] \gets $ interpolate from profile using \gls{zoh}
        \State Solve continuous-time state equations--\crefrange{eq:negliionmolesquadratic}{eq:posliionmolesquadratic} \Comment{Using $\vec{a}[k-1]$}
        \State $Q_{\text{e}_j}[k]$ \gets last time-entry  vector of soln.\  matrix \Comment{if using an adaptive solver}
        \State $\vec{a}[k]$ \gets solution of \emph{linear} system of equations--\crefrange{eq:cecontinuitynegsep}{eq:Qepbyintegration} \Comment{Using $Q_{\text{e}_j}[k-1]$}
        \State Compute $c_{\text{e}_j}$ as per \crefrange{eq:cenquadreduced}{eq:cepquadreduced} \Comment{Quadratic polynomial expressions for concentration}
        \EndFor
        \EndFunction
    \end{algorithmic}
\end{algorithm}


\begin{figure}[!htb]
    \centering
    \includegraphics[width=\textwidth]{quadratic_ce_approx_spatial_1C.pdf}
    \caption[Spatial distribution of electrolyte concentration for 1C~discharge]{Spatial distribution of ionic concentration in electrolyte along
        cell thickness at various snapshots of time for a 1C~discharge. The
        concentration profile obtained from simulating the \glsfmtshort{p2d}
        model is used as the reference. The performance of the quadratic model
        is quite poor during the initial transient duration, but improves over
    time as a quasi-steady state is reached.}
    \label{fig:spatialionicconc1C}
\end{figure}


\subsubsection*{Simulation results}\label{subsubsec:simresultsbaselinequad}

\Cref{fig:spatialionicconc1C} shows  the spatial distribution of  \ch{Li^+} ions
in electrolyte  along the  thickness of  the cell at  various snapshots  of time
obtained by simulation  of the \gls{p2d} and the  quadratic approximation models
using a  1C~discharge current. The  \gls{p2d} model's response is  considered as
the reference benchmark.  During the initial transient  phase, the concentration
profile within each electrode region exhibits a characteristic inflection point.
During this phase,  the concentration profile computed by  the parabolic profile
exhibits  a  large deviation  in  terms  of  percentage  error at  each  spatial
location. However, with the passage of time, as a \gls{qss}~is established, this
inflection point flattens out, and the quadratic approximation becomes closer to
the  true  concentration value  at  each  spatial  location. Similar  trends  in
behaviour is  exhibited for discharge and  charging at higher C-rates  and these
results are  therefore omitted here  in the  interest of keeping  the discussion
succinct.

It  is important  to  note that  while having  a  spatial concentration  profile
is  useful,  as  seen  in \cref{eq:electrolytepdwithce}, it  is  the  values  of
concentration  at   the  \emph{current  collector  interfaces}   that  are  most
influential in  computation of the  electrolyte overpotential and hence,  in the
voltage accuracy of the enhanced \gls{spm}. Thus, it is important to obtain this
alternative perspective  of time-evolution  of the electrolyte  concentration at
the two current collectors.

\begin{figure}[!htbp]
    \centering
    \includegraphics[width=\textwidth]{ce_at_cc_1Cdischg.pdf}
    \caption[Ionic concentrations at current collector
    interfaces over time for 1C~discharge]{Evolution of ionic concentration over
        time at the two current collector interfaces for a 1C~discharge (top
        row). The time evolution of the corresponding error variables is also
    shown (bottom row).}
    \label{fig:temporalcequadratic}
\end{figure}

\Cref{fig:temporalcequadratic} shows  the time evolution of  ionic concentration
in  the electrolyte  at the  two current  collector interfaces  computed by  the
\gls{p2d} and  quadratic approximation  models for a  1C~discharge  current. The
concentration  values computed  by  the  \gls{p2d} model  is  considered as  the
reference benchmark. At the negative electrode--current collector interface, the
ionic  concentration  exhibits  a  few oscillations  of  small  amplitude  owing
to  the complex  interactions  of  the ionic  phase  with  the porous  electrode
and  the charge  transfer process  at the  electrode-electrolyte interface.  The
concentration  evolution  predicted  by  the quadratic  approximation  model  is
rather  simplistic  and  is  unable to  capture  this  intricate  time-evolution
pattern.  This  is because  the  governing  equation predicting  time  evolution
of  concentration  in  the  quadratic  approximation  model  is  that  given  by
the  \emph{first-order}  \gls{ode} of \cref{eq:negliionmolesquadratic}  (with  a
proportional mapping from~$Q_\text{e,n}$ to~$c_\text{e}(0,t)$). Following system
theory, the step response  of a first order \gls{ode} is  that of an exponential
rise to a final settling value, which  is exactly the shape seen in the top-left
plot of \cref{fig:temporalcequadratic}.

The ionic  concentration evolution  at the positive  current collector  does not
exhibit major oscillations  and even has a subtle monotonicity  to its response.
However,  the  classical  first  order  response  predicated  by  the  quadratic
approximation model falls short of representing its complete dynamics. Errors of
similar  magnitude  are present  in  the  ionic  concentration at  both  current
collector  interfaces, with  a maximum  absolute error  of \approx\SI{250}{\mole
\per \meter  \cubed}. Similar trends are  exhibited in charging, with  a swapped
response pattern observed at the two current collector interfaces.

This  chapter concludes  with a  view that,  while the  spatial accuracy  of the
quadratic  approximation  model  is  acceptable,  its  temporal  performance  is
lacklustre. The causal factors for this  mediocre performance is analysed in the
context of the author's own attempts  in arriving at an enhanced electrolyte and
is presented next in \cref{ch:newelectrolytemodel}.

