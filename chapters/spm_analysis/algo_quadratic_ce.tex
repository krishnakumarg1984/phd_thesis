% -*- root: ../main.tex -*-
%!TEX root = ../main.tex
% this file is called up by main.tex
% content in this file will be fed into the main document
% vim:nospell

\begin{algorithm}[!htbp]
    \caption{Quadratic approximation model for spatio-temporal electrolyte concentration}\label{alg:quadraticce}
    \addtocontents{loa}{\vskip 9pt} % https://latex.org/forum/viewtopic.php?t=6218
    \begin{algorithmic}[1]
        \Require Load profile \Comment{\eg~a \texttt{csv} file of $t$ vs.\ C-rate}
        \Require Electrolyte model parameter set  \Comment{\eg~stored in a struct \texttt{ceparams}}
        \Userdata $ t_\text{f}$,  sample rate $T_s, c_\text{e,init}$
        \Function{QuadraticElectrolyteModel}{}
        \State Set $Q_{\text{e,init}_j}$ as per \crefrange{eq:Qeninit}{eq:Qepinit}
        \State $\vec{a}[1]$ \gets values from \cref{eq:coeffinit}
        \State $V_\text{cell}[1] \gets \textsc{ComputeCellVoltage}(\textbf{x}[1],I[1],\texttt{params})$ \Comment{from direct feedthrough}
        \For{$k \gets 2 : N_\text{max}$}
        \State $I[k] \gets $ interpolate from profile using \gls{zoh}
        \State Solve continuous-time state equations--\crefrange{eq:negliionmolesquadratic}{eq:posliionmolesquadratic} \Comment{Using $\vec{a}[k-1]$}
        \State $Q_{\text{e}_j}[k]$ \gets last time-entry  vector of soln.\  matrix \Comment{if using an adaptive solver}
        \State $\vec{a}[k]$ \gets solution of \emph{linear} system of equations--\crefrange{eq:cecontinuitynegsep}{eq:Qepbyintegration} \Comment{Using $Q_{\text{e}_j}[k-1]$}
        \State Compute $c_{\text{e}_j}$ as per \crefrange{eq:cenquadreduced}{eq:cepquadreduced} \Comment{Quadratic polynomial expressions for concentration}
        \EndFor
        \EndFunction
    \end{algorithmic}
\end{algorithm}
