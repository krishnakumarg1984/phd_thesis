% -*- root: ../../main.tex -*-
%!TEX root = ../../main.tex
% this file is called up by main.tex
% content in this file will be fed into the main document
% vim:textwidth=80 fo=cqt

% The state of the art implementation  for tackling these drawbacks is presented
% and their inadequacies are discussed.

\capolettera{T}{aking} into consideration the  relative strengths and weaknesses
of  all the  physics-based reduced  order modelling  families in  the literature
considered  (see  \cref{ch:littreview}),  the   overarching  simplicity  of  the
\gls{spm}  coupled  with   its  immediate  potential  to  bring   the  power  of
physics-based predictions to  an embedded environment is a  strong motivation to
pursue  an in-depth  exploration of  its  horizons. This  chapter discusses  the
performance of multiple \gls{spm} variants, ranging from basic to sophisticated.
The governing equations  of the conventional \gls{spm} are  first introduced and
its baseline performance  is evaluated. Next, an in-depth analysis  of the basic
\gls{spm}'s drawbacks is performed. Various published attempts to mitigate their
current challenges towards implementability  is presented and their inadequacies
discussed. Owing  to its  simplicity and latent  potential, a  popular modelling
strategy from the existing art that aims to enhance the performance of the basic
\gls{spm} through incorporation  of electrolyte dynamics, is  chosen for further
analysis.  The chapter  concludes with  the  author's critical  comments on  its
relative merits that come to view  through comparison against a benchmark model,
whilst simultaneously  identifying its critical weakness  whose mitigation forms
the focal point of the research discussed in \cref{ch:newelectrolytemodel}.

