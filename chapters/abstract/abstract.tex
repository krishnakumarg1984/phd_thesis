% satisfying specific acceleration and fast charging targets.
% that could potentially facilitate immediate adoption by
% industry.

% The  proposed methodology  accounts for  the  critical need  to avoid  lithium
% plating during fast charging and  searches for the optimal layer configuration
% considering a  range of  thermal conditions. A  numerical implementation  of a
% cell model using a hybrid  finite volume-spectral scheme is presented, wherein
% the model equations are suitably reformulated to directly accept power inputs,
% facilitating rapid and  accurate searching of the layer design  space. We show
% how thermal management design can limit vehicle driving range at high charging
% temperatures. We highlight how  electrode materials exhibiting increased solid
% phase  diffusion  rates  are  as  equally  important  for  extended  range  as
% developing new  materials with higher  inherent capacity. We illustrate  for a
% plug-in  hybrid  vehicle,  how  the proposed  methodology  facilitates  common
% module  design of  battery  packs,  thereby reducing  the  cost of  derivative
% vehicle models. To  facilitate model based layer optimisation,  we provide the
% open-source toolbox, BOLD (Battery Optimal Layer Design).


