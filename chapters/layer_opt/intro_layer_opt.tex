% -*- root: ../../main.tex -*
%!TEX root = ../../main.tex
% vim:textwidth=80 fo=cqt conceallevel=0


\capolettera{T}{he} issue of `range anxiety'  is a pervasive mental blockade for
potential buyers  of electric  vehicles which  in-turn hampers  their widespread
adoption. From  a consumer viewpoint,  yet another  practical issue is  the fact
that on  encountering a `low battery'  scenario in a long  distance journey, the
charging  times required  for sufficiently  replenishing the  battery to  enable
completion  of the  journey  are  prohibitively large,  to  the  point of  being
non-competitive against conventional fossil fuel powered vehicles.

Unfortunately,  the  aforementioned  scenarios  are not  unimaginable  with  the
present  state  of the  art  in  lithium  ion  batteries. Hence,  improving  the
\gls{aer} and  providing fast charging  capabilities are two near-term  goals of
manufacturers  of electric  vehicles.  Increasing the  \gls{aer} necessitates  a
battery pack with  higher energy content in it while  lowering the charging time
demands a  pack with higher  power capability.  The contrasting nature  of these
goals can  be traced  all the way  down to  the cell level  and is  presented in
\cref{sec:energypowertradeoff}. By trading  off the number of layers  in a pouch
cell  against  the content  of  active  electrode material  accommodated  within
it,  bespoke cell  designs  addressing either  the energy  demand  or the  power
demand  can  be  obtained.  In  the  absence  of  accessible  documentation  (as
either  industry white  papers or  academic literature)  on the  layer selection
methodologies employed in automotive pouch  cell designs, this author postulates
that manufacturers  iterate through  an extensive  empirical testing  process of
prototypes with  a range of  layer choices. In the  view of this  thesis author,
this  procedure is  not only  time-consuming, but  is also  likely to  result in
sub-optimal designs.  This chapter envisages a  model-based engineering solution
to more  optimal cell designs  by determining  the appropriate number  of layers
needed  to maximise  its  \emph{usable} energy  while simultaneously  satisfying
certain  power  capability constraints.  The  rest  of  the chapter  provides  a
detailed treatment of topics such  as the proposed layer optimisation framework,
its assumptions involved,  and various modifications to  standard numerical code
required to facilitate this design procedure.

