% -*- root: ../../main.tex -*
%!TEX root = ../../main.tex
% vim:textwidth=80 fo=cqt conceallevel=0

% \afterpage{\clearpage}

The  methodology adopted  by the  proposed layer  optimisation framework  can be
explained by  using the  flow diagram in  \cref{fig:fig_strategy_schematic}. The
power  handled by  the cell  during normal  operation (evaluated  by considering
various  drivecycles)  is   much  lower  than  that  experienced   by  the  cell
during  acceleration  (discharge)  and  fast-charging  (charge)  scenarios.  See
\cref{sec:resultslayeropt} for  a brief  summary of the  peak and  median powers
across all  standard drivecycles.  Therefore, from a  design perspective,  it is
sufficient  to consider  the power  requirements  for these  two extreme  cases.
Hence,  the schematic  in  \cref{fig:fig_strategy_schematic} can  be studied  by
broadly dividing the flow diagram into two parts ---
\begin{enumerate*}[label=\roman*)]
    \item an acceleration pathway primarily consisting of blocks shaded in grey, and
    \item a fast-charging pathway predominantly composed of blocks shaded in cyan.
\end{enumerate*}

\begin{figure}[p]
    \begin{minipage}[t]{\textwidth}
        \centering
        \includegraphics[angle=90, width=\textwidth]{fig_master_flow_diagram}
        \captionsetup{labelsep=note}
        \caption
        [%
        Flow diagram depicting an overview of the proposed layer optimisation methodology
        for Li-ion pouch cells.
        ]%
        {%
            Flow diagram depicting an overview of the proposed layer optimisation methodology
            for Li-ion pouch cells.
        }%
        \label{fig:fig_strategy_schematic}
        \mpfootnotes[1]
        % \vspace*{1.125cm}
        \vspace*{0.7225cm}
        \footnote{This figure was created by \mbox{Krishnakumar Gopalakrishnan} who
            asserts copyright, with intellectual contributions from and the right to
        use asserted by \mbox{Ian Campbell}.}
    \end{minipage}
\end{figure}

As indicated in the legend  for \cref{fig:fig_strategy_schematic}, blocks with a
light~grey border represent input  data/parameters for computations. Blocks with
a standard black  border represent computations common to  both acceleration and
fast charging pathways. The design output  is given by the double-bordered block
at the bottom centre of \cref{fig:fig_strategy_schematic}. Other types of blocks
and arrows are appropriately listed in  the legend key. To aid the understanding
of the layer  optimisation framework, the reader is encouraged  to correlate the
narrative in this section with the blocks and arrows in the schematic.



\subsection{Acceleration pathway}

The computations for acceleration-based layer  optimisation begins at the anchor
block labelled `Start Acc.\ Calcs.'.

\subsubsection*{Determination of acceleration rate, final speed and acceleration time}

The first step is  the determination of the rate of acceleration  to be used for
the  computing  the  power  requirements for  accelerating  the  \gls{xeV}  from
standstill.  For the  sake  of  claiming green  subsidies  and other  regulatory
reasons,  vehicular   standards  for  electrified  transport   are  codified  by
various  standardisation bodies  (\eg{} the  SAE~J1772 standard~\cite{Sae2010}).
The  standards  published  by  these regulatory  agencies  typically  specify  a
minimum required  acceleration rate  for the vehicle  under consideration  to be
certified as a road-faring electric vehicle. Vehicle manufacturers often provide
their  own  specifications,  which  typically exceed  these  minimum  standards.
However,  for certain  classes  of electric  vehicles such  as  golf carts,  the
manufacturer-specified standards might fall below  that of a roadworthy electric
vehicle. This thesis  advocates a conservative design by choosing  the higher of
the two values.

The  acceleration  rate  is  calculated   by  dividing  a  pre-determined  final
speed~$v_\text{f}$  by the  time~$t_\text{f}$ taken  to attain  that speed  from
standstill.  The  manufacturer-specified  acceleration  rate~$a_\text{man.}$  is
compared  against  the minimum  acceleration  rate  specified by  the  governing
vehicular  standards~$a_\text{std.}$. The  two  \gls{spdt}  switches assign  the
final speed~$v_\text{f}$ and acceleration time~$t_\text{f}$ to the corresponding
values from  the appropriate source depending  on which of the  two acceleration
rates is higher.

\subsubsection*{Computation of acceleration power at the wheels}

The  next  step  is  to  calculate   the  acceleration  power  required  at  the
driving wheels~$P_\text{w}$.  Using the  governing equations from  basic vehicle
dynamics~\cite{Maksimovic2012}, the power at the wheels is given by
\cref{eq:wheelpower}.
\begin{subequations}\label{eq:wheelpower}
    \begin{align}
        P_\text{w}     & = P_\text{mass} + P_\text{drag} + P_\text{roll} + P_\text{grade} \tag{\ref*{eq:wheelpower}}                  \\
        P_\text{mass}  & = \frac{1}{2} \frac{M_\text{v}(n)}{t_\text{a}} \left(v_\text{b}^2 + v_\text{f}^2\right) \label{eq:masspower} \\
        P_\text{drag}  & = \frac{1}{2} \left(\rho_\text{air} C_\text{d} A_\text{v}v_\text{f}^3\right) \label{eq:dragpower}            \\
        P_\text{roll}  & = C_\text{r} M_\text{v}(n) g v_\text{f} \label{eq:rollpower}                                                 \\
        P_\text{grade} & = M_\text{v}(n) Z g v_\text{f} \label{eq:gradepower}
    \end{align}
\end{subequations}

The  individual  components  contributing   to  the  summation  for  wheel-power
computation   presented   in    \cref{eq:wheelpower}   is   briefly   explained.
$P_\text{mass}$~represents  the  power  required  to  accelerate  the  vehicle's
mass.   $P_\text{drag}$~and  $P_\text{roll}$~denote   the  powers   required  to
overcome   air  resistance   and  rolling   resistance  respectively.   Finally,
$P_\text{grade}$~represents   the   power   required   to   negotiate   a   road
gradient.  Except  for  $M_\text{v}(n)$  which  is  described  next,  all  terms
in  the  \gls{rhs}  of  \crefrange{eq:masspower}{eq:gradepower}  are  constants.
Each  of  these  terms   is  explained  in  tables~\ref{tbl:CommonVehicleParams}
and~\ref{tbl:UniqueVehicleParams}  along with  their  numerical  values used  in
simulation.

\subsubsection*{Computation of layer-dependent vehicle mass}

Changing  the layer  count used  within a  cell changes  its mass.  This in-turn
affects  the mass  of the  pack, which  further influences  the overall  vehicle
mass. Hence, for precise  computation of~$P_\text{mass}$ in \cref{eq:masspower},
$M_\text{v}(n)$  is the  vehicle's  mass computed  as a  function  of number  of
layers~$n$.

In  the  schematic  of~\cref{fig:fig_strategy_schematic},  this  layer-dependent
calculation of  mass of all cells  in the pack~$M_\text{cells}$ is  shown by the
product of  the signal  labelled $m_\text{cell}$ and  the triangular  gain block
representing the overall number of cells in the pack~$n_\text{cells}$.
\begin{equation}\label{eq:massofallcells}
    M_\text{cells} = n \times m_\text{cell}
\end{equation}

The  mass of  the vehicle  is  given by  the sum  of chassis  mass~$M_\text{c}$,
vehicle payload $M_\text{p}$, pack overhead $M_\text{o}$ and the layer-dependent
mass   of   all   cells   in   the  pack   $M_\text{cells}$   as   computed   in
\cref{eq:massofallcells}.   The   computation  of   mass   of   a  single   cell
$m_\text{cell}$ is detailed in \cref{sec:massofonecell}.

\subsubsection*{Computation of acceleration power density per layer}

Since the \gls{p2d} equations of the \gls{dfn} model are based upon a normalised
unit area and is  applicable only to each electrochemical layer,  the goal is to
compute the  power density experienced  by each layer. This  is arrived at  by a
sequence of simple scaling steps.

Firstly, the  power demanded at  the pack terminals  is computed by  scaling the
power  at the  wheels  by the  efficiency  of the  drivetrain.  As explained  in
\cref{subsec:layeroptassumptions},  the  drivetrain  consists  of  a  number  of
components, the efficiencies of which depend on the operating point (such as the
speed and torque  of the electric motor, current drawn  by the power electronics
etc.). Following the  assumptions detailed in \cref{subsec:layeroptassumptions},
a  single lumped  efficiency~$\eta_\text{dt}$ can  be used  for the  powertrain.
The  power demand  on  the pack~$P^\text{acc}_\text{batt}$  is  then divided  by
the  number  of  cells  in  the   pack  $n_\text{cells}$  to  obtain  the  power
demand  at the  input  of the  cell's terminals~$p^\text{acc}_\text{cell}$.  For
each  layer choice~$n$,  the overall  electrochemically active  surface area  is
computed  using by  using \cref{eq:overallarea},  wherein the  surface area  per
layer~$A_\text{elec}$ listed in \cref{tbl:lcoSimParamslayeropt} is held constant
throughout.  Finally, the  acceleration power  density per  layer~$p^\text{acc}$
is  computed  by  dividing  $p^\text{acc}_\text{cell}$ by  the  overall  surface
area~$A_\text{cell}$.

\subsubsection*{Thermally-coupled \glsfmtshort{p2d} simulation and exit
conditions}\label{sec:accexitconditions}

For the  present layer choice~$n$,  a thermally-coupled \gls{p2d}  simulation is
performed for a duration of $t_\text{f}$  seconds with the applied power density
$p^\text{acc}_\text{cell}$ corresponding to the present layer choice as input.
When the simulation terminates, the cell's condition is compared against three
criteria
\begin{enumerate}
    \item the maximum possible values of cell temperature~$T_\text{max}$,
    \item the minimum allowed terminal voltage~$V_\text{min}$, and
    \item the lowest allowed cell \gls{soc}~$z_ztext{min}$.
\end{enumerate}
This helps to determine whether the cell constructed from the present layer
choice is able to successfully satisfy the acceleration power demands. These
comparison operations are represented by decision blocks placed in the leftmost
region of the schematic in~\cref{fig:fig_strategy_schematic}.

If  any one  of the  three aforementioned  exit checks  fail, the  present layer
configuration is deemed  to be not feasible and the  entire workflow is repeated
by  trialling a  new  layer  choice. A  sophisticated  search  algorithm, to  be
described  in~\cref{sec:searchalgo},  is  employed  to minimise  the  number  of
iterations needed  until a successful layer  choice~$n_\text{opt}^\text{acc}$ is
obtained. The  above process is  repeated for different combinations  of initial
and  ambient temperatures~\mbox{$(T_\text{init},  T_\text{sink})$}. The  largest
successful  layer value  from  all  temperature combinations  is  deemed as  the
canonical  optimal design  choice  when considering  acceleration demands.  This
concludes the narrative describing the acceleration-specific pathway.

\subsection{Search algorithm}\label{sec:searchalgo}

A  customised binary  search algorithm  is  employed in  the layer  optimisation
framework. The  binary search  is a  computationally efficient  search algorithm
requiring a worst-case operational count of~$\mathcal{O}(k \log k)$ where $k$ is
the  overall number  of  layer candidates  to be  searched.  From a  pedagogical
viewpoint,  the  method  is  briefly  described  here  in  the  context  of  the
aforementioned acceleration runs.  However, the same search  strategy is equally
applicable for both acceleration and fast charging pathways.

If all the  termination criteria are successfully met, the  output of the binary
search algorithm is  set to a logical~1.  If any of these  conditions fail, then
the output  is assigned  a logical~0.  Thus, the search  vector in  this bespoke
strategy  consists  of only  two  possibilities,~0  and~1. For  ultra-low  layer
counts, the  applied power densities are  extremely high. Therefore, one  of the
exit  conditions is  likely  to fail.  For  very high  layer  counts, the  power
densities are  very low,  implying that acceleration  runs shall  be successful.
There exists a critical transition point  \viz{} the first layer count for which
the exit vector toggles from~0 to~1. Through a systematic bisection of the layer
search  space, the  search algorithm  speedily  converges to  the optimal  layer
value.  An alternative  to using  the binary  search algorithm  is the  standard
linear  search  algorithm. This  is  a  simple  search strategy  which  involves
sequentially iterating  from $n_\text{min}$ to $n_\text{max}$  until arriving at
the lowest  value $n_\text{opt}^\text{acc}$  that satisfies all  the termination
criteria. However, a naive use of the linear search algorithm is computationally
expensive with a worst case operation count of $\mathcal{O}(n)$. Therefore, this
thesis  author  recommends  the  use  of the  bespoke  binary  search  algorithm
presented here. The choice of minimum and  maximum values of the search space is
discussed in \cref{sec:layersearchbounds}.

\subsection{Upper and lower bounds on search space}\label{sec:layersearchbounds}


It is  helpful to determine  the highest possible number  of layers that  can be
physically  accommodated in  a stack  of height~$L_\text{stack}$.  For instance,
this value may be used as the  upper bound to the binary search algorithm (which
compulsorily requires  such an upper bound)  described in \cref{sec:searchalgo}.
This can be obtained by defining a  simple integer optimisation task as shown in
\cref{eq:maxlayersopt}. The objective function here  is to maximise the value of
the layer  count~$n$ subject to  the physical  constraint that the  thickness of
negative and positives electrodes remains positive.
\begin{equation}
    \begin{aligned}
        \underset{n \, \in \, \mathbb{N}}{\mathbf{max}} \quad & n                                                                                                                                              \\
        \text{s.t.} \quad                                     & l_\text{pos} = \left(\frac{L_\text{stack} - L_\text{Al}(n) - L_\text{Cu}(n) - n l_\text{sep}}{n(1 + l_\text{ratio})}\right) > 0 \\
                                                              & l_\text{neg} = \left(\frac{l_\text{ratio}(L_\text{stack} - L_\text{Al}(n) - L_\text{Cu}(n) - n
l_\text{sep})}{n(1 + l_\text{ratio})}\right) > 0
\end{aligned}
\end{equation}


\Cref{eq:analytical_nmax} represents the analytical  closed-form solution to the
integer optimisation  problem of  \cref{eq:maxlayersopt}. The first  argument to
the $\max$ function  in \cref{eq:maxlayersopt} represents the  maximum number of
physically feasible odd layers while  the second argument represents the maximum
such value  when considering  odd layers.  Therefore, the  greater of  these two
values is chosen  as the maximum permissible layer  choice~$n_\text{max}$ and is
used as an input to the search algorithm.

\begin{equation}
    \label{eq:analytical_nmax}
    n_\text{max} = \max \left(\floor*{\frac{2\left(L_\text{stack} - l_\text{Cu}\right)}{l_\text{Al}+l_\text{Cu}+2 l_\text{sep}}}, \floor*{\frac{2L_\text{stack} - l_\text{Al} - l_\text{Cu}}{l_\text{Al} + l_\text{Cu} + 2l_\text{sep}}} \right)
\end{equation}

In the absence of  published information on how minimum layers  is set, there is
currently no unique  way to determine $n_\text{min}$. Therefore,  for this layer
optimisation study, the value of $n_\text{min}$  used by the search algorithm is
currently  set to  one ---  the minimum  physically feasible  number of  layers.
Design engineers from industry may opt to  tweak this to a higher value based on
insights gained from present empirical designs.


% \subsubsection*{\hypertarget{href:layerdependentcellmass}{Computation of layer-dependent cell mass}}\label{sec:massofonecell}
\subsection{Computation of layer-dependent cell mass}\label{sec:massofonecell}

\subsection{Electrode thickness ratio for capacity balancing}\label{sec:electroderatio}

A key  idea of the layer  optimisation scheme is that  for computing the
physical  lengths  of electrode  as  a  function  of number  of  layers,
the  ratios  of  their thicknesses~$l_\text{ratio}$  is  held  constant.
This  coefficient is  germane to  the concept  of capacity  balancing of
electrodes  to equalise  their loading and is computed as follows.

Equating  the active material volume of both electrodes,
\begin{equation}
    A_\text{elec,pos}l_\text{pos}  \varepsilon_\text{s,pos} = A_\text{elec,neg}l_\text{neg}  \varepsilon_\text{s,neg} \label{eq:electrodeCapacity}\\
\end{equation}

Neglecting  overhangs   of  the  negative  electrode   (typically  below
$\SI{2}{\milli\meter}$ to  avoid plating  at the edges),  both electrode
materials have the same cross-sectional area~$A_\text{elec}$. Therefore,
\cref{eq:electrodeCapacity} reduces to
\begin{align}
    \cancel{A_\text{elec}}l_\text{pos}  \varepsilon_\text{s,pos} & = \cancel{A_\text{elec}}l_\text{neg}  \varepsilon_\text{s,neg}  \\
    l_\text{pos}  \varepsilon_\text{s,pos}                       & = l_\text{neg}  \varepsilon_\text{s,neg}\label{eq:electrodeCapequalarea}
\end{align}

For the reference cell under consideration, the length and breadth of the cell's
pouch is  obtained from the \gls{bev}  manufacturer~\cite{GMBoltBatteryDims} and
are listed in \cref{tbl:lcoSimParamslayeropt}.

% % double-check if electrode area or overall area or what ? And uncomment later
% on


% As  a  first-order  approximation  the
% product of  these dimensions can  be assumed  to be the  active material
% cross-sectional  area   (although  in  practice,  the   planar  area  of
% the  stack  needs to  be  slightly  smaller  to be  accommodated  inside
% the  pouch).  Finally,  in  line   with  the  assumptions  discussed  in
% \cref{subsec:layeroptassumptions},  similar  to  the pouch  height,  the
% cross-sectional geometry (length  and breadth) of the cell  is also held
% constant throughout the layer optimisation process.

Owing to the reasons  outlined in \cref{subsec:layeroptassumptions}, the
volume fractions of the electrode  materials are assumed to be constant,
which  implies  that their  ratio  is  also  a constant.  The  electrode
thickness ratio~$l_\text{ratio}$ is obtained as
\begin{alignat}{2}
    l_\text{ratio} & = \frac{l_\text{neg}}{l_\text{pos}}                                                                                  &  & \qquad\text{(by definition)}                                          \\
    {}             & = \frac{\varepsilon_\text{s,pos}}{\varepsilon_\text{s,neg}}                                                          &  & \qquad\text{(rearranging \cref{eq:electrodeCapequalarea})}           \\
    {}             & = \frac{1-\varepsilon_\text{pos} - \varepsilon_\text{fi,pos}}{1-\varepsilon_\text{neg} - \varepsilon_\text{fi,neg}}  &  & \qquad\text{(by definition)}                                          \\
    {}             & = \frac{1- 0.385 - 0.025}{1 - 0.485 - 0.033}                                                                         &  & \qquad\text{(substituting values from \cref{tbl:lcoSimParamsSPMp2d})} \\
    l_\text{ratio} & = 1.22\label{eq:electrodeThicknessRatio}
\end{alignat}

% Do not forget to quote the value of cs_sat from layer opt paper table and show
% calculation inline. Might even go into the code

\subsection{Power inputs considered}

P2D model  equations are developed  for bi-directional input power.  xEV battery
pack has  constant power input,  if the  installed power capability  of charging
stations  is  fully  utilised.  Equation  (1)  describes  vehicle  dynamics  for
acceleration  and  power  demand.  For  fast  charging,  the  power  electronics
components of all grid chargers possess  a peak power delivery rating.
