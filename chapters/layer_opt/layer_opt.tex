% -*- root: ../../main.tex -*
%!TEX root = ../../main.tex
% vim:textwidth=80 fo=cqt conceallevel=0

\newpage
\thispagestyle{plain} % empty
\mbox{}
\graphicspath{{chapters/layer_opt/figures/}}
% ----------------------- contents from here ------------------------

\chapter[Model-based Design of Pouch Cells]{%
    Model-based  Design Of  Pouch Cells\footnote{\textbf{Attribution  of content}  \
        The  groundwork  for converting  the  existing  computer code  (LIONSIMBA~v1.0x)
        into  a suitable  form  for  layer optimisation  was  initiated  by this  thesis
        author, \mbox{Krishnakumar  Gopalakrishnan}. However, with the  exception of the
        spectral scheme,  for which \mbox{Krishnakumar Gopalakrishnan}  was responsible,
        and  the  zero-dimensional  thermal  model for  which  \mbox{Ian  D.\  Campbell}
        (PhD  student,  Imperial  College  London)  was  responsible,  the  advancements
        inherent to the  enhanced computer software (LIONSIMBA~v2.0) were  made in equal
        parts by  \mbox{Krishnakumar Gopalakrishnan} and \mbox{Ian  D.\ Campbell}. These
        advancements would not have been  possible without the contributions and support
        of \mbox{Dr~Davide M.~Raimondo} (Associate  Professor, University of Pavia) who
        served as  an unofficial supervisor for  the work reported in  this chapter. The
        concept  of layer  reconfiguration for  energy  and power  trade-off, the  layer
        optimisation  framework,  and  the  source  code  by  which  it  is  implemented
        were  co-developed  in equal  parts  by  \mbox{Krishnakumar Gopalakrishnan}  and
        \mbox{Ian  D.\  Campbell}.  \mbox{Krishnakumar  Gopalakrishnan}  was  the  major
        contributor  to  the  development  of  the binary  search  while  \mbox{Ian  D.\
            Campbell} was the  major contributor in the analysis  of results. \mbox{Parvathy
        Chittur Subramanianprasad}  (MSc student, Queen  Mary University of  London) was
        instrumental in  developing the analytical  expression for the  maximum possible
    number of layers~$n_\text{max}$.}%
}\label{ch:modelbaseddesign}
\ChapFrame

\vspace*{-1em}
\renewcommand{\baselinestretch}{1.0}\normalsize % for toc
\startcontents[chapters]
\printcontents[chapters]{}{1}{\setcounter{tocdepth}{1}}
\setstretch{1.348361657291667} % golden-ratio stretch (1.2 x 1.348 = 1.618)

% \bigskip
\vfill
\pagebreak

\glsunset{xeV}


\section[Introduction]{Introduction}\label{sec:layeroptintro}
% -*- root: ../../main.tex -*
%!TEX root = ../../main.tex
% vim:textwidth=80 fo=cqt conceallevel=0


\capolettera{T}{he} issue of `range anxiety'  is a pervasive mental blockade for
potential buyers  of electric  vehicles which  in-turn hampers  their widespread
adoption. From  a consumer viewpoint,  yet another  practical issue is  the fact
that on  encountering a `low battery'  scenario in a long  distance journey, the
charging  times required  for sufficiently  replenishing the  battery to  enable
completion  of the  journey  are  prohibitively large,  to  the  point of  being
non-competitive against conventional fossil fuel powered vehicles.

Unfortunately,  the  aforementioned  scenarios  are not  unimaginable  with  the
present  state  of the  art  in  lithium  ion  batteries. Hence,  improving  the
\gls{aer} and  providing fast charging  capabilities are two near-term  goals of
manufacturers  of electric  vehicles.  Increasing the  \gls{aer} necessitates  a
battery pack with  higher energy content in it while  lowering the charging time
demands a  pack with higher  power capability.  The contrasting nature  of these
goals can  be traced  all the way  down to  the cell level  and is  presented in
\cref{sec:energypowertradeoff}. By trading  off the number of layers  in a pouch
cell  against  the content  of  active  electrode material  accommodated  within
it,  bespoke cell  designs  addressing either  the energy  demand  or the  power
demand  can  be  obtained.  In  the  absence  of  accessible  documentation  (as
either  industry white  papers or  academic literature)  on the  layer selection
methodologies employed in automotive pouch  cell designs, this author postulates
that manufacturers  iterate through  an extensive  empirical testing  process of
prototypes with  a range of  layer choices. In the  view of this  thesis author,
this  procedure is  not only  time-consuming, but  is also  likely to  result in
sub-optimal designs.  This chapter envisages a  model-based engineering solution
to more  optimal cell designs  by determining  the appropriate number  of layers
needed  to maximise  its  \emph{usable} energy  while simultaneously  satisfying
certain  power  capability constraints.  The  rest  of  the chapter  provides  a
detailed treatment of topics such  as the proposed layer optimisation framework,
its assumptions involved,  and various modifications to  standard numerical code
required to facilitate this design procedure.



\section{Energy/Power Trade-off in Pouch Cells by Layer Selection}\label{sec:energypowertradeoff}
% -*- root: ../../main.tex -*
%!TEX root = ../../main.tex
% vim:textwidth=80 fo=cqt conceallevel=0


Varying the  number of electrochemical  layers stacked  within a pouch  cell has
contrasting effects  on its energy  storage and power handling  capabilities. In
this section,  a high-level  intuitive explanation of  this phenomenon  is first
offered, before  delving into  a detailed  presentation of  this effect  and its
implications for a specific  example cell in \cref{sec:energypowertradeoffdemo}.
Interwoven  into  the  narrative  is  a set  of  simplifying  assumptions  which
establishes  the broader  context  within which  a  computational framework  for
determining the optimum  number of layers for a specific  target design shall be
formalised (to be discussed in \cref{sec:layeroptframework}).

\subsection{Preliminary assumptions}\label{subsec:layeroptassumptions}

To  obtain a  balanced  loading of  both electrodes  and  to avoid  asymmetrical
exhaustion  of lithium  from  one  of the  electrodes  during  operation, it  is
desirable to carefully calculate the  volume of electrochemical active materials
to be accommodated  within the cell. This concept is  well-known and is commonly
discussed  in  standard  textbooks in  the  field  such  as  those by  Rahn  and
Wang~\cite{Rahn2013} wherein  example calculations are presented  for non-porous
electrodes. The study  by Ramadesigan~\etal~\cite{Ramadesigan2012} also supports
the  statement that  capacity  matching  of anode  and  cathode  materials is  a
standard practice in cell design.


In  the  case  of  lithium  ion   cells  with  porous  electrodes,  the  concept
of  electrode-balancing  involves an  additional  variable  \viz{} the  porosity
of  the  active materials.  The  role  of  porosity  and its  corollaries  \ie{}
the  material  volume  fraction  and  filler/binder  fraction  is  discussed  in
\cref{subsec:spmp2dparametrisation}.  In this  work,  a  major assumption  about
material porosities  (and hence active-material/filler volume  fraction) is that
they are held constant. The rationale behind using this simplified assumption is
as follows.

This  author  visualises  the  integration  of  cell-level  design  optimisation
(through  an optimal  layer  selection procedure)  into  the overall  drivetrain
design  by the  \emph{cell manufacturer}  before  a custom  design is  delivered
to  vehicle/system  integrators.   Cell  manufacturers,  especially  small-scale
manufacturers do not necessarily  synthesize each electrochemical component, but
instead may opt  to source certain raw-materials from  an upstream supply-chain.
From  a  manufacturing  viewpoint,  the  porosity  of  the  electrode  materials
is  governed  by  the  extent  of  calendaring  of  the  electrode  reel.  Using
pre-calendered electrode materials or sourcing  large volumes of electrode reels
with a fixed extent of calendaring can help to keep costs low. Since researchers
in  the  field are  typically  not  privy to  the  specifics  of the  industrial
procurement process,  in the absence  of further information, the  assumption of
constant  porosities provides  a  good starting  point  for this  model-oriented
design study.

From a technical  viewpoint, there exists another redeeming  argument to support
the constant  porosity assumption. Keeping material  porosities constant enables
to  eliminate  one  degree  of  freedom  from  the  design  optimisation  study,
thereby narrowing  the dimensionality of  the search space.  To the best  of the
author's knowledge,  there has not  yet been  any published work  tackling layer
optimisation  of  pouch  cells.  Building an  initial  infrastructure  in  terms
of  a  computational  framework  that  is  based  upon  this  constant  porosity
approximation  shall at  least  provide a  solid foundation  to  build upon  for
such  real-life  use-cases.  The  author  foresees  this  study  as  a  vanguard
research into  cell engineering and therefore  places a high value  in obtaining
ballpark estimates of  an optimal layer count, albeit  with constant porosities.
Ramadesigan~\etal~\cite{Ramadesigan2012} present  an opinion that the  choice of
porosities of electrode  materials is currently being done on  a trial and error
basis. Nevertheless, for  real-world use, the influence of  varying the material
porosities  on the  cell's  performance is  to be  quantified.  Hence, prior  to
adopting  this  model-based  methodology  for  production  yields  at  scale,  a
fully-integrated design optimisation process with  variable porosities has to be
developed. Therefore, in this work, the study is restricted to constant porosity
values, whilst  acknowledging variable porosity  designs as an  important aspect
for future studies.

At  a  system level,  the  efficiency  of the  drivetrain  is  considered to  be
constant. The  drivetrain of  an electric  vehicle consists of  a whole  host of
electrical and mechanical components such as power electronics, electric motors,
gearing, differential shaft and other  transmission systems. The efficiencies of
each  of these  individual  components has  a cascading  effect  on the  overall
drivetrain efficiency.  The efficiency of  each component is  strongly dependent
upon the operating point. For instance, the efficiency of an electric motor is a
function  of  its  torque-speed  curve.  In  practice,  it  is  rarely  easy  to
decouple  these efficiencies  at  least  during the  initial  design stage.  The
datasheet/technical specification of each component  in the platform is required
to make  a comprehensive multiphysics-based  design optimisation study.  This is
well  beyond the  scope  of this  work  and requires  access  to various  design
blueprints. Therefore, a constant lumped  efficiency value for the drivetrain is
adopted  for this  work. However,  the  proposed optimisation  methodology is  a
modular one  which implies that  it can be suitably  adapted \eg{} to  include a
efficiency  value dependent  upon power  delivered at  the wheels.  However, the
biggest redeeming  aspect (observed after the  completion of the study)  is that
using a constant  efficiency value did not influence the  final layer choice for
the cell  design. As  seen in  \cref{sec:accpathway}, the  drivetrain efficiency
plays a role  only during acceleration studies. As per  the results presented in
\cref{sec:resultslayeropt}, the  layers required  for satisfying even  the basic
fast-charging  requirements far  exceed  the layers  required  for handling  the
acceleration power demands. Therefore, this  assumption is justified for keeping
the computations tractable.

From  a  pack  perspective,  the   primary  assumption  in  the  formulation  of
the   proposed  optimisation   methodology  is   that  the   pack  configuration
(series/parallel arrangement  of modules, number  of cells per module  and other
system-level specifications) are held constant  throughout. The validity of this
assumption  is easily  justified  since  a cell-level  design  may be  performed
independently of  the larger drivetrain  design. In fact, the  author postulates
that  present design  process for  electrified transportation  is a  modular one
\ie{} empirical cell designs are  developed based on certain specifications laid
out by vehicle  manufacturers and is not integrated into  the drivetrain design.
This  modularity  in the  design  approach  enables  to keep  such  system-level
parameters constant.

A further assumption  in this study is  that the overall height of  the pouch is
held constant. In  the absence of this constraint, any  arbitrary pouch size can
be chosen,  leading to an  infinite-dimensional optimisation problem  wherein no
unique optimality  criterion exists.  This assumption is  in-fact enforced  by a
current  trend  in the  automotive  industry  \viz{} adoption  of  common-module
designs wherein  the physical  dimensions of  the pack  are chosen  a~priori and
modularising the  pack helps in  tailoring them  suitably to cater  to different
market segments. Extending this philosophy down to the cell level, it is easy to
visualise the  benefits of  having cells of  identical exterior  dimensions. For
instance, having a  common inventory helps a vehicle manufacturer  to keep costs
in check  for subsequent designs  \eg{} for  derivative model families  of their
product  portfolio. This  means that,  for  any layer  choice to  be tried,  the
constituent components  of the cell is  to be arranged and  contained within the
same  pouch  (of  fixed  exterior  dimensions).  This  naturally  leads  to  the
assumption that the  thickness of the pouch material used  shall remain constant
throughout, which in-turn implies that the overall height of the electrochemical
stack  within the  pouch is  constant. The  detailed calculations  of the  stack
height is presented in \cref{sec:surfareaperlayer}.

The  current collectors  and the  separator  in each  electrochemical layer  are
assumed  to  have  uniform  thickness  irrespective  of  the  number  of  layers
used.  Barring  minor manufacturing  variability  and  tolerances, these  values
are  merely  factual data  requiring  no  further justification.  For  instance,
a  constant  separator thickness  was  used  in  the design  optimisation  study
by  Newman~\cite{Newman1995}.  The  final  assumption  from  an  electrochemical
point  of view,  introduced specifically  for the  first time  in literature  by
this  thesis author,  is  that the  relative thicknesses  of  each electrode  is
held  constant to  a  fixed ratio.  This warrants  further  explanation, but  is
ill-suited  for this  introductory discussion.  The details  of this  aspect are
discussed  in  \cref{sec:electroderatio}. Certain  assumptions  are  to be  made
about the  temperature distribution  within the  layers owing  to the  choice of
cooling arrangement.  These aspects merit  more than  a cursory listing  in this
introductory section and hence is discussed in \cref{sec:celllevelxeVinfo}.

\subsection{Motivation}\label{subsec:layeroptmotivation}

This section aims to provide a  qualitative description of the effect of varying
the number of layers  within a pouch cell and presents  the motivation to embark
upon this layer optimisation effort.

Given the  assumptions listed in \cref{subsec:layeroptassumptions},  it is clear
that  changing the  number  of layers  in  a pouch  of  fixed thickness  results
in  different  absolute  electrode  thicknesses.   This  in  turn,  affects  the
electrochemical-thermal behaviour  of the  cell. Having a  low number  of layers
means  that the  proportion of  energy-storing  materials within  the volume  is
higher, leading  to greater nominal capacity.  However, owing to the  large time
constants necessary for  lithium ions to diffuse through thick  domains, not all
of this stored energy might be available for utilisation. With thick electrodes,
the power  handling capability  of the  cell suffers  since lithium  deficits at
electrode surfaces shall lead to the collapse of its terminal voltage.

Prima~facie,  based  on the  above  discussion,  although  it appears  that  the
absolute lowest possible  number of layers (\ie{} one layer)  is the best choice
for maximising  driving range, there  exist two  other goals that  conflict with
this design  intent. Firstly, there must  be a minimum electrode  active surface
area  to  handle the  power  demands,  and hence,  a  minimum  number of  layers
typically far  greater than one. Higher  power capability is achieved  by way of
larger electrode surface  area and higher electrical  and thermal conductivities
owing to the presence of more current collectors. Secondly, the acceptable limit
on  lifetime  degradation of  cells  places  an  upper  bound on  the  allowable
temperature rise during vehicle operation. Increasing the number of layers has a
two-fold mitigating effect on the  temperature-rise experienced by the cell. The
\emph{power  density} within  each  layer  is diminished  due  to the  increased
available  surface area,  leading to  reduced ohmic  heat generation  within the
cell. With  each layer requiring a  Al-Cu current-collector pair, the  number of
heat conduction  pathways increases  linearly with the  number of  layers. Thus,
increasing the number of layers has a beneficial effect on pack lifetime.


In summary,  for very low number  of layers, there exists  more active material,
leading  to a  high  energy  capacity. However,  the  reaction  surface area  is
diminished proportionately leading to lower power capability. Furthermore, owing
to the presence  of very thick electrodes, the current  density within the solid
conductive  matrix shall  not  be  homogeneous~\cite{Pals1995}, nullifying  some
fundamental modelling assumptions of the  standard \gls{dfn} model. On the other
hand,  very  high  number  of  layers  imply  vanishingly  thin  electrodes  and
correspondingly  less  active material  accommodated  within  the cell,  thereby
resulting  in  a lower  energy  capacity.  \Cref{fig:energyvspowercell} shows  a
qualitative comparison of the construction of one layer of an energy cell versus
power cell which helps to illustrate all the aspects discussed thus far.

\begin{figure}[!htbp]
    \centering
    \includegraphics{energy_vs_powercell}
    \caption[%
    Qualitative comparison  of the  construction of one  layer of  a high-energy
    cell versus a high-power cell
    ]%
    {%
        Schematic  depicting a  qualitative  comparison of  the construction  of
        one  layer  of  a  high-energy   cell  versus  a  high-power  cell.  The
        illustration  at top  depicts one  layer of  a high-energy  cell wherein
        thick  electrodes  are  used.  The bottom-left  illustration  depicts  a
        single layer  of a high-power  cell wherein very thin  electrode regions
        are  used.  Both  cell  diagrams  are  drawn  to  the  same  scale.  The
        bottom  right  plot  qualitatively indicates  the  relationship  between
        C-rate  and  the nominal  cell  capacity.  Illustration reproduced  from
        \mbox{von~Srbik~\cite{VonSrbik2015}.}

    }%
    \label{fig:energyvspowercell}
\end{figure}


Therefore, there exists  a research question on what constitutes  the best layer
choice  that straddles  this  trade-off  with the  least  penalty  to the  power
capability  of  the  cell  whilst simultaneously  having  the  maximum  possible
capacity. This  saddle point determination needs  to be performed for  a curated
set of power input/output conditions to the cell. This niche problem has not yet
been  tackled by  researchers  and therefore  motivates the  need  to perform  a
careful design study which is documented in this chapter.

\FloatBarrier

\subsection{Quantitative demonstration of energy/power trade-off}\label{sec:energypowertradeoffdemo}

The discussion in \cref{subsec:layeroptmotivation} has motivated the need for an
in-depth exploration of the energy to power trade-off expressed as a function of
the number of  layers. Before embarking on constructing a  framework to optimise
the layer choice by formalising various constraints that govern this optimality,
this section aims to quantitatively  demonstrate this relationship by applying a
fixed galvanostatic discharge to an example cell. Additionally, the crucial idea
of \emph{usable} energy versus \emph{total} stored energy is also introduced.

A  \gls{lco}  cell  whose  physical  properties  and  simulation  parameters  is
drawn from  the combined set of  data from tables~\ref{tbl:lcoSimParamslayeropt}
and~\ref{tbl:lcoSimParamsSPMp2d} is  used as the  example cell. The only  set of
values that overlap between these two tables are ---
\begin{enumerate*}[label=\itshape\alph*\upshape)]
    \item the cut-off voltages, and
    \item the number of nodes  used for numerical  discretisation of  the governing  \gls{pdae} equations.
\end{enumerate*}
For these conflicting quantities,  the values in \cref{tbl:lcoSimParamslayeropt}
prevail for all simulation studies  in this chapter. Furthermore, the individual
electrode thicknesses from \cref{tbl:lcoSimParamsSPMp2d}  are not directly used,
but  instead  calculated  for  every  layer  choice  by  keeping  the  ratio  of
their  relative  thicknesses  constant.  This   aspect  shall  be  explained  in
\cref{sec:electroderatio}.

\Cref{fig:fig_CC_discharge_curves}  illustrates  the  influence  of  the  number
of  layers   on  the  energy   and  power   capability  of  the   example  cell.
Starting  at  \SI{100}{\percent}  \gls{soc},  a constant  current  discharge  of
\SI{60}{\ampere}\footnotemark{}  is applied  to a  \gls{dfn} model  of the  cell
until reaching the  lower cut-off voltage. For each discharge  run, the model is
reconfigured with  a different  layer choice. Five  distinct layer  choices have
been carefully chosen so as to  provide a clear illustration of the energy/power
trade-off phenomenon.

\begin{figure}[!bp]
    \begin{minipage}[t]{\textwidth}
        \centering
        \includegraphics[trim=4 4 2 4,clip]{fig_CC_discharge_curves.pdf}
        \captionsetup{labelsep=note}
        \caption
        [%
        Voltage curves for a \SI{60}{\ampere} galvanostatic discharge from
        \SI{100}{\percent} \glsfmtshort{soc} until cut-off voltage for a few layer
        choices, in a pouch cell of fixed exterior height.
        ]%
        {%
            Terminal voltage curves of a Li-ion cell (with parameters
            given in \cref{tbl:lcoSimParamslayeropt}) under a \SI{60}{\ampere}
            galvanostatic discharge beginning from \SI{100}{\percent}
            \glsfmtshort{soc} until lower cut-off voltage for a few layer
            choices~$n$, in a pouch cell of fixed exterior height. The maximum
            usable energy is achieved for an intermediate choice of $n$
            that corresponds to neither the highest nominal capacity layer
            configuration ($n$=\num{10}) nor the highest electrode surface area
            configuration ($n$=\num{90}).
        }%
        \label{fig:fig_CC_discharge_curves}
        \mpfootnotes[1]
        \footnotetext{{The   rationale  behind  choosing   this  specific
                magnitude  of  applied  current  is   explained  in  the  section  dealing  with
        \hyperlink{refcellselection}{selection of a  suitable reference capacity cell} (also see \cref{sec:surfareaperlayer}).}}

        \footnote{This figure was created by \mbox{Ian D.\ Campbell} who asserts copyright,
            with intellectual contributions from and the right to use asserted by
        \mbox{Krishnakumar Gopalakrishnan}.}
    \end{minipage}
\end{figure}

As  seen  in \cref{fig:fig_CC_discharge_curves},  during  the  initial phase  of
discharge, the terminal voltage  of the cell is the highest  for the two highest
layer choices  \ie{} $n  = 90$  and $n=70$. Consistent  with the  explanation in
\cref{subsec:layeroptmotivation}, these  two layer choices have  thin electrodes
and  hence  comparatively low  resistances  leading  to  only a  small  internal
overpotential drop. However,  as expected, their total energy is  lower than the
cell with  $n=50$ layers as  evidenced by  their relative run-times  until lower
cut-off voltage.  This is to  be expected as the  thin electrodes of  these high
layer count cells cannot  store a large volume of active  material. Based on the
explanation  from \cref{subsec:layeroptmotivation},  it  is  expected that  this
trend  will continue  \mbox{\ie{} the}  lower the  layer count,  the higher  the
run-time until  cut-off. If this  were the case,  prima~facie it seems  that the
layer optimisation task is trivial.

Inspecting the  discharge curves of  lower layer  choices brings into  light the
concept of \emph{usable} energy. Contrary  to expectations, the discharge curves
corresponding  to very  low layer  counts in  \cref{fig:fig_CC_discharge_curves}
terminate even earlier than $n=50$. This is  owing to the fact that although the
total  stored energy  in cells  with low  layer counts  is much  higher, only  a
fraction  of it  is usable.  This  aspects introduces  non-trivial dynamics  (as
discussed below) to an otherwise linear optimisation task.

For  instance,  when $n  =  10$,  the terminal  voltage  of  the cell  collapses
instantaneously, reaching cut-off  voltage whilst its \gls{soc}  remains as high
as \SI{96}{\percent}. At very low layer  counts, the thickness of each electrode
is high. This presents a high resistance  to the flow of charges thereby leading
to  high overpotential  drops  within the  cell. The  usable  energy under  this
\SI{60}{\ampere} galvanostatic  discharge for various layer  choices is compared
in \cref{tbl:CC_discharge_curves_table}. It can be  seen that for very low layer
counts,  the usable  energy that  can be  extracted is  miniscule, albeit  their
theoretical  capacity~$Q_n$  are  in-fact  the highest.  The  usable  energy  in
\SI{}{\watthour} reported in \cref{tbl:CC_discharge_curves_table} is obtained by
multiplying the integral  of the area under each discharge  curve by the applied
current (\SI{60}{\ampere}) with the appropriate scaling of the time-base ( \ie{}
conversion from minutes to hours).

% -*- root: ../../main.tex -*-
%!TEX root = ../../main.tex

\begin{table}[!htbp]
    \caption
    [%
    Theoretical  capacity \&  usable energy  of a  Li-ion cell  for a  few layer
    choices under a \SI{60}{\ampere} galvanostatic discharge
    ]
    {%
        Theoretical capacity and usable energy of a Li-ion cell (with parameters
        given in \cref{tbl:lcoSimParamslayeropt}) for  a few layer choices under
        a \SI{60}{\ampere} galvanostatic discharge.
    }%
    \label{tbl:CC_discharge_curves_table}
    \centering
    \begin{tabular}{@{} S[table-format=2.0] S[table-format=1.2] S[table-format=2.2]  S[table-format=3.2] S[table-format=2.2] S[table-format=2.2] @{}}
        \toprule
        \multicolumn{1}{@{} l}{$n$} &  \multicolumn{1}{c}{\footnotesize C-rate} & \multicolumn{1}{c}{\footnotesize \makecell{Theoretical \\ Capacity  (\si{Ah})}} & \multicolumn{1}{c}{\footnotesize \makecell{Usable \\ Energy \si{(Wh)}}} & \multicolumn{1}{c @{}}{\footnotesize \makecell{Remaining \\ SOC  (\si{\percent})}} & \multicolumn{1}{c @{}}{\footnotesize \makecell{Resistance at \\ cutoff  (\si{\milli\ohm})}} \\
        \midrule
        90 & 1.24 & 48.25 & 166.46 & 9.84  & 0.97  \\
        70 & 1.11 & 53.99 & 184.80 & 10.26 & 1.35  \\
        50 & 1.00 & 59.73 & 195.47 & 13.51 & 3.44  \\
        30 & 0.92 & 65.47 & 101.20 & 58.95 & 10.24 \\
        10 & 0.84 & 71.21 & 10.15  & 96.22 & 11.18 \\
        \bottomrule
    \end{tabular}
\end{table}


\Cref{tbl:CC_discharge_curves_table}  also brings  into view  the fact  that the
\mbox{C-rate} of the  cell becomes a variable quantity even  for a galvanostatic
discharge,  due to  the dependence  of  its nominal  capacity on  the number  of
layers~$n$. This represents a departure from the norm in the modelling community
wherein the  performance of cells  are quantified as  a function of  the applied
C-rate  \eg{} in  chapters~\ref{ch:spmanalysis} and~\ref{ch:newelectrolytemodel}
of  this thesis.  However, the  preliminary investigation  thus far  has quickly
revealed that  this normalised  quantity does  not hold  much importance  in any
study where the number of layers within a pouch cell is varied.

Taking into account  these factors, a reasonable choice of  the number of layers
in  this specific  \SI{60}{\ampere} galvanostatic  application for  this example
cell  could  be $n=50$.  This  represents  a  practical compromise  between  the
surface area  available for  reaction and  the total  volume of  active material
accommodated.  Out of  the finite  layer configurations  considered, this  layer
choice offers the highest usable energy for the given discharge rate.

In this  sample study, only  a handful of  layer choices were  considered, which
represents  only  a  small  possibility  of  the  overall  design  space  to  be
considered. Furthermore,  thermal considerations were  not explored so  far. For
robust  cell design,  manufacturers  shall need  a  widely applicable  model-led
design tool  that can tackle the  various scenarios that can  occur in real-life
operating conditions. A  deterministic set of optimality criteria  for the layer
selection  is also  to  be formulated.  The choice  $n=50$,  therefore does  not
represent the general optimal layer choice  even for this example cell. However,
this sample study  serves as an illustrative demonstration of  the trade-offs in
energy  versus  power handling  capability  of  a cell  for  a  specific set  of
conditions. Furthermore, it introduces  the complicating aspect of \emph{usable}
capacity into what would have otherwise been a trivial exercise, thereby setting
the tone for the development of a general layer optimisation framework for pouch
cells.



\section{Scope and Context within \glsfmtshort{xeV} Powertrain}
% -*- root: ../../main.tex -*
%!TEX root = ../../main.tex
% vim:textwidth=80 fo=cqt conceallevel=0


It is  important to provide the  contextual setting for this  layer optimisation
work since it is nestled deep  within the broader horizon of electric drivetrain
optimisation. \Cref{fig:fig_PowertrainSchematic}  provides a  graphical overview
depicting the hierarchical architecture of  a typical \gls{xeV} powertrain, from
the  system level  down to  a  single electrochemical  layer. The  rest of  this
section describes the scope of this  layer optimisation work and its integration
into this  overall architecture. A further  set of assumptions that  were deemed
inopportune to be discussed  in \cref{subsec:layeroptassumptions}, is introduced
at apropos junctures  throughout this narrative. The overall  architecture of an
\gls{xeV}  powertrain  can  be  studied through  a  systematic,  hierarchical
evaluation at ---
\begin{enumerate*}[label=\itshape\alph*\upshape)]
    \item the system-level,
    \item the pack-level, and
    \item the cell-level.
\end{enumerate*}

\begin{figure}[!bp]
    \begin{minipage}[t]{\textwidth}
        \centering
        \includegraphics[width=\textwidth]{hierarchical_powertrain_to_cell_layer.pdf}
        \captionsetup{labelsep=note}
        \caption
        [%
        Vehicle-to-cell hierarchical overview of an electrified powertrain architecture
        ]%
        {%
            Schematic depicting the vehicle-to-cell hierarchical overview of
            a typical electrified powertrain architecture. This represents the
            system-level context within which the proposed layer optimisation framework
            has been developed. Two \glsfmtshort{xeV} powertrains ---
            \begin{enumerate*}[label=\itshape\alph*\upshape)]
                \item a \gls{bev}, and
                \item a series \gls{phev}
            \end{enumerate*}
            are chosen as examples to demonstrate how the methodology facilitates
            common module designs for such battery packs.
        }%
        \label{fig:fig_PowertrainSchematic}
        \mpfootnotes[1]
        \footnote{This figure was created by \mbox{Krishnakumar Gopalakrishnan} who
            asserts copyright, with intellectual contributions from and the right to
        use asserted by \mbox{Ian D.\ Campbell}.}
    \end{minipage}
\end{figure}

\subsection{System-level --- vehicular platforms}

The top row of  \cref{fig:fig_PowertrainSchematic} represents the typical layout
of a \emph{series}-hybrid powertrain~\cite{Maksimovic2012}. Partly to supply the
mechanical  power and/or  partly  to  charge the  battery  during propulsion,  a
downsized \gls{ice} is  employed. The \gls{ice} is coupled to  the pack's DC bus
through  a  generator  and  three-phase  rectifier.  While  tackling  the  power
handling  requirements,  irrespective  of  whether  a  \gls{bev}  or  \gls{phev}
powertrain is  being considered, the  cells in the pack  are to be  designed for
the  worst-case operating  scenario \ie~without  any power  support from  the
\gls{ice}. This implies that all discharge  simulations of the \gls{phev} are to
be conducted with  the powertrain operating in all-electric mode  resulting in a
net charge-depletion. The only distinction is that the \emph{magnitude} of power
to  be handled  by the  pack in  this worst  case scenario  is vastly  different
between the \gls{bev} and \gls{phev}  cases. This allows for some simplification
as explained  below and  helps to  narrow down the  scope of  the problem  to be
tackled.

Omitting the  components to the  left of the  battery pack (represented  as text
boxes with light grey border) shall render a powertrain corresponding to that of
a  \gls{bev}.  The proposed  layer  optimisation  methodology is  developed  and
presented in the context of this  \gls{bev} powertrain. However, being a modular
framework, the optimisation methodology may  be readily extended to a \gls{phev}
powertrain.

As  shown   in  \cref{fig:fig_PowertrainSchematic},  the   \gls{bev}  powertrain
typically comprises of ---
\begin{enumerate*}[label=\itshape\alph*\upshape)]
    \item a battery pack,
    \item a three phase inverter,
    \item a \gls{pmsm},
    \item a gearbox for  torque multiplication, and
    \item the rest of the  powertrain  (differential shaft  and driven  wheels).
\end{enumerate*}
Considering the  worst-case scenarios, the  power to  be handled by  the battery
pack arises due to  ---
\begin{enumerate*}[label=\roman*)]
    \item fast charging from  the mains~$P^\text{fastchg}_\text{batt}$~(charge), or
    \item acceleration  from standstill~$P^\text{acc}_\text{batt}$~(discharge).
\end{enumerate*}
The   acceleration  power   is  computed   from  the   power  required   at  the
wheels~$P_\text{w}$.  The   details  of  this  calculation   is  presented  in
\cref{sec:accpathway}.  The  sign  convention  used  in  this  chapter  is  that
the  charging power  is positive  (and  consequently, the  discharging power  is
negative).

\subsection{Pack-level --- strings, modules \& cells}\label{sec:packlevelhierarchy}

Delving  into the  battery pack  under  consideration, this  thesis considers  a
standard  modular  layout wherein  the  \gls{phev}  pack  has one  string  while
the  \gls{bev}  pack  has  three  parallel strings.  Within  each  string,  both
vehicular  platforms employ  8~series-connected  modules.  Taking cognizance  of
the  benefits  of common  module  design,  identical  pack modules  are  assumed
across  both  \gls{xeV}  platforms  which  is  then  extrapolated  to  impose  a
stronger  condition  of  identical  geometry  for  the  constituent  cells  (see
\cref{subsec:layeroptassumptions}). The  exterior dimensions of the  pouch cells
under consideration are listed in \cref{tbl:lcoSimParamslayeropt}.

Each module consists  of 12~identical series-connected cells  denoted by battery
circuit symbols (cyan-filled  blocks in \cref{fig:fig_PowertrainSchematic}). The
\gls{phev} pack is  smaller and consists of only \ordfrac{1}{3}  of the cells in
the \gls{bev} pack. Assuming that the \gls{bev} pack consists of a \mbox{96S-3P}
cell  assembly,  this implies  that  the  \gls{phev}  pack  shall conform  to  a
\mbox{96S-1P} layout. The DC~bus voltage is unaltered since both packs have same
amount of  series cells. The power  flow is assumed to  be uniformly distributed
across all  the cells within  the pack(s). At first,  the power required  at the
terminals of the  pack is computed. From this, a  first-order design ballparking
of the layers of the cell is made through a single cell simulation. This process
enables reduced simulation runtime with the conditions of one cell assumed to be
representative of all cells in the pack.


% At a  high-level, the  essence of  the layer  optimisation methodology  can be
% distilled down to the following sequence.

While the aforementioned assumption of identical cell conditions across the pack
seems infeasible at first glance, three careful considerations have been made to
justify  this assumption.  Firstly,  power  (and not  current)  is  used as  the
stimulus to the cell. This  implies that, despite the \emph{parallel} connection
of cells  (in groups of three  cells within each module),  each cell experiences
the same  power. Even  across parallel-connected strings,  the power  handled by
each cell shall be the same.  This necessitates the modification of the standard
\gls{dfn}  model in  order to  accommodate power  inputs which  is discussed  in
\cref{sec:innatepowerinput}. Secondly,  although the current  in all cells  of a
\emph{series}  string remain  the same  (by virtue  of Kirchoff's  current law),
their  terminal voltage  levels could  drift away  from each  other and  becomes
unbalanced over  time~\cite{Andrea2010}. This naturally raises  questions on the
assumption  of  identical  conditions  for  all  cells.  However,  this  voltage
unbalance is mitigated with the help of modern \glspl{bms} that employ balancing
techniques  such as  passive  bleeder resistors  or  sophisticated active  dc/dc
converters. Yet another adverse effect that  poses a threat to the assumption of
identical  conditions  is  the  uneven distribution  of  cell  temperatures.  In
automotive packs employing natural convection, cells that are physically located
innermost in  the string tend  to get hotter  than the outermost  cells. Through
good thermal management  design \eg~forced cooling through  circulation of the
coolant through conduits grooved into the pack, thermal balance may be achieved.
Therefore,  it  can be  argued  that,  when  operating  in a  well-designed  and
controlled environment,  cell-to-cell deviations  are minimised.  This justifies
the  global representation  of  all  cells in  the  pack  through a  single-cell
simulation, although modifications to the  simulation model are deemed necessary
to facilitate power inputs.


% Since  thermal effects  need to  be considered  for robust  cell design,

\subsection{Cell-level --- layers, cooling, electrochemical \& thermal models}\label{sec:celllevelxeVinfo}

The    illustration    at     the    centre    of    the     bottom    row    in
\cref{fig:fig_PowertrainSchematic} shows  a schematic  representation of  a cell
arranged within  each module. In practice,  the physical layout of  cells within
a  module  is  slightly  more  complex.  For  instance,  a  typical  arrangement
consists  of  groups   of  3  parallel  cells.  However,   the  illustration  in
\cref{fig:fig_PowertrainSchematic}  suffices to  explain  the necessary  details
required for the specific task at hand.

Each cell in the pack consists of a number of identical layers~$n$. Each layer
consists of ---
\begin{enumerate*}[label=\roman*)]
    \item a~positive current-collector,
    \item a~positive electrode region,
    \item a~separator material,
    \item a~negative electrode region, and
    \item a~negative current-collector.
\end{enumerate*}
Particular attention is called out in  regard to the distribution of temperature
within  the cell.  In  the schematic  of \cref{fig:fig_PowertrainSchematic}  the
shading scheme  is such that greener  tints represent the hotter  regions of the
cell while bluer tints represent colder regions. Furthermore, heat exchange with
the surroundings is also graphically  illustrated through cooling plates mounted
at the tabs  of the cell. This  highlights the specific type  of cooling assumed
\viz~\emph{tab-cooling}  as  opposed to  conventional  \emph{surface-cooling}
historically employed for automotive applications. The assumption of tab-cooling
is an  essential requirement for  upholding the  validity of the  proposed layer
optimisation scheme, and therefore warrants further justification.

An  experimental  study  by   Hunt~\etal~\cite{Hunt2016}  compared  tab  cooling
of  cells  against conventional  surface  cooling.  It  was found  that  \approx
\SI{8}{\percent} increase in the usable  capacity of pristine cells was achieved
with tab cooling relative to that  achieved with surface cooling. Secondly, with
surface  cooling, the  loss rate  of usable  capacity over  thousand cycles  was
nearly thrice of that with tab cooling.  This implies that using tab cooling can
potentially help to extend the lifetime of  the pack by three times. Thirdly, at
higher discharge rates,  surface cooling resulting in a loss  of usable capacity
of \SI{9.2}{\percent} compared  to just \SI{1.2}{\percent} for  tab cooling. The
simulations  discussed  as  part  of  the  optimisation  framework  reported  in
\cref{sec:layeroptframework} are intended to  obtain robust cell designs capable
of handling  worst-case power inputs.  In these  scenarios, tab cooling  is more
appropriate.  Therefore,  this author  has  no  qualms about  recommending  this
specific cooling mechanism  to be used in conjunction with  the results reported
(see  \cref{sec:resultslayeropt}) by  applying the  proposed layer  optimisation
scheme.

Apart  from  its  aforementioned  beneficial   effects  on  cell  longevity  and
performance,  with the  integral  assumption  of tab  cooling,  there exists  an
important side effect  that affects the very  core of the numerics  of the layer
optimisation methodology.  Carefully examining the  shading scheme used  for the
schematic  in the  centre-bottom  of  \cref{fig:fig_PowertrainSchematic}, it  is
clear that  at any vertical  co-ordinate in space  within the cell,  the shading
across the entire cell width remains  uniform throughout. This implies that each
layer  along  a  one-dimensional  cross-section  of the  cell  is  at  the  same
temperature. Based  on the inferences from  Hunt~\etal~\cite{Hunt2016}, with tab
cooling, only  small thermal gradients are  induced in the planar  direction. In
this unique scenario,  the thermal effects within the cell  are not large enough
to warrant a detailed numerical discretisation.  On the other hand, ignoring the
temperature distribution  of the  cell shall  not lead  to robust  cell designs,
especially given  that design simulations  typically involve high  magnitudes of
power.

In  situations akin  to  aforementioned circumstances,  a  lumped thermal  model
of  the cell  has  been  recommended by  Pals  and Newman~\cite{Pals1995}.  This
represents a good trade-off between accuracy and simplicity and hence, is deemed
to  be  appropriate for  this  design  application.  A  suitable value  for  the
convective heat transfer  coefficient~$h$ (see \cref{tbl:lcoSimParamslayeropt}),
comparable  to the  typical  magnitudes in  forced air  convection,  is used  to
represent the heat transfer from the  cell to the environment. The heat exchange
area is the combined  surface area of the two cooling tabs  that are situated at
either end  of the  cell. The temperature  of the coolant  (a thermal  `sink' in
thermodynamic  terminology) is  denoted by~$T_\text{sink}$.  In this  work, this
ambient temperature is held constant during  the course of a simulation run, but
is allowed  to change to  different constant  values between set  of simulations
as  per relevant  vehicle  testing  standards. The  details  of  this aspect  is
discussed  in  \cref{sec:layeroptframework}.  The  material  properties  of  the
constituent components  of each layer  coupled with  the total number  of layers
used  determine  the  lumped  mass  and specific  heat  capacity  of  the  pouch
cell.  These  computations  are  discussed  in  sections~\ref{sec:massofonecell}
and~\ref{sec:spheat} respectively. The  assumption of tab cooling  thus leads to
this qualitative  description of  the lumped  thermal model to  be used  for the
design simulations.  Further quantification by  way of relevant  model equations
and  the computation  of  the constituent  parameters of  the  thermal model  is
embedded as an integral aspect of  the layer optimisation framework and shall be
presented in \cref{sec:spheat}.

As a final observation, all layers  within the cell are electrically in parallel
which implies that their terminal voltages are identical. The current (or power)
at the  cell terminals is  shared equally  among each layer.  The aforementioned
considerations have important  ramifications on the cell modelling  and helps to
drastically  simplify  it. Specifically,  these  considerations  imply that  the
electrochemical  performance  of any  one  layer  is  identical to  every  other
layer.  Therefore,  in conjunction  with  a  lumped  thermal model,  a  standard
\gls{p2d}  discretisation  of  the  \gls{dfn}  model  suffices  to  capture  the
electrochemical behaviour of  the entire cell. The  bottom-right illustration of
\cref{fig:fig_PowertrainSchematic} presents a  one-dimensional discretisation of
the  cell layer  across its  thickness. The  spheres along  the axial  direction
represent computational nodes  wherein the solid-phase diffusion  equation is to
be solved  (see \cref{sec:dfnmodel} for a  brief overview). It is  this standard
\gls{dfn}  model,  suitably  amended  to  accept  power  inputs,  that  will  be
the  backbone of  the  electrochemical  aspects of  the  design simulation.  The
electrochemical model shall be strongly coupled  in a bidirectional sense to the
lumped thermal model  \ie~the temperature of the cell  shall influence various
cell parameters  (see \cref{tbl:lcoSimParamslayeropt}) while  the overpotentials
and currents in  the cell shall play a  role in the rate of  heat generation and
cell temperature simultaneously.

Thus, through a  systematic set of simplifying assumptions  that are justifiable
in a real-world  design, the system-level requirements at the  pack-level can be
suitably  scaled  down  to  power-density  inputs at  the  layer  level.  Having
established the  contextual setting and  scope of  this work within  the broader
landscape of drivetrain  optimisation, it is now possible to  proceed to the set
of numerical enhancements  required to be incorporated into  the \gls{dfn} model
to handle the specific requirements of this layer optimisation task.



\section{Enhancements/Modifications to Standard \glsfmtshort{dfn} Model}\label{sec:numericalenhancements}
% -*- root: ../../main.tex -*
%!TEX root = ../../main.tex
% vim:textwidth=80 fo=cqt conceallevel=0

\subsection{Augmentation of parameters to standard \glsfmtshort{dfn} model}

\subsubsection*{Cell capacity and electrochemically active surface area}

The  \gls{p2d} implementation  of  the standard  \gls{dfn}  model lacks  certain
parameters that are vital to the  layer optimisation process. The cell's nominal
capacity is a fundamental quantity that gets  altered as the number of layers is
varied. However, it may be surprising  to discover that this parameter is absent
in present  research literature  discussing the  \gls{p2d} model.  The rationale
behind this glaring omission becomes clear  upon closer examination of the model
equations presented in  \cref{tbl:dfneqns}. These equations do not  operate on a
cell level, but  instead operate on a  normalised basis \ie{} only  one layer is
modelled on a  \emph{unit area} basis wherein the stimulus  driving the model is
the applied current  \emph{density} rather than the total  external current. The
layer  optimisation task  here faces  a unique  predicament of  adhering to  the
present modelling paradigm  to retain compatibility with  standard models whilst
still incorporating  the concept of cell's  capacity as a function  of number of
layers.

To  tackle the  aforementioned quandary,  it  is key  to realise  that the  core
parameter  that  varies with  the  number  of layers  in  a  pouch cell  is  the
\emph{electrochemically  active  cross-sectional surface  area}~$A_\text{cell}$.
Curiously, published  literature on  physics-based cell  modelling do  not place
rigorous emphasis on  this key parameter. Most often, to  this author's chagrin,
this  parameter is  simply  listed in  a  standard table  of  parameters and  is
typically  sourced from  a historic  parameter-set with  no further  explanation
provided.

For  a pouch  cell, the  overall electrochemically  active surface  area can  be
defined as
\begin{equation}\label{eq:overallarea}
    A_\text{cell} = n \times A_\text{elec}
\end{equation}
where $n$ is  the number of layers and $A_\text{elec}$  is the electrochemically
active surface area per layer.

\subsubsection*{Surface area per layer}\label{sec:surfareaperlayer}

A literature search reveals that akin  to cell capacity, there is no information
of  cross-sectional geometry  provided in  articles dealing  with the  \gls{p2d}
implementation of  the \gls{dfn} model. To  determine the surface area  per face
(per layer), the author proposes  a new methodology/process involving a sequence
of  steps, based  on  certain  assumptions and  literature  search. The  process
involves  selection of  a  real-world cell,  and ultimately  mapping  it to  the
surface area per  unit face. To the  best knowledge of the  author, this reverse
parametrisation process,  mapping from  a real-world cell  to a  \gls{p2d} model
parameter  is unique  and is  claimed  as a  contribution  to the  art which  is
explained next.

\begin{enumerate}[ label=\textbf{\arabic*}), leftmargin=0pt, itemindent=20pt, labelwidth=15pt, labelsep=5pt, listparindent=0.7cm, align=left]
    \item \hypertarget{refcellselection}{\textbf{Selection of a suitable reference capacity cell}}

        Although  this value  shall not  be used  in the  actual layer  optimisation
        procedure itself, this is a  crucial first-step towards obtaining a complete
        parameter set, in particular, to obtain the surface area per layer.

        The focus  of this chapter is  to provide ready-to-use solution  to industry
        extend  with the  aim of  improving  the current  state of  the art  through
        optimal layer configuration  of pouch cells. There is a  clear motivation to
        further increase  cell capacities so as  to maximize driving range,  as laid
        out in the beginning of this chapter (see \cref{sec:layeroptintro}).

        With  this  guiding  principle,  as  a starting  point  towards  choosing  a
        reference capacity cell,  a survey was performed to  identify the production
        \gls{bev}  with  the  highest  driving  range.  As  of  2018,~the  Chevrolet
        Bolt~\gls{bev} bears this distinction  with a range of \SI{383}{\kilo\meter}
        as rated  by the  United States Environmental  Protection Agency  (EPA). The
        specifications of  its battery pack is  listed in Liu~\etal~\cite{Liu2016a}.
        The  battery pack  of  this  vehicle consists  of  288~cells  arranged in  a
        96S-3P configuration, in  agreement with the configuration  discussed in the
        drivetrain  hierarchy  of  \cref{sec:packlevelhierarchy}.

        The    Chevrolet   Bolt    \gls{bev}   pack    has   an    energy   capacity
        of   \SI{60.0}{\kilo\watthour}    with   a    nominal   pack    voltage   of
        \SI{350}{\volt}~\cite{Liu2016a}.  However,  for   this  specific  task,  the
        \si{\amphour} capacity is required. This can be obtained as
        \begin{align}
            \text{\si{\amphour} capacity of Bolt cell} & = \frac{\text{Pack Energy } (\si{\watthour})}{\text{Nominal pack voltage (V)} \times {\text{Number of cells in parallel}}} \\
            {}                                         & = \frac{60000}{350 \times 3} \\
            {}                                         & = \SI{57.14}{\amphour}
        \end{align}

        \begin{itemize}[ leftmargin=10pt, itemindent=15pt, labelwidth=5pt, labelsep=5pt, listparindent=0.7cm, align=left]
            \item \textbf{DC bus voltage and revised cell capacity}

                Even  without  a  dc/dc  boost   converter,  robust  design  of  the
                powertrain during  brown-outs should  allow for  continued operation
                even   with   a  slightly   diminished   DC   bus  voltage   \approx
                \SIrange{4}{5}{\percent} lower than nominal~\cite{Maksimovic2012}.
                Considering a maximum permissible dip of \SI{4}{\percent} in the bus
                voltage \ie{} \SI{336}{\volt}, the cell's capacity may be refined as
                \begin{align}
                    \text{\si{\amphour} capacity of Bolt cell} & = \frac{\text{Pack Energy } (\si{\watthour})}{\text{Lowest pack voltage (V)} \times {\text{Number of cells in parallel}}} \\
                    {}                                         & = \frac{60000}{336 \times 3} \\
                    {}                                         & = \SI{59.52}{\amphour}
                \end{align}

                The reference cell's \si{\amphour} capacity is therefore rounded to
                \textbf{\SI{60}{\amphour}}\footnote{In the interest of maintaining consistency, this computed capacity is
                    retained for the cell used in
                    \crefrange{ch:spmanalysis}{ch:newelectrolytemodel} of this thesis. This also explains the use of \SI{60}{\ampere} for
                    the simulations demonstrating energy/power trade-off of
                    \cref{sec:energypowertradeoffdemo} since this current level corresponds to
                the 1C-rate of the cell.}.

            \item \hypertarget{celllowercutoff}{\textbf{Lower cutoff voltage for cells}}

                In  this  layer  optimisation  work, following  the  assumptions  of
                \cref{subsec:layeroptassumptions},  the  overall pack  configuration
                remains  unchanged  and  independent   of  layers  within  a  pouch.
                This  implies that  the undervoltage  threshold for  DC bus  voltage
                throughout  this   work  shall  remain  fixed   at  \SI{336}{\volt}.
                Therefore, with  96~series connected  cells in  a string,  the lower
                cut-off voltage for an  individual cell is \textbf{\SI{3.5}{\volt}}.
                This  value is  reported in  \cref{tbl:lcoSimParamslayeropt} and  is
                used as a termination condition  for all simulations as explained in
                \cref{sec:layeroptframework}.

        \end{itemize}

    \item \textbf{Computation of electrochemically overall active surface area for reference cell}

        For             the            cell             properties            in
        \crefrange{tbl:lcoSimParamsSPMp2d}{tbl:lcoSimParamslayeropt},        the
        majority                of                 parameters                are
        sourced           from          Subramanian~\etal~\cite{Subramanian2009}
        and                Northrop~\etal~\cite{Northrop2011}.                In
        Northrop~\etal~\cite{Northrop2011},  the   \emph{current  density}  that
        corresponds to  a 1C-rate discharge  of a  cell with this  parameter set
        is  reported to  be \approx  \SI{30}{\ampere\per\meter\squared}. In  the
        author's carefully  designed numerical simulations (very  slow discharge
        at C/500 from fully charged state until charge depletion), this value is
        refined to \SI{29.23}{\ampere\per\meter\squared}.

        The task of determining the electrochemically active overall surface
        area of the reference cell is now straightforward
        \begin{align}
            \text{Overall surface area of reference cell}, A_\text{refcell} &= \frac{\text{Cell capacity (\si{\amphour})}}{\text{1C-rate density (\si{\ampere\per\meter\squared})}} \\
                                                                            &= \frac{60}{29.23} \\
                                                                            &= \SI{2.053}{\meter\squared}
        \end{align}

        This  value  is  listed  in  \cref{tbl:lcoSimParamsSPMp2d}  for  use  in
        \crefrange{ch:spmanalysis}{ch:newelectrolytemodel},  but  is  \emph{not}
        used  in  this  layer  optimisation   work.  This  is  because,  as  the
        number  of  layers change,  the  overall  surface  area changes  as  per
        \cref{eq:overallarea}.

    \item \textbf{Setting the pouch height for the reference cell}

        Although the  official press release~\cite{GMBoltBatteryDims}  from the
        manufacturer  of this  \gls{bev} contains  data on  the cross-sectional
        geometry  of the  cell,  it does  not report  its  height. Hence,  this
        information  needs to  be assumed  by extrapolation  from an  alternate
        source wherein the conditions are similar and therefore, can be applied
        for this reference cell.

        The  review   article  by   Gr\"oger~\etal~\cite{Groger2015}  discusses
        various    states    of    the    art   in    energy    densities    of
        electrode   materials  used   in  various   lithium  ion   chemistries.
        At   the   time   of   its   publication,   the   areal   capacity   of
        cells  were  \approx  \SI{2.0}{\milli\amphour\per\centi\meter\squared}.
        These    authors   recommended    an   areal    capacity   target    of
        \SI{4.0}{\milli\amphour\per\centi\meter\squared} for  future automotive
        applications. This  publication also  considers the aspect  of stacking
        layers  into  pouches  of  certain  geometries.  In  particular,  table
        \romanletter{4}  of Gr\"oger~\etal~\cite{Groger2015}  considers a  pouch
        of  \SI{10}{\milli\meter} height,  for which  the aforementioned  areal
        capacities were calculated.

        In       the        case       of       the        reference       cell
        under       consideration,       the      areal       capacity       is
        \begin{equation}
            \text{Areal capacity of reference cell } = \frac{\SI{60000}{\milli\amphour}}{\SI{20527}{\centi\meter\squared}}  = \SI{2.92}{\milli\amphour\per\centi\meter\squared}
        \end{equation}
        which is close  to the desired value in automotive  applications as per
        the recommendations  in Gr\"oger~\etal. Considering that  the reference
        cell is  drawn from  the high energy  density Chevrolet  Bolt \gls{bev}
        cell,  a pouch  height of  \SI{10}{\milli\meter} is  justifiable for
        this task  and is  reported in \cref{tbl:lcoSimParamslayeropt}.  As per
        the assumptions discussed in \cref{subsec:layeroptassumptions}, this is
        held constant throughout the layer optimisation process.

    \item \hypertarget{stackthickness}{\textbf{Compute stack thickness of reference cell}}

        In a  pouch of given height,  the available space to  accommodate layers
        therein is restricted  by a number of factors. For  instance, the wiring
        from the  current collectors  to the tabs,  protective elements  such as
        fuses etc.\ consume space. For instance, the pouch material itself has a
        finite  thickness  and  hence  after  accounting  for  this,  the  stack
        thickness  available  for  layer  placement  is  lower  than  the  pouch
        thickness.
        \begin{align}
            \text{Stack thickness}, L_\text{stack} & = \text{Pouch height} - 2\times \text{pouch thickness} \\
            \text{Stack thickness}, L_\text{stack} & = H_\text{pouch} - 2 T_\text{pouch} \\
            L_\text{stack}(\si{\milli\meter})      & = 10.0 - 2\times(\num{160e-3}) \\
            L_\text{stack}                         & = \SI{9.68}{\milli\meter}
        \end{align}

        The value of stack thickness of  the reference cell is held constant for
        all layer choices used throughout the entire layer optimisation process.

    \item \textbf{Determination of number  of layers within  reference cell}

        The  next  step  is  to  determine  the  number  of  layers  within  the
        reference pouch  cell~$n_\text{refcell}$.

        The thickness of a complete electrochemical sandwich multiplied by the
        number of layers yields the total stack height
        \begin{equation}\label{eq:stackheightrefcell}
            n_\text{refcell}\, (l_\text{Al} + l_\text{pos} + l_\text{sep} + l_\text{neg} + l_\text{Cu}) = L_\text{stack}
        \end{equation}

        The product of  layers and the thickness of  an electrochemical sandwich
        cannot exceed the  overall stack height. This implies  that the equality
        in \cref{eq:stackheightrefcell} is  to be changed to  an inequality with
        the upper  bound of the  computation on the  \gls{lhs} set to  the stack
        height.
        \begin{align}
            n_\text{refcell}\, (l_\text{Al} + l_\text{pos} + l_\text{sep} + l_\text{neg} + l_\text{Cu}) & \le L_\text{stack} \\
            n_\text{refcell}                                                                            & \le \frac{L_\text{stack}}{l_\text{Al} + l_\text{pos} + l_\text{sep} + l_\text{neg} + l_\text{Cu}}\label{eq:stackheightrefcellmod}
        \end{align}

        Since fractional layers do not have  any physical meaning, the number of
        layers that  can be  accommodated within  any pouch  must be  an integer
        quantity. Therefore,  $n_\text{refcell}$ is  computed as the  `floor' of
        the quantity in the \gls{rhs} of \cref{eq:stackheightrefcellmod}
        \begin{align}
            n_\text{refcell} &= \floor*{\frac{L_\text{stack}}{l_\text{Al} + l_\text{pos} + l_\text{sep} + l_\text{neg} + l_\text{Cu}}} \\
            {} &= \floor*{\frac{9.68}{(15 +  72 + 25 + 88 + 10) \times 10^{-3}}} \\
            n_\text{refcell} &= 46
        \end{align}

        The reference cell is thus deemed to consist of 46~layers.

    \item \textbf{Computation of surface area per layer}

        Substituting  the values  of  $n_\text{refcell}$ and  $A_\text{refcell}$
        into  \cref{eq:overallarea}, the  electrochemically active  surface area
        per layer is obtained as
        \begin{align}
            A_\text{elec} & = \frac{A_\text{refcell}}{n_\text{refcell}} \\
            {}            & = \frac{2.053}{46}                          \\
        \end{align}

\end{enumerate}


%\subsection{Modelling Platform and Preconditioning}

%Couple of statements  about why LIONSIMBA was chosen as  the modelling platform
%for implementing the p2d dynamics. The  cell parameters used are shown in table
%xx. This cell is henceforth known as the LIONSIMBA cell or Northrop cell.

%Discuss the  missing elements  in LIONSIMBA  only with  respect to  the present
%problem at hand, \viz{the stoichiometries}.

%\subsubsection*{Stoichiometry Augmentation}

%Discuss the problem first. How LIONSIMBA started always at 85.51 percentage and
%needed to  do a  discharge down to  zero percent before  having the  ability to
%charge. For this project, stoichiometries  are vital for capacity determination
%and  the 1C  current density.  Explain how  stoichiometries were  refined until
%cut-off for  infinitesimal bleeding discharge current  achieved. Noted relevant
%values. Explain refinement  of how approximate capacities  reported by Northrop
%and Subramanian were  refined precisely. Explanation of  remnant capacities and
%stoichiometries computation. Explanation of  1C current density.parameters init
%capacity computation code.

%\subsection{Layer Assembly within Pouch Cells}
%With  the  help  of  Northrop's   layer  assembly  figure,  explain  the  layer
%configuration/arrangement within a pouch cell. The next task is then identified
%as computing the number of layers within the reference pouch cell.

%\subsubsection*{Number of Layers of LIONSIMBA aka Northrop cell}
%Krishna came up with the idea of using integer optimisation for this task. The
%software MIDACO was also selected by Krishna and explained to Ian. The MIDACO
%result of the number of layers within the standard cell was now available.

%\subsubsection*{Computation of Surface Area per face, \protect{$A_\text{cell}$}}
%Show the simple algebraic computation of overall surface area $A$ and the
%per-face area $A_\text{cell}$. Explain how the area per face shall be a key
%quantity in the layer optimisation framework discussed later on.

%Layerphoto showing face areas and anode/cathode verhand etc will be shown here

%This concludes the augmented set of parameters added by the author to the basic
%parameter set  of the DFN  model. The added  numerical value of  parameters are
%summaried in table xx. Lots The layer optimisation framework and assumptions is
%described next.

\subsection{Modification of Standard \glsfmtshort{dfn} Model to Handle Power Inputs}\label{sec:innatepowerinput}

%%Figure showing power input

\subsection{Hybrid Spectral-\glsfmtshort{fv} Scheme}\label{sec:hybridfv-spectral}

Fast  and  accurate  estimation  of   the  solid  phase  lithium  concentration,
particularly  its   value  at   the  surface  of   electrode  particles   is  an
inherent  requirement   of  the   layer  optimisation  procedure   presented  in
\cref{sec:layeroptframework}.   The  high   power  densities   to  be   handled,
particularly  at  low   layer  counts  necessitate  this   requirement.  It  has
been   acknowledged  that   solid-phase  concentration   calculations  employing
polynomial    approximations   lack    fidelity    at   high    charge/discharge
rates~\cite{Santhanagopalan2006}.  Hence,  a  conventional  full-order  solution
based on Fick's law of diffusion is required for this layer optimisation task.

With full-order  solid phase  diffusion dynamics,  applying the  \gls{fv} scheme
(that has been  employed in LIONSIMBA~v1.0x to  discretise all through-thickness
\gls{pde}s of the \gls{p2d} model) results  in a very large system of equations.
This is due to the requirement of using a high radial node density per spherical
particle for improved accuracy. Consequently, the computational cost is high and
simulation  runtime  becomes prohibitive  when  exploring  the search  space  of
all  possible  layer configurations.  Moreover,  with  a cell-centered  \gls{fv}
discretisation,  it is  non-trivial to  directly apply  the ionic  flux boundary
condition at the particle surface, since it involves extrapolation from at least
two other nodes within the particle. While such extrapolations are acceptable in
the axial  dimension --- particularly  with high node densities  providing small
values of $\frac{\Delta x}{2}$ --- they are undesirable in the radial dimension.
This  is because  cell's  open  circuit and  terminal  voltages strongly  depend
on  the  concentration  at  the  particle  surface.  Spectral  methods  offer  a
combination  of high  accuracy and  speed while  permitting the  use of  a lower
number  of radial  discretisation nodes.  To implement  a spectral  scheme on  a
non-periodic  domain,  a  Chebyshev discretisation~\cite{Trefethen2000}  may  be
applied.  Bizeray~\etal{}~\cite{Bizeray2015} discretised  all  of the  \gls{p2d}
model  equations using  this approach.  However, this  entails a  bi-directional
mapping of all  variables between the physical and  Chebyshev domains, incurring
computational overhead.

In this chapter, a hybrid formulation of the \gls{p2d} model is proposed wherein
a standard \gls{fv} scheme  in the axial dimension and a  spectral scheme in the
radial domain  are used.  Exploiting this  natural separation  of the  axial and
radial domains enables to ---
\begin{enumerate*}[label=\roman*)]
    \item retain  the   ability  to  easily   couple  the   molar  flux  density   at  the particle  surface  through  reformulation  of the  boundary  conditions  of  the solid  diffusion \gls{pde},  and
    \item solve  for  solid-phase  lithium  concentration  in  the  Chebyshev  domain  and locally transform  to physical  domain, without requiring  system-wide Chebyshev reformulations.
\end{enumerate*}
Although the proposed implementation does  not \emph{globally} employ a spectral
scheme, the  combined beneficial  effects of  radial-domain spectral  scheme and
automatic differentiation of system equations using CasADi~\cite{Andersson2013b}
facilitates rapid simulation, enabling  layer optimisation on short time-scales.
\Crefrange{eq:defineChebNodes}{eq:solidDiffEqChebDomain}   detail  the   steps
leading to  the reformulated solid  phase diffusion and its  associated boundary
condition in the Chebyshev domain.

The $N_\text{r}$~Chebyshev collocation nodes, defined on a 1D mesh in the radial
direction, are given by \cref{eq:defineChebNodes}~\cite{Trefethen2000}.
\begin{equation}\label{eq:defineChebNodes}
    \widetilde{r} = \cos\left(\frac{i\pi}{N_\text{r}}\right), \qquad i = 0, 1, \dots N_\text{r} \quad \widetilde{r} \in [-1, 1]
\end{equation}

Assuming  constant diffusivity,  and expanding  the derivative  in the  standard
form of  the Fickian  spherical diffusion equation  (see \cref{eq:dfnsoliddiff})
for  each particle,  we  obtain  \cref{eq:quotientappliedpde}, presented  along
with  its   Neumann  boundary   conditions \cref{eq:quotientappliedpdeBCcentre}
and \cref{eq:quotientappliedpdeBCsurface}.  $j$~is   the  molar   flux  density
(\si{mol.m^{-2}.s^{-1}}) and $R_\text{p}$ is the particle radius (\si{m}).
\begin{subequations}\label{eq:quotientappliedpde}
    \begin{align}
        \frac{\partial c_\text{s}}{\partial t} &= D^\text{eff}_\text{s} \left( \frac{\partial}{\partial r} \frac{\partial c_\text{s}}{\partial r} + \frac{\partial^2 c_\text{s}}{\partial r^2} \right) \qquad r \in [0, R_\text{p}]\tag{\ref*{eq:quotientappliedpde}}\\
        \diffp{c_\text{s}}{r}{\mathrlap{r = 0}} &= 0\label{eq:quotientappliedpdeBCcentre}\\
        D^\text{eff}_\text{s}\diffp{c_\text{s}}{r}{\mathrlap{r = R_\text{p}}} &= -j\label{eq:quotientappliedpdeBCsurface}
    \end{align}
\end{subequations}

Mapping $r \in [0,R_\text{p}] \mapsto \widetilde{r} \in [-1, 1]$,
\begin{equation}\label{mappingChebDomain}
    r = \frac{R_\text{p}}{2}(\widetilde{r} + 1)
\end{equation}

Applying \cref{mappingChebDomain} to
\crefrange{eq:quotientappliedpde}{eq:quotientappliedpdeBCsurface} whilst
retaining $c_\text{s}$ in the physical space yields
\crefrange{eq:solidDiffEqChebDomain}{eq:solidDiffEqChebDomainBCsurface}.

\begin{subequations}\label{eq:solidDiffEqChebDomain}
    \begin{align}
	    \frac{\partial c_\text{s}}{\partial t}                         & = 4 \frac{D^\text{eff}_\text{s}}{R_\text{p}^2} \left( \frac{2}{\widetilde{r} + 1} \frac{\partial c_\text{s}}{\partial \widetilde{r}} + \frac{\partial^2 c_\text{s}}{\partial {\widetilde{r}}^2} \right)\tag{\ref*{eq:solidDiffEqChebDomain}} \\
        \diffp{c_\text{s}}{\widetilde{r}}{\mathrlap{\widetilde{r}=-1}} & = 0\label{eq:solidDiffEqChebDomainBCcentre}                                                                                                                                                                                                                \\
        2 \frac{D^\text{eff}_\text{s}}{R_\text{p}} \diffp{c_\text{s}}{\widetilde{r}}{\mathrlap{\widetilde{r}=1}} & = -j\label{eq:solidDiffEqChebDomainBCsurface}
    \end{align}
\end{subequations}

During  the iterative  solution process,  the spatial  gradients of  solid phase
lithium concentration  in \cref{eq:solidDiffEqChebDomain}  in this case  are not
computed through  an explicit  differentiation procedure  as usual,  but instead
evaluated by pre-multiplying  the concentration values at  the collocation nodes
by a Chebyshev differentiation matrix. This particular aspect is responsible for
the inherent reduction of simulation  runtime achieved by introducing a spectral
method. In the updated version  of LIONSIMBA~v2.0 (created specifically for this
layer  optimisation work),  differentiation matrices  of suitable  dimensions as
well as the Chebyshev collocation nodes  are generated using the MATLAB function
\texttt{cheb.m}~\cite{Trefethen2000}.



\section{Computational Framework}\label{sec:layeroptframework}
% -*- root: ../../main.tex -*
%!TEX root = ../../main.tex
% vim:textwidth=80 fo=cqt conceallevel=0

% \afterpage{\clearpage}

The  methodology adopted  by the  proposed layer  optimisation framework  can be
explained by  using the  flow diagram in  \cref{fig:fig_strategy_schematic}. The
power  handled by  the cell  during normal  operation (evaluated  by considering
various  drivecycles)  is   much  lower  than  that  experienced   by  the  cell
during  acceleration  (discharge)  and  fast-charging  (charge)  scenarios.  See
\cref{sec:resultslayeropt} for  a brief  summary of the  peak and  median powers
across all  standard drivecycles.  Therefore, from a  design perspective,  it is
sufficient  to consider  the power  requirements  for these  two extreme  cases.
Hence,  the schematic  in  \cref{fig:fig_strategy_schematic} can  be studied  by
broadly dividing the flow diagram into two parts ---
\begin{enumerate*}[label=\roman*)]
    \item an acceleration pathway primarily consisting of blocks shaded in grey, and
    \item a fast-charging pathway predominantly composed of blocks shaded in cyan.
\end{enumerate*}

\begin{figure}[p]
    \begin{minipage}[t]{\textwidth}
        \centering
        \includegraphics[angle=90, width=\textwidth]{fig_master_flow_diagram}
        \captionsetup{labelsep=note}
        \caption
        [%
        Flow diagram depicting an overview of the proposed layer optimisation methodology
        for Li-ion pouch cells.
        ]%
        {%
            Flow diagram depicting an overview of the proposed layer optimisation methodology
            for Li-ion pouch cells.
        }%
        \label{fig:fig_strategy_schematic}
        \mpfootnotes[1]
        % \vspace*{1.125cm}
        \vspace*{0.7225cm}
        \footnote{This figure was created by \mbox{Krishnakumar Gopalakrishnan} who
            asserts copyright, with intellectual contributions from and the right to
        use asserted by \mbox{Ian Campbell}.}
    \end{minipage}
\end{figure}

As explained in the legend key of \cref{fig:fig_strategy_schematic}, blocks with
a light  grey border  represent input  data/parameters for  computations. Blocks
with a standard black border  represent computations common to both acceleration
and fast  charging pathways. The design  output is given by  the double-bordered
block at the bottom centre  of \cref{fig:fig_strategy_schematic}. Other types of
blocks  and arrows  are  appropriately listed  in  the legend  key.  To aid  the
understanding of the  layer optimisation framework, the reader  is encouraged to
correlate  the narrative  in this  section  with the  blocks and  arrows in  the
schematic.



\subsection{Acceleration pathway}

The computations for acceleration-based layer  optimisation begins at the anchor
block labelled `Start Acc.\ Calcs.'.

\subsubsection*{Determination of acceleration rate, final speed and acceleration time}

The  first   step  is  the  determination   of  the  rate  of   acceleration  to
be  used  for  the  computing   the  power  requirements  for  accelerating  the
\gls{xeV}  from  standstill.  For  the  sake of  claiming  green  subsidies  and
other  regulatory reasons,  vehicular  standards for  electrified transport  are
codified by  various regulatory agencies  and standardisation bodies  (\eg{} the
SAE~J1772 standard~\cite{Sae2010}).  These typically specify a  minimum required
acceleration  rate for  the vehicle  under consideration  to be  certified as  a
road-faring electric vehicle. Manufacturers have their own specifications, which
typically exceed the minimum standards. However, for certain classes of electric
vehicles such  as golf  carts, the  manufacturer-specified standards  might fall
below that of a  roadworthy electric vehicle. In this case,  this thesis errs on
the side of conservative design by choosing the higher of the two values.

The   acceleration   rate   is   taken   to   be   the   simple   ratio   of   a
pre-determined  final  speed~$v_\text{f}$  to  the time  taken  to  attain  that
speed  from  standstill~$t_\text{f}$.  The  manufacturer-specified  acceleration
rate~$a_\text{man.}$ is compared against the minimum acceleration rate specified
by vehicular  standards~$a_\text{std.}$. The two \gls{spdt}  switches assign the
final speed~$v_\text{f}$  and acceleration time~$t_\text{f}$ to  the values from
the  appropriate source  depending on  which of  the two  acceleration rates  is
higher.

\subsubsection*{Computation of acceleration power at the wheels}

The  next  step  is  to  calculate   the  acceleration  power  required  at  the
driving wheels~$P_\text{w}$.  Using the  governing equations from  basic vehicle
dynamics~\cite{Maksimovic2012}, the power at the wheels is given by
\cref{eq:wheelpower}.
\begin{subequations}\label{eq:wheelpower}
    \begin{align}
        P_\text{w}     & = P_\text{mass} + P_\text{drag} + P_\text{roll} + P_\text{grade} \tag{\ref*{eq:wheelpower}}                  \\
        P_\text{mass}  & = \frac{1}{2} \frac{M_\text{v}(n)}{t_\text{a}} \left(v_\text{b}^2 + v_\text{f}^2\right) \label{eq:masspower} \\
        P_\text{drag}  & = \frac{1}{2} \left(\rho_\text{air} C_\text{d} A_\text{v}v_\text{f}^3\right) \label{eq:dragpower}            \\
        P_\text{roll}  & = C_\text{r} M_\text{v}(n) g v_\text{f} \label{eq:rollpower}                                                 \\
        P_\text{grade} & = M_\text{v}(n) Z g v_\text{f} \label{eq:gradepower}
    \end{align}
\end{subequations}

The  individual  components  contributing   to  the  summation  for  wheel-power
computation   presented   in    \cref{eq:wheelpower}   is   briefly   explained.
$P_\text{mass}$~represents  the  power  required  to  accelerate  the  vehicle's
mass.  $P_\text{drag}$~denotes the  power  required to  overcome air  resistance
while $P_\text{roll}$~represents  that required to overcome  rolling resistance.
Finally,  $P_\text{grade}$~represents the  power  required to  negotiate a  road
gradient.  Except  for  $M_\text{v}(n)$  which  is  described  next,  all  terms
in  the  \gls{rhs}  of  \crefrange{eq:masspower}{eq:gradepower}  are  constants.
Each  of  these  terms   is  explained  in  tables~\ref{tbl:CommonVehicleParams}
and~\ref{tbl:UniqueVehicleParams}  along with  their  numerical  values used  in
simulation.

\subsubsection*{Computation of layer-dependent vehicle mass}

Changing  the layer  count used  in  each cell  changes  the mass  of the  cell.
This  in-turn  affects the  mass  of  the  pack,  which further  influences  the
overall  vehicle  mass.  Thus,  for precise  computation  of~$P_\text{mass}$  in
\cref{eq:masspower},  $M_\text{v}(n)$  is  the  vehicle's  mass  computed  as  a
function of number of layers~$n$.

In  the  schematic  of~\cref{fig:fig_strategy_schematic},  this  layer-dependent
calculation of  mass of all cells  in the pack~$M_\text{cells}$ is  shown by the
product of  the signal  labelled $m_\text{cell}$ and  the triangular  gain block
representing the overall number of cells in the pack~$n_\text{cells}$.
\begin{equation}\label{eq:massofallcells}
    M_\text{cells} = n \times m_\text{cell}
\end{equation}

The   mass    of   the   vehicle    is   given    by   the   sum    of   chassis
mass~$M_\text{c}$,  vehicle  payload  $M_\text{p}$, pack  overhead  $M_\text{o}$
and  the  layer-dependent  mass  of  all  cells  in  the  pack  $M_\text{cells}$
shown   in   \cref{eq:massofallcells}.   The    computation   of   mass   of   a
single   cell    $m_\text{cell}$   is    detailed   in   the    section   titled
\hyperlink{href:layerdependentcellmass}{`Computation  of   layer-dependent  cell
mass'}.

\subsubsection*{Computation of acceleration power density per layer}

Since the \gls{p2d} equations of the \gls{dfn} model are based upon a normalised
unit area and is  applicable only to each electrochemical layer,  the goal is to
compute the  power density experienced  by each layer. This  is arrived at  by a
sequence of simple scaling steps.

Firstly, the  power demanded at  the pack terminals  is computed by  scaling the
power  at the  wheels  by the  efficiency  of the  drivetrain.  As explained  in
\cref{subsec:layeroptassumptions},  the  drivetrain  consists  of  a  number  of
components, the efficiencies of which depend on the operating point (such as the
speed and torque  of the electric motor, current drawn  by the power electronics
etc.). Following the  assumptions detailed in \cref{subsec:layeroptassumptions},
a  single lumped  efficiency~$\eta_\text{dt}$ can  be used  for the  powertrain.
The  pack  power  demand  $P^\text{acc}_\text{batt}$  is  then  divided  by  the
number  of cells  in the  pack $n_\text{cells}$  to obtain  the power  demand at
the  input of  the cell's  terminals~$p^\text{acc}_\text{cell}$. For  each layer
choice~$n$, the electrochemically active surface area is computed using by using
\cref{eq:overallarea},  wherein the  electrochemically active  surface area  per
layer~$A_\text{elec}$ listed in \cref{tbl:lcoSimParamslayeropt} is held constant
throughout.  Finally, the  acceleration power  density per  layer~$p^\text{acc}$
is  computed  by  dividing  $p^\text{acc}_\text{cell}$ by  the  overall  surface
area~$A_\text{cell}$.

\subsubsection*{\hypertarget{href:layerdependentcellmass}{Computation of layer-dependent cell mass}}\label{sec:massofonecell}

\subsection{Electrode thickness ratio for capacity balancing}\label{sec:electroderatio}

A key  idea of the layer  optimisation scheme is that  for computing the
physical  lengths  of electrode  as  a  function  of number  of  layers,
the  ratios  of  their thicknesses~$l_\text{ratio}$  is  held  constant.
This  coefficient is  germane to  the concept  of capacity  balancing of
electrodes  to equalise  their loading and is computed as follows.

Equating  the active material volume of both electrodes,
\begin{equation}
    A_\text{elec,pos}l_\text{pos}  \varepsilon_\text{s,pos} = A_\text{elec,neg}l_\text{neg}  \varepsilon_\text{s,neg} \label{eq:electrodeCapacity}\\
\end{equation}

Neglecting  overhangs   of  the  negative  electrode   (typically  below
$\SI{2}{\milli\meter}$ to  avoid plating  at the edges),  both electrode
materials have the same cross-sectional area~$A_\text{elec}$. Therefore,
\cref{eq:electrodeCapacity} reduces to
\begin{align}
    \cancel{A_\text{elec}}l_\text{pos}  \varepsilon_\text{s,pos} & = \cancel{A_\text{elec}}l_\text{neg}  \varepsilon_\text{s,neg}  \\
    l_\text{pos}  \varepsilon_\text{s,pos}                       & = l_\text{neg}  \varepsilon_\text{s,neg}\label{eq:electrodeCapequalarea}
\end{align}

For the reference cell under consideration, the length and breadth of the cell's
pouch is  obtained from the \gls{bev}  manufacturer~\cite{GMBoltBatteryDims} and
are listed in \cref{tbl:lcoSimParamslayeropt}.

% % double-check if electrode area or overall area or what ? And uncomment later
% on


% As  a  first-order  approximation  the
% product of  these dimensions can  be assumed  to be the  active material
% cross-sectional  area   (although  in  practice,  the   planar  area  of
% the  stack  needs to  be  slightly  smaller  to be  accommodated  inside
% the  pouch).  Finally,  in  line   with  the  assumptions  discussed  in
% \cref{subsec:layeroptassumptions},  similar  to  the pouch  height,  the
% cross-sectional geometry (length  and breadth) of the cell  is also held
% constant throughout the layer optimisation process.

Owing to the reasons  outlined in \cref{subsec:layeroptassumptions}, the
volume fractions of the electrode  materials are assumed to be constant,
which  implies  that their  ratio  is  also  a constant.  The  electrode
thickness ratio~$l_\text{ratio}$ is obtained as
\begin{alignat}{2}
    l_\text{ratio} & = \frac{l_\text{neg}}{l_\text{pos}}                                                                                  &  & \qquad\text{(by definition)}                                          \\
    {}             & = \frac{\varepsilon_\text{s,pos}}{\varepsilon_\text{s,neg}}                                                          &  & \qquad\text{(rearranging \cref{eq:electrodeCapequalarea})}           \\
    {}             & = \frac{1-\varepsilon_\text{pos} - \varepsilon_\text{fi,pos}}{1-\varepsilon_\text{neg} - \varepsilon_\text{fi,neg}}  &  & \qquad\text{(by definition)}                                          \\
    {}             & = \frac{1- 0.385 - 0.025}{1 - 0.485 - 0.033}                                                                         &  & \qquad\text{(substituting values from \cref{tbl:lcoSimParamsSPMp2d})} \\
    l_\text{ratio} & = 1.22\label{eq:electrodeThicknessRatio}
\end{alignat}

% Do not forget to quote the value of cs_sat from layer opt paper table and show
% calculation inline. Might even go into the code

\subsection{Power inputs considered}

P2D model  equations are developed  for bi-directional input power.  xEV battery
pack has  constant power input,  if the  installed power capability  of charging
stations  is  fully  utilised.  Equation  (1)  describes  vehicle  dynamics  for
acceleration  and  power  demand.  For  fast  charging,  the  power  electronics
components of all grid chargers possess  a peak power delivery rating.


\section{Results and Discussion}\label{sec:resultslayeropt}
% -*- root: ../../main.tex -*
%!TEX root = ../../main.tex
% vim:textwidth=80 fo=cqt conceallevel=0


At the outset, it is worth mentioning that the focus of this chapter
is on the layer optimisation \emph{methodology} itself. The results as
such do not stand alone outside of the modelling universe with all the
inherent assumptions discussed thus far. Presently, the value added by
this work is its ready adaptability to industry through its modular
design. A numerical implementation in the form of a toolbox\footnote{The
\gls{bold} Toolbox is made available for download from GitHub. \\
\mbox{\href{https://github.com/ImperialCollegeESE/BOLD_Toolbox}{\includegraphics
[width=0.025\textwidth]{github.pdf}}} \url{
https://github.com/ImperialCollegeESE/BOLD_Toolbox}} is also provided which
is immediately available for download and use by industry. This author
recommends that until the tool matures, cell manufacturers substitute their own
parameters and adjust other numerical coefficients suitably so that the toolbox
supplements, rather than supplants their empirical layer designs. Hence, the
results presented in this section must be interpreted in the backdrop of the
context within which the methodology was developed implying that the reader
must consciously strive to interpret all numerical values in relative terms of
magnitude. To aid this thought process, this author chooses to deliberately
limit the discussion on the absolute magnitude of numbers presented here.

%\subsection{Modelling Platform and Preconditioning}

%Couple of statements  about why LIONSIMBA was chosen as  the modelling platform
%for implementing the p2d dynamics. The  cell parameters used are shown in table
%xx. This cell is henceforth known as the LIONSIMBA cell or Northrop cell.

%Discuss the  missing elements  in LIONSIMBA  only with  respect to  the present
%problem at hand, \viz{the stoichiometries}.

%%\subsubsection*{Stoichiometry Augmentation}

%%Discuss the problem first. How LIONSIMBA started always at 85.51 percentage and
%%needed to  do a  discharge down to  zero percent before  having the  ability to
%%charge. For this project, stoichiometries  are vital for capacity determination
%%and  the 1C  current density.  Explain how  stoichiometries were  refined until
%%cut-off for  infinitesimal bleeding discharge current  achieved. Noted relevant
%%values. Explain refinement  of how approximate capacities  reported by Northrop
%%and Subramanian were  refined precisely. Explanation of  remnant capacities and
%%stoichiometries computation. Explanation of  1C current density.parameters init
%%capacity computation code.

%%% -*- root: ../../main.tex -*-
%!TEX root = ../../main.tex
% vim:nospell

\begin{table}[!htbp]
    \small
    \caption[%
    System-level simulation conditions \& thermal parameters of  an \glsfmtshort{lco} cell
    ]%
    {%
        Cell   parameters   and   system   conditions  for   a   simulating   an
        \glsfmtshort{lco} cell  with the  \gls{dfn} electrochemical model  and a
        lumped thermal model. The parameters  presented here when augmented with
        the  values  of  the  kinetic, geometric  and  transport  properties  of
        the  cell (from  \cref{tbl:lcoSimParamsSPMp2d}  represents the  complete
        information  required for  all  simulations in  this layer  optimisation
        framework.
    }%
    \label{tbl:lcoSimParamslayeropt}
    \vspace{-2.6229525pt}
    \begin{threeparttable}
        \centering
        \textbf{System Conditions} \\ \smallskip
        \begin{varwidth}[t]{0.48\linewidth}
            \begin{tabular*}{\textwidth}{@{} l @{\extracolsep{\fill}} S[table-format=1.2,table-space-text-pre=\Tnote{a} ,table-align-text-pre=false] @{}}
                \toprule
                \multicolumn{1}{@{}l}{Parameter} \\
                \midrule

                Lower cutoff cell voltage, $V_\text{min}$ (\si{\volt}) & \Tnote{a} 3.50   \\
                Upper cutoff cell voltage, $V_\text{max}$ (\si{\volt}) & \Tnote{b} 4.22   \\

                \bottomrule
            \end{tabular*}
        \end{varwidth}
        \hfill
        \begin{varwidth}[t]{0.48\linewidth}
            \begin{tabular*}{\textwidth}{@{} l @{\extracolsep{\fill}} S[table-format=2.2,table-space-text-pre=\Tnote{a} ,table-align-text-pre=false] @{}}
                \toprule
                \multicolumn{1}{@{}l}{Parameter} \\
                \midrule

                Target cell SOC for fast charge, $z^\ast$ \si{(\%)}                & \Tnote{c} 80.00 \\
                Cell upper temperature limit, $T_\text{max}$ \si{(\degreeCelsius)} & \Tnote{d} 55.00 \\

                \bottomrule
            \end{tabular*}
        \end{varwidth}

        \medskip
        \begin{tabular*}{\textwidth}{@{} l @{\extracolsep{\fill}} r @{}}
            \multicolumn{2}{c}{\textbf{Geometric Parameters}} \\
            \toprule
            \multicolumn{1}{@{}l}{Parameter} \\
            \midrule
            Surface area of pos.\ \& neg.\ electrode overlap within a layer, {$A_\text{elec}$} \si{(m^2)} & \textsuperscript{b}\num{4.19e-2}   \\
            Exterior pouch length, $L_\text{pouch}$ \si{(m)}                                              & \textsuperscript{e}\num{332.74e-3} \\
            Exterior pouch width, $W_\text{pouch}$ \si{(m)}                                               & \textsuperscript{e}\num{99.06e-3}  \\
            Exterior pouch height, $H_\text{pouch}$ \si{(m)}                                              & \textsuperscript{f}\num{10.00e-3}  \\
            Pouch material thickness, $T_\text{pouch}$ \si{(m)}                                           & \textsuperscript{g}\num{160.00e-6} \\
            Stack thickness, $L_\text{stack}$ \si{(m)}                                                    & \textsuperscript{r}\num{9.68e-3}  \\
            \bottomrule
        \end{tabular*}
        \medskip
        \centering \textbf{Thermal Parameters} \\ \smallskip
        \resizebox{\textwidth}{!}{%
            \begin{tabular}{@{} l S[table-format=4.0,table-space-text-pre=\Tnote{m} ,table-align-text-pre=false] S[table-format=4.1,table-space-text-pre=\Tnote{m} ,table-align-text-pre=false] S[table-format=4.1,table-space-text-pre=\Tnote{m} ,table-align-text-pre=false] S[table-format=4.2,table-space-text-pre=\Tnote{m} ,table-align-text-pre=false] S[table-format=4.0,table-space-text-pre=\Tnote{m} ,table-align-text-pre=false] S[table-format=4.1,table-space-text-pre=\Tnote{m} ,table-align-text-pre=false] S[table-format=4.1,table-space-text-pre=\Tnote{m} ,table-align-text-pre=false] @{}}
                \toprule
                \multicolumn{1}{@{}l}{Parameter} & \multicolumn{1}{c}{Al.\ CC} & \multicolumn{1}{c}{Pos} & \multicolumn{1}{c}{Sep} & \multicolumn{1}{c}{Neg} & \multicolumn{1}{c}{Cu.\ CC} & \multicolumn{1}{c}{\ch{LiPF_6}} & \multicolumn{1}{r@{}}{Pouch}\\
                \midrule

                Sp.\ heat capacity, $c_j$ (\si{\joule\per\kilogram\per\kelvin})   & \Tnote{h} 903  & \Tnote{h} 1269.2 & \Tnote{h} 1978.2 & \Tnote{h} 1437.4 & \Tnote{h} 385  & \Tnote{h} 2055.1 & \Tnote{i} 1464.8 \\
                Density, $\rho_j$ (\si{\kilogram\per\meter\cubed})                & \Tnote{j} 2700 & \Tnote{k} 2291.6 & \Tnote{b} 1100.0 & \Tnote{j} 2660.0 & \Tnote{l} 8960 & \Tnote{j} 1290.0 & \Tnote{m} 1150.0 \\
                Activ.\ energy, diff. ${E_\text{act,s}}_j$ (\si{\joule\per\mole}) & {---}                   & \Tnote{p} 5000   & {---}                     & \Tnote{p} 5000   & {---}                   & {---}                     & \multicolumn{1}{c}{---}   \\
                Activ.\ energy, rxn. ${E_\text{act,k}}_j$ (\si{\joule\per\mole})  & {---}                   & \Tnote{p} 5000   & {---}                     & \Tnote{p} 5000   & {---}                   & {---}                     & \multicolumn{1}{c}{---}   \\

                \bottomrule
            \end{tabular}
        }
        \medskip
        \begin{tabular*}{\textwidth}{@{} l @{\extracolsep{\fill}} r @{}}
            \multicolumn{2}{c}{\textbf{Other Geometric/Cell-Level Parameters}} \\
            \toprule
            \multicolumn{1}{@{}l}{Parameter} \\
            \midrule

            Thickness of pos.\ current collector, $l_\text{Al}$ \si{(m)}                    & \textsuperscript{f}\num{15e-6}   \\
            Thickness of neg.\ current collector, $l_\text{Cu}$ \si{(m)}                    & \textsuperscript{p}\num{10e-6}   \\
            Total tab area, $A_\text{tabs}$ \si{(m^2)}                                      & \textsuperscript{b}\num{5.94e-3} \\
            Lumped heat transfer coefficient, $h$ (\si{\watt\per\meter\squared\per\kelvin}) & \textsuperscript{b}150           \\
            Initial electrolyte concentration, $c_\text{e,0}$ (\si{\mole\per\meter\cubed})  & \textsuperscript{q}1000          \\

            \bottomrule
        \end{tabular*}

        \medskip
        \begin{tabular*}{\textwidth}{@{} =P{7.5cm}  +l@{\extracolsep{\fill}}+c +r @{}}
            \multicolumn{4}{c}{\textbf{Spatial Discretisation}} \\
            \toprule
            \multicolumn{1}{@{}l}{Parameter} & \multicolumn{1}{l}{Pos} & \multicolumn{1}{c}{Sep} & \multicolumn{1}{r@{}}{Neg}\\
            \midrule

            Nodes, through-thickness (axial), $N_{\text{a}_j}$          & \num{40} & \num{40} & \num{40} \\
            Nodes, within spherical particle (radial), $N_{\text{r}_j}$ & \num{15} & ---      & \num{15} \\

            \bottomrule
        \end{tabular*}

        \smallskip
        % \vspace{-2.6229525pt}
        \vspace*{-5pt}
        \begin{tablenotes}[para,flushleft]
            \begin{footnotesize}
            \item[a] Calculated in section \hyperlink{celllowercutoff}{`Lower cutoff voltage for cells'} (also see \cref{sec:surfareaperlayer})
            \item[b] Assumed
            \item[c] Ref.~\cite{Sae2010}
            \item[d] Ref.~\cite{Kizilel2009}
            \item[e] Converted from imperial units reported in~Ref.~\cite{GMBoltBatteryDims}
		    \item[f] Table~\romanletter{4} of~Ref.~\cite{Groger2015} \\
            \item[g] Sum of values in table~1 of~Ref.~\cite{Svens2013}
            \item[h] Ref.~\cite{Chen2005} \\
            \item[i] Computed from values of constituents (see~\cite{Svens2013}) using Ref.~\cite{martienssen2006springer} \\
            \item[j] Ref.~\cite{Guo2010}
            \item[k] Ref.~\cite{Jeon2011}
            \item[l] Ref.~\cite{Worwood2017,Song2000}
            \item[m] Ref.~\cite{Kim2009}
            \item[p] Ref.~\cite{Northrop2011}
            \item[q] Ref.~\cite{Subramanian2009} \\
            \item[r] See section \hyperlink{stackthickness}{`Compute stack thickness of reference cell'}
            \end{footnotesize}
        \end{tablenotes}
    \end{threeparttable}
\end{table}


%%%% -*- root: ../../main.tex -*-
%!TEX root = ../../main.tex
% vim:nospell


\begin{table}[!htbp]
	\renewcommand{\thetable}{\arabic{table}a}
	\centering
	\caption{Acceleration test parameters (common across xEV platforms)}
	\label{tbl:CommonVehicleParams}
	\sisetup{table-format=3.2, table-number-alignment=center, table-space-text-pre=\textsuperscript{a}, table-space-text-post=\textsuperscript{a}, table-align-text-post=false}
	\begin{threeparttable}[t]
		\centering
		\begin{tabular}{@{} l  S @{}}
			\toprule
			Parameter \\
			\midrule

			% Coefficient of drag for xEV body, $C_\mathrm{d}$                           & {\makebox*{00}[r]{\tnote{a}}} 0.31                \\
			% Frontal area of xEV, $A_\mathrm{v}$ \si{(m^2)}                             & {\makebox*{00}[r]{\tnote{b}}} 2.40                \\
			% Acc.\ time specified by manufacturer, $t_\mathrm{f,man}$ \si{(s)}          & {\makebox*{00}[r]{\tnote{d}}} 6.50                \\
			% Acc.\ time dictated by standards, $t_\mathrm{f,std}$ \si{(s)}              & {\makebox*{00}[r]{\tnote{c}}} 6.00                \\
			% Speed, end of acc. (standards), $v_\mathrm{f,std}$ \si{(m.s^{-1})}         & {\makebox*{00}[r]{\tnote{e}}} 8.94                \\
			% Speed, end of acc. (manufacturer), $v_\mathrm{f,man}$ \si{(m.s^{-1})}      & {\makebox*{0}[r]{\tnote{f}}} 26.82                \\
			% Base speed of  xEV, $v_\mathrm{b}$ \si{(m.s^{-1})}                         & {\makebox*{\hspace*{0.5mm}0}[r]{\tnote{e}}} 13.41 \\
			% Air density at acc.\ test conditions, $\rho_\mathrm{air}$ \si{(kg.m^{-3})} & {\makebox*{\hspace*{0.5mm}00}[r]{\tnote{f}}} 1.20 \\
			% Drivetrain efficiency, $\eta_\mathrm{dt}$                                  & {\makebox*{00}[r]{\tnote{g}}} 0.75                \\
			% Payload, $M_\mathrm{p}$ \si{(kg)}                                          & {\hspace*{0.00005mm}{\tnote{c}}} 150.60 \\
			% Rolling resistance coefficient of road surface, $C_\mathrm{r}$             & {\makebox*{00}[r]{\tnote{f}}} 0.01                \\
			% Road gradient, $Z$                                                         & {\makebox*{00}[r]{\tnote{g}}} 0.00                \\

			Coefficient of drag for xEV body, $C_\mathrm{d}$                           & 0.31   {\tnote{a}} \\
			Frontal area of xEV, $A_\mathrm{v}$ \si{(m^2)}                             & 2.40   {\tnote{b}} \\
			Acc.\ time specified by manufacturer, $t_\mathrm{f,man}$ \si{(s)}          & 6.50   {\tnote{d}} \\
			Acc.\ time dictated by standards, $t_\mathrm{f,std}$ \si{(s)}              & 6.00   {\tnote{c}} \\
			Speed, end of acc. (standards), $v_\mathrm{f,std}$ \si{(m.s^{-1})}         & 8.94   {\tnote{e}} \\
			Speed, end of acc. (manufacturer), $v_\mathrm{f,man}$ \si{(m.s^{-1})}      & 26.82  {\tnote{f}} \\
			Base speed of  xEV, $v_\mathrm{b}$ \si{(m.s^{-1})}                         & 13.41  {\tnote{e}} \\
			Air density at acc.\ test conditions, $\rho_\mathrm{air}$ \si{(kg.m^{-3})} & 1.20   {\tnote{f}} \\
			Drivetrain efficiency, $\eta_\mathrm{dt}$                                  & 0.75   {\tnote{g}} \\
			Payload, $M_\mathrm{p}$ \si{(kg)}                                          & 150.60 {\tnote{c}} \\
			Rolling resistance coefficient of road surface, $C_\mathrm{r}$             & 0.01   {\tnote{f}} \\
			Road gradient, $Z$                                                         & 0.00   {\tnote{g}} \\

			\bottomrule
		\end{tabular}
        \begin{tablenotes}[para,flushleft]
        \item[a]Ref.~\cite{HybridCars2017Drag}
        \item[b]Calculated from typical \gls{bev} dimensions in~\cite{BoltDimensions}
        \item[c]Ref.~\cite{ETANTP002-2004}
        \item[d]Ref.~\cite{BoltOverview}
        \item[e]Ref.~\cite{Liu2016a}
        \item[f]Ref.~\cite{EmadiElectric}
        \item[g]Assumed
        \end{tablenotes}
	\end{threeparttable}
\end{table}

%%%% -*- root: ../../main.tex -*-
%!TEX root = ../../main.tex
% vim:nospell


\begin{table}[!htbp] % Parameters unique to each of the BEV & PHEV
	% \addtocounter{table}{-1}
	% \renewcommand{\thetable}{\arabic{table}b}
	\caption{Acceleration test parameters (specific to each \glsfmtshort{xeV})}
	\label{tbl:UniqueVehicleParams}
	\centering
    \sisetup{table-format=4.1, table-number-alignment=center, table-space-text-pre=\textsuperscript{a}, table-align-text-pre=false}
	\begin{threeparttable}[t]
		\begin{tabular*}{0.675\textwidth}{@{} l @{\extracolsep{\fill}}  S S @{}}	% Works with Tnote
			\toprule
			\multicolumn{1}{@{} l}{Parameter} & \multicolumn{1}{c}{BEV} & \multicolumn{1}{c@{}}{PHEV} \\
			\midrule

			Mass of xEV chassis, $M_\mathrm{c}$ \si{(kg)}               & \Tnote{a} 1340.0 & \Tnote{b} 1438.0 \\
			Mass of pack overhead (w/o cells), $M_\mathrm{o}$ \si{(kg)} & \Tnote{a} 196.4  & \Tnote{c} 65.5   \\
			Upper cutoff SOC of cell, $z_\mathrm{max}$ \si{(\%)}        & \Tnote{d} 95.0   & \Tnote{d} 90.0   \\
			Lower cutoff SOC of cell, $z_\mathrm{min}$ \si{(\%)}        & \Tnote{d} 5.0    & \Tnote{e} 30.0   \\

			\bottomrule
		\end{tabular*}
		\begin{tablenotes}[para,flushleft]
		\item[a]Calculated based on~\cite{ChevyBoltSpecs}
		\item[b]Calculated based on~\cite{motortrendEcotec,ChevyBoltSpecs}
		\item[c]Calculated see \cref{sec:Configurations}
		\item[d]Assumed
		\item[e]Ref.~\cite{EmadiElectric}
		\end{tablenotes}

	\end{threeparttable}
\end{table}

%%%% -*- root: ../../main.tex -*-
%!TEX root = ../../main.tex
% vim:nospell

\begin{table}[htb!]
    \caption{\glsfmtshort{xeV} acceleration test results}
    \label{tbl:accResults}
    \centering
	\begin{tabular}{c c c}
        \toprule
        \multicolumn{1}{@{} l}{\makecell{($T_\text{init},T_\text{sink}$) \\ \footnotesize (degC)}} & \makecell{$n^\text{acc}_\text{opt}$ \\ \footnotesize \glsfmtshort{bev}}&  \multicolumn{1}{c @{}}{\makecell{$n^\text{acc}_\text{opt}$ \\ \footnotesize \gls{phev}}}  \\
        \midrule

        (38,5)  & \num{21} & \num{55} \\
        (38,49) & \num{21} & \num{57} \\
        (25,25) & \num{23} & \num{63} \\
        (15,5)  & \num{27} & \num{69} \\

        \bottomrule
    \end{tabular}
\end{table}


%%\begin{figure}[!bp]
%%    \begin{minipage}[t]{\textwidth}
%%        \centering
%%        \includegraphics[width=\textwidth]{fig_generate_heatmap_BEV}
%%        \captionsetup{labelsep=note}
%%        \caption[Optimal cell layer configurations for the \gls{bev}, presented for a range of fast charging powers and thermal conditions]{Optimal cell layer configurations for the \gls{bev}, presented for a range of fast charging powers and thermal conditions.}
%%        \label{fig:fig_generate_heatmap_BEV}
%%        \mpfootnotes[1]
%%        \footnote{This figure was created by \mbox{Ian Campbell} who asserts copyright,
%%            with intellectual contributions from and the right to use asserted by
%%        \mbox{Krishnakumar Gopalakrishnan}.}
%%    \end{minipage}
%%\end{figure}

%%\begin{figure*}[!bp]
%%    \begin{minipage}[t]{\textwidth}
%%        \centering \includegraphics[width=\textwidth,trim=4 2 3 4,clip]{fig_capacity_quadrants.pdf}
%%        \captionsetup{labelsep=note}
%%        \caption[The plots in the right column show the nominal cell capacity and charge passed
%%        during \gls{xeV} \gls{cp} charging. Increased rate capability and cell utilisation are positively
%%        correlated with $n$, while the maximum-$q$ layer configuration clearly shifts to higher
%%        values of $n$ with increasing charging powers. The plots in the left column depict
%%        galvanostatic charging scenarios at various currents to highlight the similarity with the
%%        \gls{cp} process. All data obtained at $T_\text{init} =$ \SI{25}{\degreeCelsius},
%%        $T_\text{sink} =$ \SI{25}{\degreeCelsius}.]{The plots in the right column show the nominal cell capacity and charge passed
%%            during \gls{xeV} \gls{cp} charging. Increased rate capability and cell utilisation are positively
%%            correlated with $n$, while the maximum-$q$ layer configuration clearly shifts to higher
%%            values of $n$ with increasing charging powers. The plots in the left column depict
%%            galvanostatic charging scenarios at various currents to highlight the similarity with the
%%            \gls{cp} process. All data obtained at $T_\text{init} =$ \SI{25}{\degreeCelsius},
%%        $T_\text{sink} =$ \SI{25}{\degreeCelsius}.}
%%        \label{fig:fig_CapacityQuadrants}
%%        \mpfootnotes[1]
%%        \footnote{This figure was created by \mbox{Ian Campbell} who asserts copyright,
%%            with intellectual contributions from and the right to use asserted by
%%        \mbox{Krishnakumar Gopalakrishnan}.}
%%    \end{minipage}
%%\end{figure*}


%%\begin{figure}[!bp]
%%    \begin{minipage}[t]{\textwidth}
%%        \centering
%%        \includegraphics[width=\textwidth]{fig_generate_heatmap_PHEV}
%%        \captionsetup{labelsep=note}
%%        \caption[Optimal cell layer configurations for the \gls{phev}, presented for a range of
%%        fast charging powers and thermal conditions]{Optimal cell layer configurations for the \gls{phev}, presented for a range of
%%        fast charging powers and thermal conditions.}
%%        \label{fig:fig_generate_heatmap_PHEV}
%%        \mpfootnotes[1]
%%        \footnote{This figure was created by \mbox{Ian Campbell} who asserts copyright,
%%            with intellectual contributions from and the right to use asserted by
%%        \mbox{Krishnakumar Gopalakrishnan}.}
%%    \end{minipage}
%%\end{figure}

\subsection{Common Module Design}\label{sec:commonmodulelayeropt}



\section{Conclusions}

A methodology to design the number of layers within pouch cells so as to
maximise their energy density whilst simultaneously being capable of high
charge-acceptance rate at the plating boundary has been developed. The proposed
methodology may be employed by designers of \gls{xeV} powertrains to mitigate
the inefficiencies that plague the iterative experimental designs for each
powertrain in a vehicle manufacturer's product portfolio.  The methodology
discussed in this chapter also paves the way to incorporate common module
designs for packs through layer optimisation. By suitably changing the number of
layers within a cell, while keeping the exterior module geometry unchanged,
shall lead to cost reduction and lower the time for derivative product designs.

\addlines[0.5]
A mathematical procedure to adapt the standard galvanostatic-driven \gls{p2d}
model to accept direct power inputs has been elucidated. This reformulation is
based on a quadratic boundary condition for the discretised solid-phase charge
conservation equation, coupled with a positivity constraint on the cell's
terminal voltage and an algebraic residual equation. A key numerical aspect of
the \glspl{pbm} used in this work is the incorporation of a hybrid
pseudo-spectral/Finite Volume scheme, wherein the solid phase diffusion equation
is solved in the Chebyshev domain while a finite volume discretisation is
employed for rest of the axial-direction \glspl{pde}. This hybrid combination
helps to achieve fast simulation runs without sacrificing model fidelity.

The methodology proposed here for pouch cells may be considered for application
to cells of other form factors with suitable modifications. While considering
the cylindrical form factor, instead of a discrete set of layers, the electrode
sheets are wound into one continuous jelly roll and placed inside a cylindrical
can.  A suitable mathematical curve such as the logarithmic spiral could be
employed wherein the number of turns could be used as a proxy for the layer
count. The optimisation problem is also simplified to a certain extent since the
cylindrical can's dimensions have already been standardised across the
automotive battery industry. However, certain additional challenges could arise
in directly translating the assumptions of the cooling methodology employed. For
instance, a three dimensional discretisation of the thermal grid is not uncommon
for cylindrical form factors, which requires careful considerations for
bidirectionally coupling to the electrochemical model. Any differences in power
densities at the interior of the core versus the extremities of the electrode
spirals need to be accounted for.

This concludes the design-oriented aspects of the thesis. Next,
\cref{ch:improveddra} deals with the aspects of analysing the computational
bottlenecks present in one of the notable \glspl{rom} and proposes a mitigation
strategy for them.



% The outcome is that a ready to  use tool is made available to validate empirical
% layer choices.  In the  absence of  access to  cell manufacturing  facilities to
% confirm and test  the layer Immediate adoption  in industry, this is  the best I
% can do. Needs to be prototyped and tested.

\glsreset{xeV}

