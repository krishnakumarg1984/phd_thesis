% -*- root: ../../main.tex -*-
%!TEX root = ../../main.tex
% vim:textwidth=80 fo=cqt

\graphicspath{{chapters/layer_opt/figures/}}
% ----------------------- contents from here ------------------------

\chapter{Model-based Design Of Pouch Cells}\label{ch:modelbaseddesign}
\vspace*{-1em}
\startcontents[chapters]
\printcontents[chapters]{}{1}{\setcounter{tocdepth}{1}}

\bigskip


% A brief  1-2 sentences throwback to  the literature review, referring  what we
% will say in this chapter.

\section[Introduction]{Introduction\protect\footnote{\textbf{Attribution      of
        content}  \   The  groundwork   for  converting   the  existing   computer  code
        (LIONSIMBA~v1.0) into  a suitable form  for layer optimisation was  initiated by
        this  thesis  author,  \mbox{Krishnakumar  Gopalakrishnan}.  However,  with  the
        exception of  the spectral scheme, for  which \mbox{Krishnakumar Gopalakrishnan}
        was  responsible, and  the zero-dimensional  thermal model  for which  \mbox{Ian
        Campbell} was  responsible, the advancements  inherent to the  enhanced computer
        software  (LIONSIMBA~v2.0)  were  made  in  equal  parts  by  \mbox{Krishnakumar
        Gopalakrishnan} and \mbox{Ian Campbell}. These  advancements would not have been
        possible without the contributions and support of Dr.~Davide Raimondo who served
        as an unofficial  supervisor for the work reported in  this chapter. The concept
        of layer reconfiguration for energy  and power trade-off, the layer optimisation
        framework, and the  source code by which it is  implemented were co-developed in
        equal  parts  by  \mbox{Krishnakumar Gopalakrishnan}  and  \mbox{Ian  Campbell}.
        \mbox{Krishnakumar Gopalakrishnan} was the  major contributor to the development
        of the binary search while \mbox{Ian  Campbell} was the major contributor in the
        analysis of results. \mbox{Parvathy  Chittur Subramanianprasad} was instrumental
        in  developing the  analytical expression  for  the maximum  possible number  of
layers~$n_\text{max}$.}}

The issue of `range anxiety' is a pervasive mental blockade for potential buyers
of electric  vehicles which  in-turn hampers their  widespread adoption.  From a
consumer viewpoint, yet another practical issue is the fact that on encountering
a  `low  battery' scenario  in  a  long  distance  journey, the  charging  times
required for sufficiently  replenishing the battery to enable  completion of the
journey are prohibitively  large, to the point of  being non-competitive against
conventional fossil  fuel powered  vehicles. Unfortunately, these  scenarios are
not unimaginable with the present state of the art in lithium ion batteries.

Hence, improving the \gls{aer} and  providing fast charging capabilities are two
near-term goals of manufacturers of  electric vehicles. Increasing the \gls{aer}
necessitates  a battery  pack  with  higher energy  content  in  it while  lower
charging  times demand  a pack  with  higher power  capability. The  contrasting
nature of these  goals can be traced all  the way down to the cell  level and is
presented in \cref{sec:energypowertradeoff}. By trading off the number of layers
in a  pouch cell against the  content of active electrode  material accommodated
within it, bespoke cell designs addressing either the energy demand or the power
demand  can  be  obtained.  In  the  absence  of  accessible  documentation  (as
either  industry white  papers or  academic literature)  on the  layer selection
methodologies employed in automotive pouch  cell designs, this author postulates
that manufacturers  iterate through  an extensive  empirical testing  process of
prototypes with  a range of  layer choices. In the  view of this  thesis author,
this  procedure  is not  only  time-consuming,  but  also  likely to  result  in
sub-optimal designs.  This chapter envisages a  model-based engineering solution
to more  optimal cell designs  by determining  the appropriate number  of layers
needed  to maximise  its  \emph{usable} energy  while simultaneously  satisfying
certain  power  capability constraints.  The  rest  of  the chapter  provides  a
detailed treatment of topics such  as the proposed layer optimisation framework,
its assumptions involved,  and various modifications to  standard numerical code
required to facilitate this design procedure.

\section{Energy to Power Trade-off}\label{sec:energypowertradeoff}
%Impart knowledge on energy cell versus power cell with the help of a diagram.
%Explain the effect of useable energy and power for a given cell with the help of
%a figure and table. How layers place a role in controlling this trade-off shall
%be discussed in the subsequent sections.

%\section{Augmentation of p2d parameters}

%Shed more light on the p2d model. Explain how they do not model a cell of given
%capacity, but instead work on a  normalised basis driven by the applied current
%densities rather than the external current.  The parameter that layers within a
%cell  change  is the  overall  electrochemically  active cross-sectional  area.
%Criticise  how published  literature  curiously omit  this important  geometric
%parameter, however  they may be  forgiven in the scope  of their work  since it
%needs only normalised dynamics. This  parameter comes into light when modelling
%anything  involving geometries  as  in this  project. This  is  the product  of
%surface area per face  and the number of layers. To  determine the surface area
%per face, the author has derived a new methodology/process involving a sequence
%of  steps, based  on assumptions  and literature  search. The  process involves
%selection of a  real-world cell, and ultimately mapping it  to the surface area
%per  unit face.  To the  best  knowledge of  the  author, this  mapping from  a
%physical cell to the Newman model is  unique and is claimed as the author's own
%contribution to the art.

%Describe briefly  the process of  mapping a  real-world cell onto  the standard
%\gls{p2d} model.

%Note: Krishna wrote the Northrop to EV mapping code and explained the concept to
%the mapping to Ian. Ian's help on the matter is identified where relevant.

%\subsection{Modelling Platform and Preconditioning}

%Couple of statements about why LIONSIMBA was chosen as the modelling platform
%for implementing the p2d dynamics. The cell parameters used are shown in table
%xx. This cell is henceforth known as the LIONSIMBA cell or Northrop cell.

%Discuss the missing elements in LIONSIMBA only with respect to the present
%problem at hand, \viz{the stoichiometries}. Thanks to Ian for running the
%time-intensive and memory-intensive simulation process on his workstation and
%offering comments on datalogging.

%\subsubsection*{Stoichiometry Augmentation}
%Discuss the problem first. How LIONSIMBA started always at 85.51 percentage and
%needed to do a discharge down to zero percent before having the ability to
%charge. For this project, stoichiometries are vital for capacity determination
%and the 1C current density. Explain how stoichiometries were refined until
%cut-off for infinitesimal bleeding discharge current achieved. Noted relevant
%values. Explanation with detailed figure on how stoichiometries were found to be
%missing. Explain refinement of how approximate capacities reported by Northrop
%and Subramanian were refined precisely. Explanation of remnant capacities and
%stoichiometries computation. Explanation of 1C current density. Note: Krishna
%wrote the parameters init capacity computation code. Thanks to Ian for running
%the simulation

%\subsection{Selection of a Suitable Reference Capacity Cell}

%Explanation of how the Bolt EV cell was selected (state of the art in driving
%range) In particular, explain how the critical computation of 60 Ah capacity was
%done. Furthermore, convincingly explain how a pouch cell thickness of 10 mm was
%used with all the background references.

%Note: Krishna did this thorough literature review, whose proof is available with
%timestamp in Box. Krishna would like to thank Ian for his time in brainstorming
%the exact BEV to use. This phase lasted a couple of weeks. For the PHEV, Krishna
%and Ian jointly did the PHEV literature search, and since Krishna does not
%intend to go in depth for PHEV results, Ian may choose to use it and simply
%acknowledge Krishna.

%\subsection{Layer Assembly within Pouch Cells}
%With the help of Northrop's layer assembly figure, explain the layer
%configuration/arrangement within a pouch cell. The next task is then identified
%as computing the number of layers within the pouch cell.

%\subsubsection*{Number of Layers of LIONSIMBA aka Northrop cell}
%Krishna came up with the idea of using integer optimisation for this task. The
%software MIDACO was also selected by Krishna and explained to Ian. The MIDACO
%result of the number of layers within the standard cell was now available.

%\subsubsection*{Computation of Surface Area per face, \protect{$A_\text{cell}$}}
%Show the simple algebraic computation of overall surface area $A$ and the
%per-face area $A_\text{cell}$. Explain how the area per face shall be a key
%quantity in the layer optimisation framework discussed later on.

%Layerphoto showing face areas and anode/cathode verhand etc will be shown here

%This concludes the augmented set of parameters added by the author to the basic
%parameter set of the DFN model. The added numerical value of parameters are
%summaried in table xx. Lots The layer optimisation framework and assumptions is
%described next. % table: pouch length, width, tab area, stack thickness etc.

%\section{Layer Optimisation Framework}

%Basically the pack configuration and the universe in which we worked shall be
%discussed here with the help of the hierarchical powertrain-to-cell schematic.
%This will set the context of the problem being tackled. Important concepts that
%enable the creation of the universe in which we define the problem to be solved
%in discussed and the assumptions involved shall be explained.

%The layer optimisation framework hinges upon the concept of capacity balancing
%between electrodes. Explain this nicely with references and citations on why
%this is important. Helps us make us of the most of the active material invested
%into the cell.


%\subsection{Capacity Balancing}
%Formula for capacity balance. Show how the length of the negative electrode is
%slightly larger than the positive electrode, but compensated for by the reduced
%volume fraction. Cite other parameter sets where this is observed. They are
%nearly equal, but made to be 1.10

%Note: This 1.1 ratio was fixed by Krishna. I can explain in person how
%this happened (reason: numerical convergence). The thickness of the positive
%electrode was adjusted accordingly. If Ian is asked this question and details of
%its origin, Krishna is sure he cannot explain it.

%\subsection{Electrode Thicknesses per layer}
%Talk briefly about separator and how its thickness remains constant.

%With this, show the mathematical derivation of the expression for number of
%layers Explain the significance of the outermost layer and how it affects the
%formula used Explain how the stack is formed. Report its numerical value. Pouch
%thickness and its role along with suitable reference. Show briefly a clever
%computer code snippet that encapsulates both cases of odd and even.

%At this stage, can explain clearly the relationship between the theoretical
%capacity and number of layers. Use a graph or a table to highlight the
%idea. Useable capacity shall be lower than total capacity due to cutoff
%considerations. Explanation.

%Note: Again, Krishna explained the layer calculation formulae to Ian.
%Especially, the derivation of the combined electrode thickness concept, and the
%usage of the ratio to obtain one from the other. Ian had to write it down
%multiple times drawing the layers and verify that my computation was correct.
%Krishna explained how the cases environment can be used for theoretical
%description while the computer code for both the cases uses ceil and floor. Ian
%naively used the mathematical symbol for ceil, until Krishna explained to him
%about the cases environment. Krishna also typed up these equations (basically
%any equation that required a derivation) in the paper manuscript.

%\subsection{Derivation of analytical \protect{$n_\text{max}$}}

%The search space spans a finite number of layers. An initial alternative
%considered was MIDACO. The detailed equations and how they are derived go here.
%Discuss for both odd and even cases. However, closed form analytical expressions
%are now available. This is so as to enable a binary search as discussed in the
%next section.

%The optimisation formula and analytical solution shall also be discussed here.

%Note: Krishna is claiming the idea generation and coding in entirety. The mixed
%integer optimisation to zero thickness was some fancy coding. The zero thickness
%idea was also quite fancy. Krishna came up with the whole of this section.
%Initiated the idea of narrowing down the search window. Ian was content with
%doing a for loop since it will also work. The optimal layer in this case is the
%earliest choice of $n$ in the for loop for which the termination conditions are
%correctly satisfied. Krishna was always in favour of an optimisation approach
%from the beginning.

%\subsection{Customised Binary Search}

%A bi-section based algorithm. Algorithm/binary tree based description with
%figure/algo. Discuss the O(log n) speedup.

%Note: One fine evening, Krishna increased the computational speedup by two orders of magnitude. Proof in github and email. While Ian was always interested in a for-loop approach right from beginning, Krishna always suggested to use an optimisation approach.

%\subsection{Mass recomputation as a function of layers}
%Explanation with graph.

%\subsection{Specific-heat recomputation as a function of layers}
%Suitable explanation and plot.

%Note: Ian congratulated Krishna for mass and specifc heat computation. In his
%words, ``You were clever coming up with a scheme wherein mass varies as a
%number of layers''. Packaged this up as a function and all. Refer to
%computelumpedmassandCpavgforgivenlayerfcn. Can prove git history for this file.
%Not only mass recomputations but also mass initial computation was done by
%Krishna

%\subsection{Thermal space permutation of layers.}
%Explanation of how the four corner temperatures are tried here. 5 lines of code snippet that packs a punch

%Note: Krishna read the acceleration specification description,educated Ian and implemented this.

%\begin{quotation}
%Ian: Krishna!. Whoa that small block of code does so much.
%\end{quotation}

%\section{Workflow for Optimal Layer Computation}

%This section discusses the actual methodology or the procedure in which the
%aforementioned ideas are incorporated in a process-like workflow. Introduce the
%full-page crazy flow diagram here. Explain how the flow diagram goes through a
%methodical approach in arriving at the optimal number of layers. Extensive
%forward and backward referencing to sections discussing each modular idea
%encountered in the flow path. Discuss how the applied current and power
%densities change due to change in overall surface area while the applied
%external current/power remains the same.

%% This will help us to apply current in units rather in current density.

%The salient cases to be covered are the following.

%\subsection{Case 1 --- Analysis of Drivecycle Powers}

%Although from Colorado Boulder lecture notes, Krishna already knew that
%acceleration demands the highest power demand on the cell, it needs to be proven
%that drive cycles, which are the basis for various fuel efficiency calculations
%do not represent this case. Ian did the comparative study of various drivecyles.
%Some plots and charts showing various power levels were generated by him.
%Krishna does not explicity need to use this section, and is optional to the
%story, other than the sake of well-roundedness and demonstrating thoroughness of
%the study.

%Explain that this case is not present in the flow diagram.

%However, it was Krishna who generated the drivecycle speed vs time and
%acceleration versus time results alone and then informed Ian afterwards that it
%is now available in the repository. If Ian uses this section, an acknowledgement
%will be appropriate.


%\subsection{Case 2 --- Acceleration from standstill}

%Explanation and computations based on standard vehicle dynamics. Lecture notes
%and videos given to Ian by Krishna from Univ of Colorado Boulder The sole value
%addition in this case is the specification to adhere to and its implementation
%through a cases study. Krishna shall explain this through either-or if-then
%case, explanation along with a short code snippet which he implemented alone.

%Explain the acceleration run in detail, what it entails etc. Walk through the
%flow diagram till acc layer results are discussed.

%Note: The acceleration specification for electrified transport was unearthed by
%Krishna (all the interlibrary loan stuff and educated Ian about it). Two
%passengers and all that thing. Complete coding.

%\subsection{Case 3 --- Fast Charging}

%A brief explanation of the whys and the hows and implications. Explanation of
%prevalent standards. Present table of standards etc etc

%One paragraph review of control algorithms and how this algorithm was chosen.
%Brief overview of algorithm. The idea of introducing saturation and pulsed
%charging profile. Based on patent at Auburn university. Krishna did literature
%review. Flag introduced in code. Code snippet.enablecsnegsaturationlimit.
%% Sinusoidal excitation charging.

%Explanation of why power demand is important (based on charger power electronics).
%Finally, walk through of the fast charging section of the schematic.

%Note: Krishna performed the litt search of this section entirely (proof with
%timestamp available). However, Krishna is tentatively not using a detailed litt
%review. In kind consideration of Ian's own overarching thesis topic, which
%Krishna understands to be something pertinent to fast charging, Krishna can let
%Ian use the literature provided proper attribution is in place. For the sake of
%completion, the MIDACO based approaching to fast charging showing the pulsing
%power is also possible by Ian (just a suggestion).

%Note: The charger power electronics limit is again my contribution from an EEE
%background

%% \item Review of model-based fast charging control algorithms (how does this go into litt review)

%\section{Simulation Environment}
%Discuss parameters of simulation environment. Discuss with the help of a table
%for the BEV case only. Ian can take the PHEV Number of BEV calls designed to
%make up the series string.

%Krishna also came up with the cell's cutoff study with respect to the the
%system's bus bar voltage, and will be cited here with examples and strong
%backing. There are a few preconditioning steps needed to amend the p2d model
%before numerical implementation of the flow diagram is possible. 1. Fixing
%aspects of the code in LIONSIMBA v1.023 used as the baseline by this thesis
%author. 2. Addition of the capability to apply power input instead of current
%input as is common in traditional p2d simulations.

%Note: All crucial parameters were sourced by Krishna through a literature
%review. Especially, the vehicular parameters like classis mass, coefficient of
%drag, base speed. Krishna also had to educate Ian about base speed. If an on the
%spot quiz is conducted at a conceptual level, this truth can easily be deduced
%by the judging authority. Krishna claims the literature review of this topic and
%is happy to let Ian use this with proper attribution.

%\subsection{Preconditioning of Computer Code}

%Briefly allude to the stoichiometries that was introduced into the computer code
%through capacity characterisation simulation.Now possible to start at any SoC.
%The other salient amendments and enhancements to the software toolbox is shown
%below. Released as 2.0 and link to LIONSIMBA github repo (not BOLD toolbox repo)

%\subsubsection{Re-parameterisation}

%Discuss the extensive Reparameterisation undertaken by Krishna by studying
%relevant literature. Discuss the changed parameters with respect to Northrop
%cell and why. Maybe show a table. This is quite substantial.

%Conductivity/diffusivity changes. Show how the isothermal and thermal variants
%in existing Northrop cell were bogus with plots.

%Performed comprehensive literature review to replace the dubious/bogus
%parameters of the electrode and electrolyte specific heat capacities email proof
%PS: Every thermal/material property of the Al/Cu current collectors is very
%clear and have been traced out, hand-calculated and validated.

%\subsubsection{Deterministic initialisation of algebraic variables}

%Replaced silly fsolve. Numerical explanation here.

%\subsubsection{Linear interpolation for field variables}

%Explain how linear interpolation was performed for phis and phie at the edges
%of control volumes in electrode and separator. Draw a schematic for this
%explanation.

%\subsubsection*{Convergence Analysis of Computation Mesh}
%Report only if Krishna has sufficient time left. Explain simplifying
%assumptions. Refer to the table etc Show plot of how terminal voltage is
%converging for chosen mesh as a function of number of nodes. Another plot of how
%simulation end time converges as a function of number of nodes. Lots of analysis
%by changing the number of nodes within each mesh to justify the validity of the
%results. Prove that mesh independence is reached.

%\subsection{Hybrid fv--Spectral Scheme}

%Completely numerical section. Highly mathemetical. Explain how the rational to
%decimal truncation of the unbalanced twelfth order finite difference scheme
%messes up numerical conditioning. Show matrices and discuss the drawbacks.
%Fornberg matrix did not help. So, spectral scheme was needed.

%Background study, explanation, analysis, literature review and equation
%derivation of spectral scheme Complete contribution of solid-phase diffusion
%with spectral methods. Reading textbook, understanding concept, investigation of
%applicability, hunting relevant literature, text-writing, hand-derivation of
%equations. This complete section was done by Krishna.

%\subsection{Lumped Thermal Model}

%Basic discussion of lumped thermal model and justify through citations why here
%it might be sufficient for this application. Not originally present in
%LIONSIMBA. Present the thermal model here equations here. Discuss what cp avg
%means and how it is computed as a here function of number of layers. Explanation
%of how tab cooling helps here.

%Note: Cp avg was calculated and coded by Krishna, as a function of number of
%layers. I shall leave the detailed thermal model to Ian, especially since a biot
%analysis was performed by him. Anyway, suddenly discussing the thermal model at
%depth does not fit the story of my thesis. My hint to Ian would be to emphasise
%the thermal model at depth, since anyway he seems to be confident in the biot
%analysis. Specifically the value of heat transfer coefficient was empirically
%chosen by Ian through simulations. So, I will let him explain that stuff.
%However, tab area idea computation using twice the Bolt's tab area was proposed
%by Krishna, but overall this thermal stuff, Krishna is willing to bequeath to
%Ian since the whole thing doesn't fit the story of Krishna's work. The
%polarisation heat concept was initially described by Greg, but anyway let Ian
%explain it, no problem. Ian may wish to discuss entropic heat generation and a
%lot of other things we investigated together, but I am not going to dwell on
%them in the thesis.


%\subsection{Power Input Boundary Conditions}

%Discuss why power input is needed for layer optimisation. Explain how it is done
%currently in state of the art case.

%Krishna educated Ian about Pletts existing work. Accompanied Ian to the library
%and told him to check out that book which I had ordered.

%Krishna claims a very important thing here. A critical argument on how the
%present schemes do not fit into the layer opt methodology.

%Detailed mathematical derivation. Both Ian and Krishna acknowledge each other's
%help in shared derivation and also acknowledge Davide appropriately.

%Note: Ian might wish to report the two intermediate steps before this solution
%was obtained. The 2nd of the 3 approaches was matching current and voltage at
%discrete intervals through numerical integration. This was our third approach.
%The first approach was too simplistic. So, Ian might wish to skip it.



%\section{Layer Optimisation Results}

%Show the table only with the extra parameters not present in isothermal model.
%In particular, I can remember the thermal parameters of the two electrodes,
%separator, pouch, current collectors and electrolyte.

%Present the results in a horribly bland table format steering well clear of the
%heatmap. Krishna's view is that the results do not stand alone by themselves and
%if you plonk a layer choice from the heatmap/table into your cell, things are
%not expected to work as is. The numbers herein are the result of quite a few
%assumptions and hold validity only within the universe in which it was created.

%However, the framework itself is transferable and is the most valuable component
%of the work. A user reading this thesis can easily substitute their own cell
%parameters and system-level considerations into the code and obtain results from
%it accordingly. This was the working premise of the paper as well, until Ian
%decided to write up a bloated explanatory section.

%Plots of acceleration and fast charging for successful layer count.

%For PHEV, although common module design will be alluded to, it will not be dwelled on or explained or analysed in depth.


%Note: In Ian's thesis, Krishna seeks attribution for the idea to use a heatmap
%since it arose out of Krishna's heavy criticism, bordering on the offensive,
%about continuously complaining that our group's plots/charts/graphs do not
%exhibit any ``interesting'' twists and turns and fancy illustrations. Apologies
%for this. Anyway, the heatmap idea was my suggestion. However, I shall not take
%an inch of credit for the implementation as it was entirely Ian's work in coming
%up with fancy schemes.

%Note: Having obtained BEV results, Krishna also set up the full set of
%assumptions for the PHEV simulations too before leaving actual numerical
%simulations to Ian (and on vacation to India followed by Konstanz). So an
%acknowledgement is required if Ian choses to list these assumptions. Ian can
%discuss monotonicity issues in PHEV simulation that he found out about.

%\section{Appendix}
\begin{figure}[!bp]
    \begin{minipage}[t]{\textwidth}
        \centering
        \includegraphics[trim=4 4 2 4,clip]{fig_CC_discharge_curves.pdf}
        \caption
        [%
        Voltage curves for a \SI{60}{\ampere} galvanostatic discharge from
        \SI{100}{\percent} \glsfmtshort{soc} until cut-off voltage for a few layer
        choices, in a pouch cell of fixed exterior height.
        ]%
        {%
            Terminal voltage curves of a Li-ion cell (with parameters
            given in \cref{tbl:lcoSimParamslayeropt}) under a \SI{60}{\ampere}
            galvanostatic discharge beginning from \SI{100}{\percent}
            \glsfmtshort{soc} until lower cut-off voltage for a few layer
            choices~$n$, in a pouch cell of fixed exterior height. The maximum
            usable energy is achieved for an intermediate choice of $n$
            that corresponds to neither the highest nominal capacity layer
            configuration ($n$=\num{10}) nor the highest electrode surface area
            configuration ($n$=\num{90})\footnotemark.
        }%
        \label{fig:fig_CC_discharge_curves}
        \mpfootnotes[1]
        \footnote{This figure was created by \mbox{Ian Campbell} who asserts copyright,
            with intellectual contributions from and the right to use asserted by
        \mbox{Krishnakumar Gopalakrishnan}.}
    \end{minipage}
\end{figure}

% -*- root: ../../main.tex -*-
%!TEX root = ../../main.tex

\begin{table}[!htbp]
    \caption
    [%
    Theoretical  capacity \&  usable energy  of a  Li-ion cell  for a  few layer
    choices under a \SI{60}{\ampere} galvanostatic discharge
    ]
    {%
        Theoretical capacity and usable energy of a Li-ion cell (with parameters
        given in \cref{tbl:lcoSimParamslayeropt}) for  a few layer choices under
        a \SI{60}{\ampere} galvanostatic discharge.
    }%
    \label{tbl:CC_discharge_curves_table}
    \centering
    \begin{tabular}{@{} S[table-format=2.0] S[table-format=1.2] S[table-format=2.2]  S[table-format=3.2] S[table-format=2.2] S[table-format=2.2] @{}}
        \toprule
        \multicolumn{1}{@{} l}{$n$} &  \multicolumn{1}{c}{\footnotesize C-rate} & \multicolumn{1}{c}{\footnotesize \makecell{Theoretical \\ Capacity  (\si{Ah})}} & \multicolumn{1}{c}{\footnotesize \makecell{Usable \\ Energy \si{(Wh)}}} & \multicolumn{1}{c @{}}{\footnotesize \makecell{Remaining \\ SOC  (\si{\percent})}} & \multicolumn{1}{c @{}}{\footnotesize \makecell{Resistance at \\ cutoff  (\si{\milli\ohm})}} \\
        \midrule
        90 & 1.24 & 48.25 & 166.46 & 9.84  & 0.97  \\
        70 & 1.11 & 53.99 & 184.80 & 10.26 & 1.35  \\
        50 & 1.00 & 59.73 & 195.47 & 13.51 & 3.44  \\
        30 & 0.92 & 65.47 & 101.20 & 58.95 & 10.24 \\
        10 & 0.84 & 71.21 & 10.15  & 96.22 & 11.18 \\
        \bottomrule
    \end{tabular}
\end{table}


\begin{figure}[!bp]
    \begin{minipage}[t]{\textwidth}
        \centering
        \includegraphics[width=\textwidth]{hierarchical_powertrain_to_cell_layer.pdf}
        \caption
        [%
        Vehicle-to-cell hierarchical overview of an electrified powertrain architecture.
        ]%
        {%
            Schematic depicting the vehicle-to-cell hierarchical overview of
            a typical electrified powertrain architecture. This represents the
            system-level context within which the proposed layer optimisation framework
            has been developed. Two \glsfmtshort{xeV} powertrains ---
            \begin{enumerate*}[label=\itshape\alph*\upshape)]
                \item a \gls{bev}, and
                \item a series \gls{phev}
            \end{enumerate*}
            are chosen as examples to demonstrate how the methodology facilitates
            common module designs for such battery packs\footnotemark.
        }%
        \label{fig:fig_PowertrainSchematic}
        \mpfootnotes[1]
        \footnote{This figure was created by \mbox{Krishnakumar Gopalakrishnan} who
            asserts copyright, with intellectual contributions from and the right to
        use asserted by \mbox{Ian Campbell}.}
    \end{minipage}
\end{figure}

\begin{figure}[p]
    \begin{minipage}[t]{\textwidth}
        \centering
        \includegraphics[angle=90, width=\textwidth]{fig_master_flow_diagram}
        \caption
        [%
        Flow diagram depicting an overview of the proposed layer optimisation methodology
        for Li-ion pouch cells.
        ]%
        {%
            Flow diagram depicting an overview of the proposed layer optimisation methodology
            for Li-ion pouch cells\footnotemark.
        }%
        \label{fig:fig_strategy_schematic}
        \mpfootnotes[1]
        % \vspace*{1.125cm}
        \vspace*{0.7225cm}
        \footnote{This figure was created by \mbox{Krishnakumar Gopalakrishnan} who
            asserts copyright, with intellectual contributions from and the right to
        use asserted by \mbox{Ian Campbell}.}
    \end{minipage}
\end{figure}

\section{Lower Cutoff Voltage}\label{sec:cutoff}
% -*- root: ../../main.tex -*-
%!TEX root = ../../main.tex
% vim:nospell

\begin{table}[!htbp]
    \small
    \caption[%
    System-level simulation conditions \& thermal parameters of  an \glsfmtshort{lco} cell
    ]%
    {%
        Cell   parameters   and   system   conditions  for   a   simulating   an
        \glsfmtshort{lco} cell  with the  \gls{dfn} electrochemical model  and a
        lumped thermal model. The parameters  presented here when augmented with
        the  values  of  the  kinetic, geometric  and  transport  properties  of
        the  cell (from  \cref{tbl:lcoSimParamsSPMp2d}  represents the  complete
        information  required for  all  simulations in  this layer  optimisation
        framework.
    }%
    \label{tbl:lcoSimParamslayeropt}
    \vspace{-2.6229525pt}
    \begin{threeparttable}
        \centering
        \textbf{System Conditions} \\ \smallskip
        \begin{varwidth}[t]{0.48\linewidth}
            \begin{tabular*}{\textwidth}{@{} l @{\extracolsep{\fill}} S[table-format=1.2,table-space-text-pre=\Tnote{a} ,table-align-text-pre=false] @{}}
                \toprule
                \multicolumn{1}{@{}l}{Parameter} \\
                \midrule

                Lower cutoff cell voltage, $V_\text{min}$ (\si{\volt}) & \Tnote{a} 3.50   \\
                Upper cutoff cell voltage, $V_\text{max}$ (\si{\volt}) & \Tnote{b} 4.22   \\

                \bottomrule
            \end{tabular*}
        \end{varwidth}
        \hfill
        \begin{varwidth}[t]{0.48\linewidth}
            \begin{tabular*}{\textwidth}{@{} l @{\extracolsep{\fill}} S[table-format=2.2,table-space-text-pre=\Tnote{a} ,table-align-text-pre=false] @{}}
                \toprule
                \multicolumn{1}{@{}l}{Parameter} \\
                \midrule

                Target cell SOC for fast charge, $z^\ast$ \si{(\%)}                & \Tnote{c} 80.00 \\
                Cell upper temperature limit, $T_\text{max}$ \si{(\degreeCelsius)} & \Tnote{d} 55.00 \\

                \bottomrule
            \end{tabular*}
        \end{varwidth}

        \medskip
        \begin{tabular*}{\textwidth}{@{} l @{\extracolsep{\fill}} r @{}}
            \multicolumn{2}{c}{\textbf{Geometric Parameters}} \\
            \toprule
            \multicolumn{1}{@{}l}{Parameter} \\
            \midrule
            Surface area of pos.\ \& neg.\ electrode overlap within a layer, {$A_\text{elec}$} \si{(m^2)} & \textsuperscript{b}\num{4.19e-2}   \\
            Exterior pouch length, $L_\text{pouch}$ \si{(m)}                                              & \textsuperscript{e}\num{332.74e-3} \\
            Exterior pouch width, $W_\text{pouch}$ \si{(m)}                                               & \textsuperscript{e}\num{99.06e-3}  \\
            Exterior pouch height, $H_\text{pouch}$ \si{(m)}                                              & \textsuperscript{f}\num{10.00e-3}  \\
            Pouch material thickness, $T_\text{pouch}$ \si{(m)}                                           & \textsuperscript{g}\num{160.00e-6} \\
            Stack thickness, $L_\text{stack}$ \si{(m)}                                                    & \textsuperscript{r}\num{9.68e-3}  \\
            \bottomrule
        \end{tabular*}
        \medskip
        \centering \textbf{Thermal Parameters} \\ \smallskip
        \resizebox{\textwidth}{!}{%
            \begin{tabular}{@{} l S[table-format=4.0,table-space-text-pre=\Tnote{m} ,table-align-text-pre=false] S[table-format=4.1,table-space-text-pre=\Tnote{m} ,table-align-text-pre=false] S[table-format=4.1,table-space-text-pre=\Tnote{m} ,table-align-text-pre=false] S[table-format=4.2,table-space-text-pre=\Tnote{m} ,table-align-text-pre=false] S[table-format=4.0,table-space-text-pre=\Tnote{m} ,table-align-text-pre=false] S[table-format=4.1,table-space-text-pre=\Tnote{m} ,table-align-text-pre=false] S[table-format=4.1,table-space-text-pre=\Tnote{m} ,table-align-text-pre=false] @{}}
                \toprule
                \multicolumn{1}{@{}l}{Parameter} & \multicolumn{1}{c}{Al.\ CC} & \multicolumn{1}{c}{Pos} & \multicolumn{1}{c}{Sep} & \multicolumn{1}{c}{Neg} & \multicolumn{1}{c}{Cu.\ CC} & \multicolumn{1}{c}{\ch{LiPF_6}} & \multicolumn{1}{r@{}}{Pouch}\\
                \midrule

                Sp.\ heat capacity, $c_j$ (\si{\joule\per\kilogram\per\kelvin})   & \Tnote{h} 903  & \Tnote{h} 1269.2 & \Tnote{h} 1978.2 & \Tnote{h} 1437.4 & \Tnote{h} 385  & \Tnote{h} 2055.1 & \Tnote{i} 1464.8 \\
                Density, $\rho_j$ (\si{\kilogram\per\meter\cubed})                & \Tnote{j} 2700 & \Tnote{k} 2291.6 & \Tnote{b} 1100.0 & \Tnote{j} 2660.0 & \Tnote{l} 8960 & \Tnote{j} 1290.0 & \Tnote{m} 1150.0 \\
                Activ.\ energy, diff. ${E_\text{act,s}}_j$ (\si{\joule\per\mole}) & {---}                   & \Tnote{p} 5000   & {---}                     & \Tnote{p} 5000   & {---}                   & {---}                     & \multicolumn{1}{c}{---}   \\
                Activ.\ energy, rxn. ${E_\text{act,k}}_j$ (\si{\joule\per\mole})  & {---}                   & \Tnote{p} 5000   & {---}                     & \Tnote{p} 5000   & {---}                   & {---}                     & \multicolumn{1}{c}{---}   \\

                \bottomrule
            \end{tabular}
        }
        \medskip
        \begin{tabular*}{\textwidth}{@{} l @{\extracolsep{\fill}} r @{}}
            \multicolumn{2}{c}{\textbf{Other Geometric/Cell-Level Parameters}} \\
            \toprule
            \multicolumn{1}{@{}l}{Parameter} \\
            \midrule

            Thickness of pos.\ current collector, $l_\text{Al}$ \si{(m)}                    & \textsuperscript{f}\num{15e-6}   \\
            Thickness of neg.\ current collector, $l_\text{Cu}$ \si{(m)}                    & \textsuperscript{p}\num{10e-6}   \\
            Total tab area, $A_\text{tabs}$ \si{(m^2)}                                      & \textsuperscript{b}\num{5.94e-3} \\
            Lumped heat transfer coefficient, $h$ (\si{\watt\per\meter\squared\per\kelvin}) & \textsuperscript{b}150           \\
            Initial electrolyte concentration, $c_\text{e,0}$ (\si{\mole\per\meter\cubed})  & \textsuperscript{q}1000          \\

            \bottomrule
        \end{tabular*}

        \medskip
        \begin{tabular*}{\textwidth}{@{} =P{7.5cm}  +l@{\extracolsep{\fill}}+c +r @{}}
            \multicolumn{4}{c}{\textbf{Spatial Discretisation}} \\
            \toprule
            \multicolumn{1}{@{}l}{Parameter} & \multicolumn{1}{l}{Pos} & \multicolumn{1}{c}{Sep} & \multicolumn{1}{r@{}}{Neg}\\
            \midrule

            Nodes, through-thickness (axial), $N_{\text{a}_j}$          & \num{40} & \num{40} & \num{40} \\
            Nodes, within spherical particle (radial), $N_{\text{r}_j}$ & \num{15} & ---      & \num{15} \\

            \bottomrule
        \end{tabular*}

        \smallskip
        % \vspace{-2.6229525pt}
        \vspace*{-5pt}
        \begin{tablenotes}[para,flushleft]
            \begin{footnotesize}
            \item[a] Calculated in section \hyperlink{celllowercutoff}{`Lower cutoff voltage for cells'} (also see \cref{sec:surfareaperlayer})
            \item[b] Assumed
            \item[c] Ref.~\cite{Sae2010}
            \item[d] Ref.~\cite{Kizilel2009}
            \item[e] Converted from imperial units reported in~Ref.~\cite{GMBoltBatteryDims}
		    \item[f] Table~\romanletter{4} of~Ref.~\cite{Groger2015} \\
            \item[g] Sum of values in table~1 of~Ref.~\cite{Svens2013}
            \item[h] Ref.~\cite{Chen2005} \\
            \item[i] Computed from values of constituents (see~\cite{Svens2013}) using Ref.~\cite{martienssen2006springer} \\
            \item[j] Ref.~\cite{Guo2010}
            \item[k] Ref.~\cite{Jeon2011}
            \item[l] Ref.~\cite{Worwood2017,Song2000}
            \item[m] Ref.~\cite{Kim2009}
            \item[p] Ref.~\cite{Northrop2011}
            \item[q] Ref.~\cite{Subramanian2009} \\
            \item[r] See section \hyperlink{stackthickness}{`Compute stack thickness of reference cell'}
            \end{footnotesize}
        \end{tablenotes}
    \end{threeparttable}
\end{table}



Figure showing power input

\section{Hybrid Spectral-\glsfmtshort{fv} Scheme}\label{sec:hybridfv-spectral}

Fast and accurate estimation of the solid phase lithium concentration,
particularly its value at the surface of electrode particles is an
inherent requirement of the layer optimisation procedure presented in
Section~\ref{sec:Framework}. The high power densities that result from using low
layer counts necessitate this requirement. It has been acknowledged that
concentration calculations employing polynomial approximations such as those
proposed in~\cite{Santhanagopalan2006a} lack fidelity at high charge/discharge
rates~\cite{Santhanagopalan2006}. Hence, a conventional full-order solution
based on Fick's law of diffusion is appropriate.

With full-order solid phase diffusion dynamics, applying the \gls{fv} scheme
(that has been employed to discretise all through-thickness \gls{pde}s in the
\gls{p2d} model) results in a large system of equations. This is due to the
requirement of using a high radial node density per spherical particle
for improved accuracy. Consequently, the computational cost is high and
simulation runtime becomes prohibitive when exploring the search space of
all possible layer configurations. Moreover, with a cell-centered \gls{fv}
discretisation, it is non-trivial to directly apply the ionic flux boundary
condition at the particle surface, since it involves extrapolation from at
least two other nodes within the particle. While such extrapolations are
acceptable in the axial dimension --- particularly with high node densities
providing small values of $\frac{\Delta x}{2}$ --- they are undesirable in
the radial dimension. This is because cell's open circuit and terminal
voltages strongly depend on the concentration at the particle surface. Spectral
methods offer a combination of high accuracy and speed while permitting
the use of a lower number of radial discretisation nodes. To implement a
spectral scheme on a non-periodic domain, a Chebyshev discretisation may be
applied~\cite{Trefethen2000}. Bizeray~\etal discretised all of the \gls{p2d}
model equations using this approach~\cite{Bizeray2015a}. However, this entails a
bi-directional mapping of all variables between the physical and Chebyshev
domains, incurring computational overhead.

Here we propose the use of a hybrid formulation of the \gls{p2d} model wherein a
standard \gls{fv} scheme in the axial dimension and a spectral scheme in the
radial domain are used. By exploiting the natural separation of the axial and
radial domains, we \romanletter{1}) retain the ability to easily couple the
molar flux density at the particle surface through reformulation of the
boundary conditions of the solid diffusion pde and \romanletter{2}) solve for
solid-phase lithium concentration in the Chebyshev domain and locally transform
to physical domain, without requiring system-wide Chebyshev reformulations.
Although the proposed implementation does not globally employ a spectral
scheme, the combined beneficial effects of radial-domain spectral scheme and
automatic differentiation of system equations using CasADi~\cite{Andersson2013b}
facilitates rapid simulation, enabling layer optimisation on short time-scales.
Eqns\cref{eqn:defineChebNodes}--\cref{eqn:solidDiffEqChebDomain} detail the
steps leading to the reformulated solid phase diffusion and its associated
boundary condition in the Chebyshev domain.

The Chebyshev collocation nodes defined on a 1D mesh in the radial direction are
given by\cref{eqn:defineChebNodes}~\cite{Trefethen2000}.

\begin{equation}\label{eqn:defineChebNodes}
    \tilde{r} = \cos\left(\frac{i\pi}{N_\text{r}}\right), \qquad i = 0, 1, \dots N_\text{r} \quad \tilde{r} \in [-1, 1]
\end{equation}

Assuming constant diffusivity, and expanding the derivative in the standard
form of the Fickian spherical diffusion equation for each particle
(refer~\ref{sec:Supplementary}) we obtain \cref{eqn:quotientappliedpde},
presented along with its Neumann boundary conditions. $j$ is the molar flux
density (\si{mol.m^{-2}.s^{-1}}) and $R_\text{p}$ is the particle radius
(\si{m}).

\begin{align}
    \frac{\partial c_\text{s}}{\partial t} &= D^\text{eff}_\text{s} \left( \frac{\partial}{\partial r} \frac{\partial c_\text{s}}{\partial r} + \frac{\partial^2 c_\text{s}}{\partial r^2} \right) \qquad r \in [0, R_\text{p}]\label{eqn:quotientappliedpde}\\
\frac{\partial c_\text{s}}{\partial r}\bigg\rvert_{r=0} &= 0\tag{\ref{eqn:quotientappliedpde}a}\\
 D^\text{eff}_\text{s}\frac{\partial c_\text{s}}{\partial r}\bigg\rvert_{r=R_\text{p}} &= {}-j\tag{\ref{eqn:quotientappliedpde}b}
\end{align}

{Mapping} $r \in [0,R_\text{p}] \mapsto \tilde{r} \in [-1, 1]$,
\begin{align}\label{mappingChebDomain}
    r = \frac{R_\text{p}}{2}(\tilde{r} + 1)
\end{align}

Applying \cref{mappingChebDomain} to \cref{eqn:quotientappliedpde}
whilst retaining $c_\text{s}$ in the physical space yields
\cref{eqn:solidDiffEqChebDomain}.

\begin{align}
	\frac{\partial c_\text{s}}{\partial t} &= 4 \frac{D^\text{eff}_\text{s}}{R_\text{p}^2} \left( \frac{2}{\tilde{r} + 1} \frac{\partial c_\text{s}}{\partial \tilde{r}} + \frac{\partial^2 c_\text{s}}{\partial {\tilde{r}}^2} \right)\label{eqn:solidDiffEqChebDomain}\\
\frac{\partial c_\text{s}}{\partial \tilde{r}}\bigg\rvert_{\tilde{r}=-1} &= 0\tag{\ref{eqn:solidDiffEqChebDomain}a}\\
	2 \frac{D^\text{eff}_\text{s}}{R_\text{p}} \frac{\partial c_\text{s}}{\partial \tilde{r}}\bigg\rvert_{\tilde{r}=1} &= -j\tag{\ref{eqn:solidDiffEqChebDomain}b}
\end{align}

During the iterative solution process, the spatial gradients of solid phase
lithium concentration in \cref{eqn:solidDiffEqChebDomain} are not computed
through an explicit differentiation procedure, but instead evaluated by
pre-multiplying the concentration values at the collocation nodes by a
Chebyshev differentiation matrix. This particular fact is responsible for
the inherent reduction of simulation runtime achieved by introducing a
spectral method. In the modified version of LIONSIMBA used in the layer
optimisation methodology, differentiation matrices of suitable dimension as well
as the Chebyshev collocation nodes are generated using the MATLAB function
\texttt{cheb.m}~\cite{Trefethen2000}.

% Intend to provide a call-graph in the appendix



% -*- root: ../../main.tex -*-
%!TEX root = ../../main.tex
% vim:nospell


\begin{table}[!htbp]
	\renewcommand{\thetable}{\arabic{table}a}
	\centering
	\caption{Acceleration test parameters (common across xEV platforms)}
	\label{tbl:CommonVehicleParams}
	\sisetup{table-format=3.2, table-number-alignment=center, table-space-text-pre=\textsuperscript{a}, table-space-text-post=\textsuperscript{a}, table-align-text-post=false}
	\begin{threeparttable}[t]
		\centering
		\begin{tabular}{@{} l  S @{}}
			\toprule
			Parameter \\
			\midrule

			% Coefficient of drag for xEV body, $C_\mathrm{d}$                           & {\makebox*{00}[r]{\tnote{a}}} 0.31                \\
			% Frontal area of xEV, $A_\mathrm{v}$ \si{(m^2)}                             & {\makebox*{00}[r]{\tnote{b}}} 2.40                \\
			% Acc.\ time specified by manufacturer, $t_\mathrm{f,man}$ \si{(s)}          & {\makebox*{00}[r]{\tnote{d}}} 6.50                \\
			% Acc.\ time dictated by standards, $t_\mathrm{f,std}$ \si{(s)}              & {\makebox*{00}[r]{\tnote{c}}} 6.00                \\
			% Speed, end of acc. (standards), $v_\mathrm{f,std}$ \si{(m.s^{-1})}         & {\makebox*{00}[r]{\tnote{e}}} 8.94                \\
			% Speed, end of acc. (manufacturer), $v_\mathrm{f,man}$ \si{(m.s^{-1})}      & {\makebox*{0}[r]{\tnote{f}}} 26.82                \\
			% Base speed of  xEV, $v_\mathrm{b}$ \si{(m.s^{-1})}                         & {\makebox*{\hspace*{0.5mm}0}[r]{\tnote{e}}} 13.41 \\
			% Air density at acc.\ test conditions, $\rho_\mathrm{air}$ \si{(kg.m^{-3})} & {\makebox*{\hspace*{0.5mm}00}[r]{\tnote{f}}} 1.20 \\
			% Drivetrain efficiency, $\eta_\mathrm{dt}$                                  & {\makebox*{00}[r]{\tnote{g}}} 0.75                \\
			% Payload, $M_\mathrm{p}$ \si{(kg)}                                          & {\hspace*{0.00005mm}{\tnote{c}}} 150.60 \\
			% Rolling resistance coefficient of road surface, $C_\mathrm{r}$             & {\makebox*{00}[r]{\tnote{f}}} 0.01                \\
			% Road gradient, $Z$                                                         & {\makebox*{00}[r]{\tnote{g}}} 0.00                \\

			Coefficient of drag for xEV body, $C_\mathrm{d}$                           & 0.31   {\tnote{a}} \\
			Frontal area of xEV, $A_\mathrm{v}$ \si{(m^2)}                             & 2.40   {\tnote{b}} \\
			Acc.\ time specified by manufacturer, $t_\mathrm{f,man}$ \si{(s)}          & 6.50   {\tnote{d}} \\
			Acc.\ time dictated by standards, $t_\mathrm{f,std}$ \si{(s)}              & 6.00   {\tnote{c}} \\
			Speed, end of acc. (standards), $v_\mathrm{f,std}$ \si{(m.s^{-1})}         & 8.94   {\tnote{e}} \\
			Speed, end of acc. (manufacturer), $v_\mathrm{f,man}$ \si{(m.s^{-1})}      & 26.82  {\tnote{f}} \\
			Base speed of  xEV, $v_\mathrm{b}$ \si{(m.s^{-1})}                         & 13.41  {\tnote{e}} \\
			Air density at acc.\ test conditions, $\rho_\mathrm{air}$ \si{(kg.m^{-3})} & 1.20   {\tnote{f}} \\
			Drivetrain efficiency, $\eta_\mathrm{dt}$                                  & 0.75   {\tnote{g}} \\
			Payload, $M_\mathrm{p}$ \si{(kg)}                                          & 150.60 {\tnote{c}} \\
			Rolling resistance coefficient of road surface, $C_\mathrm{r}$             & 0.01   {\tnote{f}} \\
			Road gradient, $Z$                                                         & 0.00   {\tnote{g}} \\

			\bottomrule
		\end{tabular}
        \begin{tablenotes}[para,flushleft]
        \item[a]Ref.~\cite{HybridCars2017Drag}
        \item[b]Calculated from typical \gls{bev} dimensions in~\cite{BoltDimensions}
        \item[c]Ref.~\cite{ETANTP002-2004}
        \item[d]Ref.~\cite{BoltOverview}
        \item[e]Ref.~\cite{Liu2016a}
        \item[f]Ref.~\cite{EmadiElectric}
        \item[g]Assumed
        \end{tablenotes}
	\end{threeparttable}
\end{table}

% -*- root: ../../main.tex -*-
%!TEX root = ../../main.tex
% vim:nospell


\begin{table}[!htbp] % Parameters unique to each of the BEV & PHEV
	% \addtocounter{table}{-1}
	% \renewcommand{\thetable}{\arabic{table}b}
	\caption{Acceleration test parameters (specific to each \glsfmtshort{xeV})}
	\label{tbl:UniqueVehicleParams}
	\centering
    \sisetup{table-format=4.1, table-number-alignment=center, table-space-text-pre=\textsuperscript{a}, table-align-text-pre=false}
	\begin{threeparttable}[t]
		\begin{tabular*}{0.675\textwidth}{@{} l @{\extracolsep{\fill}}  S S @{}}	% Works with Tnote
			\toprule
			\multicolumn{1}{@{} l}{Parameter} & \multicolumn{1}{c}{BEV} & \multicolumn{1}{c@{}}{PHEV} \\
			\midrule

			Mass of xEV chassis, $M_\mathrm{c}$ \si{(kg)}               & \Tnote{a} 1340.0 & \Tnote{b} 1438.0 \\
			Mass of pack overhead (w/o cells), $M_\mathrm{o}$ \si{(kg)} & \Tnote{a} 196.4  & \Tnote{c} 65.5   \\
			Upper cutoff SOC of cell, $z_\mathrm{max}$ \si{(\%)}        & \Tnote{d} 95.0   & \Tnote{d} 90.0   \\
			Lower cutoff SOC of cell, $z_\mathrm{min}$ \si{(\%)}        & \Tnote{d} 5.0    & \Tnote{e} 30.0   \\

			\bottomrule
		\end{tabular*}
		\begin{tablenotes}[para,flushleft]
		\item[a]Calculated based on~\cite{ChevyBoltSpecs}
		\item[b]Calculated based on~\cite{motortrendEcotec,ChevyBoltSpecs}
		\item[c]Calculated see \cref{sec:Configurations}
		\item[d]Assumed
		\item[e]Ref.~\cite{EmadiElectric}
		\end{tablenotes}

	\end{threeparttable}
\end{table}


\section{To figure out name}\label{sec:Configurations}

% -*- root: ../../main.tex -*-
%!TEX root = ../../main.tex
% vim:nospell

\begin{table}[htb!]
    \caption{\glsfmtshort{xeV} acceleration test results}
    \label{tbl:accResults}
    \centering
	\begin{tabular}{c c c}
        \toprule
        \multicolumn{1}{@{} l}{\makecell{($T_\text{init},T_\text{sink}$) \\ \footnotesize (degC)}} & \makecell{$n^\text{acc}_\text{opt}$ \\ \footnotesize \glsfmtshort{bev}}&  \multicolumn{1}{c @{}}{\makecell{$n^\text{acc}_\text{opt}$ \\ \footnotesize \gls{phev}}}  \\
        \midrule

        (38,5)  & \num{21} & \num{55} \\
        (38,49) & \num{21} & \num{57} \\
        (25,25) & \num{23} & \num{63} \\
        (15,5)  & \num{27} & \num{69} \\

        \bottomrule
    \end{tabular}
\end{table}


\begin{figure}[!bp]
    \begin{minipage}[t]{\textwidth}
        \centering
        \includegraphics[width=\textwidth]{fig_generate_heatmap_BEV}
        \caption[Optimal cell layer configurations for the \gls{bev}, presented for a range of fast charging powers and thermal conditions]{Optimal cell layer configurations for the \gls{bev}, presented for a range of fast charging powers and thermal conditions\footnotemark.}
        \label{fig:fig_generate_heatmap_BEV}
        \mpfootnotes[1]
        \footnote{This figure was created by \mbox{Ian Campbell} who asserts copyright,
            with intellectual contributions from and the right to use asserted by
        \mbox{Krishnakumar Gopalakrishnan}.}
    \end{minipage}
\end{figure}

\begin{figure*}[!bp]
    \begin{minipage}[t]{\textwidth}
        \centering \includegraphics[width=\textwidth,trim=4 2 3 4,clip]{fig_capacity_quadrants.pdf}
        \caption[The plots in the right column show the nominal cell capacity and charge passed
        during \gls{xeV} \gls{cp} charging. Increased rate capability and cell utilisation are positively
        correlated with $n$, while the maximum-$q$ layer configuration clearly shifts to higher
        values of $n$ with increasing charging powers. The plots in the left column depict
        galvanostatic charging scenarios at various currents to highlight the similarity with the
        \gls{cp} process. All data obtained at $T_\text{init} =$ \SI{25}{\degreeCelsius},
        $T_\text{sink} =$ \SI{25}{\degreeCelsius}.]{The plots in the right column show the nominal cell capacity and charge passed
            during \gls{xeV} \gls{cp} charging. Increased rate capability and cell utilisation are positively
            correlated with $n$, while the maximum-$q$ layer configuration clearly shifts to higher
            values of $n$ with increasing charging powers. The plots in the left column depict
            galvanostatic charging scenarios at various currents to highlight the similarity with the
            \gls{cp} process. All data obtained at $T_\text{init} =$ \SI{25}{\degreeCelsius},
        $T_\text{sink} =$ \SI{25}{\degreeCelsius}\footnotemark.}\label{fig:fig_CapacityQuadrants}
        \mpfootnotes[1]
        \footnote{This figure was created by \mbox{Ian Campbell} who asserts copyright,
            with intellectual contributions from and the right to use asserted by
        \mbox{Krishnakumar Gopalakrishnan}.}
    \end{minipage}
\end{figure*}

\begin{figure}[!bp]
    \begin{minipage}[t]{\textwidth}
        \centering
        \includegraphics[width=\textwidth]{fig_generate_heatmap_PHEV}
        \caption[Optimal cell layer configurations for the \gls{phev}, presented for a range of
        fast charging powers and thermal conditions]{Optimal cell layer configurations for the \gls{phev}, presented for a range of
        fast charging powers and thermal conditions\footnotemark.}
        \label{fig:fig_generate_heatmap_PHEV}
        \mpfootnotes[1]
        \footnote{This figure was created by \mbox{Ian Campbell} who asserts copyright,
            with intellectual contributions from and the right to use asserted by
        \mbox{Krishnakumar Gopalakrishnan}.}
    \end{minipage}
\end{figure}


% satisfying specific acceleration and fast charging targets.
% that could potentially facilitate immediate adoption by
% industry.

% The proposed methodology accounts for the critical need to avoid lithium
% plating during fast charging and searches for the optimal layer configuration
% considering a range of thermal conditions. A numerical implementation of a cell
% model using a hybrid finite volume-spectral scheme is presented, wherein the
% model equations are suitably reformulated to directly accept power inputs,
% facilitating rapid and accurate searching of the layer design space. We show how
% thermal management design can limit vehicle driving range at high charging
% temperatures. We highlight how electrode materials exhibiting increased solid
% phase diffusion rates are as equally important for extended range as developing
% new materials with higher inherent capacity. We illustrate for a plug-in hybrid
% vehicle, how the proposed methodology facilitates common module design of
% battery packs, thereby reducing the cost of derivative vehicle models. To
% facilitate model based layer optimisation, we provide the open-source toolbox,
% BOLD (Battery Optimal Layer Design).

% Explain the goal of this chapter, to arrive at the number
% of layers to meet a given energy and power demand. The outcome is that a ready
% to use tool is made available to validate empirical layer choices. Special
% emphasis is placed on the \emph{methodology} since the results per se do not
% stand alone outside of the modelling regimen (universe we created). However, the
% value is on the methodology and its implementation in a toolbox which is
% immediately available for download and use by industry to confirm their
% empirical layer designs.


%%In the absence of access to cell manufacturing facilities to confirm and test the layer
%% Immediate adoption in industry.

