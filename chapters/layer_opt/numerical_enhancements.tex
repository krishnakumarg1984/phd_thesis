% -*- root: ../../main.tex -*
%!TEX root = ../../main.tex
% vim:textwidth=80 fo=cqt conceallevel=0

\subsection{Augmentation of parameters to standard \glsfmtshort{dfn} model}

\subsubsection*{Cell capacity and electrochemically active surface area}

The  \gls{p2d} implementation  of  the standard  \gls{dfn}  model lacks  certain
parameters that are vital to the  layer optimisation process. The cell's nominal
capacity is a fundamental quantity that gets  altered as the number of layers is
varied. However, it may be surprising  to discover that this parameter is absent
in present  research literature  discussing the  \gls{p2d} model.  The rationale
behind this glaring omission becomes clear  upon closer examination of the model
equations presented in  \cref{tbl:dfneqns}. These equations do not  operate on a
cell level, but  instead operate on a  normalised basis \ie{} only  one layer is
modelled on a  \emph{unit area} basis wherein the stimulus  driving the model is
the applied current  \emph{density} rather than the total  external current. The
layer  optimisation task  here faces  a unique  predicament of  adhering to  the
present modelling paradigm  to retain compatibility with  standard models whilst
still incorporating  the concept of cell's  capacity as a function  of number of
layers.

To  tackle the  aforementioned quandary,  it  is key  to realise  that the  core
parameter  that  varies with  the  number  of layers  in  a  pouch cell  is  the
\emph{electrochemically  active  cross-sectional surface  area}~$A_\text{cell}$.
Curiously, published  literature on  physics-based cell  modelling do  not place
rigorous emphasis on  this key parameter. Most often, to  this author's chagrin,
this  parameter is  simply  listed in  a  standard table  of  parameters and  is
typically  sourced from  a historic  parameter-set with  no further  explanation
provided.

For  a pouch  cell, the  overall electrochemically  active surface  area can  be
defined as
\begin{equation}\label{eq:overallarea}
    A_\text{cell} = n \times A_\text{elec}
\end{equation}
where $n$ is  the number of layers and $A_\text{elec}$  is the electrochemically
active surface area per layer.

\subsubsection*{Surface area per layer}\label{sec:surfareaperlayer}

A literature search reveals that akin  to cell capacity, there is no information
of  cross-sectional geometry  provided in  articles dealing  with the  \gls{p2d}
implementation of  the \gls{dfn} model. To  determine the surface area  per face
(per layer), the author proposes  a new methodology/process involving a sequence
of  steps, based  on  certain  assumptions and  literature  search. The  process
involves  selection of  a  real-world cell,  and ultimately  mapping  it to  the
surface area per  unit face. To the  best knowledge of the  author, this reverse
parametrisation process,  mapping from  a real-world cell  to a  \gls{p2d} model
parameter  is unique  and is  claimed  as a  contribution  to the  art which  is
explained next.

\begin{enumerate}[
    leftmargin=0pt, itemindent=20pt,
    labelwidth=15pt, labelsep=5pt, listparindent=0.7cm,
    align=left]
\item \hypertarget{refcellselection}{\textbf{Selection of a suitable reference capacity cell}}

    Although  this value  shall not  be used  in the  actual layer  optimisation
    procedure itself, this is a  crucial first-step towards obtaining a complete
    parameter set, in particular, to obtain the surface area per layer.

    The focus  of this chapter is  to provide ready-to-use solution  to industry
    extend  with the  aim of  improving  the current  state of  the art  through
    optimal layer configuration  of pouch cells. There is a  clear motivation to
    further increase  cell capacities so as  to maximize driving range,  as laid
    out in the beginning of this chapter (see \cref{sec:layeroptintro}).

    With  this  guiding  principle,  as  a starting  point  towards  choosing  a
    reference capacity cell,  a survey was performed to  identify the production
    \gls{bev}  with  the  highest  driving  range.  As  of  2018,~the  Chevrolet
    Bolt~\gls{bev} bears this distinction  with a range of \SI{383}{\kilo\meter}
    as rated  by the  United States Environmental  Protection Agency  (EPA). The
    specifications of  its battery pack is  listed in Liu~\etal~\cite{Liu2016a}.
    The  battery pack  of  this  vehicle consists  of  288~cells  arranged in  a
    96S-3P configuration, in  agreement with the configuration  discussed in the
    drivetrain  hierarchy  of  \cref{sec:packlevelhierarchy}.

    The    Chevrolet   Bolt    \gls{bev}   pack    has   an    energy   capacity
    of   \SI{60.0}{\kilo\watthour}    with   a    nominal   pack    voltage   of
    \SI{350}{\volt}~\cite{Liu2016a}.  However,  for   this  specific  task,  the
    \si{\amphour} capacity is required. This can be obtained
    \begin{align}
        \text{\si{\amphour} capacity of Bolt cell} & = \frac{\text{Pack Energy } (\si{\watthour})}{\text{Nominal pack voltage (V)} \times {\text{Number of cells in parallel}}} \\
        {}                                         & = \frac{60000}{350 \times 3} \\
        {}                                         & = \SI{57.14}{\amphour}
    \end{align}

    \begin{itemize}[ leftmargin=10pt, itemindent=15pt, labelwidth=5pt, labelsep=5pt, listparindent=0.7cm, align=left]
        \item \textbf{DC bus voltage and revised cell capacity}

            Even  without  a  dc/dc  boost   converter,  robust  design  of  the
            powertrain during  brown-outs should  allow for  continued operation
            even   with   a  slightly   diminished   DC   bus  voltage   \approx
            \SIrange{4}{5}{\percent} lower than nominal~\cite{Maksimovic2012}.
            Considering a maximum permissible dip of \SI{4}{\percent} in the bus
            voltage \ie{} \SI{336}{\volt}, the cell's capacity may be refined as
            \begin{align}
                \text{\si{\amphour} capacity of Bolt cell} & = \frac{\text{Pack Energy } (\si{\watthour})}{\text{Lowest pack voltage (V)} \times {\text{Number of cells in parallel}}} \\
                {}                                         & = \frac{60000}{336 \times 3} \\
                {}                                         & = \SI{59.52}{\amphour}
            \end{align}

            The reference cell's \si{\amphour} capacity is therefore rounded to
            \textbf{\SI{60}{\amphour}}\footnote{In the interest of maintaining consistency, this computed capacity is
                retained for the cell's used in
                \crefrange{ch:spmanalysis}{ch:newelectrolytemodel} of this thesis. This also explains the use of \SI{60}{\ampere} for
        the simulations demonstrating energy/power trade-off of
    \cref{sec:energypowertradeoffdemo} since this current level corresponds to
the 1C-rate of the cell.}.

        \item \hypertarget{celllowercutoff}{\textbf{Lower cutoff voltage for cells}}

            In  this  layer  optimisation  work, following  the  assumptions  of
            \cref{subsec:layeroptassumptions},  the  overall pack  configuration
            remains  unchanged  and  independent   of  layers  within  a  pouch.
            This  implies that  the undervoltage  threshold for  DC bus  voltage
            throughout  this   work  shall  remain  fixed   at  \SI{336}{\volt}.
            Therefore, with  96~series connected  cells in  a string,  the lower
            cut-off voltage for an  individual cell is \textbf{\SI{3.5}{\volt}}.
            This  value is  reported in  \cref{tbl:lcoSimParamslayeropt} and  is
            used as a termination condition  for all simulations as explained in
            \cref{sec:layeroptframework}.

    \end{itemize}


    \end{enumerate}

 Furthermore, convincingly explain how a pouch cell thickness of 10 mm
was used with all the background references.



%Describe briefly  the process of  mapping a  real-world cell onto  the standard
\gls{p2d} model. %Note: Northrop to EV mapping code

%\subsection{Modelling Platform and Preconditioning}

%Couple of statements  about why LIONSIMBA was chosen as  the modelling platform
%for implementing the p2d dynamics. The  cell parameters used are shown in table
%xx. This cell is henceforth known as the LIONSIMBA cell or Northrop cell.

%Discuss the  missing elements  in LIONSIMBA  only with  respect to  the present
%problem at hand, \viz{the stoichiometries}.

%\subsubsection*{Stoichiometry Augmentation}

%Discuss the problem first. How LIONSIMBA started always at 85.51 percentage and
%needed to  do a  discharge down to  zero percent before  having the  ability to
%charge. For this project, stoichiometries  are vital for capacity determination
%and  the 1C  current density.  Explain how  stoichiometries were  refined until
%cut-off for  infinitesimal bleeding discharge current  achieved. Noted relevant
%values. Explain refinement  of how approximate capacities  reported by Northrop
%and Subramanian were  refined precisely. Explanation of  remnant capacities and
%stoichiometries computation. Explanation of  1C current density.parameters init
%capacity computation code.

%\subsection{Layer Assembly within Pouch Cells}
%With  the  help  of  Northrop's   layer  assembly  figure,  explain  the  layer
%configuration/arrangement within a pouch cell. The next task is then identified
%as computing the number of layers within the reference pouch cell.

%\subsubsection*{Number of Layers of LIONSIMBA aka Northrop cell}
%Krishna came up with the idea of using integer optimisation for this task. The
%software MIDACO was also selected by Krishna and explained to Ian. The MIDACO
%result of the number of layers within the standard cell was now available.

%\subsubsection*{Computation of Surface Area per face, \protect{$A_\text{cell}$}}
%Show the simple algebraic computation of overall surface area $A$ and the
%per-face area $A_\text{cell}$. Explain how the area per face shall be a key
%quantity in the layer optimisation framework discussed later on.

%Layerphoto showing face areas and anode/cathode verhand etc will be shown here

%This concludes the augmented set of parameters added by the author to the basic
%parameter set  of the DFN  model. The added  numerical value of  parameters are
%summaried in table xx. Lots The layer optimisation framework and assumptions is
%described next.
