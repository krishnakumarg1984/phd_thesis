% -*- root: ../../main.tex -*
%!TEX root = ../../main.tex
% vim:textwidth=80 fo=cqt conceallevel=0


At the outset, it  is worth mentioning that the focus of this  chapter is on the
layer optimisation \emph{methodology}  itself. The results as such  do not stand
alone  outside of  the  modelling  universe with  all  its inherent  assumptions
discussed  thus far.  Presently,  the value  added  by this  work  is its  ready
adaptability to industry through its  modular design. A numerical implementation
in the form of a toolbox
\footnote{As an accompaniment to this chapter, an open-source software toolbox,
(co-created with \mbox{Ian D.\ Campbell} and \mbox{Davide M.\ Raimondo}) for optimal layer selection in pouch cells \viz~\gls{bold} is made available for download from GitHub. \\ \mbox{\href{https://github.com/ImperialCollegeESE/BOLD_Toolbox}{\includegraphics [width=0.025\textwidth]{github.pdf}}} \url{ https://github.com/ImperialCollegeESE/BOLD_Toolbox}}
is  also  provided which  is  immediately  available  for  download and  use  by
relevant stakeholders. This author recommends  that until the tool matures, cell
manufacturers  substitute  their  own  parameters  and  adjust  other  numerical
coefficients suitably  so that  the toolbox  supplements, rather  than supplants
present empirical  layer designs. Hence,  the results presented in  this section
must be interpreted in the backdrop  of the context within which the methodology
was developed implying that the reader  must consciously strive to interpret all
numerical  values in  \emph{relative} terms  of magnitude.  To aid  this thought
process, this  author chooses  to deliberately limit  the discussion  around the
\emph{absolute} magnitude of numbers presented here.

\subsection{Modelling Platform and Preconditioning}\label{sec:platformlionsimba}

% -*- root: ../../main.tex -*-
%!TEX root = ../../main.tex
% vim:nospell

\begin{table}[!htbp]
    \small
    \caption[%
    System-level simulation conditions \& thermal parameters of  an \glsfmtshort{lco} cell
    ]%
    {%
        Cell   parameters   and   system   conditions  for   a   simulating   an
        \glsfmtshort{lco} cell  with the  \gls{dfn} electrochemical model  and a
        lumped thermal model. The parameters  presented here when augmented with
        the  values  of  the  kinetic, geometric  and  transport  properties  of
        the  cell (from  \cref{tbl:lcoSimParamsSPMp2d}  represents the  complete
        information  required for  all  simulations in  this layer  optimisation
        framework.
    }%
    \label{tbl:lcoSimParamslayeropt}
    \vspace{-2.6229525pt}
    \begin{threeparttable}
        \centering
        \textbf{System Conditions} \\ \smallskip
        \begin{varwidth}[t]{0.48\linewidth}
            \begin{tabular*}{\textwidth}{@{} l @{\extracolsep{\fill}} S[table-format=1.2,table-space-text-pre=\Tnote{a} ,table-align-text-pre=false] @{}}
                \toprule
                \multicolumn{1}{@{}l}{Parameter} \\
                \midrule

                Lower cutoff cell voltage, $V_\text{min}$ (\si{\volt}) & \Tnote{a} 3.50   \\
                Upper cutoff cell voltage, $V_\text{max}$ (\si{\volt}) & \Tnote{b} 4.22   \\

                \bottomrule
            \end{tabular*}
        \end{varwidth}
        \hfill
        \begin{varwidth}[t]{0.48\linewidth}
            \begin{tabular*}{\textwidth}{@{} l @{\extracolsep{\fill}} S[table-format=2.2,table-space-text-pre=\Tnote{a} ,table-align-text-pre=false] @{}}
                \toprule
                \multicolumn{1}{@{}l}{Parameter} \\
                \midrule

                Target cell SOC for fast charge, $z^\ast$ \si{(\%)}                & \Tnote{c} 80.00 \\
                Cell upper temperature limit, $T_\text{max}$ \si{(\degreeCelsius)} & \Tnote{d} 55.00 \\

                \bottomrule
            \end{tabular*}
        \end{varwidth}

        \medskip
        \begin{tabular*}{\textwidth}{@{} l @{\extracolsep{\fill}} r @{}}
            \multicolumn{2}{c}{\textbf{Geometric Parameters}} \\
            \toprule
            \multicolumn{1}{@{}l}{Parameter} \\
            \midrule
            Surface area of pos.\ \& neg.\ electrode overlap within a layer, {$A_\text{elec}$} \si{(m^2)} & \textsuperscript{b}\num{4.19e-2}   \\
            Exterior pouch length, $L_\text{pouch}$ \si{(m)}                                              & \textsuperscript{e}\num{332.74e-3} \\
            Exterior pouch width, $W_\text{pouch}$ \si{(m)}                                               & \textsuperscript{e}\num{99.06e-3}  \\
            Exterior pouch height, $H_\text{pouch}$ \si{(m)}                                              & \textsuperscript{f}\num{10.00e-3}  \\
            Pouch material thickness, $T_\text{pouch}$ \si{(m)}                                           & \textsuperscript{g}\num{160.00e-6} \\
            Stack thickness, $L_\text{stack}$ \si{(m)}                                                    & \textsuperscript{r}\num{9.68e-3}  \\
            \bottomrule
        \end{tabular*}
        \medskip
        \centering \textbf{Thermal Parameters} \\ \smallskip
        \resizebox{\textwidth}{!}{%
            \begin{tabular}{@{} l S[table-format=4.0,table-space-text-pre=\Tnote{m} ,table-align-text-pre=false] S[table-format=4.1,table-space-text-pre=\Tnote{m} ,table-align-text-pre=false] S[table-format=4.1,table-space-text-pre=\Tnote{m} ,table-align-text-pre=false] S[table-format=4.2,table-space-text-pre=\Tnote{m} ,table-align-text-pre=false] S[table-format=4.0,table-space-text-pre=\Tnote{m} ,table-align-text-pre=false] S[table-format=4.1,table-space-text-pre=\Tnote{m} ,table-align-text-pre=false] S[table-format=4.1,table-space-text-pre=\Tnote{m} ,table-align-text-pre=false] @{}}
                \toprule
                \multicolumn{1}{@{}l}{Parameter} & \multicolumn{1}{c}{Al.\ CC} & \multicolumn{1}{c}{Pos} & \multicolumn{1}{c}{Sep} & \multicolumn{1}{c}{Neg} & \multicolumn{1}{c}{Cu.\ CC} & \multicolumn{1}{c}{\ch{LiPF_6}} & \multicolumn{1}{r@{}}{Pouch}\\
                \midrule

                Sp.\ heat capacity, $c_j$ (\si{\joule\per\kilogram\per\kelvin})   & \Tnote{h} 903  & \Tnote{h} 1269.2 & \Tnote{h} 1978.2 & \Tnote{h} 1437.4 & \Tnote{h} 385  & \Tnote{h} 2055.1 & \Tnote{i} 1464.8 \\
                Density, $\rho_j$ (\si{\kilogram\per\meter\cubed})                & \Tnote{j} 2700 & \Tnote{k} 2291.6 & \Tnote{b} 1100.0 & \Tnote{j} 2660.0 & \Tnote{l} 8960 & \Tnote{j} 1290.0 & \Tnote{m} 1150.0 \\
                Activ.\ energy, diff. ${E_\text{act,s}}_j$ (\si{\joule\per\mole}) & {---}                   & \Tnote{p} 5000   & {---}                     & \Tnote{p} 5000   & {---}                   & {---}                     & \multicolumn{1}{c}{---}   \\
                Activ.\ energy, rxn. ${E_\text{act,k}}_j$ (\si{\joule\per\mole})  & {---}                   & \Tnote{p} 5000   & {---}                     & \Tnote{p} 5000   & {---}                   & {---}                     & \multicolumn{1}{c}{---}   \\

                \bottomrule
            \end{tabular}
        }
        \medskip
        \begin{tabular*}{\textwidth}{@{} l @{\extracolsep{\fill}} r @{}}
            \multicolumn{2}{c}{\textbf{Other Geometric/Cell-Level Parameters}} \\
            \toprule
            \multicolumn{1}{@{}l}{Parameter} \\
            \midrule

            Thickness of pos.\ current collector, $l_\text{Al}$ \si{(m)}                    & \textsuperscript{f}\num{15e-6}   \\
            Thickness of neg.\ current collector, $l_\text{Cu}$ \si{(m)}                    & \textsuperscript{p}\num{10e-6}   \\
            Total tab area, $A_\text{tabs}$ \si{(m^2)}                                      & \textsuperscript{b}\num{5.94e-3} \\
            Lumped heat transfer coefficient, $h$ (\si{\watt\per\meter\squared\per\kelvin}) & \textsuperscript{b}150           \\
            Initial electrolyte concentration, $c_\text{e,0}$ (\si{\mole\per\meter\cubed})  & \textsuperscript{q}1000          \\

            \bottomrule
        \end{tabular*}

        \medskip
        \begin{tabular*}{\textwidth}{@{} =P{7.5cm}  +l@{\extracolsep{\fill}}+c +r @{}}
            \multicolumn{4}{c}{\textbf{Spatial Discretisation}} \\
            \toprule
            \multicolumn{1}{@{}l}{Parameter} & \multicolumn{1}{l}{Pos} & \multicolumn{1}{c}{Sep} & \multicolumn{1}{r@{}}{Neg}\\
            \midrule

            Nodes, through-thickness (axial), $N_{\text{a}_j}$          & \num{40} & \num{40} & \num{40} \\
            Nodes, within spherical particle (radial), $N_{\text{r}_j}$ & \num{15} & ---      & \num{15} \\

            \bottomrule
        \end{tabular*}

        \smallskip
        % \vspace{-2.6229525pt}
        \vspace*{-5pt}
        \begin{tablenotes}[para,flushleft]
            \begin{footnotesize}
            \item[a] Calculated in section \hyperlink{celllowercutoff}{`Lower cutoff voltage for cells'} (also see \cref{sec:surfareaperlayer})
            \item[b] Assumed
            \item[c] Ref.~\cite{Sae2010}
            \item[d] Ref.~\cite{Kizilel2009}
            \item[e] Converted from imperial units reported in~Ref.~\cite{GMBoltBatteryDims}
		    \item[f] Table~\romanletter{4} of~Ref.~\cite{Groger2015} \\
            \item[g] Sum of values in table~1 of~Ref.~\cite{Svens2013}
            \item[h] Ref.~\cite{Chen2005} \\
            \item[i] Computed from values of constituents (see~\cite{Svens2013}) using Ref.~\cite{martienssen2006springer} \\
            \item[j] Ref.~\cite{Guo2010}
            \item[k] Ref.~\cite{Jeon2011}
            \item[l] Ref.~\cite{Worwood2017,Song2000}
            \item[m] Ref.~\cite{Kim2009}
            \item[p] Ref.~\cite{Northrop2011}
            \item[q] Ref.~\cite{Subramanian2009} \\
            \item[r] See section \hyperlink{stackthickness}{`Compute stack thickness of reference cell'}
            \end{footnotesize}
        \end{tablenotes}
    \end{threeparttable}
\end{table}



The   complete   parameter   set   used   for   simulation   is   presented   in
\cref{tbl:lcoSimParamslayeropt}.  All   cells  are   assumed  to  be   in  their
equilibrium  state prior  to  beginning of  simulations. The  thermally-coupled,
\gls{p2d}  electrochemical  model  used  for simulating  each  layer  choice  is
implemented  in MATLAB~\cite{matlab}  using  a heavily-modified  version of  the
LIONSIMBA toolbox~\cite{Torchio2016}.  The work reported in  this chapter helped
to advance the  toolbox from~v1.0x to~v2.0. The updated computer  codes to which
this author heavily contributed,
is available from the project's official repository\footnote{LIONSIMBA~v2:
\mbox{\href{https://github.com/lionsimbatoolbox/LIONSIMBA}{\includegraphics
[width=0.025\textwidth]{github.pdf}}} \url{
https://github.com/lionsimbatoolbox/LIONSIMBA}}.

The  rationale  behind  choosing  this  specific  software  to  implement  layer
optimisation  is  as  follows.  The LIONSIMBA~v1.0x  toolbox  has  already  been
validated against  the results of the  DUALFOIL~\cite{Dualfoil1998} codes (which
can be considered as the present benchmark standard). The toolbox is implemented
in the  MATLAB programming language.  Since this  chapter has a  strong industry
focus,  the  omnipresence  of  MATLAB  in industry,  its  mature  code-base  and
familiarity  was  a strong  motivator  in  the  adoption  of this  toolbox.  The
simulation  speeds using  LIONSIMBA  have been  shown to  be  comparable to  the
FORTRAN  implementation  of  DUALFOIL,  primarily  owing  to  the  sophisticated
computation  of  the  analytical  Jacobian   of  the  system  through  automatic
differentiation~\cite{Torchio2016}.  In  addition  to  fundamental  enhancements
to  the   modelling  platform  presented   in  \cref{sec:numericalenhancements},
numerous  bug fixes  and  other  minor enhancements  to  the original  LIONSIMBA
code-base  have  been  provided  by   this  thesis  author.  Interested  readers
may  peruse   these  from  the   \texttt{README.md}  file  from   the  project's
\href{https://github.com/lionsimbatoolbox/LIONSIMBA}{repository}.

\subsection{\glsfmtshort{xeV} configurations}

% -*- root: ../../main.tex -*-
%!TEX root = ../../main.tex
% vim:nospell


\begin{table}[!htbp]
	\renewcommand{\thetable}{\arabic{table}a}
	\centering
	\caption{Acceleration test parameters (common across xEV platforms)}
	\label{tbl:CommonVehicleParams}
	\sisetup{table-format=3.2, table-number-alignment=center, table-space-text-pre=\textsuperscript{a}, table-space-text-post=\textsuperscript{a}, table-align-text-post=false}
	\begin{threeparttable}[t]
		\centering
		\begin{tabular}{@{} l  S @{}}
			\toprule
			Parameter \\
			\midrule

			% Coefficient of drag for xEV body, $C_\mathrm{d}$                           & {\makebox*{00}[r]{\tnote{a}}} 0.31                \\
			% Frontal area of xEV, $A_\mathrm{v}$ \si{(m^2)}                             & {\makebox*{00}[r]{\tnote{b}}} 2.40                \\
			% Acc.\ time specified by manufacturer, $t_\mathrm{f,man}$ \si{(s)}          & {\makebox*{00}[r]{\tnote{d}}} 6.50                \\
			% Acc.\ time dictated by standards, $t_\mathrm{f,std}$ \si{(s)}              & {\makebox*{00}[r]{\tnote{c}}} 6.00                \\
			% Speed, end of acc. (standards), $v_\mathrm{f,std}$ \si{(m.s^{-1})}         & {\makebox*{00}[r]{\tnote{e}}} 8.94                \\
			% Speed, end of acc. (manufacturer), $v_\mathrm{f,man}$ \si{(m.s^{-1})}      & {\makebox*{0}[r]{\tnote{f}}} 26.82                \\
			% Base speed of  xEV, $v_\mathrm{b}$ \si{(m.s^{-1})}                         & {\makebox*{\hspace*{0.5mm}0}[r]{\tnote{e}}} 13.41 \\
			% Air density at acc.\ test conditions, $\rho_\mathrm{air}$ \si{(kg.m^{-3})} & {\makebox*{\hspace*{0.5mm}00}[r]{\tnote{f}}} 1.20 \\
			% Drivetrain efficiency, $\eta_\mathrm{dt}$                                  & {\makebox*{00}[r]{\tnote{g}}} 0.75                \\
			% Payload, $M_\mathrm{p}$ \si{(kg)}                                          & {\hspace*{0.00005mm}{\tnote{c}}} 150.60 \\
			% Rolling resistance coefficient of road surface, $C_\mathrm{r}$             & {\makebox*{00}[r]{\tnote{f}}} 0.01                \\
			% Road gradient, $Z$                                                         & {\makebox*{00}[r]{\tnote{g}}} 0.00                \\

			Coefficient of drag for xEV body, $C_\mathrm{d}$                           & 0.31   {\tnote{a}} \\
			Frontal area of xEV, $A_\mathrm{v}$ \si{(m^2)}                             & 2.40   {\tnote{b}} \\
			Acc.\ time specified by manufacturer, $t_\mathrm{f,man}$ \si{(s)}          & 6.50   {\tnote{d}} \\
			Acc.\ time dictated by standards, $t_\mathrm{f,std}$ \si{(s)}              & 6.00   {\tnote{c}} \\
			Speed, end of acc. (standards), $v_\mathrm{f,std}$ \si{(m.s^{-1})}         & 8.94   {\tnote{e}} \\
			Speed, end of acc. (manufacturer), $v_\mathrm{f,man}$ \si{(m.s^{-1})}      & 26.82  {\tnote{f}} \\
			Base speed of  xEV, $v_\mathrm{b}$ \si{(m.s^{-1})}                         & 13.41  {\tnote{e}} \\
			Air density at acc.\ test conditions, $\rho_\mathrm{air}$ \si{(kg.m^{-3})} & 1.20   {\tnote{f}} \\
			Drivetrain efficiency, $\eta_\mathrm{dt}$                                  & 0.75   {\tnote{g}} \\
			Payload, $M_\mathrm{p}$ \si{(kg)}                                          & 150.60 {\tnote{c}} \\
			Rolling resistance coefficient of road surface, $C_\mathrm{r}$             & 0.01   {\tnote{f}} \\
			Road gradient, $Z$                                                         & 0.00   {\tnote{g}} \\

			\bottomrule
		\end{tabular}
        \begin{tablenotes}[para,flushleft]
        \item[a]Ref.~\cite{HybridCars2017Drag}
        \item[b]Calculated from typical \gls{bev} dimensions in~\cite{BoltDimensions}
        \item[c]Ref.~\cite{ETANTP002-2004}
        \item[d]Ref.~\cite{BoltOverview}
        \item[e]Ref.~\cite{Liu2016a}
        \item[f]Ref.~\cite{EmadiElectric}
        \item[g]Assumed
        \end{tablenotes}
	\end{threeparttable}
\end{table}


Tables~\ref{tbl:CommonVehicleParams} and~\ref{tbl:UniqueVehicleParams}  show the
\gls{xeV}  parameters used  in simulations.  The  power demands  on the  battery
pack  during  normal  operation  are   found  to  be  significantly  lower  than
that  experienced during  the  two  extreme cases  of  discharging and  charging
\viz~\emph{acceleration} and \emph{fast  charging} respectively. For instance,
\SI{50.83}{\kilo\watt} is the peak  discharge power while \SI{14.20}{\kilo\watt}
is  the median  discharge power  for various  standard drive  cycles. Even  with
the  assumption that  \SI{100}{\percent}  of braking  energy  can be  recovered,
the  peak  and  median  charging  powers  are  only  \SI{43.13}{\kilo\watt}  and
\SI{26.03}{\kilo\watt}  respectively.

The   discharging  and   charging  powers   experienced  by   the  pack   during
acceleration and  fast charge  are significantly  higher than  those experienced
with  any  standard  drivecycle.  Considering  the  acceleration  parameters  in
\cref{tbl:CommonVehicleParams} for  the \gls{bev}  pack, \SI{181.45}{\kilo\watt}
is   the  power   requirement  for   acceleration  of   a  fixed   vehicle  mass
on   a   flat  road   surface.   Four   distinct  fast-charging   power   levels
\viz~\SI{50}{\kilo\watt},   \SI{80}{\kilo\watt},  \SI{110}{\kilo\watt}   and
\SI{130}{\kilo\watt}  are considered  in this  study.  This is  in adherence  to
the  minimum  and maximum  values  of  level~3  rating  as suggested  by  Yilmaz
and~Krein~\cite{Yilmaz2012}. Furthermore, near-term fast charging goals laid out
in  literature~\cite{Ashique2017,Srdic2016} and  the  peak  power capability  of
charging infrastructure further justify these choices.

% -*- root: ../../main.tex -*-
%!TEX root = ../../main.tex
% vim:nospell


\begin{table}[!htbp] % Parameters unique to each of the BEV & PHEV
	% \addtocounter{table}{-1}
	% \renewcommand{\thetable}{\arabic{table}b}
	\caption{Acceleration test parameters (specific to each \glsfmtshort{xeV})}
	\label{tbl:UniqueVehicleParams}
	\centering
    \sisetup{table-format=4.1, table-number-alignment=center, table-space-text-pre=\textsuperscript{a}, table-align-text-pre=false}
	\begin{threeparttable}[t]
		\begin{tabular*}{0.675\textwidth}{@{} l @{\extracolsep{\fill}}  S S @{}}	% Works with Tnote
			\toprule
			\multicolumn{1}{@{} l}{Parameter} & \multicolumn{1}{c}{BEV} & \multicolumn{1}{c@{}}{PHEV} \\
			\midrule

			Mass of xEV chassis, $M_\mathrm{c}$ \si{(kg)}               & \Tnote{a} 1340.0 & \Tnote{b} 1438.0 \\
			Mass of pack overhead (w/o cells), $M_\mathrm{o}$ \si{(kg)} & \Tnote{a} 196.4  & \Tnote{c} 65.5   \\
			Upper cutoff SOC of cell, $z_\mathrm{max}$ \si{(\%)}        & \Tnote{d} 95.0   & \Tnote{d} 90.0   \\
			Lower cutoff SOC of cell, $z_\mathrm{min}$ \si{(\%)}        & \Tnote{d} 5.0    & \Tnote{e} 30.0   \\

			\bottomrule
		\end{tabular*}
		\begin{tablenotes}[para,flushleft]
		\item[a]Calculated based on~\cite{ChevyBoltSpecs}
		\item[b]Calculated based on~\cite{motortrendEcotec,ChevyBoltSpecs}
		\item[c]Calculated see \cref{sec:Configurations}
		\item[d]Assumed
		\item[e]Ref.~\cite{EmadiElectric}
		\end{tablenotes}

	\end{threeparttable}
\end{table}


For  the   acceleration  tests,  the   initial  cell  \gls{soc}  has   been  set
to~\SI{40}{\percent}. This  is in  conformity with  the test  criterion~${(50\pm
10)}$~\%  of  the  SAE~J1666  standard~\cite{Sae2010}.  By  choosing  the  worst
case  starting  \gls{soc}  \ie~\SI{40}{\percent},   a  conservative  design  can
be  achieved.  The  chassis  mass  of  the  vehicle  as  well  as  the  mass  of
two  passengers   at  75.3  kg   each~\cite{Sae2010}  is  considered   for  both
\gls{xeV}  platforms. The  pack mass  is  computed as  a function  of number  of
layers  as  described  in  \cref{sec:layeroptframework}.  Vehicle  manufacturers
General  Motors  Inc.\, provide  the  mass  value of  the  GM  Ecotec series  of
engines~\cite{motortrendEcotec} that can  be used for the  \gls{phev} case which
consists of  a range-extending \gls{ice}.  The mass  of the Bolt  \gls{bev} pack
reported in~\cite{ChevyBoltSpecs} minus  the computed mass of  the overall cells
used in the pack  gives the overhead mass of the  \gls{bev} pack. The \gls{phev}
pack's  overhead  mass  is  determined  by suitably  scaling  the  mass  by  the
proportion of reduction in the number of cells used.


For the \gls{bev}  platform, a fast-charging scheme operated on  a \gls{cp} mode
with an initial  \gls{soc} of \SI{20}{\percent} is employed. In  the case of the
\gls{phev}, an initial \gls{soc}  of \SI{30}{\percent} (\SI{10}{\percent} higher
than that for \gls{bev}) is used. This facilitates a smaller \gls{soc} window by
taking  into account  the higher  number  of charge-discharge  cycles which  are
typical with \gls{phev}  designs~\cite{Maksimovic2012}. Both \gls{xeV} platforms
are fast  charged to a target  \gls{soc} of \SI{80}{\percent} in  \gls{cp} mode.
This \gls{soc} value corresponds to the end-of-charge target in level~3 charging
standards~\cite{SAECharging2011}.

\subsection{Acceleration studies}

For both vehicle platforms under study, the acceleration at a worst-case rate of
\SI{4.13}{\meter\per\second\squared} is assumed for simulation. This corresponds
to the  manufacturer's acceleration specifications  for the \gls{bev}  listed in
\cref{tbl:CommonVehicleParams}.  The  acceleration  rate  corresponding  to  the
SAE~J1772 standards is lower than this  rate. Therefore, to obtain a robust cell
design, the higher of the two acceleration rates needs to be considered.

\Cref{tbl:accResults}  gives the  simulation  results  for various  combinations
of~${(T_\text{init},  T_\text{sink})}$  for  both the  \gls{bev}  and  \gls{phev}
platforms. The following discussion is applicable for both vehicular platforms.

% -*- root: ../../main.tex -*-
%!TEX root = ../../main.tex
% vim:nospell

\begin{table}[htb!]
    \caption{\glsfmtshort{xeV} acceleration test results}
    \label{tbl:accResults}
    \centering
	\begin{tabular}{c c c}
        \toprule
        \multicolumn{1}{@{} l}{\makecell{($T_\text{init},T_\text{sink}$) \\ \footnotesize (degC)}} & \makecell{$n^\text{acc}_\text{opt}$ \\ \footnotesize \glsfmtshort{bev}}&  \multicolumn{1}{c @{}}{\makecell{$n^\text{acc}_\text{opt}$ \\ \footnotesize \gls{phev}}}  \\
        \midrule

        (38,5)  & \num{21} & \num{55} \\
        (38,49) & \num{21} & \num{57} \\
        (25,25) & \num{23} & \num{63} \\
        (15,5)  & \num{27} & \num{69} \\

        \bottomrule
    \end{tabular}
\end{table}


The  specific combinations  of temperatures  for traversing  the thermal  design
space are  chosen following the  SAE~J1772 guidelines. The high  power densities
resulting from  low numbers of layers  lead to large overpotentials  causing the
cell's terminal  voltage to  drop lower than~$V_\text{min}$, thereby  unable to
satisfy acceleration requirements. However,  at higher~$T_\text{init}$, owing to
the  reduction in  overpotentials, a  larger  voltage overhead  is available  to
accommodate the  internal polarisation  drop. For all  temperature combinations,
the largest deviation from~$T_\text{init}$  experienced by the \gls{bev} cell is
a \SI{0.48}{\degreeCelsius} increase.  Consequently, it can be  concluded that a
single isolated acceleration event does not heat the \gls{bev} battery pack, and
therefore the cell  temperature remains close to that of  the initial value. The
\gls{phev} cell  experiences higher  power levels  and although  its temperature
increases  much  higher  than  the corresponding  \gls{bev}  cell,  the  maximum
temperature  during acceleration  remains  well below  the  upper cutoff  limit.
Furthermore, even  for the worst  case simulation  run, the cell's  \gls{soc} is
depleted only by  a maximum value of \SI{0.32}{\percent} for  the \gls{bev} cell
and by a slightly higher value for the \gls{phev} cell.

The foregoing  discussion has revealed  that the lower cut-off  voltage strongly
influences  layer   configuration  for   acceleration  tests.   Therefore,  when
considering acceleration requirements, ${n = 27}$ and ${n=69}$ represent the optimal
layer choices for the \gls{bev} and \gls{phev} platforms respectively .

\subsection{Fast-charging studies}

\begin{figure}[p]
    \begin{minipage}[t]{\textwidth}
        \centering
        \includegraphics[width=\textwidth]{fig_generate_heatmap_BEV}
        \captionsetup{labelsep=note}
        \caption[Optimal cell layer configurations for the \glsfmtshort{bev} for a range of fast charging powers and thermal conditions]{Optimal cell layer configurations for the \gls{bev}}
        \label{fig:fig_generate_heatmap_BEV}
        \setcounter{footnote}{8}
        \vspace*{\floatsep}
        \includegraphics[width=\textwidth]{fig_generate_heatmap_PHEV}
        \caption[Optimal cell layer configurations for the \glsfmtshort{phev} for a range of
        fast charging powers and thermal conditions]{Optimal cell layer configurations for the \gls{phev}}
        \label{fig:fig_generate_heatmap_PHEV}
        \mpfootnotes[1]
        \footnote{These figures were created by \mbox{Ian D.\ Campbell} who asserts copyright,
            with intellectual contributions from and the right to use asserted by
        \mbox{Krishnakumar Gopalakrishnan}.}
    \end{minipage}
\end{figure}

Figures~\ref{fig:fig_generate_heatmap_BEV}
and~\ref{fig:fig_generate_heatmap_PHEV} shows the results  produced by the layer
optimisation framework  for the \gls{bev} and  \gls{phev} platforms respectively
when  considering fast  charging requirements.  Each heat~map  in these  figures
show  the optimal  number  of  layers~$n^\text{fastchg}_\text{opt}$ for  various
combinations of  initial and  ambient temperatures  for four  different charging
powers. In each case, the  values of~$n^\text{fastchg}_\text{opt}$ correspond to
the  temperature  combination~${(T_\text{init},T_\text{sink})  =  (15,  5)
\si{\degreeCelsius}}$  as  shown  in  \cref{fig:fig_generate_heatmap_BEV}.  This
represents the  least number of  layers required to  fast charge the  pack under
\gls{cp} conditions until  the target \gls{soc} is reached.  The charging scheme
additionally considers the constraint that the cell temperature must stay
within~${T_\text{max}=  \SI{55}{\degreeCelsius}}$. Furthermore,  its voltage  must
remain less than or equal  to~${V_\text{max} = \SI{4.22}{\volt}}$. Finally,
the charging algorithm is plating-aware \ie~the charging stops as soon as the
concentration at the particle surface reaches the maximum possible concentration
limit, thereby preventing  lithium plating at the surface  of negative electrode
particles.


Thus,   using    the   model-based    design   strategy   presented    in   this
chapter,   an   effective    cell   design   is   achieved    which   helps   to
maximise   energy   density  and   \gls{bev}   range,   without  forgoing   fast
charging   power    targets.   From   figures~\ref{fig:fig_generate_heatmap_BEV}
and~\ref{fig:fig_generate_heatmap_PHEV},       it       is       seen       that
$n^\text{fastchg}_\text{opt}$  increases with  increase in  the charging  power.
This is because,  as the charging power increases, the  minimum number of layers
required to maintain  the cell voltage below the maximum  permissible value also
increases.  This requires  higher interfacial  surface area  to accommodate  the
increased  power demand.  Furthermore, rapid  surface saturation  occurs due  to
steep  concentration  profiles in  the  negative  electrode particles  when  the
charging power is  high which causes plating. With higher  layers, the resulting
electrodes  are thinner,  thereby allowing  faster diffusion  of lithium  in the
solid  particles  and  avoiding  steep concentration  gradients  in  them.  This
suggests that the number of layers must be large enough to prevent plating.

\Cref{fig:fig_CapacityQuadrants} shows the nominal  capacity of cells and charge
passed versus  the number of  layers during fast  charging. In these  plots, the
theoretical  capacity~$Q_\text{n}$ of  the cell  versus the  layer count~$n$  is
represented by the linear downward-sloping line.

\begin{figure}[!bp]
    \begin{minipage}[t]{\textwidth}
        \centering
        \includegraphics[width=0.998046875\textwidth,trim=4 2 3 4,clip]{fig_capacity_quadrants.pdf}
        \captionsetup{labelsep=note}
        \caption[
        Nominal capacity and charge passed versus layer count for constant current  and constant power  charging
        ]
        {
            The right column shows nominal cell capacity and charge passed
            during \gls{cp} charging. Rate capability and cell utilisation are positively
            correlated with~$n$. With increasing power levels, the optimal layer configuration shifts to higher
            values of~$n$. Similar behaviour is observed for galvanostatic
            charging (left column). Plotted for~${T_\text{init} =
            T_\text{sink} = \SI{25}{\degreeCelsius}}$.
        }
        \label{fig:fig_CapacityQuadrants}
        \mpfootnotes[1]
        \footnote{This figure was created by \mbox{Ian D.\ Campbell} who asserts copyright,
            with intellectual contributions from and the right to use asserted by
        \mbox{Krishnakumar Gopalakrishnan}.}
    \end{minipage}
\end{figure}

For any layer  choice, $Q_\text{n}$~therefore represents the upper  bound on the
charge  that can  be  passed  during charging.  For  both  constant current  and
constant power  charging, the locii of  actual charge passed~$q$ lie  much below
this theoretical nominal  capacity. For very low layer counts,  as the number of
layers decreases, the power density drops rapidly which implies that the rate of
heating is low. This allows for more  charge to be passed. However, at ultra-low
layer counts, the  overpotential due to the cell's internal  resistance is quite
high.  Therefore,  hitting the  upper  bound  on  the  terminal voltage  is  the
reason  for  the failure  of  these  layer choices.  This  is  indicated by  the
narrow \mbox{$V_\text{max}$-limited} region in \cref{fig:fig_CapacityQuadrants}. For an
intermediate range of  layer choices, the rate of power-density  drop with layer
count begins to flatten, thereby leading  to a plating-limited region. For these
layer choices, the  surface concentration starts to exceed  the saturation value
before any  thermal or voltage  limits are reached. Finally,  further increasing
the layer count beyond  an intermediate optimal value leads to  a linear drop in
the cell's  charge accepting  capability. During fast  charging with  the chosen
power levels, although these layer choices  do not reach the thermal, voltage or
concentration  limits,  they  are  unable  to attain  the  target  \gls{soc}  of
\SI{80}{\percent}. This is simply due to the lower nominal capacity of the these
cells.  There is  no  benefit  whatsoever in  designing  cells  in this  region.
\Cref{fig:fig_CapacityQuadrants} provides  clues to  the design engineer  on the
degree  of optimisation  that can  be achieved  by careful  design choices.  For
instance, by tuning certain design parameters, such as using electrode materials
capable  of  operating at  higher  plating  voltage  or with  higher  saturation
concentration, the optimisation  point can be appropriately adjusted  as per the
application's demands.


From  the results  discussed thus  far, it  is evident  that it  is the  thermal
environment  that  governs   the  optimal  cell  layer   configuration  in  both
acceleration and fast charging studies and for both vehicular platforms. For all
charging powers simulated, $n^\text{fastchg}_\text{opt}$~is the highest for the
coldest temperature  combination~${(T_\text{init},T_\text{sink}) =  (15, 5)
\si{\degreeCelsius}}$. This is due to  the slow rate of electrochemical reaction
and diffusion  at cold  temperatures. The thinner  electrodes from  using higher
layer count enable fast charging without saturating the surface of the electrode
particles. For the  fast charging scenarios considered here,  the optimal number
of layers to use  is 89~for the \gls{bev} cell and  153~for the \gls{phev} cell.
The globally optimal  layer choice to be  used for cell design  is therefore the
higher of the two values corresponding  to acceleration and fast charging cases.

Therefore, this model-based design framework recommends the use of 89~layers for
the cells to be  used in the \gls{bev} platform and 153~layers  for the cells to
be used in the \gls{phev} platform.  This concludes the discussion of results of
this chapter as well as all design-related aspects of this thesis.

