% -*- root: ../../main.tex -*
%!TEX root = ../../main.tex
% vim:textwidth=80 fo=cqt conceallevel=0


At the outset, it is worth mentioning that the focus of this chapter
is on the layer optimisation \emph{methodology} itself. The results as
such do not stand alone outside of the modelling universe with all the
inherent assumptions discussed thus far. Presently, the value added by
this work is its ready adaptability to industry through its modular
design. A numerical implementation in the form of a toolbox\footnote{The
\gls{bold} Toolbox is made available for download from GitHub. \\
\mbox{\href{https://github.com/ImperialCollegeESE/BOLD_Toolbox}{\includegraphics
[width=0.025\textwidth]{github.pdf}}} \url{
https://github.com/ImperialCollegeESE/BOLD_Toolbox}} is also provided which
is immediately available for download and use by industry. This author
recommends that until the tool matures, cell manufacturers substitute their own
parameters and adjust other numerical coefficients suitably so that the toolbox
supplements, rather than supplants their empirical layer designs. Hence, the
results presented in this section must be interpreted in the backdrop of the
context within which the methodology was developed implying that the reader
must consciously strive to interpret all numerical values in relative terms of
magnitude. To aid this thought process, this author chooses to deliberately
limit the discussion on the absolute magnitude of numbers presented here.

\subsection{Modelling Platform and Preconditioning}

The thermally-coupled, \gls{p2d} electrochemical cell model used for
simulating each layer choice is implemented in MATLAB~\cite{matlab}, using
a heavily-modified version of the LIONSIMBA toolbox~\cite{Torchio2016}.
The work reported in this chapter helped to advance the toolbox from~v1.0x
to~v2.0. The updated computer codes to which this author heavily contributed,
is available from the project's official repository\footnote{LIONSIMBA~v2:
\mbox{\href{https://github.com/lionsimbatoolbox/LIONSIMBA}{\includegraphics
[width=0.025\textwidth]{github.pdf}}} \url{
https://github.com/lionsimbatoolbox/LIONSIMBA}}.

The  rationale  behind  choosing  this  specific  software  to  implement  layer
optimisation  is  as  follows.  The LIONSIMBA~v1.0x  toolbox  has  already  been
validated against  the results of DUALFOIL~\cite{Dualfoil1998}  codes (which can
be considered as the present benchmark  standard). The toolbox is implemented in
the  MATLAB programming  language.  Since  this chapter  has  a strong  industry
focus,  the  omnipresence  of  MATLAB  in industry,  its  mature  code-base  and
familiarity  was  a strong  motivator  in  the  adoption  of this  toolbox.  The
simulation  speeds using  LIONSIMBA  have been  shown to  be  comparable to  the
FORTRAN  implementation  of  DUALFOIL,  primarily  owing  to  the  sophisticated
computation  of  the  analytical  Jacobian   of  the  system  through  automatic
differentiation~\cite{Torchio2016}.  In  addition  to  fundamental  enhancements
to   the  modelling   platform  presented   in~\cref{sec:numericalenhancements},
numerous  bug fixes  and  other  minor enhancements  to  the original  LIONSIMBA
code-base   have  been   provided  by   this  thesis   author.  The   interested
reader  may  peruse  these  from  the  \texttt{README.md}  file  hosted  on  the
\href{https://github.com/lionsimbatoolbox/LIONSIMBA}{project's  repository}. The
complete  parameter set  used  for  simulation of  this  model  is presented  in
table~\ref{tbl:lcoSimParamslayeropt}. All cells are assumed to be in their state
of equilibrium prior to beginning of simulations.

% -*- root: ../../main.tex -*-
%!TEX root = ../../main.tex
% vim:nospell

\begin{table}[!htbp]
    \small
    \caption[%
    System-level simulation conditions \& thermal parameters of  an \glsfmtshort{lco} cell
    ]%
    {%
        Cell   parameters   and   system   conditions  for   a   simulating   an
        \glsfmtshort{lco} cell  with the  \gls{dfn} electrochemical model  and a
        lumped thermal model. The parameters  presented here when augmented with
        the  values  of  the  kinetic, geometric  and  transport  properties  of
        the  cell (from  \cref{tbl:lcoSimParamsSPMp2d}  represents the  complete
        information  required for  all  simulations in  this layer  optimisation
        framework.
    }%
    \label{tbl:lcoSimParamslayeropt}
    \vspace{-2.6229525pt}
    \begin{threeparttable}
        \centering
        \textbf{System Conditions} \\ \smallskip
        \begin{varwidth}[t]{0.48\linewidth}
            \begin{tabular*}{\textwidth}{@{} l @{\extracolsep{\fill}} S[table-format=1.2,table-space-text-pre=\Tnote{a} ,table-align-text-pre=false] @{}}
                \toprule
                \multicolumn{1}{@{}l}{Parameter} \\
                \midrule

                Lower cutoff cell voltage, $V_\text{min}$ (\si{\volt}) & \Tnote{a} 3.50   \\
                Upper cutoff cell voltage, $V_\text{max}$ (\si{\volt}) & \Tnote{b} 4.22   \\

                \bottomrule
            \end{tabular*}
        \end{varwidth}
        \hfill
        \begin{varwidth}[t]{0.48\linewidth}
            \begin{tabular*}{\textwidth}{@{} l @{\extracolsep{\fill}} S[table-format=2.2,table-space-text-pre=\Tnote{a} ,table-align-text-pre=false] @{}}
                \toprule
                \multicolumn{1}{@{}l}{Parameter} \\
                \midrule

                Target cell SOC for fast charge, $z^\ast$ \si{(\%)}                & \Tnote{c} 80.00 \\
                Cell upper temperature limit, $T_\text{max}$ \si{(\degreeCelsius)} & \Tnote{d} 55.00 \\

                \bottomrule
            \end{tabular*}
        \end{varwidth}

        \medskip
        \begin{tabular*}{\textwidth}{@{} l @{\extracolsep{\fill}} r @{}}
            \multicolumn{2}{c}{\textbf{Geometric Parameters}} \\
            \toprule
            \multicolumn{1}{@{}l}{Parameter} \\
            \midrule
            Surface area of pos.\ \& neg.\ electrode overlap within a layer, {$A_\text{elec}$} \si{(m^2)} & \textsuperscript{b}\num{4.19e-2}   \\
            Exterior pouch length, $L_\text{pouch}$ \si{(m)}                                              & \textsuperscript{e}\num{332.74e-3} \\
            Exterior pouch width, $W_\text{pouch}$ \si{(m)}                                               & \textsuperscript{e}\num{99.06e-3}  \\
            Exterior pouch height, $H_\text{pouch}$ \si{(m)}                                              & \textsuperscript{f}\num{10.00e-3}  \\
            Pouch material thickness, $T_\text{pouch}$ \si{(m)}                                           & \textsuperscript{g}\num{160.00e-6} \\
            Stack thickness, $L_\text{stack}$ \si{(m)}                                                    & \textsuperscript{r}\num{9.68e-3}  \\
            \bottomrule
        \end{tabular*}
        \medskip
        \centering \textbf{Thermal Parameters} \\ \smallskip
        \resizebox{\textwidth}{!}{%
            \begin{tabular}{@{} l S[table-format=4.0,table-space-text-pre=\Tnote{m} ,table-align-text-pre=false] S[table-format=4.1,table-space-text-pre=\Tnote{m} ,table-align-text-pre=false] S[table-format=4.1,table-space-text-pre=\Tnote{m} ,table-align-text-pre=false] S[table-format=4.2,table-space-text-pre=\Tnote{m} ,table-align-text-pre=false] S[table-format=4.0,table-space-text-pre=\Tnote{m} ,table-align-text-pre=false] S[table-format=4.1,table-space-text-pre=\Tnote{m} ,table-align-text-pre=false] S[table-format=4.1,table-space-text-pre=\Tnote{m} ,table-align-text-pre=false] @{}}
                \toprule
                \multicolumn{1}{@{}l}{Parameter} & \multicolumn{1}{c}{Al.\ CC} & \multicolumn{1}{c}{Pos} & \multicolumn{1}{c}{Sep} & \multicolumn{1}{c}{Neg} & \multicolumn{1}{c}{Cu.\ CC} & \multicolumn{1}{c}{\ch{LiPF_6}} & \multicolumn{1}{r@{}}{Pouch}\\
                \midrule

                Sp.\ heat capacity, $c_j$ (\si{\joule\per\kilogram\per\kelvin})   & \Tnote{h} 903  & \Tnote{h} 1269.2 & \Tnote{h} 1978.2 & \Tnote{h} 1437.4 & \Tnote{h} 385  & \Tnote{h} 2055.1 & \Tnote{i} 1464.8 \\
                Density, $\rho_j$ (\si{\kilogram\per\meter\cubed})                & \Tnote{j} 2700 & \Tnote{k} 2291.6 & \Tnote{b} 1100.0 & \Tnote{j} 2660.0 & \Tnote{l} 8960 & \Tnote{j} 1290.0 & \Tnote{m} 1150.0 \\
                Activ.\ energy, diff. ${E_\text{act,s}}_j$ (\si{\joule\per\mole}) & {---}                   & \Tnote{p} 5000   & {---}                     & \Tnote{p} 5000   & {---}                   & {---}                     & \multicolumn{1}{c}{---}   \\
                Activ.\ energy, rxn. ${E_\text{act,k}}_j$ (\si{\joule\per\mole})  & {---}                   & \Tnote{p} 5000   & {---}                     & \Tnote{p} 5000   & {---}                   & {---}                     & \multicolumn{1}{c}{---}   \\

                \bottomrule
            \end{tabular}
        }
        \medskip
        \begin{tabular*}{\textwidth}{@{} l @{\extracolsep{\fill}} r @{}}
            \multicolumn{2}{c}{\textbf{Other Geometric/Cell-Level Parameters}} \\
            \toprule
            \multicolumn{1}{@{}l}{Parameter} \\
            \midrule

            Thickness of pos.\ current collector, $l_\text{Al}$ \si{(m)}                    & \textsuperscript{f}\num{15e-6}   \\
            Thickness of neg.\ current collector, $l_\text{Cu}$ \si{(m)}                    & \textsuperscript{p}\num{10e-6}   \\
            Total tab area, $A_\text{tabs}$ \si{(m^2)}                                      & \textsuperscript{b}\num{5.94e-3} \\
            Lumped heat transfer coefficient, $h$ (\si{\watt\per\meter\squared\per\kelvin}) & \textsuperscript{b}150           \\
            Initial electrolyte concentration, $c_\text{e,0}$ (\si{\mole\per\meter\cubed})  & \textsuperscript{q}1000          \\

            \bottomrule
        \end{tabular*}

        \medskip
        \begin{tabular*}{\textwidth}{@{} =P{7.5cm}  +l@{\extracolsep{\fill}}+c +r @{}}
            \multicolumn{4}{c}{\textbf{Spatial Discretisation}} \\
            \toprule
            \multicolumn{1}{@{}l}{Parameter} & \multicolumn{1}{l}{Pos} & \multicolumn{1}{c}{Sep} & \multicolumn{1}{r@{}}{Neg}\\
            \midrule

            Nodes, through-thickness (axial), $N_{\text{a}_j}$          & \num{40} & \num{40} & \num{40} \\
            Nodes, within spherical particle (radial), $N_{\text{r}_j}$ & \num{15} & ---      & \num{15} \\

            \bottomrule
        \end{tabular*}

        \smallskip
        % \vspace{-2.6229525pt}
        \vspace*{-5pt}
        \begin{tablenotes}[para,flushleft]
            \begin{footnotesize}
            \item[a] Calculated in section \hyperlink{celllowercutoff}{`Lower cutoff voltage for cells'} (also see \cref{sec:surfareaperlayer})
            \item[b] Assumed
            \item[c] Ref.~\cite{Sae2010}
            \item[d] Ref.~\cite{Kizilel2009}
            \item[e] Converted from imperial units reported in~Ref.~\cite{GMBoltBatteryDims}
		    \item[f] Table~\romanletter{4} of~Ref.~\cite{Groger2015} \\
            \item[g] Sum of values in table~1 of~Ref.~\cite{Svens2013}
            \item[h] Ref.~\cite{Chen2005} \\
            \item[i] Computed from values of constituents (see~\cite{Svens2013}) using Ref.~\cite{martienssen2006springer} \\
            \item[j] Ref.~\cite{Guo2010}
            \item[k] Ref.~\cite{Jeon2011}
            \item[l] Ref.~\cite{Worwood2017,Song2000}
            \item[m] Ref.~\cite{Kim2009}
            \item[p] Ref.~\cite{Northrop2011}
            \item[q] Ref.~\cite{Subramanian2009} \\
            \item[r] See section \hyperlink{stackthickness}{`Compute stack thickness of reference cell'}
            \end{footnotesize}
        \end{tablenotes}
    \end{threeparttable}
\end{table}



\subsection{\glsfmtshort{xeV} configurations}


% -*- root: ../../main.tex -*-
%!TEX root = ../../main.tex
% vim:nospell


\begin{table}[!htbp]
	\renewcommand{\thetable}{\arabic{table}a}
	\centering
	\caption{Acceleration test parameters (common across xEV platforms)}
	\label{tbl:CommonVehicleParams}
	\sisetup{table-format=3.2, table-number-alignment=center, table-space-text-pre=\textsuperscript{a}, table-space-text-post=\textsuperscript{a}, table-align-text-post=false}
	\begin{threeparttable}[t]
		\centering
		\begin{tabular}{@{} l  S @{}}
			\toprule
			Parameter \\
			\midrule

			% Coefficient of drag for xEV body, $C_\mathrm{d}$                           & {\makebox*{00}[r]{\tnote{a}}} 0.31                \\
			% Frontal area of xEV, $A_\mathrm{v}$ \si{(m^2)}                             & {\makebox*{00}[r]{\tnote{b}}} 2.40                \\
			% Acc.\ time specified by manufacturer, $t_\mathrm{f,man}$ \si{(s)}          & {\makebox*{00}[r]{\tnote{d}}} 6.50                \\
			% Acc.\ time dictated by standards, $t_\mathrm{f,std}$ \si{(s)}              & {\makebox*{00}[r]{\tnote{c}}} 6.00                \\
			% Speed, end of acc. (standards), $v_\mathrm{f,std}$ \si{(m.s^{-1})}         & {\makebox*{00}[r]{\tnote{e}}} 8.94                \\
			% Speed, end of acc. (manufacturer), $v_\mathrm{f,man}$ \si{(m.s^{-1})}      & {\makebox*{0}[r]{\tnote{f}}} 26.82                \\
			% Base speed of  xEV, $v_\mathrm{b}$ \si{(m.s^{-1})}                         & {\makebox*{\hspace*{0.5mm}0}[r]{\tnote{e}}} 13.41 \\
			% Air density at acc.\ test conditions, $\rho_\mathrm{air}$ \si{(kg.m^{-3})} & {\makebox*{\hspace*{0.5mm}00}[r]{\tnote{f}}} 1.20 \\
			% Drivetrain efficiency, $\eta_\mathrm{dt}$                                  & {\makebox*{00}[r]{\tnote{g}}} 0.75                \\
			% Payload, $M_\mathrm{p}$ \si{(kg)}                                          & {\hspace*{0.00005mm}{\tnote{c}}} 150.60 \\
			% Rolling resistance coefficient of road surface, $C_\mathrm{r}$             & {\makebox*{00}[r]{\tnote{f}}} 0.01                \\
			% Road gradient, $Z$                                                         & {\makebox*{00}[r]{\tnote{g}}} 0.00                \\

			Coefficient of drag for xEV body, $C_\mathrm{d}$                           & 0.31   {\tnote{a}} \\
			Frontal area of xEV, $A_\mathrm{v}$ \si{(m^2)}                             & 2.40   {\tnote{b}} \\
			Acc.\ time specified by manufacturer, $t_\mathrm{f,man}$ \si{(s)}          & 6.50   {\tnote{d}} \\
			Acc.\ time dictated by standards, $t_\mathrm{f,std}$ \si{(s)}              & 6.00   {\tnote{c}} \\
			Speed, end of acc. (standards), $v_\mathrm{f,std}$ \si{(m.s^{-1})}         & 8.94   {\tnote{e}} \\
			Speed, end of acc. (manufacturer), $v_\mathrm{f,man}$ \si{(m.s^{-1})}      & 26.82  {\tnote{f}} \\
			Base speed of  xEV, $v_\mathrm{b}$ \si{(m.s^{-1})}                         & 13.41  {\tnote{e}} \\
			Air density at acc.\ test conditions, $\rho_\mathrm{air}$ \si{(kg.m^{-3})} & 1.20   {\tnote{f}} \\
			Drivetrain efficiency, $\eta_\mathrm{dt}$                                  & 0.75   {\tnote{g}} \\
			Payload, $M_\mathrm{p}$ \si{(kg)}                                          & 150.60 {\tnote{c}} \\
			Rolling resistance coefficient of road surface, $C_\mathrm{r}$             & 0.01   {\tnote{f}} \\
			Road gradient, $Z$                                                         & 0.00   {\tnote{g}} \\

			\bottomrule
		\end{tabular}
        \begin{tablenotes}[para,flushleft]
        \item[a]Ref.~\cite{HybridCars2017Drag}
        \item[b]Calculated from typical \gls{bev} dimensions in~\cite{BoltDimensions}
        \item[c]Ref.~\cite{ETANTP002-2004}
        \item[d]Ref.~\cite{BoltOverview}
        \item[e]Ref.~\cite{Liu2016a}
        \item[f]Ref.~\cite{EmadiElectric}
        \item[g]Assumed
        \end{tablenotes}
	\end{threeparttable}
\end{table}


Tables~\ref{tbl:CommonVehicleParams} and~\ref{tbl:UniqueVehicleParams}  show the
\gls{xeV}  parameters  used  in  simulations. In  this  author's  analysis,  the
power  demands on  the battery  pack  during normal  operation are  found to  be
significantly  lower than  that  experienced  during the  two  extreme cases  of
discharging  and charging  \viz{} \emph{acceleration}  and \emph{fast  charging}
respectively. For  instance, \SI{50.83}{\kilo\watt} is the  peak discharge power
while \SI{14.20}{\kilo\watt} is the median  discharge power for various standard
drive  cycles.  Even with  the  assumption  that \SI{100}{\percent}  of  braking
energy  can  be  recovered,  the  peak  and  median  charging  powers  are  only
\SI{43.13}{\kilo\watt} and \SI{26.03}{\kilo\watt}  respectively. Considering the
acceleration  parameters  in~\cref{tbl:CommonVehicleParams}, for  the  \gls{bev}
pack, \SI{181.45}{\kilo\watt}  is the  power requirement  for acceleration  of a
fixed vehicle  mass on a  flat road  surface. Four distinct  fast-charging power
levels \viz{} \SI{50}{\kilo\watt}, \SI{80}{\kilo\watt}, \SI{110}{\kilo\watt} and
\SI{130}{\kilo\watt}  are considered  in this  study.  This is  in adherence  to
the  minimum  and maximum  values  of  level~3  rating  as suggested  by  Yilmaz
and~Krein~\cite{Yilmaz2012}. Furthermore, near-term fast charging goals laid out
in  literature~\cite{Ashique2017,Srdic2016} and  the  peak  power capability  of
charging infrastructure further justify these choices.

% -*- root: ../../main.tex -*-
%!TEX root = ../../main.tex
% vim:nospell


\begin{table}[!htbp] % Parameters unique to each of the BEV & PHEV
	% \addtocounter{table}{-1}
	% \renewcommand{\thetable}{\arabic{table}b}
	\caption{Acceleration test parameters (specific to each \glsfmtshort{xeV})}
	\label{tbl:UniqueVehicleParams}
	\centering
    \sisetup{table-format=4.1, table-number-alignment=center, table-space-text-pre=\textsuperscript{a}, table-align-text-pre=false}
	\begin{threeparttable}[t]
		\begin{tabular*}{0.675\textwidth}{@{} l @{\extracolsep{\fill}}  S S @{}}	% Works with Tnote
			\toprule
			\multicolumn{1}{@{} l}{Parameter} & \multicolumn{1}{c}{BEV} & \multicolumn{1}{c@{}}{PHEV} \\
			\midrule

			Mass of xEV chassis, $M_\mathrm{c}$ \si{(kg)}               & \Tnote{a} 1340.0 & \Tnote{b} 1438.0 \\
			Mass of pack overhead (w/o cells), $M_\mathrm{o}$ \si{(kg)} & \Tnote{a} 196.4  & \Tnote{c} 65.5   \\
			Upper cutoff SOC of cell, $z_\mathrm{max}$ \si{(\%)}        & \Tnote{d} 95.0   & \Tnote{d} 90.0   \\
			Lower cutoff SOC of cell, $z_\mathrm{min}$ \si{(\%)}        & \Tnote{d} 5.0    & \Tnote{e} 30.0   \\

			\bottomrule
		\end{tabular*}
		\begin{tablenotes}[para,flushleft]
		\item[a]Calculated based on~\cite{ChevyBoltSpecs}
		\item[b]Calculated based on~\cite{motortrendEcotec,ChevyBoltSpecs}
		\item[c]Calculated see \cref{sec:Configurations}
		\item[d]Assumed
		\item[e]Ref.~\cite{EmadiElectric}
		\end{tablenotes}

	\end{threeparttable}
\end{table}


For  the  acceleration  tests,  the  initial cell  \gls{soc}  has  been  set  to
\SI{40}{\percent}.  This  is  in  conformity with  the  test  criterion  $(50\pm
10)$~\%)  of  the  SAE  J1666 standard~\cite{Sae2010}.  By  choosing  the  worst
case  starting \gls{soc}  \ie{} \SI{40}{\percent},  a conservative  design shall
be  achieved.  The  chassis  mass  of  the  vehicle  as  well  as  the  mass  of
two  passengers   at  75.3  kg   each~\cite{Sae2010}  is  considered   for  both
\gls{xeV}  platforms. The  pack mass  is  computed as  a function  of number  of
layers  as described  in~\cref{sec:layeroptframework}. The  vehicle manufacturer
General  Motors  Inc.\ suggests  the  mass  value of  the  GM  Ecotec series  of
engines~\cite{motortrendEcotec}  to use  for  the \gls{phev}  which consists  of
a  range-extending  \gls{ice}.  The  mass  of  the  Bolt  \gls{bev}  pack  given
in~\cite{ChevyBoltSpecs} minus  the computed mass  of the overall cells  used in
the pack  gives the overhead mass  of the \gls{bev} pack.  The \gls{phev} pack's
overhead mass  is determined by suitably  scaling the mass by  the proportion of
reduction in the number of cells used.

For the \gls{bev}  platform, a fast-charging scheme operated on  a \gls{cp} mode
with an initial  \gls{soc} of \SI{20}{\percent} is employed. In  the case of the
\gls{phev}, an initial \gls{soc}  of \SI{30}{\percent} (\SI{10}{\percent} higher
than that for \gls{bev}) is used. This facilitates a smaller \gls{soc} window to
take into account the higher number of charge-discharge cycles which are typical
with \gls{phev} designs~\cite{Maksimovic2012}. Both \gls{xeV} platforms are fast
charged  to a  target  \gls{soc}  of \SI{90}{\percent}  in  \gls{cp} mode.  This
\gls{soc}  value corresponds  to the  end-of-charge target  in level~3  charging
standards~\cite{SAECharging2011}.

\subsection{Acceleration studies --- \gls{bev}}

\gls{bev}       acceleration      at       a       worst-case      rate       of
\SI{4.13}{\meter\per\second\squared} is assumed for simulation. This corresponds
to the  manufacturer's acceleration specifications  for the \gls{bev}  listed in
\cref{tbl:CommonVehicleParams}.  The  acceleration  rate  corresponding  to  the
SAE~J1772 standards is lower than this  rate. Therefore, to obtain a robust cell
design, the higher of the two acceleration rates needs to be considered.

\Cref{tbl:accResults}  gives the  simulation  results  for various  combinations
of  $(T_\text{init}, T_\text{sink})$.  These  combinations  of temperatures  for
traversing  the  thermal  design  space   are  chosen  following  the  SAE~J1772
guidelines.  The power  densities  with  low numbers  of  layers  lead to  large
overpotentials resulting in the cell's terminal  voltage to drop lower down than
$v_\text{min}$, thereby unable to satisfy acceleration requirements. However, at
higher  $T_\text{init}$, owing  to  the reduction  in  overpotentials, a  larger
voltage overhead is  available to accommodate polarisation.  For all temperature
combinations,  the  largest  deviation  from $T_\text{init}$  experienced  by  a
\gls{bev} cell is a \SI{0.48}{\degreeCelsius}  increase. Consequently, it can be
concluded that a single isolated acceleration  event does not heat the \gls{bev}
battery pack, and therefore it can  be assumed that the cell temperature remains
close  to  that  of the  initial  value.  Similarly,  even  for the  worst  case
simulation run,  the cell's  \gls{soc} is  depleted only by  a maximum  value of
\SI{0.32}{\percent}.

% -*- root: ../../main.tex -*-
%!TEX root = ../../main.tex
% vim:nospell

\begin{table}[htb!]
    \caption{\glsfmtshort{xeV} acceleration test results}
    \label{tbl:accResults}
    \centering
	\begin{tabular}{c c c}
        \toprule
        \multicolumn{1}{@{} l}{\makecell{($T_\text{init},T_\text{sink}$) \\ \footnotesize (degC)}} & \makecell{$n^\text{acc}_\text{opt}$ \\ \footnotesize \glsfmtshort{bev}}&  \multicolumn{1}{c @{}}{\makecell{$n^\text{acc}_\text{opt}$ \\ \footnotesize \gls{phev}}}  \\
        \midrule

        (38,5)  & \num{21} & \num{55} \\
        (38,49) & \num{21} & \num{57} \\
        (25,25) & \num{23} & \num{63} \\
        (15,5)  & \num{27} & \num{69} \\

        \bottomrule
    \end{tabular}
\end{table}


The foregoing  discussion has revealed  that the lower cut-off  voltage strongly
influences  layer  configuration for  acceleration  tests.  Therefore, for  this
specific design,  $n =  27$ represents  the optimal  layer when  considering the
acceleration requirements for the \gls{bev} platform.


%%\begin{figure}[!bp]
%%    \begin{minipage}[t]{\textwidth}
%%        \centering
%%        \includegraphics[width=\textwidth]{fig_generate_heatmap_BEV}
%%        \captionsetup{labelsep=note}
%%        \caption[Optimal cell layer configurations for the \gls{bev}, presented for a range of fast charging powers and thermal conditions]{Optimal cell layer configurations for the \gls{bev}, presented for a range of fast charging powers and thermal conditions.}
%%        \label{fig:fig_generate_heatmap_BEV}
%%        \mpfootnotes[1]
%%        \footnote{This figure was created by \mbox{Ian Campbell} who asserts copyright,
%%            with intellectual contributions from and the right to use asserted by
%%        \mbox{Krishnakumar Gopalakrishnan}.}
%%    \end{minipage}
%%\end{figure}

%%\begin{figure*}[!bp]
%%    \begin{minipage}[t]{\textwidth}
%%        \centering \includegraphics[width=\textwidth,trim=4 2 3 4,clip]{fig_capacity_quadrants.pdf}
%%        \captionsetup{labelsep=note}
%%        \caption[The plots in the right column show the nominal cell capacity and charge passed
%%        during \gls{xeV} \gls{cp} charging. Increased rate capability and cell utilisation are positively
%%        correlated with $n$, while the maximum-$q$ layer configuration clearly shifts to higher
%%        values of $n$ with increasing charging powers. The plots in the left column depict
%%        galvanostatic charging scenarios at various currents to highlight the similarity with the
%%        \gls{cp} process. All data obtained at $T_\text{init} =$ \SI{25}{\degreeCelsius},
%%        $T_\text{sink} =$ \SI{25}{\degreeCelsius}.]{The plots in the right column show the nominal cell capacity and charge passed
%%            during \gls{xeV} \gls{cp} charging. Increased rate capability and cell utilisation are positively
%%            correlated with $n$, while the maximum-$q$ layer configuration clearly shifts to higher
%%            values of $n$ with increasing charging powers. The plots in the left column depict
%%            galvanostatic charging scenarios at various currents to highlight the similarity with the
%%            \gls{cp} process. All data obtained at $T_\text{init} =$ \SI{25}{\degreeCelsius},
%%        $T_\text{sink} =$ \SI{25}{\degreeCelsius}.}
%%        \label{fig:fig_CapacityQuadrants}
%%        \mpfootnotes[1]
%%        \footnote{This figure was created by \mbox{Ian Campbell} who asserts copyright,
%%            with intellectual contributions from and the right to use asserted by
%%        \mbox{Krishnakumar Gopalakrishnan}.}
%%    \end{minipage}
%%\end{figure*}


%%\begin{figure}[!bp]
%%    \begin{minipage}[t]{\textwidth}
%%        \centering
%%        \includegraphics[width=\textwidth]{fig_generate_heatmap_PHEV}
%%        \captionsetup{labelsep=note}
%%        \caption[Optimal cell layer configurations for the \gls{phev}, presented for a range of
%%        fast charging powers and thermal conditions]{Optimal cell layer configurations for the \gls{phev}, presented for a range of
%%        fast charging powers and thermal conditions.}
%%        \label{fig:fig_generate_heatmap_PHEV}
%%        \mpfootnotes[1]
%%        \footnote{This figure was created by \mbox{Ian Campbell} who asserts copyright,
%%            with intellectual contributions from and the right to use asserted by
%%        \mbox{Krishnakumar Gopalakrishnan}.}
%%    \end{minipage}
%%\end{figure}

\subsection{Common Module Design}\label{sec:commonmodulelayeropt}

