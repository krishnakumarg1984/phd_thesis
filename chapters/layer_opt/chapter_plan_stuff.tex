% -*- root: ../../main.tex -*
%!TEX root = ../../main.tex
% vim:textwidth=80 fo=cqt conceallevel=0

\section{Layer Optimisation Framework}

The layer optimisation framework hinges  upon the concept of capacity balancing
between electrodes.  Explain this nicely  with references and citations  on why
this is important. Helps us make us of the most of the active material invested
into the cell.

\subsection{Capacity Balancing}

Formula for capacity balance. Show how  the length of the negative electrode is
slightly larger than the positive electrode, but compensated for by the reduced
volume fraction.  Cite other parameter  sets where  this is observed.  They are
nearly equal, but made to be 1.10.% The thickness of the positive electrode was
adjusted accordingly.

\subsection{Electrode Thicknesses per layer}
Talk briefly about separator and how its thickness remains constant.

With this, show the mathematical derivation of the expression for number of
layers Explain the significance of the outermost layer and how it affects the
formula used Explain how the stack is formed. Report its numerical value. Pouch
thickness and its role along with suitable reference. Show briefly a clever
computer code snippet that encapsulates both cases of odd and even.

At this stage, can explain clearly the relationship between the theoretical
capacity and number of layers. Use a graph or a table to highlight the
idea. Useable capacity shall be lower than total capacity due to cutoff
considerations. Explanation.

Especially, the derivation of the combined electrode thickness concept, and the
usage of the ratio to obtain one from the other.

\subsection{Derivation of analytical \protect{$n_\text{max}$}}

The  search space  spans  a finite  number of  layers.  An initial  alternative
considered  was MIDACO.  The detailed  equations and  how they  are derived  go
here. Discuss  for both  odd and  even cases.  However, closed  form analytical
expressions are  now available.  This is  so as  to enable  a binary  search as
discussed in the next section.

The optimisation formula and analytical solution shall also be discussed here.

%Note: Krishna  is claiming  the idea  generation and  coding in  entirety. The
mixed  %integer  optimisation to  zero  thickness.  narrowing down  the  search
window.  for loop  will also  work.  The optimal  layer  in %this  case is  the
earliest choice  of~$n$ in the  for loop for which  the %termination conditions
are correctly satisfied.

\subsection{Customised Binary Search}

A bi-section based algorithm. Algorithm/binary tree based description with
figure/algo. Discuss the O(log n) speedup.

%Note:  One fine  evening, Krishna  increased the  computational speedup  by two
orders of magnitude. Proof in github  and email. While Ian was always interested
in a for-loop approach right from  beginning, Krishna always suggested to use an
optimisation approach.

\subsection{Mass recomputation as a function of layers}
Explanation with graph.

\subsection{Specific-heat recomputation as a function of layers}
Suitable explanation and plot.

Note: Ian congratulated Krishna for mass and specifc heat computation. In his
words, ``You were clever coming up with a scheme wherein mass varies as a
number of layers''. Packaged this up as a function and all. Refer to
computelumpedmassandCpavgforgivenlayerfcn. Can prove git history for this file.
Not only mass recomputations but also mass initial computation was done by
Krishna

\subsection{Thermal space permutation of layers.}
Explanation of how the four corner temperatures are tried here. 5 lines of code snippet that packs a punch

Note: Krishna read the acceleration specification description,educated Ian and implemented this.

\begin{quotation}
Ian: Krishna!. Whoa that small block of code does so much.
\end{quotation}

\section{Workflow for Optimal Layer Computation}

This section discusses the actual methodology or the procedure in which the
aforementioned ideas are incorporated in a process-like workflow. Introduce the
full-page crazy flow diagram here. Explain how the flow diagram goes through a
methodical approach in arriving at the optimal number of layers. Extensive
forward and backward referencing to sections discussing each modular idea
encountered in the flow path. Discuss how the applied current and power
densities change due to change in overall surface area while the applied
external current/power remains the same.

% This will help us to apply current in units rather in current density.

The salient cases to be covered are the following.

\subsection{Case 1 --- Analysis of Drivecycle Powers}

Although from Colorado Boulder lecture notes, Krishna already knew that
acceleration demands the highest power demand on the cell, it needs to be proven
that drive cycles, which are the basis for various fuel efficiency calculations
do not represent this case. Ian did the comparative study of various drivecyles.
Some plots and charts showing various power levels were generated by him.
Krishna does not explicity need to use this section, and is optional to the
story, other than the sake of well-roundedness and demonstrating thoroughness of
the study.

Explain that this case is not present in the flow diagram.

However, it was Krishna who generated the drivecycle speed vs time and
acceleration versus time results alone and then informed Ian afterwards that it
is now available in the repository. If Ian uses this section, an acknowledgement
will be appropriate.


\subsection{Case 2 --- Acceleration from standstill}

Explanation and computations based on standard vehicle dynamics. Lecture notes
and videos given to Ian by Krishna from Univ of Colorado Boulder The sole value
addition in this case is the specification to adhere to and its implementation
through a cases study. Krishna shall explain this through either-or if-then
case, explanation along with a short code snippet which he implemented alone.

Explain the acceleration run in detail, what it entails etc. Walk through the
flow diagram till acc layer results are discussed.

Note: The acceleration specification for electrified transport was unearthed by
Krishna (all the interlibrary loan stuff and educated Ian about it). Two
passengers and all that thing. Complete coding.

\subsection{Case 3 --- Fast Charging}

A brief explanation of the whys and the hows and implications. Explanation of
prevalent standards. Present table of standards etc etc

One paragraph review of control algorithms and how this algorithm was chosen.
Brief overview of algorithm. The idea of introducing saturation and pulsed
charging profile. Based on patent at Auburn university. Krishna did literature
review. Flag introduced in code. Code snippet.enablecsnegsaturationlimit.
% Sinusoidal excitation charging.

Explanation of why power demand is important (based on charger power electronics).
Finally, walk through of the fast charging section of the schematic.

Note: Krishna performed the litt search of this section entirely (proof with
timestamp available). However, Krishna is tentatively not using a detailed litt
review. In kind consideration of Ian's own overarching thesis topic, which
Krishna understands to be something pertinent to fast charging, Krishna can let
Ian use the literature provided proper attribution is in place. For the sake of
completion, the MIDACO based approaching to fast charging showing the pulsing
power is also possible by Ian (just a suggestion).

Note: The charger power electronics limit is again my contribution from an EEE
background

% \item Review of model-based fast charging control algorithms (how does this go into litt review)

\section{Simulation Environment}
Discuss parameters of simulation environment. Discuss with the help of a table
for the BEV case only. Ian can take the PHEV Number of BEV calls designed to
make up the series string.

Krishna also came up with the cell's cutoff study with respect to the the
system's bus bar voltage, and will be cited here with examples and strong
backing. There are a few preconditioning steps needed to amend the p2d model
before numerical implementation of the flow diagram is possible. 1. Fixing
aspects of the code in LIONSIMBA v1.023 used as the baseline by this thesis
author. 2. Addition of the capability to apply power input instead of current
input as is common in traditional p2d simulations.

Note: All crucial parameters were sourced by Krishna through a literature
review. Especially, the vehicular parameters like classis mass, coefficient of
drag, base speed. Krishna also had to educate Ian about base speed. If an on the
spot quiz is conducted at a conceptual level, this truth can easily be deduced
by the judging authority. Krishna claims the literature review of this topic and
is happy to let Ian use this with proper attribution.

\subsection{Preconditioning of Computer Code}

Briefly allude to the stoichiometries that was introduced into the computer code
through capacity characterisation simulation.Now possible to start at any SoC.
The other salient amendments and enhancements to the software toolbox is shown
below. Released as 2.0 and link to LIONSIMBA github repo (not BOLD toolbox repo)

\subsubsection{Re-parameterisation}

Discuss the extensive Reparameterisation undertaken by Krishna by studying
relevant literature. Discuss the changed parameters with respect to Northrop
cell and why. Maybe show a table. This is quite substantial.

Conductivity/diffusivity changes. Show how the isothermal and thermal variants
in existing Northrop cell were bogus with plots.

Performed comprehensive literature review to replace the dubious/bogus
parameters of the electrode and electrolyte specific heat capacities email proof
PS: Every thermal/material property of the Al/Cu current collectors is very
clear and have been traced out, hand-calculated and validated.

\subsubsection{Deterministic initialisation of algebraic variables}

Replaced silly fsolve. Numerical explanation here.

\subsubsection{Linear interpolation for field variables}

Explain how linear interpolation was performed for phis and phie at the edges
of control volumes in electrode and separator. Draw a schematic for this
explanation.

\subsubsection*{Convergence Analysis of Computation Mesh}
Report only if Krishna has sufficient time left. Explain simplifying
assumptions. Refer to the table etc Show plot of how terminal voltage is
converging for chosen mesh as a function of number of nodes. Another plot of how
simulation end time converges as a function of number of nodes. Lots of analysis
by changing the number of nodes within each mesh to justify the validity of the
results. Prove that mesh independence is reached.

\subsection{Hybrid fv--Spectral Scheme}

Completely numerical section. Highly mathemetical. Explain how the rational to
decimal truncation of the unbalanced twelfth order finite difference scheme
messes up numerical conditioning. Show matrices and discuss the drawbacks.
Fornberg matrix did not help. So, spectral scheme was needed.

Background study, explanation, analysis, literature review and equation
derivation of spectral scheme Complete contribution of solid-phase diffusion
with spectral methods. Reading textbook, understanding concept, investigation of
applicability, hunting relevant literature, text-writing, hand-derivation of
equations. This complete section was done by Krishna.

\subsection{Lumped Thermal Model}

Basic discussion of lumped thermal model and justify through citations why here
it might be sufficient for this application. Not originally present in
LIONSIMBA. Present the thermal model here equations here. Discuss what cp avg
means and how it is computed as a here function of number of layers. Explanation
of how tab cooling helps here.

Note: Cp avg was calculated and coded by Krishna, as a function of number of
layers. I shall leave the detailed thermal model to Ian, especially since a biot
analysis was performed by him. Anyway, suddenly discussing the thermal model at
depth does not fit the story of my thesis. My hint to Ian would be to emphasise
the thermal model at depth, since anyway he seems to be confident in the biot
analysis. Specifically the value of heat transfer coefficient was empirically
chosen by Ian through simulations. So, I will let him explain that stuff.
However, tab area idea computation using twice the Bolt's tab area was proposed
by Krishna, but overall this thermal stuff, Krishna is willing to bequeath to
Ian since the whole thing doesn't fit the story of Krishna's work. The
polarisation heat concept was initially described by Greg, but anyway let Ian
explain it, no problem. Ian may wish to discuss entropic heat generation and a
lot of other things I investigated together, but I am not going to dwell on
them in the thesis.


\subsection{Power Input Boundary Conditions}

Discuss why power input is needed for layer optimisation. Explain how it is done
currently in state of the art case.

Krishna educated Ian about Pletts existing work. Accompanied Ian to the library
and told him to check out that book which I had ordered.

Krishna claims a very important thing here. A critical argument on how the
present schemes do not fit into the layer opt methodology.

Detailed mathematical derivation. Both Ian and Krishna acknowledge each other's
help in shared derivation and also acknowledge Davide appropriately.

Note: Ian might wish to report the two intermediate steps before this solution
was obtained. The 2nd of the 3 approaches was matching current and voltage at
discrete intervals through numerical integration. This was our third approach.
The first approach was too simplistic. So, Ian might wish to skip it.


\section{Layer Optimisation Results}

Show the table only with the extra parameters not present in isothermal model.
In particular, I can remember the thermal parameters of the two electrodes,
separator, pouch, current collectors and electrolyte.

Present the results in a horribly bland table format steering well clear of the
heatmap. Krishna's view is that the results do not stand alone by themselves and
if you plonk a layer choice from the heatmap/table into your cell, things are
not expected to work as is. The numbers herein are the result of quite a few
assumptions and hold validity only within the universe in which it was created.

However, the framework itself is transferable and is the most valuable component
of the work. A user reading this thesis can easily substitute their own cell
parameters and system-level considerations into the code and obtain results from
it accordingly. This was the working premise of the paper as well, until Ian
decided to write up a bloated explanatory section.

Plots of acceleration and fast charging for successful layer count.

For PHEV, although common module design will be alluded to, it will not be dwelled on or explained or analysed in depth.


Note: In Ian's thesis, Krishna seeks attribution for the idea to use a heatmap
since it arose out of Krishna's heavy criticism, bordering on the offensive,
about continuously complaining that our group's plots/charts/graphs do not
exhibit any ``interesting'' twists and turns and fancy illustrations. Apologies
for this. Anyway, the heatmap idea was my suggestion. However, I shall not take
an inch of credit for the implementation as it was entirely Ian's work in coming
up with fancy schemes.

Note: Having obtained BEV results, Krishna also set up the full set of
assumptions for the PHEV simulations too before leaving actual numerical
simulations to Ian (and on vacation to India followed by Konstanz). So an
acknowledgement is required if Ian choses to list these assumptions. Ian can
discuss monotonicity issues in PHEV simulation that he found out about.

\section{Appendix}
