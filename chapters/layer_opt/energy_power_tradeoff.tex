% -*- root: ../../main.tex -*
%!TEX root = ../../main.tex
% vim:textwidth=80 fo=cqt conceallevel=0


Varying the  number of electrochemical  layers stacked  within a pouch  cell has
contrasting effects  on its energy  storage and power handling  capabilities. In
this section,  a high-level  intuitive explanation of  this phenomenon  is first
offered, before  delving into  a detailed  presentation of  this effect  and its
implications for a specific  example cell in \cref{sec:energypowertradeoffdemo}.
Interwoven  into  the  narrative  is  a set  of  simplifying  assumptions  which
establishes  the broader  context  within which  a  computational framework  for
determining the optimum  number of layers for a specific  target design shall be
formalised (to be discussed in \cref{sec:layeroptframework}).

\subsection{Preliminary assumptions}\label{subsec:layeroptassumptions}

To  obtain a  balanced  loading of  both electrodes  and  to avoid  asymmetrical
exhaustion  of lithium  from  one  of the  electrodes  during  operation, it  is
desirable to carefully calculate the  volume of electrochemical active materials
to be accommodated  within the cell. This concept is  well-known and is commonly
discussed  in  standard  textbooks in  the  field  such  as  those by  Rahn  and
Wang~\cite{Rahn2013} wherein  example calculations are presented  for non-porous
electrodes. The study  by Ramadesigan~\etal~\cite{Ramadesigan2012} also supports
the  statement that  capacity  matching  of anode  and  cathode  materials is  a
standard practice in cell design.


In  the  case  of  lithium  ion   cells  with  porous  electrodes,  the  concept
of  electrode-balancing  involves an  additional  variable  \viz{} the  porosity
of  the  active materials.  The  role  of  porosity  and its  corollaries  \ie{}
the  material  volume  fraction  and  filler/binder  fraction  is  discussed  in
\cref{subsec:spmp2dparametrisation}.  In this  work,  a  major assumption  about
material porosities  (and hence active-material/filler volume  fraction) is that
they are held constant. The rationale behind using this simplified assumption is
as follows.

This  author  visualises  the  integration  of  cell-level  design  optimisation
(through  an optimal  layer  selection procedure)  into  the overall  drivetrain
design  by the  \emph{cell manufacturer}  before  a custom  design is  delivered
to  vehicle/system  integrators.   Cell  manufacturers,  especially  small-scale
manufacturers do not necessarily  synthesize each electrochemical component, but
instead may opt  to source certain raw-materials from  an upstream supply-chain.
From  a  manufacturing  viewpoint,  the  porosity  of  the  electrode  materials
is  governed  by  the  extent  of  calendaring  of  the  electrode  reel.  Using
pre-calendered electrode materials or sourcing  large volumes of electrode reels
with a fixed extent of calendaring can help to keep costs low. Since researchers
in  the  field are  typically  not  privy to  the  specifics  of the  industrial
procurement process,  in the absence  of further information, the  assumption of
constant  porosities provides  a  good starting  point  for this  model-oriented
design study.

From a technical  viewpoint, there exists another redeeming  argument to support
the constant  porosity assumption. Keeping material  porosities constant enables
to  eliminate  one  degree  of  freedom  from  the  design  optimisation  study,
thereby narrowing  the dimensionality of  the search space.  To the best  of the
author's knowledge,  there has not  yet been  any published work  tackling layer
optimisation  of  pouch  cells.  Building an  initial  infrastructure  in  terms
of  a  computational  framework  that  is  based  upon  this  constant  porosity
approximation  shall at  least  provide a  solid foundation  to  build upon  for
such  real-life  use-cases.  The  author  foresees  this  study  as  a  vanguard
research into  cell engineering and therefore  places a high value  in obtaining
ballpark estimates of  an optimal layer count, albeit  with constant porosities.
Ramadesigan~\etal~\cite{Ramadesigan2012} present  an opinion that the  choice of
porosities of electrode  materials is currently being done on  a trial and error
basis. Nevertheless, for  real-world use, the influence of  varying the material
porosities  on the  cell's  performance is  to be  quantified.  Hence, prior  to
adopting  this  model-based  methodology  for  production  yields  at  scale,  a
fully-integrated design optimisation process with  variable porosities has to be
developed. Therefore, in this work, the study is restricted to constant porosity
values, whilst  acknowledging variable porosity  designs as an  important aspect
for future studies.

At  a  system level,  the  efficiency  of the  drivetrain  is  considered to  be
constant. The  drivetrain of  an electric  vehicle consists of  a whole  host of
electrical and mechanical components such as power electronics, electric motors,
gearing, differential shaft and other  transmission systems. The efficiencies of
each  of these  individual  components has  a cascading  effect  on the  overall
drivetrain efficiency.  The efficiency of  each component is  strongly dependent
upon the operating point. For instance, the efficiency of an electric motor is a
function  of  its  torque-speed  curve.  In  practice,  it  is  rarely  easy  to
decouple  these efficiencies  at  least  during the  initial  design stage.  The
datasheet/technical specification of each component  in the platform is required
to make  a comprehensive multiphysics-based  design optimisation study.  This is
well  beyond the  scope  of this  work  and requires  access  to various  design
blueprints. Therefore, a constant lumped  efficiency value for the drivetrain is
adopted  for this  work. However,  the  proposed optimisation  methodology is  a
modular one  which implies that  it can be suitably  adapted \eg{} to  include a
efficiency  value dependent  upon power  delivered at  the wheels.  However, the
biggest redeeming  aspect (observed after the  completion of the study)  is that
using a constant  efficiency value did not influence the  final layer choice for
the cell  design. As  seen in  \cref{sec:accpathway}, the  drivetrain efficiency
plays a role  only during acceleration studies. As per  the results presented in
\cref{sec:resultslayeropt}, the  layers required  for satisfying even  the basic
fast-charging  requirements far  exceed  the layers  required  for handling  the
acceleration power demands. Therefore, this  assumption is justified for keeping
the computations tractable.

From  a  pack  perspective,  the   primary  assumption  in  the  formulation  of
the   proposed  optimisation   methodology  is   that  the   pack  configuration
(series/parallel arrangement  of modules, number  of cells per module  and other
system-level specifications) are held constant  throughout. The validity of this
assumption  is easily  justified  since  a cell-level  design  may be  performed
independently of  the larger drivetrain  design. In fact, the  author postulates
that  present design  process for  electrified transportation  is a  modular one
\ie{} empirical cell designs are  developed based on certain specifications laid
out by vehicle  manufacturers and is not integrated into  the drivetrain design.
This  modularity  in the  design  approach  enables  to keep  such  system-level
parameters constant.

A further assumption  in this study is  that the overall height of  the pouch is
held constant. In  the absence of this constraint, any  arbitrary pouch size can
be chosen,  leading to an  infinite-dimensional optimisation problem  wherein no
unique optimality  criterion exists.  This assumption is  in-fact enforced  by a
current  trend  in the  automotive  industry  \viz{} adoption  of  common-module
designs wherein  the physical  dimensions of  the pack  are chosen  a~priori and
modularising the  pack helps in  tailoring them  suitably to cater  to different
market segments. Extending this philosophy down to the cell level, it is easy to
visualise the  benefits of  having cells of  identical exterior  dimensions. For
instance, having a  common inventory helps a vehicle manufacturer  to keep costs
in check  for subsequent designs  \eg{} for  derivative model families  of their
product  portfolio. This  means that,  for  any layer  choice to  be tried,  the
constituent components  of the cell is  to be arranged and  contained within the
same  pouch  (of  fixed  exterior  dimensions).  This  naturally  leads  to  the
assumption that the  thickness of the pouch material used  shall remain constant
throughout, which in-turn implies that the overall height of the electrochemical
stack  within the  pouch is  constant. The  detailed calculations  of the  stack
height is presented in \cref{sec:surfareaperlayer}.

The  current collectors  and the  separator  in each  electrochemical layer  are
assumed  to  have  uniform  thickness  irrespective  of  the  number  of  layers
used.  Barring  minor manufacturing  variability  and  tolerances, these  values
are  merely  factual data  requiring  no  further justification.  For  instance,
a  constant  separator thickness  was  used  in  the design  optimisation  study
by  Newman~\cite{Newman1995}.  The  final  assumption  from  an  electrochemical
point  of view,  introduced specifically  for the  first time  in literature  by
this  thesis author,  is  that the  relative thicknesses  of  each electrode  is
held  constant to  a  fixed ratio.  This warrants  further  explanation, but  is
ill-suited  for this  introductory discussion.  The details  of this  aspect are
discussed  in  \cref{sec:electroderatio}. Certain  assumptions  are  to be  made
about the  temperature distribution  within the  layers owing  to the  choice of
cooling arrangement.  These aspects merit  more than  a cursory listing  in this
introductory section and hence is discussed in \cref{sec:celllevelxeVinfo}.

\subsection{Motivation}\label{subsec:layeroptmotivation}

This section aims to provide a  qualitative description of the effect of varying
the number of layers  within a pouch cell and presents  the motivation to embark
upon this layer optimisation effort.

Given the  assumptions listed in \cref{subsec:layeroptassumptions},  it is clear
that  changing the  number  of layers  in  a pouch  of  fixed thickness  results
in  different  absolute  electrode  thicknesses.   This  in  turn,  affects  the
electrochemical-thermal behaviour  of the  cell. Having a  low number  of layers
means  that the  proportion of  energy-storing  materials within  the volume  is
higher, leading  to greater nominal capacity.  However, owing to the  large time
constants necessary for  lithium ions to diffuse through thick  domains, not all
of this stored energy might be available for utilisation. With thick electrodes,
the power  handling capability  of the  cell suffers  since lithium  deficits at
electrode surfaces shall lead to the collapse of its terminal voltage.

Prima~facie,  based  on the  above  discussion,  although  it appears  that  the
absolute lowest possible  number of layers (\ie{} one layer)  is the best choice
for maximising  driving range, there  exist two  other goals that  conflict with
this design  intent. Firstly, there must  be a minimum electrode  active surface
area  to  handle the  power  demands,  and hence,  a  minimum  number of  layers
typically far  greater than one. Higher  power capability is achieved  by way of
larger electrode surface  area and higher electrical  and thermal conductivities
owing to the presence of more current collectors. Secondly, the acceptable limit
on  lifetime  degradation of  cells  places  an  upper  bound on  the  allowable
temperature rise during vehicle operation. Increasing the number of layers has a
two-fold mitigating effect on the  temperature-rise experienced by the cell. The
\emph{power  density} within  each  layer  is diminished  due  to the  increased
available  surface area,  leading to  reduced ohmic  heat generation  within the
cell. With  each layer requiring a  Al-Cu current-collector pair, the  number of
heat conduction  pathways increases  linearly with the  number of  layers. Thus,
increasing the number of layers has a beneficial effect on pack lifetime.


In summary,  for very low number  of layers, there exists  more active material,
leading  to a  high  energy  capacity. However,  the  reaction  surface area  is
diminished proportionately leading to lower power capability. Furthermore, owing
to the presence  of very thick electrodes, the current  density within the solid
conductive  matrix shall  not  be  homogeneous~\cite{Pals1995}, nullifying  some
fundamental modelling assumptions of the  standard \gls{dfn} model. On the other
hand,  very  high  number  of  layers  imply  vanishingly  thin  electrodes  and
correspondingly  less  active material  accommodated  within  the cell,  thereby
resulting  in  a lower  energy  capacity.  \Cref{fig:energyvspowercell} shows  a
qualitative comparison of the construction of one layer of an energy cell versus
power cell which helps to illustrate all the aspects discussed thus far.

\begin{figure}[!htbp]
    \centering
    \includegraphics{energy_vs_powercell}
    \caption[%
    Qualitative comparison  of the  construction of one  layer of  a high-energy
    cell versus a high-power cell
    ]%
    {%
        Schematic  depicting a  qualitative  comparison of  the construction  of
        one  layer  of  a  high-energy   cell  versus  a  high-power  cell.  The
        illustration  at top  depicts one  layer of  a high-energy  cell wherein
        thick  electrodes  are  used.  The bottom-left  illustration  depicts  a
        single layer  of a high-power  cell wherein very thin  electrode regions
        are  used.  Both  cell  diagrams  are  drawn  to  the  same  scale.  The
        bottom  right  plot  qualitatively indicates  the  relationship  between
        C-rate  and  the nominal  cell  capacity.  Illustration reproduced  from
        \mbox{von~Srbik~\cite{VonSrbik2015}.}

    }%
    \label{fig:energyvspowercell}
\end{figure}


Therefore, there exists  a research question on what constitutes  the best layer
choice  that straddles  this  trade-off  with the  least  penalty  to the  power
capability  of  the  cell  whilst simultaneously  having  the  maximum  possible
capacity. This  saddle point determination needs  to be performed for  a curated
set of power input/output conditions to the cell. This niche problem has not yet
been  tackled by  researchers  and therefore  motivates the  need  to perform  a
careful design study which is documented in this chapter.

\FloatBarrier

\subsection{Quantitative demonstration of energy/power trade-off}\label{sec:energypowertradeoffdemo}

The discussion in \cref{subsec:layeroptmotivation} has motivated the need for an
in-depth exploration of the energy to power trade-off expressed as a function of
the number of  layers. Before embarking on constructing a  framework to optimise
the layer choice by formalising various constraints that govern this optimality,
this section aims to quantitatively  demonstrate this relationship by applying a
fixed galvanostatic discharge to an example cell. Additionally, the crucial idea
of \emph{usable} energy versus \emph{total} stored energy is also introduced.

A  \gls{lco}  cell  whose  physical  properties  and  simulation  parameters  is
drawn from  the combined set of  data from tables~\ref{tbl:lcoSimParamslayeropt}
and~\ref{tbl:lcoSimParamsSPMp2d} is  used as the  example cell. The only  set of
values that overlap between these two tables are ---
\begin{enumerate*}[label=\itshape\alph*\upshape)]
    \item the cut-off voltages, and
    \item the number of nodes  used for numerical  discretisation of  the governing  \gls{pdae} equations.
\end{enumerate*}
For these conflicting quantities,  the values in \cref{tbl:lcoSimParamslayeropt}
prevail for all simulation studies  in this chapter. Furthermore, the individual
electrode thicknesses from \cref{tbl:lcoSimParamsSPMp2d}  are not directly used,
but  instead  calculated  for  every  layer  choice  by  keeping  the  ratio  of
their  relative  thicknesses  constant.  This   aspect  shall  be  explained  in
\cref{sec:electroderatio}.

\Cref{fig:fig_CC_discharge_curves}  illustrates  the  influence  of  the  number
of  layers   on  the  energy   and  power   capability  of  the   example  cell.
Starting  at  \SI{100}{\percent}  \gls{soc},  a constant  current  discharge  of
\SI{60}{\ampere}\footnotemark{}  is applied  to a  \gls{dfn} model  of the  cell
until reaching the  lower cut-off voltage. For each discharge  run, the model is
reconfigured with  a different  layer choice. Five  distinct layer  choices have
been carefully chosen so as to  provide a clear illustration of the energy/power
trade-off phenomenon.

\begin{figure}[!bp]
    \begin{minipage}[t]{\textwidth}
        \centering
        \includegraphics[trim=4 4 2 4,clip]{fig_CC_discharge_curves.pdf}
        \captionsetup{labelsep=note}
        \caption
        [%
        Voltage curves for a \SI{60}{\ampere} galvanostatic discharge from
        \SI{100}{\percent} \glsfmtshort{soc} until cut-off voltage for a few layer
        choices, in a pouch cell of fixed exterior height.
        ]%
        {%
            Terminal voltage curves of a Li-ion cell (with parameters
            given in \cref{tbl:lcoSimParamslayeropt}) under a \SI{60}{\ampere}
            galvanostatic discharge beginning from \SI{100}{\percent}
            \glsfmtshort{soc} until lower cut-off voltage for a few layer
            choices~$n$, in a pouch cell of fixed exterior height. The maximum
            usable energy is achieved for an intermediate choice of $n$
            that corresponds to neither the highest nominal capacity layer
            configuration ($n$=\num{10}) nor the highest electrode surface area
            configuration ($n$=\num{90}).
        }%
        \label{fig:fig_CC_discharge_curves}
        \mpfootnotes[1]
        \footnotetext{{The   rationale  behind  choosing   this  specific
                magnitude  of  applied  current  is   explained  in  the  section  dealing  with
        \hyperlink{refcellselection}{selection of a  suitable reference capacity cell} (also see \cref{sec:surfareaperlayer}).}}

        \footnote{This figure was created by \mbox{Ian D.\ Campbell} who asserts copyright,
            with intellectual contributions from and the right to use asserted by
        \mbox{Krishnakumar Gopalakrishnan}.}
    \end{minipage}
\end{figure}

As  seen  in \cref{fig:fig_CC_discharge_curves},  during  the  initial phase  of
discharge, the terminal voltage  of the cell is the highest  for the two highest
layer choices  \ie{} $n  = 90$  and $n=70$. Consistent  with the  explanation in
\cref{subsec:layeroptmotivation}, these  two layer choices have  thin electrodes
and  hence  comparatively low  resistances  leading  to  only a  small  internal
overpotential drop. However,  as expected, their total energy is  lower than the
cell with  $n=50$ layers as  evidenced by  their relative run-times  until lower
cut-off voltage.  This is to  be expected as the  thin electrodes of  these high
layer count cells cannot  store a large volume of active  material. Based on the
explanation  from \cref{subsec:layeroptmotivation},  it  is  expected that  this
trend  will continue  \mbox{\ie{} the}  lower the  layer count,  the higher  the
run-time until  cut-off. If this  were the case,  prima~facie it seems  that the
layer optimisation task is trivial.

Inspecting the  discharge curves of  lower layer  choices brings into  light the
concept of \emph{usable} energy. Contrary  to expectations, the discharge curves
corresponding  to very  low layer  counts in  \cref{fig:fig_CC_discharge_curves}
terminate even earlier than $n=50$. This is  owing to the fact that although the
total  stored energy  in cells  with low  layer counts  is much  higher, only  a
fraction  of it  is usable.  This  aspects introduces  non-trivial dynamics  (as
discussed below) to an otherwise linear optimisation task.

For  instance,  when $n  =  10$,  the terminal  voltage  of  the cell  collapses
instantaneously, reaching cut-off  voltage whilst its \gls{soc}  remains as high
as \SI{96}{\percent}. At very low layer  counts, the thickness of each electrode
is high. This presents a high resistance  to the flow of charges thereby leading
to  high overpotential  drops  within the  cell. The  usable  energy under  this
\SI{60}{\ampere} galvanostatic  discharge for various layer  choices is compared
in \cref{tbl:CC_discharge_curves_table}. It can be  seen that for very low layer
counts,  the usable  energy that  can be  extracted is  miniscule, albeit  their
theoretical  capacity~$Q_n$  are  in-fact  the highest.  The  usable  energy  in
\SI{}{\watthour} reported in \cref{tbl:CC_discharge_curves_table} is obtained by
multiplying the integral  of the area under each discharge  curve by the applied
current (\SI{60}{\ampere}) with the appropriate scaling of the time-base ( \ie{}
conversion from minutes to hours).

% -*- root: ../../main.tex -*-
%!TEX root = ../../main.tex

\begin{table}[!htbp]
    \caption
    [%
    Theoretical  capacity \&  usable energy  of a  Li-ion cell  for a  few layer
    choices under a \SI{60}{\ampere} galvanostatic discharge
    ]
    {%
        Theoretical capacity and usable energy of a Li-ion cell (with parameters
        given in \cref{tbl:lcoSimParamslayeropt}) for  a few layer choices under
        a \SI{60}{\ampere} galvanostatic discharge.
    }%
    \label{tbl:CC_discharge_curves_table}
    \centering
    \begin{tabular}{@{} S[table-format=2.0] S[table-format=1.2] S[table-format=2.2]  S[table-format=3.2] S[table-format=2.2] S[table-format=2.2] @{}}
        \toprule
        \multicolumn{1}{@{} l}{$n$} &  \multicolumn{1}{c}{\footnotesize C-rate} & \multicolumn{1}{c}{\footnotesize \makecell{Theoretical \\ Capacity  (\si{Ah})}} & \multicolumn{1}{c}{\footnotesize \makecell{Usable \\ Energy \si{(Wh)}}} & \multicolumn{1}{c @{}}{\footnotesize \makecell{Remaining \\ SOC  (\si{\percent})}} & \multicolumn{1}{c @{}}{\footnotesize \makecell{Resistance at \\ cutoff  (\si{\milli\ohm})}} \\
        \midrule
        90 & 1.24 & 48.25 & 166.46 & 9.84  & 0.97  \\
        70 & 1.11 & 53.99 & 184.80 & 10.26 & 1.35  \\
        50 & 1.00 & 59.73 & 195.47 & 13.51 & 3.44  \\
        30 & 0.92 & 65.47 & 101.20 & 58.95 & 10.24 \\
        10 & 0.84 & 71.21 & 10.15  & 96.22 & 11.18 \\
        \bottomrule
    \end{tabular}
\end{table}


\Cref{tbl:CC_discharge_curves_table}  also brings  into view  the fact  that the
\mbox{C-rate} of the  cell becomes a variable quantity even  for a galvanostatic
discharge,  due to  the dependence  of  its nominal  capacity on  the number  of
layers~$n$. This represents a departure from the norm in the modelling community
wherein the  performance of cells  are quantified as  a function of  the applied
C-rate  \eg{} in  chapters~\ref{ch:spmanalysis} and~\ref{ch:newelectrolytemodel}
of  this thesis.  However, the  preliminary investigation  thus far  has quickly
revealed that  this normalised  quantity does  not hold  much importance  in any
study where the number of layers within a pouch cell is varied.

Taking into account  these factors, a reasonable choice of  the number of layers
in  this specific  \SI{60}{\ampere} galvanostatic  application for  this example
cell  could  be $n=50$.  This  represents  a  practical compromise  between  the
surface area  available for  reaction and  the total  volume of  active material
accommodated.  Out of  the finite  layer configurations  considered, this  layer
choice offers the highest usable energy for the given discharge rate.

In this  sample study, only  a handful of  layer choices were  considered, which
represents  only  a  small  possibility  of  the  overall  design  space  to  be
considered. Furthermore,  thermal considerations were  not explored so  far. For
robust  cell design,  manufacturers  shall need  a  widely applicable  model-led
design tool  that can tackle the  various scenarios that can  occur in real-life
operating conditions. A  deterministic set of optimality criteria  for the layer
selection  is also  to  be formulated.  The choice  $n=50$,  therefore does  not
represent the general optimal layer choice  even for this example cell. However,
this sample study  serves as an illustrative demonstration of  the trade-offs in
energy  versus  power handling  capability  of  a cell  for  a  specific set  of
conditions. Furthermore, it introduces  the complicating aspect of \emph{usable}
capacity into what would have otherwise been a trivial exercise, thereby setting
the tone for the development of a general layer optimisation framework for pouch
cells.

