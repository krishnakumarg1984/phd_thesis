% -*- root: ../main.tex -*- Krishna was leading at this point.
%!TEX root = ../main.tex

\graphicspath{{chapters/layer_opt/figures/}}
% ----------------------- contents from here ------------------------

\chapter{Model-based Design Of Pouch Cells}\label{ch:modelbaseddesign}
\vspace*{-1em}
\startcontents[chapters]
\printcontents[chapters]{}{1}{\setcounter{tocdepth}{1}}

\bigskip

Honour  code pledge:  This  chapter  plan was  written  without  looking at  the
manuscript even  once. Only the actual  code commits of LIONSIMBA,  BOLD toolbox
and the data from shared Box folder was used in writing this up.

The literature review for this chapter shall be split over two parts - the first
part shall be in a separate literature review chapter. The other part will be in
this chapter,  but peppered in  multiple areas/sections. To clarify,  the second
part  shall deal  with the  literature  pertinent to  the section  whence a  new
concept is introduced.

\section{Literature  Review Chapter}
This shall  present an  inverted pyramidal  overview on  the topic  of model-led
design of lithium ion  cells. The review exercise shall aims  to bring to light,
the glaring  contrast between  the cornucopia  of literature  discussing various
aspects  of drivetrain  design versus  the relative  paucity of  published works
dealing with cell-level design optimisation for electrified transport.

The literature review shall be broken up into sections as follows.

\section{Drivetrain design Optimisation}

Two to three paragraphs discussing  the published works (actual journal articles
in  SAE etc.\  authored by  automotive companies)  dealing with  all aspects  of
drivetrain  design. This  includes two-three  sentences describing  special care
touching topics of choice of gearing ratios, motor speed-torque-efficiency curve
discussion, choice  of rare-earth materials  for motors, choice of  motors, body
chassis for minimizing drag.

A detailed publication on  the ins and outs of the  Prius drivetrain analysed by
OakRidge  National  Labs  will  be  reviewed. Its  figures  shall  be  used  for
highlighting motor-speed torque curves,  choice of power electronics, schematics
of bi-directional converters etc, and any (there are many) relevant figures that
may be used.

A  remark  or  two  about  typical  efficiencies  of  components.  Basically  my
drivetrain course  and the Emadi  textbook cited and  relied upon as  source for
material

My  own  course  on  drivetrain  design shall  be  cited  and  the  optimisation
exercise/homework be quoted.

This author  recognizes that this  is a huge  topic of interest  in conventional
mechanical and  automotive engineering, the  idea is  not to do  a comprehensive
literature review on  this well-established field, but nevertheless use  it as a
crutch  to  bring  out  the  stark  contrast  in  the  published  literature  on
battery-specific  design. It  will become  clear that  manufacturers and  system
integrators  rely  mainly  on  equivalent  circuit or  even  bucket  models  for
drivetrain design optimisation. They cannot be entirely blamed, since this is an
inherently  multi  scale problem,  how  design  optimisation  can be  done  with
fine-grained physics-based pack models is still a research question.

In particular, at the  end of this section, it should drive  home the point that
design  optimization  is  being  performed  in  every  field  other  than  at  a
battery/cell level. This  author hypothesises the cause to be  attributed to the
relative  infancy  of  the  subject  as well  as  possible  secrecy  among  cell
manufacturers/battery systems  engineers. A  need to study  this field  shall be
pointed out.

\section{Battery Design Optimisation}

Group  of cells  arranged into  modules  and then  connected in  series/parallel
strings to form battery pack. The state  of the art in battery pack design shall
be reviewed  in a  few paragraphs. Only  the relevant works  will be  cited. Our
focus is on optimisation of pack design.

The  literature  covering   pack  design  optimisation  shall   be  covered.  In
particular,  about adherence  to  standard  to design  modules  below  50 V  for
safe  handling by  technicians  can be  viewed  as a  constraint  on the  design
optimisation.  Very historic  views  shall  only be  mentioned  and not  covered
in-depth.

\subsection{Thermal Design}

System level pack thermal design shall be acknowledged as a complex topic beyond
the scope of this work. How inner cells  in the pack are typically a few degrees
hotter than the outermost cells. Finite-Volume based CFD analysis is the norm in
this field  and is  usually done  by a dedicated  thermal engineering  team. The
interested reader shall be referred to the relevant literature.

However,  one important  takeaway that  will  later influence  and dictate  this
author's study  with respect  to pack  thermal design is  the choice  of cooling
mechanism. In the view of the  experimental results revealed by Ian Hunt's paper
on tab  cooling, particularly, the uniform  thermal gradient across layers  in a
pouch cell, particularly helps to represent an individual cell's electrochemical
properties  (composed of  many layers)  through  a representative  study on  one
layer  using the  P2D model.  For a  standard surface-cooled  pack design,  this
implies  that for  cell modelling,  two dimensions  must be  considered for  the
electrochemical and coupled thermal phenomena.

\subsection{Pack Layout}
Common system  bus voltages  shall be  listed, which will  entail the  number of
series connected cells  in each string. Choice of number  of parallel strings to
determine the  energy content of  the pack, and hence  the driving range  of the
vehicle. With  this considerations  in place, the  literature on  arrangement of
packs shall be presented.

There   are  a   couple  of   literature  dealing   with  optimisation   of  the
series-parallel  matrix arrangement.  Some  real-world constraints  and how  the
literature handled them shall be discussed.

\section{Cell Design}

Cell Design (even within the narrower  lithium ion chemistries) is a complex and
hugely active area of  research. The wealth of literature is  much more than one
could possibly review  in a thesis, and  spans topics in synthesis  of new anode
and cathodes (cite some recent research here), new electrolyte salts and organic
solvents  that  remain  stable  and  inflammable  beyond  5V,  intelligent/smart
separators that shut down their pores in  the event of thermal runaway etc. Very
brief listing only of  the most important papers in the last  2 years only, just
for roundedness.  This broad  cell design  overview concludes  with a  view that
while  the future  is  exciting  and certainly  deserves  investment,  a lot  of
breakthroughs are still in the pipeline and not ready for primetime.


Squeezing out  every last ounce of  juice left in batteries,  while not damaging
them  is of  prime interest  to manufacturers  today to  quell range  anxiety of
customers.  Similarly, the  charging time  is a  bottleneck in  real-life usage.
Therefore, to tackle  an immediate demand, we focus on  the optimisation of cell
designs  based  on  existing  materials  for which  supply  chains  are  already
well-established.

\subsection{Practical Approaches to Cell Design Optimisation}

In the view of this thesis author, cell design optimisation can be carried out through
\begin{itemize}
    \item Empirical approaches
    \item Systematic Experimental approaches
    \item Model-led approach
    \item and/or an iterative combination of the three above
\end{itemize}

Give  one  example each  of  empirical  design  and  experimental design  and  a
combination of them. Cite through review of relevant literature on how they have
successfully taken off.

Explain  why model-led  design optimisation  is still  lagging behind  empirical
optimisation.  Physics-based  models  have  too many  parameters.  Criticism  on
parameterisation.  Each cell  manufacturer  has  a custom  cell  design that  is
shrouded  in secrecy.  Examples  include in-house  ingredients  for binders  and
fillers and enhance the electronic  conductivities of the active material matrix
while ensuring that the electrodes remain  stable. Such art are never published.
Even system  integrated tasked with optimisation  of the drivetrain do  not have
access  to  the  individual  electrode-level/electrolyte-level  enhancements.  A
teardown of the cell can at most reveal the geometric parameters of the cell and
may offer a  clue to some physical properties of  the material. Parameterisation
involve specalised  lab equipment  that is often  available only  to large-scale
test  facilities (SEM,  porosimeters)  in specialised  academic  labs. There  is
also  a culture  of lack  of  information sharing  or NDAs  signed that  prevent
publishing of  such information  . However, without  definitive parameterisation
and co-operation  between cell  manufacturers and system  integrators, model-led
cell  design  has  suffered,  although  it   is  of  need  of  the  hour  (until
more  material-level  fundamental  breakthroughs  are  available).  Explain  how
this  interdisciplinary  field  involving   numerical  expertise  and  intricate
electrochemical knowledge can  lead to better design of  batteries that benefits
all.

Dwell  on  the  advantages  of   a  model-led  design  optimisation  like  rapid
prototyping  and minimising  time  to first  production  yield. Offer  parallels
through  published  literature  on  the  areas  where  this  has  been  deployed
successfully.

Now,  discuss  literature, wherein  despite  limited  parameters are  available,
academia and industry have worked hand-in-hand to optimise electrode thicknesses
and  porosities. There  are  two-three  strong literature  in  my Mendeley  (Ian
did  not  yet  give  me  access  to our  shared  Mendeley)  that  discuss  these
optimisation studies made, however, from an experimental point of view. There is
nothing missing  from a modelling point  of view. Modelling results  followed by
experimental confirmation  is the  method of  science that  can be  employed for
robust progress in the field.

Conclude with a view that there has  been an obvious skip (a jump) in literature
that  is  glaring  and  in  plain  sight.  From  an  electrified  transportation
perspective,  there has  been  plenty of  system-level  analysis discussing  all
aspects  of system-level  pack design  (electrochemical, thermal  and placement)
mostly  done by  industry folks.  At  the other  end  of the  spectrum, we  have
academicians and  government research  labs toiling  away at  microscopic scale,
synthesising materials, which  may be years before being available  at scale for
mass-adoption. The  closest thing  we have now  is experimental  optimisation of
electrode thicknesses and  porosities that is both  intensive and time-consuming
and not attractive  as a drop-in optimisation solution to  industry. The obvious
jump in scale is from the system level down to electrode-level. Pouch cells form
the majority of automotive-grade cell type (refer intro chapter). There is a key
design choice that can be optimised here \viz~the number of layers. Literature
is  missing  this in  entirety  with  not  a  single published  work  discussing
optimising  the  number  of  layers,  at-least  from  a  model-led  perspective.
The  author's work  therefore  discusses  this critical  aspect  of cell  design
optimisation in chapter so and so.


The above paragraphs were in a  dedicated chapter and the following content goes
into the chapter. This paragraph is written here only for the chapter plan.

\section{Chapter Introduction}

A statement  of thanks and  acknowledgement upfront  that this work  was jointly
undertaken with  Ian Campbell. Relevant  sections where Ian's help  was obtained
are attributed in-line,  with an attribution statement  identifying the specific
nature of his support.

A brief 1-2 sentences throwback to the literature review, referring what we will
say in this chapter.  Explain the goal of this chapter, to  arrive at the number
of layers to meet  a given energy and power demand. The outcome  is that a ready
to  use tool  is made  available to  validate empirical  layer choices.  Special
emphasis is  placed on the  \emph{methodology} since the  results per se  do not
stand alone outside of the modelling regimen (universe we created). However, the
value  is on  the  methodology and  its  implementation in  a  toolbox which  is
immediately  available  for  download  and  use by  industry  to  confirm  their
empirical layer designs.


%In the absence of access to cell manufacturing facilities to confirm and test the layer
% Immediate adoption in industry.

\section{Energy to Power Trade-off}
Impart knowledge on  energy cell versus power  cell with the help  of a diagram.
Explain the effect of useable energy and power for a given cell with the help of
a figure and table. How layers place  a role in controlling this trade-off shall
be discussed in the subsequent sections.

\section{Augmentation of P2D parameters}

Shed more light on the p2d model. Explain  how they do not model a cell of given
capacity, but instead  work on a normalised basis driven  by the applied current
densities rather than  the external current. The parameter that  layers within a
cell  change  is  the  overall electrochemically  active  cross-sectional  area.
Criticise  how  published literature  curiously  omit  this important  geometric
parameter, however  they may  be forgiven in  the scope of  their work  since it
needs only normalised  dynamics. This parameter comes into  light when modelling
anything involving geometries as in this project. This is the product of surface
area per face and the number of  layers. To determine the surface area per face,
the author has derived a new  methodology/process involving a sequence of steps,
based on assumptions and literature search.  The process involves selection of a
real-world cell, and ultimately mapping it to the surface area per unit face. To
the best  knowledge of  the author,  this mapping  from a  physical cell  to the
Newman model is  unique and is claimed  as the author's own  contribution to the
art.

Describe briefly  the process  of mapping  a real-world  cell onto  the standard
\gls{p2d} model.

Note: Krishna wrote the Northrop to EV mapping code and explained the concept to
the mapping to Ian. Ian's help on the matter is identified where relevant.

\subsection{Modelling Platform and Preconditioning}

Couple of  statements about why LIONSIMBA  was chosen as the  modelling platform
for implementing the  P2D dynamics. The cell parameters used  are shown in table
xx. This cell is henceforth known as the LIONSIMBA cell or Northrop cell.

Discuss  the missing  elements in  LIONSIMBA only  with respect  to the  present
problem  at hand  \viz{the  stoichiometries}.  Thanks to  Ian  for running  the
time-intensive and  memory-intensive simulation  process on his  workstation and
offering comments on datalogging.

\subsubsection*{Stoichiometry Augmentation}
Discuss the problem first. How LIONSIMBA  started always at 85.51 percentage and
needed to  do a  discharge down  to zero  percent before  having the  ability to
charge. For this  project, stoichiometries are vital  for capacity determination
and  the 1C~current density.  Explain  how stoichiometries  were refined  until
cut-off for  infinitesimal bleeding  discharge current achieved.  Noted relevant
values. Explanation with detailed figure on how stoichiometries were found to be
missing. Explain refinement  of how approximate capacities  reported by Northrop
and Subramanian  were refined precisely.  Explanation of remnant  capacities and
stoichiometries computation.  Explanation of  1C~current density.  Note: Krishna
wrote the parameters  init capacity computation code. Thanks to  Ian for running
the simulation

\subsection{Selection of a Suitable Reference Capacity Cell}

Explanation of how  the Bolt EV cell  was selected (state of the  art in driving
range) In particular, explain how the critical computation of 60 Ah capacity was
done. Furthermore, convincingly explain how a  pouch cell thickness of 10 mm was
used with all the background references.

Note: Krishna did this thorough literature review, whose proof is available with
timestamp in Box. Krishna would like to  thank Ian for his time in brainstorming
the exact BEV to use. This phase lasted a couple of weeks. For the PHEV, Krishna
and  Ian jointly  did the  PHEV literature  search, and  since Krishna  does not
intend to  go in depth  for PHEV results,  Ian may choose  to use it  and simply
acknowledge Krishna.

\subsection{Layer Assembly within Pouch Cells}
With  the  help   of  Northrop's  layer  assembly  figure,   explain  the  layer
configuration/arrangement within a pouch cell.  The next task is then identified
as computing the number of layers within the pouch cell.

\subsubsection*{Number of Layers of LIONSIMBA aka Northrop cell}
Krishna came up with  the idea of using integer optimisation  for this task. The
software MIDACO  was also selected by  Krishna and explained to  Ian. The MIDACO
result of the number of layers within the standard cell was now available.

\subsubsection*{Computation of Surface Area per face,~\protect{$A_\text{cell}$}}
Show  the simple  algebraic  computation of  overall surface  area~$A$ and  the
per-face area~$A_\text{cell}$. Explain  how the  area per face  shall be  a key
quantity in the layer optimisation framework discussed later on.

Layerphoto showing face areas and anode/cathode verhand etc will be shown here

This concludes the augmented set of parameters  added by the author to the basic
parameter set  of the  DFN model.  The added numerical  value of  parameters are
summaried in table xx. Lots The  layer optimisation framework and assumptions is
described next. % table: pouch length, width, tab area, stack thickness etc.

\section{Layer Optimisation Framework}

Basically the  pack configuration and the  universe in which we  worked shall be
discussed here with  the help of the  hierarchical powertrain-to-cell schematic.
This will set the context of  the problem being tackled. Important concepts that
enable the creation of the universe in  which we define the problem to be solved
in discussed and the assumptions involved shall be explained.

The layer optimisation  framework hinges upon the concept  of capacity balancing
between electrodes.  Explain this  nicely with references  and citations  on why
this is important. Helps us make us  of the most of the active material invested
into the cell.


\subsection{Capacity Balancing}
Formula for capacity  balance. Show how the length of  the negative electrode is
slightly larger than the positive electrode,  but compensated for by the reduced
volume fraction.  Cite other  parameter sets  where this  is observed.  They are
nearly equal, but made to be 1.10

Note:  This  1.1 ratio  was  fixed  by Krishna.  I  can  explain in  person  how
this happened  (reason: numerical  convergence). The  thickness of  the positive
electrode was adjusted accordingly. If Ian is asked this question and details of
its origin, Krishna is sure he cannot explain it.

\subsection{Electrode Thicknesses per layer}
Talk briefly about separator and how its thickness remains constant.

With this,  show the  mathematical derivation  of the  expression for  number of
layers Explain  the significance of the  outermost layer and how  it affects the
formula used Explain how the stack  is formed. Report its numerical value. Pouch
thickness and  its role  along with  suitable reference.  Show briefly  a clever
computer code snippet that encapsulates both cases of odd and even.

At  this stage,  can explain  clearly the  relationship between  the theoretical
capacity  and  number of  layers.  Use  a graph  or  a  table to  highlight  the
idea.  Useable  capacity shall  be  lower  than  total  capacity due  to  cutoff
considerations. Explanation.

Note:  Again,  Krishna   explained  the  layer  calculation   formulae  to  Ian.
Especially, the derivation of the  combined electrode thickness concept, and the
usage of  the ratio  to obtain  one from  the other.  Ian had  to write  it down
multiple times  drawing the layers and  verify that my computation  was correct.
Krishna  explained  how  the  cases  environment can  be  used  for  theoretical
description while the computer code for both  the cases uses ceil and floor. Ian
naively used  the mathematical symbol for  ceil, until Krishna explained  to him
about the  cases environment. Krishna  also typed up these  equations (basically
any equation that required a derivation) in the paper manuscript.

\subsection{Derivation of analytical \protect{$n_\text{max}$}}

The  search space  spans  a  finite number  of  layers.  An initial  alternative
considered was MIDACO. The detailed equations  and how they are derived go here.
Discuss for both odd and even cases. However, closed form analytical expressions
are now available. This  is so as to enable a binary search  as discussed in the
next section.

The optimisation formula and analytical solution shall also be discussed here.

Note: Krishna is claiming the idea  generation and coding in entirety. The mixed
integer optimisation to zero thickness was some fancy coding. The zero thickness
idea was  also quite  fancy. Krishna  came up  with the  whole of  this section.
Initiated the  idea of narrowing  down the search  window. Ian was  content with
doing a for loop since it will also  work. The optimal layer in this case is the
earliest choice of~$n$ in the for  loop for which the termination conditions are
correctly satisfied.  Krishna was always  in favour of an  optimisation approach
from the beginning.

\subsection{Customised Binary Search}

A  bi-section  based algorithm.  Algorithm/binary  tree  based description  with
figure/algo. Discuss the O(log n) speedup.

Note: One fine evening, Krishna increased the computational speedup by two orders of magnitude. Proof in github and email. While Ian was always interested in a for-loop approach right from beginning, Krishna always suggested to use an optimisation approach.

\subsection{Mass recomputation as a function of layers}
Explanation with graph.

\subsection{Specific-heat recomputation as a function of layers}
Suitable explanation and plot.

Note: Ian  congratulated Krishna for mass  and specifc heat computation.  In his
words,  ``You were  clever coming  up with  a scheme  wherein mass  varies as  a
number  of  layers''.  Packaged  this  up  as  a  function  and  all.  Refer  to
computelumpedmassandCpavgforgivenlayerfcn. Can prove git  history for this file.
Not  only mass  recomputations but  also mass  initial computation  was done  by
Krishna

\subsection{Thermal space permutation of layers.}
Explanation of how the four corner temperatures are tried here. 5 lines of code snippet that packs a punch

Note: Krishna read the acceleration specification description,educated Ian and implemented this.

\begin{quotation}
Ian: Krishna!. Whoa that small block of code does so much.
\end{quotation}

\section{Workflow for Optimal Layer Computation}

This section  discusses the  actual methodology  or the  procedure in  which the
aforementioned ideas are incorporated in  a process-like workflow. Introduce the
full-page crazy flow  diagram here. Explain how the flow  diagram goes through a
methodical  approach in  arriving at  the  optimal number  of layers.  Extensive
forward  and  backward referencing  to  sections  discussing each  modular  idea
encountered  in  the flow  path.  Discuss  how  the  applied current  and  power
densities  change due  to  change  in overall  surface  area  while the  applied
external current/power remains the same.

% This will help us to apply current in units rather in current density.

The salient cases to be covered are the following.

\subsection{Case 1 --- Analysis of Drivecycle Powers}

Although  from  Colorado  Boulder  lecture  notes,  Krishna  already  knew  that
acceleration demands the highest power demand on the cell, it needs to be proven
that drive cycles, which are the  basis for various fuel efficiency calculations
do not represent this case. Ian did the comparative study of various drivecyles.
Some  plots and  charts  showing various  power levels  were  generated by  him.
Krishna does  not explicity  need to use  this section, and  is optional  to the
story, other than the sake of well-roundedness and demonstrating thoroughness of
the study.

Explain that this case is not present in the flow diagram.

However,  it  was  Krishna  who  generated the  drivecycle  speed  vs  time  and
acceleration versus time results alone and  then informed Ian afterwards that it
is now available in the repository. If Ian uses this section, an acknowledgement
will be appropriate.


\subsection{Case 2 --- Acceleration from standstill}

Explanation and computations  based on standard vehicle  dynamics. Lecture notes
and videos given to Ian by Krishna  from Univ of Colorado Boulder The sole value
addition in this  case is the specification to adhere  to and its implementation
through  a cases  study. Krishna  shall explain  this through  either-or if-then
case, explanation along with a short code snippet which he implemented alone.

Explain the  acceleration run in detail,  what it entails etc.  Walk through the
flow diagram till acc layer results are discussed.

Note: The acceleration specification for  electrified transport was unearthed by
Krishna  (all the  interlibrary  loan  stuff and  educated  Ian  about it).  Two
passengers and all that thing. Complete coding.

\subsection{Case 3 --- Fast Charging}

A brief  explanation of the whys  and the hows and  implications. Explanation of
prevalent standards. Present table of standards etc etc

One paragraph  review of control algorithms  and how this algorithm  was chosen.
Brief  overview of  algorithm. The  idea  of introducing  saturation and  pulsed
charging profile. Based  on patent at Auburn university.  Krishna did literature
review.  Flag  introduced   in  code.  Code  snippet.enablecsnegsaturationlimit.
% Sinusoidal excitation charging.

Explanation of why power demand is important (based on charger power electronics).
Finally, walk through of the fast charging section of the schematic.

Note: Krishna  performed the litt  search of  this section entirely  (proof with
timestamp available). However, Krishna is  tentatively not using a detailed litt
review.  In kind  consideration of  Ian's  own overarching  thesis topic,  which
Krishna understands to be something pertinent  to fast charging, Krishna can let
Ian use the literature provided proper attribution  is in place. For the sake of
completion, the  MIDACO based approaching  to fast charging showing  the pulsing
power is also possible by Ian (just a suggestion).

Note: The charger  power electronics limit is again my  contribution from an EEE
background

% \item Review of model-based fast charging control algorithms (how does this go into litt review)

\section{Simulation Environment}
Discuss parameters of  simulation environment. Discuss with the help  of a table
for the  BEV case only. Ian  can take the PHEV  Number of BEV calls  designed to
make up the series string.

Krishna  also came  up with  the cell's  cutoff study  with respect  to the  the
system's  bus bar  voltage, and  will  be cited  here with  examples and  strong
backing. There  are a few  preconditioning steps needed  to amend the  P2D model
before  numerical implementation  of the  flow  diagram is  possible. 1.  Fixing
aspects of  the code  in LIONSIMBA v1.023  used as the  baseline by  this thesis
author. 2.  Addition of the capability  to apply power input  instead of current
input as is common in traditional p2d simulations.

Note:  All crucial  parameters  were  sourced by  Krishna  through a  literature
review. Especially, the  vehicular parameters like classis  mass, coefficient of
drag, base speed. Krishna also had to educate Ian about base speed. If an on the
spot quiz is conducted  at a conceptual level, this truth  can easily be deduced
by the judging authority. Krishna claims the literature review of this topic and
is happy to let Ian use this with proper attribution.

\subsection{Preconditioning of Computer Code}

Briefly allude to the stoichiometries that was introduced into the computer code
through capacity characterisation  simulation.Now possible to start  at any SoC.
The other salient  amendments and enhancements to the software  toolbox is shown
below. Released as 2.0 and link to LIONSIMBA github repo (not BOLD toolbox repo)

\subsubsection{Re-parameterisation}

Discuss  the  extensive Reparameterisation  undertaken  by  Krishna by  studying
relevant literature.  Discuss the  changed parameters  with respect  to Northrop
cell and why. Maybe show a table. This is quite substantial.

Conductivity/diffusivity changes.  Show how the isothermal  and thermal variants
in existing Northrop cell were bogus with plots.

Performed  comprehensive   literature  review   to  replace   the  dubious/bogus
parameters of the electrode and electrolyte specific heat capacities email proof
PS:  Every thermal/material  property of  the Al/Cu  current collectors  is very
clear and have been traced out, hand-calculated and validated.

\subsubsection{Deterministic initialisation  of algebraic variables}

Replaced silly fsolve. Numerical explanation here.

\subsubsection{Linear interpolation for field variables}

Explain how  linear interpolation was performed  for phis and phie  at the edges
of  control volumes  in  electrode  and separator.  Draw  a  schematic for  this
explanation.

\subsubsection*{Convergence Analysis of Computation Mesh}
Report  only   if  Krishna  has   sufficient  time  left.   Explain  simplifying
assumptions.  Refer to  the  table etc  Show  plot of  how  terminal voltage  is
converging for chosen mesh as a function of number of nodes. Another plot of how
simulation end time converges as a function of number of nodes. Lots of analysis
by changing the number of nodes within  each mesh to justify the validity of the
results. Prove that mesh independence is reached.

\subsection{Hybrid FV--Spectral Scheme}

Completely numerical section.  Highly mathemetical. Explain how  the rational to
decimal  truncation of  the unbalanced  twelfth order  finite difference  scheme
messes  up numerical  conditioning.  Show matrices  and  discuss the  drawbacks.
Fornberg matrix did not help. So, spectral scheme was needed.

Background  study,   explanation,  analysis,  literature  review   and  equation
derivation  of spectral  scheme Complete  contribution of  solid-phase diffusion
with spectral methods. Reading textbook, understanding concept, investigation of
applicability,  hunting relevant  literature,  text-writing, hand-derivation  of
equations. This complete section was done by Krishna.

\subsection{Lumped Thermal Model}

Basic discussion of lumped thermal model  and justify through citations why here
it  might  be  sufficient  for  this  application.  Not  originally  present  in
LIONSIMBA. Present  the thermal model here  equations here. Discuss what  cp avg
means and how it is computed as a here function of number of layers. Explanation
of how tab cooling helps here.

Note: Cp  avg was calculated and  coded by Krishna,  as a function of  number of
layers. I shall leave the detailed thermal model to Ian, especially since a biot
analysis was performed by him. Anyway,  suddenly discussing the thermal model at
depth does not fit the story of my  thesis. My hint to Ian would be to emphasise
the thermal model  at depth, since anyway  he seems to be confident  in the biot
analysis. Specifically  the value of  heat transfer coefficient  was empirically
chosen  by Ian  through simulations.  So,  I will  let him  explain that  stuff.
However, tab area idea computation using  twice the Bolt's tab area was proposed
by Krishna,  but overall this thermal  stuff, Krishna is willing  to bequeath to
Ian  since  the  whole thing  doesn't  fit  the  story  of Krishna's  work.  The
polarisation heat  concept was initially described  by Greg, but anyway  let Ian
explain it, no problem.  Ian may wish to discuss entropic  heat generation and a
lot of  other things we investigated  together, but I  am not going to  dwell on
them in the thesis.


\subsection{Power Input Boundary Conditions}

Discuss why power input is needed for layer optimisation. Explain how it is done
currently in state of the art case.

Krishna educated Ian about Pletts existing  work. Accompanied Ian to the library
and told him to check out that book which I had ordered.

Krishna  claims a  very important  thing here.  A critical  argument on  how the
present schemes do not fit into the layer opt methodology.

Detailed mathematical derivation. Both Ian  and Krishna acknowledge each other's
help in shared derivation and also acknowledge Davide appropriately.

Note: Ian might  wish to report the two intermediate  steps before this solution
was obtained. The  2nd of the 3  approaches was matching current  and voltage at
discrete intervals through  numerical integration. This was  our third approach.
The first approach was too simplistic. So, Ian might wish to skip it.



\section{Layer Optimisation Results}

Show the table  only with the extra parameters not  present in isothermal model.
In particular,  I can  remember the  thermal parameters  of the  two electrodes,
separator, pouch, current collectors and electrolyte.

Present the results in a horribly bland  table format steering well clear of the
heatmap. Krishna's view is that the results do not stand alone by themselves and
if you plonk  a layer choice from  the heatmap/table into your  cell, things are
not expected  to work as is.  The numbers herein are  the result of quite  a few
assumptions and hold validity only within the universe in which it was created.

However, the framework itself is transferable and is the most valuable component
of the  work. A user  reading this thesis can  easily substitute their  own cell
parameters and system-level considerations into the code and obtain results from
it accordingly.  This was the  working premise of the  paper as well,  until Ian
decided to write up a bloated explanatory section.

Plots of acceleration and fast charging for successful layer count.

For PHEV, although common module design will be alluded to, it will not be dwelled on or explained or analysed in depth.


Note: In Ian's thesis,  Krishna seeks attribution for the idea  to use a heatmap
since it  arose out of  Krishna's heavy  criticism, bordering on  the offensive,
about  continuously  complaining that  our  group's  plots/charts/graphs do  not
exhibit any ``interesting'' twists and  turns and fancy illustrations. Apologies
for this. Anyway, the heatmap idea was  my suggestion. However, I shall not take
an inch of credit for the implementation as it was entirely Ian's work in coming
up with fancy schemes.

Note:  Having  obtained  BEV results,  Krishna  also  set  up  the full  set  of
assumptions  for  the  PHEV  simulations too  before  leaving  actual  numerical
simulations  to Ian  (and on  vacation  to India  followed by  Konstanz). So  an
acknowledgement is  required if Ian  choses to  list these assumptions.  Ian can
discuss monotonicity issues in PHEV simulation that he found out about.

\section{Appendix}
Intend to provide a call-graph in the appendix


