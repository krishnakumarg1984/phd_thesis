 % -*- root: ../main.tex -*-
% !TEX root = ../main.tex
% this file is called up by main.tex
% content in this file will be fed into the main document
% vim:textwidth=80 fo=cqt

% \setstretch{1.348361657291667} % golden-ratio stretch (1.2 x 1.348 = 1.618)
\setstretch{1.16}
\chapter*{Acknowledgements}
\vspace*{-12mm}

Obtaining a PhD has long been a dream of mine. This dream could not have come to
fruition without the support of many of my well-wishers to whom I shall remain
eternally grateful. Here, I would like to acknowledge their invaluable support
in aiding this achievement.

I wish to incorporate \emph{Sahadharmiṇī} (companion soul), \emph{Bandhu}
(relatives) and \emph{Mitram} (friends) into the classical quartet of support
pillars --- \emph{Matah} (mother), \emph{Pitah} (father), \emph{Guru} (teacher),
\emph{Deivam} (God) --- commonly attributed as reasons for one's success in
ancient Indian scriptures such as the Vedas, Shastras, Charitras,  Itihasas and
Purāṇas. Whilst making this personal amendment, I would like to de-emphasise any
connotations of hierarchy amongst these support pillars, since they all play
equally important roles in one's life.

During these four years, the level of support accorded by my wife Parvathy
Chittur Subramanianprasad cannot be described in mere words. She shouldered the
entire burden of responsibility in the family so that I could focus solely on my
research. At the prime of her youth, she sacrificed so many material comforts
and happily lived a relatively austere life by humbly accepting the financial
constraints that are part and parcel of a student's life. In the same vein, I
owe a lot to my parents. It has been ten long years since I departed from Indian
shores in pursuit of higher studies. During all these years, despite ill-health
and missing my presence at home, they have always urged me to carry on and
complete my studies. Many a time, I have felt incomplete in my failings to
fulfil basic duties of a son towards his parents, the least of which is spending
quality time with them. However, my parents have always brushed aside their
physical and financial difficulties to put my success above rest.

% \addlines
It is needless to explain the crucial role of academic supervisors in the life
of a doctoral candidate. I was extremely lucky to be bestowed with not just one,
but two amazing supervisors here at Imperial College London --- \mbox{Dr.\
Gregory J.\ Offer} and \mbox{Dr.\ Monica Marinescu}. I have immensely benefited
from their supervision, both from technical and personal perspectives. Even
during moments of sheer despair, my supervisors had immense faith that I shall
pull through. Their unflinching support contributed in no small measure for this
success of mine. I also wish to acknowledge \mbox{Dr.\ Davide M.\ Raimondo} who
served as my unofficial supervisor, allocating months of his valuable time for
my research, many times even well outside regular working hours. I also wish to
express my gratitude to \mbox{Dr.\ Gregory L.\ Plett} who was always glad to offer
his technical expertise, opinion and guidance on all of my research ideas. I am
thankful to \mbox{Dr. Teng Zhang} who mentored me in my initial days here and
without his feedback, I could not have published my first journal article.

Although my interaction with them has sadly been on the wane in recent years, I
am lucky to be a recipient of the blessings of my thatha \mbox{(Shri.\ Hariharan
S)} and patti \mbox{(Smt.\ Parvathi Hariharan)}, my affectionate grandparents,
who have lived a life of humility, adhering to time-honoured traditions whilst
seeming to have an endless reserve of love to bestow. Although I regret not
having had a deeper relationship with him, I am immensely fortunate  to have 
benedictions of the noble soul and towering figure, Acharyan \mbox{Shri C R
Krishna Iyer}, affectionately known as thathanna, who has recently attained the
lotus feet of Lord Guruvayoorappa. Though I have not yet attempted to imbibe the
spiritual hints bequeathed by him, I hope to be a worthy beneficiary of its
legacy with the passage of time.

My father-in-law, \mbox{Dr.\ Subramaniaprasad C K} or CKSP/Chithappa, as he is
fondly known in our family circle, has been a father-like figure to me. From
procuring transcripts at my alma-mater at the start of my PhD journey to
providing regular feedbacks by proof-reading this thesis, he has been a strong
driving force behind my success.

Members of my extended family have long been staunch supporters of my
educational aspirations. I vividly recollect the day I offered
\emph{namaskaarams} to Kaimal Chithappa before my first journey outside India
prior to commencing Masters' studies at Virginia Tech. \mbox{Shri.\ Keshava
Kaimal} has been a strong motivator and was one of the earliest in my family to
instigate a passion for engineering in me. His erudite teachings to me, such as
as the translation of  ``\emph{Agnayeh, ithanna mamah} (Oh Agni! This is not for
me, this is for the society)'' still rings in my ears and I shall strive to be
worthy of this. 

I wish to express my thanks to my sister, \mbox{Radhika Balaji} for constantly
cheering me up. Not for a moment has her belief in me wavered, and I am relieved
that I did not let her down. I would like to express my gratitude to Smt.~Usha
Prasad, whose calmness and compassion were a solace during difficult personal
times and to Smt.~Radha Ramaswamy, whose enthusiasm has been a  buoyant
motivator. I wish to thank Balaji Anna, Shri.~Ramaswamy Krishnan Chittur
(Ramesh) and Dr.~Sandeep Sangameswaran, all of whom have been more than what
even brothers could be and have always lent their support across all aspects of
my life.


% \addlines[1]

Despite not having interacted much with her, the abilities of our beloved Manni
Ammal from Chittur remind me that one should not be carried away by one's
achievements. In India, a female attending school in the pre-independence era,
particularly in a rural Kerala village, is virtually unheard of. Not only did
Manni top her high-school grades, she can reel off the Nobel-winning
\emph{Gitanjali} in fluent English, despite being hampered by the age-related
afflictions of a nonagenarian. I would like to express my sincere gratitude to
my uncles and aunts (Shri.~Jayakumar, Smt.~Lalitha, Shri.~Nandakumar, Smt.~Asha,
Shri.~Kaimal, Smt.Ranjini, Shri.~Dwarakanath and Smt.~Brinda) and to all my
cousins who used to note with pride that I was the first in the family to
pursue higher studies outside India, and now the first one to complete a
doctoral degree. During  difficult times in the PhD, I am glad to have had the
cheerful support of Smt.~Remya and Smt.~Shyama, whose light-hearted approaches
to many issues is certainly worth learning from. I am thankful to Athais and
Athimber for their constant support and encouragement. During times of despair,
the next generation in the immediate family --- Arjith, Siddharth, Aditya and
little Usha --- provided a respite from the monotonicity of doctoral studies
through their carefree joy and innocence.


On a technical front outside my specific research area, I would like to
acknowledge the thousands of individuals who have coded for open-source projects
for free. It is due to their efforts that much of the computational research
work in the modern era is possible. In particular, I wish to express my
admiration of Dr.~Donald Knuth and Dr.~Leslie Lamport for giving the world, the
\TeX{} and \LaTeX{} typesetting systems respectively. I am also thankful to Bill
Joy and Bram Moolenaar for developing the \texttt{vi} and \texttt{vim} text
editors, whose immense power eased my pain during coding and thesis writing
sessions. I am thankful to the senior contributors (David Carlisle, Ulrike
Fischer, Nicola Talbot, Jonathan Spratte to just name a few) of the TeX forum in
the StackExchange family of portals, for their personalised support and patient
answers to my rather dull questions.

\addlines[2]

I would like to acknowledge members of my research group (Alexander Holland,
Wasim Sarwar, Yan Zhao, Mei-Chin Pang, Ian Campbell, Yuri Merla, Ian Hunt, Emma
Vendola, Oisin Shaw to name a few) for their companionship during my studies
here. Although far and few between, the conversations in the coffee room with my
group members as well as those with other PhD students here at Imperial (Xianyan
Zhou, Marco Da Costa Alves, Joel Henry, Manikandan Ganapathy, James Tebbutt and
many others), had always lent the shared comfort of being fellow travellers on
the same journey. I am deeply touched by the gesture of Mei-Chin Pang, who
offered prayers to the divine force to give me strength and endurance in the
hours leading up to my thesis submission. I am thankful to her for being a good
friend with whom I could share technical as well personal challenges. I would
like to thank Vivek, Krishnan, Karthik, Abhilash, Ram, Lakshmi, Anand and
Praveen for the camaraderie and time-tested friendship whose memories motivated
me to carry on the  battle, even in the bleakest of times.

Last, but not the least, I would like to thank God Almighty --- the universal
driving force well beyond specific forms of personification by humans as Krishna or
Jesus or Allah amid others --- for enabling me to complete this
doctorate degree. \emph{Tathastu!} (so let it be).

