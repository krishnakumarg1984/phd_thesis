% -*- root: ../main.tex -*-
%!TEX root = ../main.tex
% vim:textwidth=80 fo=cqt

\chapter*{Abstract\hfill}
\addcontentsline{toc}{chapter}{Abstract}

\vspace*{-1cm}

The  realisation of  \glspl{pbm}  of lithium  ion cells  in  the \glspl{bms}  of
electric  vehicles is  studied through  a three-pronged  strategy ---  analysis,
design  and  implementation.   The  survey  of  literature   undertaken  in  the
backdrop of  this broad landscape  reveals a  dearth of model-based  designs for
automotive-grade  pouch  cells,  which  is therefore  addressed  in  this  work.
Perusal of  prior art in  reduced-order battery modelling provides  key guidance
on  topics meriting  further  investigation \viz~the  \gls{dra}  scheme and  the
electrolyte-enhanced \gls{spm} which are  therefore carefully analysed here from
the perspective of  their embedded implementation. Owing to  its familiarity and
wide-spread popularity among relevant stakeholders, the \gls{p2d} implementation
of  the \gls{dfn}  model  is used  as the  \gls{pbm}  underpinning all  research
presented herein.

A  computational framework  to  optimise the  number  of electrochemical  layers
within a pouch  cell is developed. The chosen  optimality criterion specifically
addresses the  two most pertinent  issues that currently hinder  the mass-market
adoption of electric vehicles --- range anxiety and fast charging. Driven by the
need  for  a balanced  capacity  loading  at  both electrodes,  a  deterministic
criterion for  computation of  electrode thicknesses  is formulated.  The search
space of layer choices across all  thermal scenarios is traversed with the least
operation  count through  a novel  application of  the binary  search algorithm.
Numerical  simulations of  a lumped  thermal  model coupled  with the  \gls{p2d}
electrochemical  model in  conjunction with  judiciously chosen  exit conditions
help  to inform  the  number of  layers  needed to  maximise  the cell's  usable
capacity  whilst  simultaneously  satisfying  the  power  requirements  of  fast
charging.  The \gls{p2d}  model is  reformulated to  accord it  with the  innate
ability to  accept power  inputs. The model-led  optimal layer  design procedure
thus developed  is plating-aware, facilitating  the extension of  pack lifetimes
whilst helping to bypass expensive empirical prototyping.

Owing to its simplicity, the \gls{spm} family of models is deemed to be the most
promising  \gls{rom} candidate  that  can usher  in the  use  of \glspl{pbm}  in
electric vehicles.  An in-depth  analysis of the  \gls{spm} reveals  an inherent
mismatch  between  the accuracies  of  its  voltage and  \gls{soc}  predictions,
thereby rendering it unsuitable as the  plant model in state-estimation tasks. A
comprehensive evaluation  of the  salient electrolyte-enhanced  \glspl{spm} from
literature reveals that most solutions  are either mathematically intractable or
overly  simplistic. For  the ionic  concentration in  the electrolyte,  analysis
of  the quadratic  approximation  model, which  straddles  the boundary  between
computational complexity and mathematical  tractability, reveals a poor temporal
performance particularly  at the  current collector  interfaces. However,  it is
capable of  delivering acceptable  levels of accuracy  in computing  the spatial
profile of ionic concentration. Application  of the \gls{mggp} technique exposes
that the  causal factor for this  mediocre temporal performance is  the equation
deficiency of the underlying \gls{p2d} model.

From an implementation perspective, the discrete-time formulation of \glspl{spm}
is presented  by using the matrix  exponential approach and its  advantages over
its continuous-time counterparts enumerated.  Despite its inherent shortcomings,
it  is  deemed  that  operating  within the  confines  of  the  well-established
foundations of  the \gls{p2d} dynamics  represents a definitive step  forward in
bringing  into fruition  the goal  of incorporating  \glspl{pbm} into  vehicular
\glspl{bms}. Therefore,  the existing quadratic approximation  model is retained
for  the   electrolyte  spatial  concentration,  whilst   advocating  the  novel
application  of  a  system  identification method  for  its  temporal  dynamics.
After  establishing  linearity  and  time-invariance  of  the  subsystems  under
consideration, discrete-time transfer functions of  the number of moles per unit
area of  lithium ions in each  electrode region is identified  for the pertinent
range of applied  currents. The identified transfer functions  are then employed
in a  composite \gls{spm} which  demonstrates superior accuracy compared  to the
incumbent state of the art electrolyte-enhanced \gls{spm}, thereby demonstrating
a  substantial accomplishment  from  an implementation  viewpoint. Although  the
advancements herein are reported for an isothermal implementation of the models,
future  enhancement through  thermally coupled  model derivations  is advocated.
Finally,  the  importance of  parametrisation  of  the underlying  \gls{pbm}  is
acknowledged as a  crucial unsolved aspect which needs the  collective effort of
the battery research community.






