% !TEX-program = lualatex
% vim:nospell
\documentclass{article}
\usepackage[T1]{fontenc}
\usepackage{xcolor}
\definecolor{imperiallightgray}{RGB}{235,238,238}
\definecolor{imperialcoolgray}{RGB}{157,157,157}
\usepackage{algorithm}
\usepackage[noend]{algpseudocode}

%%%%%%%%%%%%%%%%%%%%%%%%%%%%%%%%%%%%%%%%%%%%%%%%%%%%%%%%%%%%%%%%%%%%%%%%%%%%%%%%
% https://tex.stackexchange.com/questions/292815/how-can-i-create-vertical-lines-indentation-in-algorithm-pseudo-code-correctly-w?noredirect=1&lq=1
\usepackage{etoolbox}

\makeatletter
% start with some helper code
% This is the vertical rule that is inserted
\newcommand*{\algrule}[1][\algorithmicindent]{%
    \makebox[#1][l]{%
        \hspace*{.2em}% <------------- This is where the rule starts from
        \color{imperialcoolgray} \vrule height .75\baselineskip depth .25\baselineskip
        % \vrule height .75\baselineskip depth .3\baselineskip % https://tex.stackexchange.com/questions/301462/why-are-vertical-rules-dashed-sometimes-with-algorithmic-package
    }
}

\newcount\ALG@printindent@tempcnta
\def\ALG@printindent{%
    \ifnum \theALG@nested>0% is there anything to print
        \ifx\ALG@text\ALG@x@notext% is this an end group without any text?
            % do nothing
    \else
        \unskip
        % draw a rule for each indent level
        \ALG@printindent@tempcnta=1
        \loop
        \algrule[\csname ALG@ind@\the\ALG@printindent@tempcnta\endcsname]%
        \advance \ALG@printindent@tempcnta 1
        \ifnum \ALG@printindent@tempcnta<\numexpr\theALG@nested+1\relax
            \repeat
        \fi
    \fi
}
% the following line injects our new indent handling code in place of the default spacing
\patchcmd{\ALG@doentity}{\noindent\hskip\ALG@tlm}{\ALG@printindent}{}{\errmessage{failed to patch}}
\patchcmd{\ALG@doentity}{\item[]\nointerlineskip}{}{}{} % no spurious vertical space
% end vertical rule patch for algorithmicx
\makeatother

% https://tex.stackexchange.com/questions/292815/how-can-i-create-vertical-lines-indentation-in-algorithm-pseudo-code-correctly-w?noredirect=1&lq=1
%%%%%%%%%%%%%%%%%%%%%%%%%%%%%%%%%%%%%%%%%%%%%%%%%%%%%%%%%%%%%%%%%%%%%%%%%%%%%%%%

\newcommand*\Let[2]{\State #1 $\gets$ #2}

\begin{document}

\begin{figure}[!htbp]
    \begin{algorithmic}[1]

        \Procedure{sum}{ $\{x\}$}

        \State $y\gets0$
        \For{$i \gets 1 : N^{x}$} \Comment{Time series $\{x\}$ has length $N^{x}$}
        \State $y\gets y+x(i)$ \Comment{Summing up.}
        \EndFor

        \State \textbf{return}  $y$
        \EndProcedure
    \end{algorithmic}
    \caption[Implementation of a algorithm for calculating a sum.]{Implementation of a algorithm for calculating a sum.}
    \label{fig:algorithm1}
\end{figure}

\begin{algorithm}
    \caption{Euclid’s algorithm}\label{euclid}
    \begin{algorithmic}[1]
        \Procedure{euclid}{$a,b$}\Comment{The g.c.d. of a and b}
        \State $r\gets a\bmod b$
        \While{$r\ne 0$}\Comment{We have the answer if r is 0}
        \State $a\gets b$
        \State $b\gets r$
        \State $r\gets a\bmod b$
        \EndWhile\label{euclidendwhile}
        \State \textbf{return} $b$\Comment{The gcd is b}
        \EndProcedure
    \end{algorithmic}
\end{algorithm}

\end{document}

% https://tex.stackexchange.com/questions/414135/how-to-reproduce-this-pseudocode?noredirect=1&lq=1
