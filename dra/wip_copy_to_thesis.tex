\section{Conclusions\label{sec:Conclusion}}  In   this  paper,   Singular  Value
Decomposition of a  large Block-Hankel matrix is identified as  a key bottleneck
in  the  classical  \gls{dra}  for   reduced-order  Li-ion  cell  modelling.  An
improved  \gls{svd} scheme  is presented,  which  employs a  combination of  the
Golyandina-Usevich  and Lanczos  algorithms.  The results  discussed in  Section
\ref{sec:Results} demonstrate  the performance  improvement achieved by  the new
method  without trading-off  model  fidelity.  At a  single  operating point  of
\gls{soc} and temperature,  for a Hankel block size of  8000, \gls{rom} workflow
incorporating  the improved  \gls{dra} is  approximately 100  times faster  than
that  employing classical  DRA. Using  the machine  specifications in  Appenndix
\ref{sec:Specifications-of-Workstation}, for 100  operating points (combinations
of 10 \gls{soc} and temperature values), computing the \gls{rom} requires only 6
hours using  the improved \gls{dra}  , whereas the classical  \gls{dra} consumes
666  hours  (27  days).  Furthermore,  for the  same  block-size,  the  improved
\gls{dra}  is  demonstrated  to  be  superior in  terms  of  memory  efficiency,
drastically reducing the  memory requirement from 112 GB down  to 2 GB. Finally,
the improved \gls{dra} demonstrates superior modelling accuracy when implemented
even in moderately equipped computing environments such as laptops.

%%%%%%%%%%%%%%%%%%%%%%%%%%%%%%%%%%%%%%%%%%%%%%%%%%%%%%%%%%%%%%%%%%%%%%
\begin{acknowledgment}
    Financial support for the research reported in this paper has been
    obtained through the Imperial College President's PhD Scholarships
    scheme. The sponsor had no role whatsoever in collection, analysis
    and interpretation of data, or in writing of the manuscript. Furthermore,
    the funding body has no role/involvement in the decision to submit
    the article for publication. The authors wish to acknowledge the support
    of The Department of Mathematics, Imperial College London for usage
    of the departmental computing cluster. The \gls{cpu} times, memory usage
    and all other computational results reported in this paper were obtained
    by using a computing node from this facility.
\end{acknowledgment}

\begin{nomenclature}
	\entry{$c_e \scriptstyle(x,t)$}{Concentration of Li$^\text{+}$ ions in the electrolyte at each spatial location within the 1-D cell geometry $(\text{mol m}^{-\text{3}})$}
	\entry{$c_{s,e} \scriptstyle(z,t) $}{Concentration of Li at the surface of each solid particle within the normalized domain length of each electrode $(\text{mol m}^{-\text{3}})$}
	\entry{$\medmuskip=0mu \tilde{c}_{\scriptscriptstyle s,e_{pos}}^*\scriptstyle(0,t) $}{Surface concentration of Li in the solid particle adjacent to positive current collector, obtained after model linearisation and subsequent removal of the integrator pole. The algorithms discussed in this paper require that all model variables have poles located strictly within the open left-half complex plane. Since the solid diffusion transfer functions have poles at the origin, it is necessary to remove this integrator pole before deriving the model $(\text{mol m}^{-\text{3}})$}
	\entry{$L_{neg}$}{Thickness of the negative electrode $(\text{m})$}
	\entry{$L_{sep}$}{Thickness of the separator domain $(\text{m})$}
	\entry{$L_{pos}$}{Thickness of the positive electrode $(\text{m})$}
	\entry{$j \scriptstyle(z,t) $}{Li molar flux density at electrode-electrolye interface of each particle within the normalized electrode domain $(\text{mol m}^{-\text{2}}s^{-\text{1}})$}
	\entry{$\phi_e \scriptstyle(x,t) $}{Electrolyte potential at each spatial location within the 1-D cell geometry $(\text{V})$}
	\entry{$\phi_{s,e} \scriptstyle(z,t) $}{Solid-electrolyte potential difference at the inteerfacial boundary for each spatial location within the 1-D cell geometry $(\text{V})$}
\end{nomenclature}
%%%%%%%%%%%%%%%%%%%%%%%%%%%%%%%%%%%%%%%%%%%%%%%%%%%%%%%%%%%%%%%%%%%%%%
\bibliographystyle{asmems4}
\bibliography{\gls{svd}_paper_Bibliography}
\newpage
%%%%%%%%%%%%%%%%%%%%%%%%%%%%%%%%%%%%%%%%%%%%%%%%%%%%%%%%%%%%%%%%%%%%%%
\appendix %%% starting appendix
\section{Specifications of Workstation Used\label{sec:Specifications-of-Workstation}}

\newpage
\section*{Listing of Table Captions}

\begin{description}
	\item[Table 1]   Parameters for \gls{rom} Computation\\
	\item[Table 2]   Salient Results - Classical vs. Improved \gls{dra} \\
	\item[Table 3]   Specifications of workstation used
\end{description}
\newpage
\section*{Listing of Figure Captions}

\begin{description}
	\item[Fig. 1.]   Reduced-order modelling (ROM) workflow using classical \gls{dra} .\\ (The shaded blocks represent computational bottlenecks).\\
	\item[Fig. 2.]   Time evolution of Markov parameters of pole-removed transfer function corresponding to surface concentration of Li in the solid particle adjacent to positive current collector.\\
	\item[Fig. 3.]   Reduced Order Modelling (ROM) Workflow using improved \gls{dra} .\\
	\item[Fig. 4.]   Comparison of singular values computed by the conventional and improved \gls{svd} methods.\\
	\item[Fig. 5.]	 Memory usage of classical and improved \gls{dra} . Overall \gls{ram} usage as well as \gls{ram} used only for \gls{svd} computation is illustrated.\\
	\item[Fig. 6.] 	 Computation times for classical and improved \gls{dra} schemes.\\
	\item[Fig. 7.]	 Comparison of singular values computed by conventional and improved \gls{svd} methods under a practical \gls{ram} limit of 10 GB.\\
	\item[Fig. 8.]	 Time-domain simulation depicting solid surface concentrations at the boundary of positive electrode and separator.\\
\end{description}



% achieve the latent  potential and wide applicability  of the
% physics-based reduced order Li-ion battery model.

