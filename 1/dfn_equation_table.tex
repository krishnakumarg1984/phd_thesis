% -*- root: ../main.tex -*-
%!TEX root = ../main.tex
% this file is called up by main.tex
% content in this file will be fed into the main document
% vim:nospell textwidth=180 foldlevelstart=3 foldlevel=3

% \begin{table}[!htbp]
    \begin{table}[p]
        \centering
        \caption[]{}
        % \label{tbl:charSimspmp2d}
        \begin{threeparttable}
            \begingroup
            \makeatletter\def\f@size{10.0625}\check@mathfonts
            % \addtolength{\jot}{0.75em}
            \addtolength{\jot}{0.875em}
            \begin{tabular*}{\textwidth}{@{} l c r l r @{}}
                \toprule
                Region & Governing equations & \multicolumn{3}{c}{Boundary conditions} \\
                \midrule
                \multicolumn{1}{l |}{\rotatebox[origin=c]{+90}{\makecell{\small Electrodes \\ \small \linnegpos}}} &
                $\begin{aligned} % placement: default is "center", options are "top" and "bottom"
                    \vphantom{\diffp{c_\slsub}{r}{\mathrlap{r = R_\pl}}}
                    \diffp{c_\slsub}{t} &= \frac{D_\slsub}{r^2}\diffp{}{r}\left(r^2 \diffp{c_\slsub}{r} \right) \\
                    \vphantom{\diffp{c_\text{e}}{x}{\mathrlap{x = l_\text{tot}}}}
                    \varepsilon_l \diffp{c_\text{e}}{t} &= \diffp{}{x}\left(D_\effl
                    \diffp{c_\text{e}}{x} \right) + (1 - t^0_\text{+}) a_\slsub j_l \\
                    \vphantom{\diffp{\phi_\text{e}}{x}{\mathrlap{x = 0}}} 0 &= \diffp{}{x}\left(\kappa_\effl \diffp{\phi_\text{e}}{x}\right) + \diffp{}{x}\left(\kappa_\effl \frac{2 R T(t)}{F} (t^0_{+}-1)\diffp{ \ln c_\text{e}}{x}\right) \\
                    \vphantom{\sigma_\effl\!\diffp{\phi_\slsub}{x}{\mathrlap{\substack{\vphantom{\displaystyle M}x = x_\text{pos/sep}\\x = x_\text{neg/sep}}}}} a_\slsub F j_l &= \diffp{}{x}\left(\sigma_\effl \diffp{\phi_\slsub}{x}\right) \\
                    % j_l &= k_\lr c^\alpha_\text{e}\left(c_\slmax - c_\slsurf\right)^\alpha c^\alpha_\slsurf\left( \exp\left(\frac{\left(1-\alpha\right)F}{R T}\eta\right) - \exp\left(-\frac{\alpha F}{R T}\eta\right) \right)
                    j_l &= 2 k_\lr \sqrt{c_\text{e}\left(c_\slmax - c_\slsurf\right) c_\slsurf} \sinh \left(\frac{0.5 F}{R T(t)} \eta_l \right)
                \end{aligned}$ &
                $\begin{aligned}
                    \vphantom{\diffp{c_\slsub}{r}{\mathrlap{r = R_\pl}}} \diffp{c_\slsub}{r}{\mathrlap{r = 0}}\hspace{1mm} &= 0, \\
                    \vphantom{\diffp{c_\text{e}}{x}{\mathrlap{x = l_\text{tot}}}} \diffp{c_\text{e}}{x}{\mathrlap{x = 0}}\hspace{1mm} &= 0, \\
                    \diffp{\phi_\text{e}}{x}{\mathrlap{x = 0}}\hspace{1mm} &= 0, \\
                    \sigma_\effl\!\diffp{\phi_\slsub}{x}{\mathrlap{\substack{\vphantom{\displaystyle M}x = x_\text{pos/sep}\\x = x_\text{neg/sep}}}}\hspace{1mm} &= 0, \\
                    % {}&\textemdash{}{}
                    {}&\xdash[1.25em]{}
                \end{aligned}$ &
                $\begin{aligned}
                    \diffp{c_\slsub}{r}{\mathrlap{r = R_\pl}}\hspace{1mm} &= \frac{-j_l}{D_\slsub} \\
                    \diffp{c_\text{e}}{x}{\mathrlap{x = l_\text{tot}}}\hspace{1mm} &= 0 \\
                \vphantom{\diffp{\phi_\text{e}}{x}{\mathrlap{x = 0}}} \phi_\text{e}\Bigr\rvert_{\mathrlap{x=l_\text{tot}}} \hspace{1mm}&= 0 \\
                % \sigma_\effl\!\diffp{\phi_\slsub}{x}{\mathrlap{\subalign{x&=0\\x&=x_\text{tot}}}} &= \frac{-I}{A} \\
                \vphantom{\sigma_\effl\!\diffp{\phi_\slsub}{x}{\mathrlap{\substack{\vphantom{\displaystyle M}x = x_\text{pos/sep}\\x = x_\text{neg/sep}}}}} \sigma_\effl\!\diffp{\phi_\slsub}{x}{\mathrlap{\substack{\!\!\!\!\!x=0\\x=x_\text{tot}}}}\hspace{1mm} &= \frac{-I}{A} \\
                % \sigma_\effl\!\diffp{\phi_\slsub}{x}{\mathrlap{\substack{\begin{mysubarray} x &=0 \\ x &=x_\text{tot}\end{mysubarray}}}} &= \frac{-I}{A} \\
                % {}&\textemdash{}{}
                {}&\xdash[1.25em]{}
            \end{aligned}$ &
            $\begin{aligned}
                \vphantom{\diffp{c_\slsub}{r}{\mathrlap{r = R_\pl}}}\refstepcounter{equation}(\theequation)\label{eq:dfnsoliddiff} \\
                \vphantom{\diffp{c_\text{e}}{x}{\mathrlap{x = l_\text{tot}}}} \refstepcounter{equation}(\theequation)\label{eq:dfnliquiddiff} \\
                \vphantom{\diffp{\phi_\text{e}}{x}{\mathrlap{x = 0}}} \refstepcounter{equation}(\theequation) \\
                \vphantom{\sigma_\effl\!\diffp{\phi_\slsub}{x}{\mathrlap{\substack{\vphantom{\displaystyle M}x = x_\text{pos/sep}\\x = x_\text{neg/sep}}}}} \refstepcounter{equation}(\theequation) \\
                \vphantom{\left(\frac{0.5 F}{R T(t)} \eta \right)} \refstepcounter{equation}(\theequation)
            \end{aligned}$ \\
            \bottomrule
        \end{tabular*}
        \endgroup
        \begin{minipage}{\textwidth}
            \bigskip
            \begin{flushleft}
                \raggedright
                \makeatletter\def\f@size{14}\check@mathfonts
                $\begin{alignedat}{2}
                    & \text{\textbullet{} } c_\text{e} &  & \coloneqq c_\text{e}(x,t), \quad  \{x \in [0,l_\text{tot}],\, (x=0)\symbol{"2259} \text{\footnotesize pos/Alcc},\, (x=l_\text{tot})\symbol{"2259} \text{\footnotesize neg/Cucc}\}  \tnote{\dagger}                                                                                                                                                                  \\
                    & \text{\textbullet{} } \phi_\text{e} &  & \coloneqq \phi_\text{e}(x,t)\\
                \end{alignedat}$
                \\[0.5em]
                \fbox{\textbf  \linnegpos}
                $\begin{alignedat}{2}
                    % & \mathbf{\textbf \linnegpos} & {} \\
                    & \text{\textbullet{} } c_\slsub   &  & \coloneqq c_\slsub(r,t), \quad  \{r \in [0,R_\pl],\, (r=0)\symbol{"2259} \text{\footnotesize  particle center},\, (r=R_\pl)\symbol{"2259} \text{\footnotesize particle surface}\}              \\
                    & \text{\textbullet{} } c_\slsurf &  & \coloneqq c_\slsub(r=R_\pl,t)\\
                    & \text{\textbullet{} } \phi_\slsub &  & \coloneqq \phi_\slsub(x,t)\\
                    & \text{\textbullet{} } j_l          &  & \coloneqq j_l(x,t) \\
                    & \text{\textbullet{} } \sigma_\effl &  & = \sigma_l \cdot \varepsilon_l \\
                    & \text{\textbullet{} } \eta_l        &  & \coloneqq \eta_l(x,t) = \phi_\slsub(x,t) - \phi_\text{e}(x,t) - \mathcal{U}_l(c_\slsurf) \\[0.5em]
                \end{alignedat}$
                % \medskip
                \fbox{\textbf  \linnegseppos}\\
                $\begin{alignedat}{2}
                    & \text{\textbullet{} } D_\effl    &  & = D \cdot \varepsilon_l^{\text{brugg}_l}                                                                                                                                                  \\
                    & \text{\textbullet{} } \kappa_\effl &  & = \kappa \cdot \varepsilon_l^{\text{brugg}_l} \\
                    & \text{\textbullet{} } D        &  & \coloneqq D(c_\text{e},T) = 10^{-4} \times 10^{-4.43 - \frac{54}{T(t) - 229 - 5\times10^{-3} c_\text{e}(x,t)} - 0.22\times10^{-3} c_\text{e}(x,t)}               \\
                \end{alignedat}$
                \\
                \makeatletter\def\f@size{14}\check@mathfonts
                $\begin{alignedat}{2}
                    & \text{\textbullet{} }  \kappa   \: \quad&  & \coloneqq \kappa(c_\text{e},T)\, = \parbox[t]{11.60cm}{\raggedright $\scriptstyle 10^{-4} \times c_\text{e}(x,t)\Big(-10.5 + 0.668\times10^{-3} c_\text{e}(x,t) + 0.494\times10^{-6} c_\text{e}^2(x,t) + \big(0.074 - 1.78\times10^{-5}c_\text{e}(x,t) - 8.86\times10^{-10}c_\text{e}^2(x,t)\big)T(t) + \big(-6.96\times10^{-5} + 2.8\times10^{-8} c_\text{e}(x,t)\big)T^2(t)\Big)^2$}\\
                \end{alignedat}$
            \end{flushleft}
        \end{minipage}
        \bigskip
        \footnoterule{}
        \begin{tablenotes}
        \item[\dagger] \footnotesize{This definition of the global axial domain $x$, spanning the cell thickness applies for all variables dependent on axial spatial position. Hence the domain definition for $x$ is not repeated elsewhere for the sake of brevity.}
        \end{tablenotes}
    \end{threeparttable}
\end{table}

