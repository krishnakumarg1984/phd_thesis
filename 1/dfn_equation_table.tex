% -*- root: ../main.tex -*-
%!TEX root = ../main.tex
% this file is called up by main.tex
% content in this file will be fed into the main document
% vim:nospell textwidth=180

\begin{table}[!htbp]
    \centering
    \caption[]{}
    % \label{tbl:charSimspmp2d}
    \begingroup
    \addtolength{\jot}{0.5em}
    \begin{tabular}{@{} c c r l r @{}}
        \toprule
        Cell region & Governing equations & \multicolumn{3}{c}{Boundary conditions} \\
        \midrule
        \makecell{Electrodes \\ \footnotesize \linnegpos} &
        $\begin{aligned} % placement: default is "center", options are "top" and "bottom"
            \vphantom{\diffp{c_\slsub}{r}{\mathrlap{r = R_\pl}}}
            \diffp{c_\slsub}{t} &= \frac{D_\slsub}{r^2}\diffp{}{r}\left(r^2 \diffp{c_\slsub}{r} \right) \\
            \vphantom{\diffp{c_\text{e}}{x}{\mathrlap{x = l_\text{tot}}}}
            \varepsilon_l \diffp{c_\text{e}}{t} &= \diffp{}{x}\left(D_\effl
            \diffp{c_\text{e}}{x} \right) + (1 - t^0_\text{+}) a_\slsub j \\
            0 &= \diffp{}{x}\left(\kappa_\effl \diffp{\phi_\text{e}}{x}\right)
        \end{aligned}$ &
        $\begin{aligned}
            \vphantom{\diffp{c_\slsub}{r}{\mathrlap{r = R_\pl}}} \diffp{c_\slsub}{r}{\mathrlap{r = 0}} &= 0, \\
            \vphantom{\diffp{c_\text{e}}{x}{\mathrlap{x = l_\text{tot}}}} \diffp{c_\text{e}}{x}{\mathrlap{x = 0}} &= 0, \\
        \end{aligned}$ &
        $\begin{aligned}
            \diffp{c_\slsub}{r}{\mathrlap{r = R_\pl}} &= -\frac{j}{D_\slsub} \\
            \diffp{c_\text{e}}{x}{\mathrlap{x = l_\text{tot}}} &= 0 \\
        \end{aligned}$ &
        $\begin{aligned}
            \vphantom{\diffp{c_\slsub}{r}{\mathrlap{r = R_\pl}}}\refstepcounter{equation}(\theequation)\label{eq:dfnsoliddiff} \\
            \vphantom{\diffp{c_\text{e}}{x}{\mathrlap{x = l_\text{tot}}}} \refstepcounter{equation}(\theequation)\label{eq:dfnliquiddiff} \\
        \end{aligned}$ \\
        \bottomrule
    \end{tabular}
    \endgroup
    \begingroup
    \raggedright
    $\begin{alignedat}{2}
        & \text{\textbullet{} } c_\slsub   &  & \coloneqq c_\slsub(r,t),\quad \{x \in [0,R_\pl],\, (r=0)\symbol{"2259} \text{\footnotesize particle center},\, (r=R_\pl)\symbol{"2259} \text{\footnotesize particle surface}\}              \\
        & \text{\textbullet{} } j          &  & \coloneqq j(x,t),\quad \{x \in [0,l_\text{tot}],\, (x=0)\symbol{"2259} \text{\footnotesize pos/Alcc},\, (x=l_\text{tot})\symbol{"2259} \text{\footnotesize neg/Cucc}\} \text{\footnotemark} \\
        & \text{\textbullet{} } c_\text{e} &  & \coloneqq c_\text{e}(x,t)                                                                                                                                                                   \\
        & \text{\textbullet{} } D_l        &  & \coloneqq D_l(x,t,c_\text{e},T)\, \text{ where } D_l = 10^{-4} \times 10^{-4.43 - \frac{54}{T(t) - 229 - 5\times10^{-3} c_\text{e}(x,t)} - 0.22\times10^{-3} c_\text{e}(x,t)}               \\
        & \text{\textbullet{} } D_\effl    &  & = D_l \cdot \varepsilon_l^{\text{brugg}_l}                                                                                                                                                  \\
    \end{alignedat}$
    % \footnotetext{For all variables dependent on the axial spatial position $x$, it is implied that $x \in [0,l_\text{tot}]$ to avoid repetition.}
    \footnotetext{This definition of the domain $x$ shall apply for all variables dependent on the axial spatial position and shall not be repeated for the sake of brevity.}
    \endgroup
\end{table}
