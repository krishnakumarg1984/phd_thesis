% -*- root: ../main.tex -*-
%!TEX root = ../main.tex
% this file is called up by main.tex
% content in this file will be fed into the main document
% vim:textwidth=80 fo=cqt

\clearpage
\chapter{Physics-Based Controls-Oriented Time-Domain Modelling}

% change according to folder and file names
\graphicspath{{5/figures/}}

% ----------------------- contents from here onwards ------------------------
\fxnote{uncomment and edit the intro in the latex source file later on}

% Firstly, the contrasting nature of this modelling objective is presented.

% A suitable  family of models from  the broad category of  reduced-order models
% is  identified  as  a  promising  candidate  for  implementation  in  controls
% applications. Next,  the drawbacks of  this family  of models is  discussed in
% detail. The  state of the art  implementation for tackling these  drawbacks is
% presented and their inadequacies are discussed.

% Prima Facie, based  on a preliminary comparison of the  strengths and weaknesses
% of  the  modelling  families  in  the  literature  considered,  the  overarching
% simplicity of  \glspl{spm} and their immediate  potential to bring the  power of
% physics-based prediction  to an embedded  environment is a strong  motivation to
% pursue their in-depth exploration. The  equations describing the single particle
% model  is introduced  in  \ldots. An  in-depth analysis  of  their drawbacks  is
% presented in \ldots. Various attempts to  fix this issue is presented in \ldots.
% The  electrolyte  concentration and  potential  fixes  is presented  in  \ldots.
% Finally, results and discussion. \fxnote{REWRITE after finishing up everything}


\fxnote{Write the chapter. Finally come back to summarize this}


% The following efforts/trials were done (failures)
% \begin{itemize}
%     \item first attempt
%     \item second attempt
% \end{itemize}
% The following successes were achieved.
% \begin{itemize}
%     \item first attempt
%     \item second attempt
% \end{itemize}


% At the  end of this  chapter, we have a  control oriented reduced  order battery
% model amenable for use in real-time applications for SOC, SOH etc.\ estimations.


 % Intro to the SPM analysis chapter


\section{The \glsfmtfull{spm}}\label{sec:spmintro}
% -*- root: ../main.tex -*-
%!TEX root = ../main.tex
% this file is called up by main.tex
% content in this file will be fed into the main document
% vim:textwidth=80 fo=cqt

% As  discussed in  \cref{sec:classificationscheme},\todo{may need  to cross-ref
% the relevant subsection}


Reducing  the   number  of   computational  dimensions   in  a   physical  model
helps  in  formulating  their  low order  approximations,  thereby  facilitating
fast   computations.   The  \gls{spm},   originally   used   in  modelling   the
Metal-Hydride    chemistry~\cite{Haran1998}   and    later   on    adapted   for
Li-ion   cells~\cite{Ning2004},  represents   the  canonical   apogee  of   such
dimension-reduction strategies.


During the initial  years following its inception, the formulation  of the basic
\gls{spm} has  been discussed  extensively within  application-specific contexts
such  as  \gls{soc}  estimation~\cite{Santhanagopalan2006a,Santhanagopalan2008},
parameter   estimation~\cite{Santhanagopalan2007},   life   cycle   and   ageing
predictions~\cite{Santhanagopalan2008a,Safari2009}.   There   have   also   been
detailed    stand-alone    presentations    of    various    facets    of    the
basic   \gls{spm},   such   as    its   inherent   assumptions   and   governing
equations~\cite{Santhanagopalan2006,Chaturvedi2010}.    The   basic    \gls{spm}
suffers    from    certain    major     limitations    which    are    discussed
in~\cref{subsec:basicspmlimitations}. Since the turn  of the decade, researchers
have attempted to tackle many of these  issues and such efforts are discussed in
\cref{sec:electrolyteinclusion}.


A  survey of  the  most recent  literature  in all  \gls{spm}  family of  models
reveals  a  diminishing rate  of  advancement  in quantifiable  improvements  to
the  underlying  plant  model.  This   nearly-static  trend  can  be  attributed
to  the  general consensus  within  the  research  community that  these  models
may  be  too  simplistic  and  not  of  suitable  accuracy  to  warrant  further
studies.   Other  than   a  small   minority   of  papers   that  propose   core
modelling  improvements  to  tackle  their modelling  inaccuracies  or  add  new
enhancements  such as  mechanical-stress physics~\cite{Li2017a,Li2018b},  latest
work  in   this  family   of  models  predominantly   pertains  to   the  topics
of    state   estimation~\cite{Chaochun2018,Lin2017,Tran2017,Moura2017,Zou2016},
optimal    charging~\cite{Perez2015},    cycling    performance~\cite{Maia2017},
conversion           to            equivalent           circuits~\cite{Li2017b},
parametrisation~\cite{Li2018,Rajabloo2017,Bizeray2017,Namor2017}, pack-balancing
studies~\cite{Docimo2018}  and   observer  design  for   joint  state--parameter
estimation~\cite{Ascencio2016}. The \gls{spm} approach  was also extended to the
case  of  composite  electrodes,  leading  to a  state  estimator  design  after
basic  observablity  analysis~\cite{Bartlett2015}.  Owing to  their  simplicity,
this  author  believes that  \gls{spm}s  hold  the  highest potential  to  bring
a  physics  based  model  to  embedded  \gls{bms}s.  With  this  goal  in  view,
this  thesis seeks  to  resurrect  interest in  this  field  by addressing  this
paucity in fundamental  model improvements. A proposed enhancement  to the basic
\gls{spm}  is the  main  contribution of  this chapter  and  shall be  discussed
in~\cref{sec:newelectrolytemodel}.


In  order  to establish  a  context  for discussing  the  author's  work, it  is
imperative to provide  a holistic presentation of the  basic \gls{spm} modelling
art. The conventional \gls{spm} is the simplest of all time domain physics-based
models and  the rest  of this  section provides an  expository treatment  of its
rubrics.


\subsection{Model Development --- Geometry}\label{subsec:basicspmgeometry}

\begin{figure}[h]
    \centering
    \includegraphics[width=\textwidth]{placeholder_images/example-image-golden.pdf}
    \caption[Schematic illustration depicting geometrical origins of the \gls{spm}]
    {Schematic illustration depicting the geometrical origins of the \gls{spm}. The \gls{spm} is obtained through a degenerate spatial discretisation of one electrochemical layer within a typical Li-ion pouch cell. The axial direction along the cell's thickness is denoted by $x(t)$, whilst the pseudo-dimension along the radial depth of each electrode particle is denoted by $r(t)$. In the basic \gls{spm}, the active material of each porous electrode is represented by one representative spherical particle, thus entirely eliminating the spatial dimension along the axial direction.}
    \label{fig:sandwichtospm}
\end{figure}


\Cref{fig:sandwichtospm}  shows the  arrangement  of  one electrochemical  layer
within a  typical Li-ion pouch cell.  A description of the  working principle of
the  cell was  presented in~\cref{ch:chapter1}  and  is not  repeated here.  The
\gls{spm}, as the name suggests,  models the electrochemical phenomena along the
thickness  $l_j$, \jinnegpos{}  of  each porous  electrode  by a  representative
spherical particle.  Thus the two  distinct solid-phase porous materials  of the
cell, \ie{} the  negative and positive electrodes, are idealised  as two spheres
of radii $r_\text{neg}$ and $r_\text{pos}$ respectively.


In  this  reformulated  arrangement,  the  spatial  dimension  along  the  axial
thickness  of  each  electrode  degenerates   to  a  single  point.  Hence,  the
concentration of Lithium within each electrode $c_{\text{s}_j}$, \jinnegpos{} is
only a  function of the radial  position $r_j$, \jinnegpos{} along  the depth of
their  representative  spherical  particle,  and time,  $t$.  The  surface  area
of  each representative  sphere  is  scaled appropriately,  such  that they  are
equivalent  to the  active area  of  the corresponding  porous electrodes.  Thus
the  \gls{spm} accounts  for  the  reduced volume-fraction  arising  due to  the
microporous structure of  the solid-phase. However, the storage  capacity of the
representative  particles  match  that  of  the  corresponding  electrodes.  The
overarching  assumption  of  the  \gls{spm} modelling  philosophy  is  that  the
electrochemical performance of these representative electrodes are sufficient to
model the behaviour of the cell at its terminals. The \gls{spm} thus employs the
coarsest possible spatial  discretisation of the cell's thickness  with the goal
of minimising computational burden.


\subsection{Model Development --- Scope and Assumptions}

Having established the geometrical representation of the model, it is imperative
to establish its  aims and scope. This section discusses  the subset of physical
phenomena that can captured by the model and enumerates the inherent assumptions
in  model derivation.The  validity of  these  assumptions and  their effects  is
discussed  in~\cref{subsec:basicspmlimitations}.  As  a  broad  outline  of  the
\gls{spm}s scope,  the model  attributes the cell  polarisation to  two dominant
physics,  \viz{} reaction  kinetics and  solid-phase transport  phenomena, \ie{}
diffusion dynamics.


The  \gls{spm}  assumes that  charge  transfer  happens throughout  the  surface
of  each  representative  spherical  particle where  intercalation  occurs.  The
electronic  conductivity of  the solid-phase  is assumed  to be  high enough  to
ignore the  spatial distribution of  charge, \ie{} the local  volumetric current
density is assumed  to be uniform along the thickness  of each porous electrode.
This assumption is  motivated by the early calculations performed  by Newman and
Tobias~\cite{Newman1962} in their stand-alone  analysis of current distributions
in porous electrodes, wherein a volume-averaged molar flux is deemed sufficient.
This uniform current density assumption implies that all of the particles in the
electrode active material are in parallel.


In the  \gls{spm}, solid-phase  diffusion dynamics are  solved by  assuming this
averaged electrochemical  reaction rate.  In the simulation  study by  Smith and
Wang~\cite{Smith2006b},  it  is  reported  that, soon  after  the  beginning  of
discharge, solid-phase concentration and ionic flux become nearly independent of
spatial position, and  Lithium diffusion in solid particles may  be driven by an
averaged molar flux at the surface.


Based  on the  discussion thus  far, it  is clear  that the  \gls{spm} does  not
attempt  to model  all physical  processes within  the cell.  The model  assumes
instantaneous  charge transport  from one  electrode  to the  other through  the
solution phase.  This implies that  electrolytic diffusion is  sufficiently fast
(relative to  diffusion in the solid  phase). Thus, mass transport  phenomena in
the  electrolyte  are  not  considered.


During the  operation of the  cell, the  \gls{spm} assumes that  the electrolyte
concentration  $c_\text{e}$ remains  constant at  its equilibrium  initial value
$c_{\text{e},0}$ throughout  the cell thickness. Neglecting  local concentration
gradients  in the  solution phase,  together  with ignoring  its mass  transport
phenomena implies that  the current in the electrolyte does  not vary over space
and time. Hence,  in the conventional \gls{spm} there is  no contribution of the
solution  phase  to internal  overpotentials  and  electroyte dynamics  have  no
influence on the cell's terminal voltage.


Finally,  the  \gls{spm}  ignores  any   variations  in  material  porosity  and
ionic-flow tortuosity  along the axial  direction of the cell.  This facilitates
the  usage of  a constant  effective diffusion  coefficient for  the electrolyte
phase. Furthermore,  the solid-phase diffusion  and kinetic parameters  are held
constant. Thermal effects are assumed to be negligible and no degradation
effects are attempted to be modelled.


These simplifying  assumptions are  made so  as to enable  the formulation  of a
physics-based model without incurring a  heavy computational cost. The impact of
these  assumptions shall  be  examined in~\cref{subsec:basicspmlimitations}  and
later  sections presents  research that  strives  to straddle  the fine  balance
between model sophistication and computational complexity.


\subsection{Model Development --- Chemistry}

This section  provides a  brief overview of  the essential  chemistry principles
that helps to provide a background context for the governing equations presented
in~\cref{subsec:basicspmgoverningeqns}.


In  a Li-ion  cell,  the  positive electrode  consists  of  porous particles  of
Lithium--Transition Metal Oxide (MO) compounds. The negative electrode typically
employs  some  variant  of  microporous  graphite.  The  porous  nature  of  the
electrodes  provide interstitial  sites  which act  as  intercalation spots  for
Lithium shuttling  between the two  electrodes. The electrolyte,  whose dynamics
are ignored  in the \gls{spm},  helps in the  conduction of \ch{Li^+}  ions. The
separator membrane allows the passage of  these ions between the two electrodes,
but prevents internal short-circuit  by inhibiting electronic conduction through
it. The current collectors facilitate  passage of electrons generated during the
charge transfer reaction  at particle surface to the external  circuit. With the
help of~\cref{fig:chargetransferprocess}  the steps involved in  this process is
detailed next.

\begin{figure}[h]
    \centering
    \includegraphics[width=0.5\textwidth]{placeholder_images/example-image-golden.pdf}
    \caption
    {Simplified representation of charge-transfer process and illustration of
    basic working mechanism of a Li-ion cell}
    \label{fig:chargetransferprocess}
\end{figure}

At fully charged  condition, majority of Lithium in the  system is packed within
the negative electrode microstructure. During discharge, \ch{Li^0} atoms diffuse
out  of  deep  interstitial  sites  towards the  surface  of  the  particles  in
the  negative electrode.  At  the surface  (electrode-electrolyte interface),  a
charge-transfer process takes place according to Butler-Volmer kinetics, leading
to  the formation  of \ch{Li^+}  ions and  electrons. The  electrons are  passed
to  the external  circuit  through  \ch{Cu} current  collectors  onto which  the
conductive matrix  composed of  the negative electrode  material and  binders is
coated. The  \ch{Li^+} ions travel  through the electrolyte phase,  crossing the
separator membrane  to the positive  electrode where they encounter  an electron
influx from the external circuit. A  charge transfer reaction takes place at the
surface of the positive electrode particles, leading to the formation of neutral
\ch{Li^0} atoms that diffuse into the positive electrode microstructure.

During  the   charging  process,  the   reverse  phenomena  occur.   Lithium  is
de-intercalated  from  the  positive  electrode and  a  similar  charge-transfer
happens  at the  surface,  leading  to the  formation  of  \ch{Li^+} ions  which
reach  the  negative  electrode  by   passing  through  the  separator.  At  the
surface  of  the  negative  electrode particles,  these  ions  absorb  electrons
from  the  external circuit,  leading  to  the  formation of  neutral  \ch{Li^0}
that   diffuses  into   interior   vacant  spaces   in   the  layered   graphite
electrode. The  charge-transfer mechanism  and sequence  of events  are depicted
in~\cref{fig:chargetransferprocess}.
~\Cref{eq:NegElectrodeRxn,eq:PosElectrodeRxn} summarise the reactions during the
charging and discharging process at the surfaces of both electrode materials.
\tikzexternaldisable
\begin{align}
    \ch{Li_{$x$} C &<=>[\tiny{discharge}][\tiny{charge}] C + $x$ Li^+ + $x$ e^-}\label{eq:NegElectrodeRxn}\\
    \ch{Li_{1-$x$} M O2 + $x$ Li^+  + $x$ e^- &<=>[\tiny{discharge}][\tiny{charge}] LiMO2}\label{eq:PosElectrodeRxn}
\end{align}
\tikzexternalenable
where \ch{M} represents a transition metal compound such as
\ch{Ni_{1/3}Co_{1/3}Mn_{1/3}} (NMC), \ch{Ni_{0.8}Co_{0.15}Al_{0.05}} (NCA)
amongst other choices~\cite{Reddy2011}. Assuming no loss  of cycleable Lithium
due to parasitic side reactions or through other mechanisms, the process is
fully reversible.


The  electric potential  at  each  electrode is  dependent  upon  the extent  of
its  lithiation. An  empirical  relationship of  each  electrode's potential  as
a  function  of  its  stoichiometry  can be  obtained,  and  is  dependent  upon
the  specific design  and  material  properties of  each  active material  under
consideration. Finally, the \gls{ocv} of the cell is obtained by subtracting the
negative electrode potential from its positive electrode counterpart.

\subsection{Model Development --- Governing Equations}\label{subsec:basicspmgoverningeqns}

As discussed  in~\cref{subsec:basicspmgeometry}, the \gls{spm}  primarily models
the phenomena of solid-phase diffusion and reaction kinetics.

Conservation of \ch{Li^0} in the electrodes can be obtained by treating that the
movement of neutral  atoms within the solid phase is  primarily due to diffusion
within particles.  This diffusion  phenomena is induced  due to  a concentration
gradient that  exists between the  surface and interior/core of  the solid-phase
particles. This  diffusion effect  can be studied  by applying  standard Fickian
dynamics given by

\begin{equation}
    \frac{\partial x}{\partial t}
\end{equation}

\subsection{Limitations and Drawbacks}\label{subsec:basicspmlimitations}

The modelling foundations of the \gls{spm} have been



\section{\glsfmtshort{spm} Model Development}\label{sec:spmmodeldevelopment}
% -*- root: ../../main.tex -*-
%!TEX root = ../../main.tex
% vim:textwidth=80 fo=cqt

In  order  to  establish  a  context  for the  author's  work  to  be  discussed
in  \cref{ch:newelectrolytemodel},  it  is  imperative  to  provide  a  holistic
presentation of the basic \gls{spm} modelling art. The conventional \gls{spm} is
the  simplest  of  all  time  domain  \glspl{pbm}.  The  rest  of  this  section
provides  this author's  digested summary  of its  modelling rubrics  based upon
the  keen  insights  gained  from  perusing  the  vanguard  literature  in  this
topic~\cite{Santhanagopalan2006,Santhanagopalan2006a,DiDomenico2010} as  well as
their relevant derivative works.

\subsection{Geometry}\label{subsec:basicspmgeometry}

A  description  of   the  working  principle  of  the  cell   was  presented  in
\cref{subsec:liionchemistry} and  is not  repeated here.  The \gls{spm}  aims to
capture the electrochemical phenomena along the thickness~$l_j$,~\jinnegpos{} of
each  porous electrode  by a  representative  spherical particle.  Thus the  two
distinct  solid  phase porous  regions  of  the  cell \ie~the  negative  and
positive electrode regions, are idealised as two spheres of radii~$R_\text{neg}$
and~$R_\text{pos}$ respectively.


\Cref{fig:sandwichtospm}   shows   the   geometrical  origins   of   the   basic
\gls{spm}.  In  this   arrangement,  the  spatial  dimension   along  the  axial
thickness  of   each  electrode   degenerates  to  a   single  point   which  is
represented  by  a sphere.  Hence,  the  concentration  of Lithium  within  each
electrode~$c_{\text{s}_j}$,~\jinnegpos{}  is  only  a  function  of  the  radial
position~$r_j$,~\jinnegpos{}  and  the  time~$t$.   The  surface  area  of  each
representative sphere is scaled appropriately,  such that they are equivalent to
the  active area  of the  corresponding  porous electrodes.  Thus the  \gls{spm}
accounts  for  the  reduced  volume-fraction  arising  due  to  the  microporous
structure  of  the   solid  phase  and  hence,  the  storage   capacity  of  the
representative  particles  match  that  of  the  corresponding  electrodes.  The
overarching  assumption  of  the  \gls{spm} modelling  philosophy  is  that  the
electrochemical performance of these representative electrodes are sufficient to
model the behaviour of the cell at its terminals. The \gls{spm} thus employs the
coarsest possible spatial  discretisation of the cell's thickness  with the goal
of minimising computational burden.

\begin{figure}[!htbp]
    \centering
    \includegraphics{spm_geometry}
    \caption[Schematic illustration depicting geometrical origins of the
    \glsfmtshort{spm}]
    {%
        Schematic illustration depicting the geometrical origins of the basic
        \gls{spm}. The model is obtained through a degenerate spatial
        discretisation of one electrochemical layer of a  pouch cell.  The
        active material of each porous electrode is represented by one
        representative spherical particle, thus entirely eliminating the spatial
        dimension along the axial direction. Illustration reproduced
        from Moura~\etal~\cite{Moura2012}.
    }%
    \label{fig:sandwichtospm}
\end{figure}

\subsection{Scope and Assumptions}\label{subsec:basicspmassumptions}

Having described the  geometrical representation of the model,  it is imperative
to establish its  aims and scope. This section discusses  the subset of physical
phenomena  that can  be  captured  by the  basic  \gls{spm}  and enumerates  the
inherent assumptions in model derivation.  The validity of these assumptions and
their effects  on model accuracy shall  be examined in the  results presented in
\cref{subsec:simresultsbasicspm}. As  a broad  outline of  its scope,  the model
attributes the cell polarisation to two dominant physics \viz~reaction kinetics
and solid phase transport phenomena \ie~diffusion dynamics.


The  \gls{spm}  assumes that  charge  transfer  happens throughout  the  surface
of  each  representative  spherical  particle where  intercalation  occurs.  The
electronic  conductivity of  the solid  phase is  assumed to  be high  enough to
ignore the  spatial distribution  of charge \ie~the local  volumetric current
density is assumed  to be uniform along the thickness  of each porous electrode.
This assumption is  motivated by the early calculations performed  by Newman and
Tobias~\cite{Newman1962} in their stand-alone  analysis of current distributions
in porous electrodes, wherein a volume-averaged molar flux was deemed sufficient
throughout  the  thickness  of  the  electrode.  This  uniform  current  density
assumption implies  that all of the  particles in the electrode  active material
are  in parallel.  Solid  phase  diffusion dynamics  within  each electrode  are
therefore solved by assuming this averaged electrochemical reaction rate. In the
simulation study  by Smith and~Wang~\cite{Smith2006},  it is reported  that soon
after  the beginning  of discharge,  solid  phase concentration  and ionic  flux
become nearly  independent of  spatial position, and  that Lithium  diffusion in
solid particles may be driven by an averaged molar flux at the surface.


Based  on the  discussion thus  far, it  is clear  that the  \gls{spm} does  not
attempt to  model all  physical processes  within the  cell. In  particular, the
model assumes  instantaneous charge  transport from one  electrode to  the other
through  the  solution  phase.  This  implies  that  electrolytic  diffusion  is
sufficiently  fast  (relative to  diffusion  in  the  solid phase).  Thus,  mass
transport  phenomena  in  the  electrolyte  have been  neglected  in  the  basic
\gls{spm}.


During   the  operation   of  the   cell,   the  \gls{spm}   assumes  that   the
electrolyte  concentration~$c_\text{e}$  remains  constant  at  its  equilibrium
initial value~$c_{\text{e},0}$  throughout the cell thickness.  Neglecting local
concentration gradients in  the solution phase, together with  ignoring its mass
transport phenomena, implies  that the current in the electrolyte  does not vary
over  space  and  time.  Hence,  in  the  conventional  \gls{spm}  there  is  no
contribution of the solution phase  to internal overpotentials \ie~electrolyte
dynamics have  no influence on the  cell's terminal voltage. The  \gls{spm} also
ignores any  variations in material  porosities in each  electrode. Furthermore,
all solid phase diffusivities and kinetic parameters are held constant. Finally,
all thermal effects are assumed to  be negligible and no degradation effects are
attempted to be modelled in the basic \gls{spm} formulation.

These  simplifying  assumptions  are  made  so as  to  facilitate  the  ease  of
implementing a  \gls{pbm}, without incurring  the heavy computational  cost that
typically accompanies it. The impact of these assumptions on the accuracy of the
model shall  be examined in \cref{subsec:simresultsbasicspm}  and later sections
presents prior research that strives to  straddle the fine balance between model
sophistication and computational complexity.

\subsection{Governing  Equations}\label{subsec:basicspmgoverningeqns}
% {Governing  Equations\protect\footnote{In  this
% section,   only  those   simplifications  to   mathematical  notations   arising
% due  to  assumptions  discussed  in \cref{subsec:basicspmassumptions}  shall  be
% introduced   in-line.   For  a   comprehensive   reference   to  the   notations
% used,   please  refer   to   nomenclature   list  in   the~\nameref{ch:glossary}
% chapter.  Notations  introduced  solely  for   this  section  are  also  covered
% here.}}


As discussed  in \cref{subsec:basicspmassumptions},  the \gls{spm}  captures the
cell's dynamics arising due to diffusion  and kinetics at the two representative
spherical electrodes. It also accounts for the contribution of their equilibrium
thermodynamics to the cell's \gls{ocp}.

\subsubsection*{Solid Phase Diffusion}

Conservation of \ch{Li^0} in the electrodes can be obtained by assuming that the
movement of neutral  atoms within the solid phase is  primarily due to diffusion
within particles.  This diffusion  phenomena is induced  due to  a concentration
gradient that  exists between the surface  and interior/core of the  solid phase
particles. Based on the geometrical assumptions of the \gls{spm} as discussed in
\cref{subsec:basicspmgeometry} \ie~owing to the lack of spatial discretisation
in  the  axial  direction~$x$,  the  concentrations  of  \ch{Li^0}  in  the  two
electrodes~$c_\sj(x,r,t)$, reduce  to a  function of the  radial co-ordinate~$r$
and time~$t$, and is denoted by~$c_\sj(r,t)$,~\jinnegpos{}. To keep the notation
tractable, this  explicit spatio-temporal radial dependence  is omitted, further
simplifying the representation  to~$c_\sj$. We start with the  derivation of the
solid-phase diffusion equation \cref{eq:dfnsoliddiff} in \cref{tbl:dfneqns}, and
then proceed  to the simplifications  of its boundary conditions  facilitated by
the \gls{spm} assumptions.

Diffusion  in the  solid phase  can be  modelled by  applying classical  Fickian
dynamics~\cite{Fick1995} given by
\begin{equation}\label{eq:cartesiandiffusion}
    \diffp{c_\sj}{t} = ∇\! ⋅ \left(D_\sj\, ∇ c_\sj \right)\qquad \jinnegpos{}%\footnotemark{}
\end{equation}
% \footnotetext{For  the  sake of  brevity,  in  rest  of  the equations  in  this
% section,  the  explicit  definition  for   each  subsequent  occurrence  of  the
% subscripted variable $j$ shall be omitted. It is implied that \jinnegpos{} since
% these  equations describe  solid phase  diffusion in  the negative  and positive
% electrodes.}
The  divergence of  a vector  field~$\mathbf{F}(r,θ,ϕ)$ can  be expressed  in
spherical co-ordinates as
\begin{equation}\label{eq:fullsphericaldiv}
    ∇ ⋅ \mathbf{F} = \frac{1}{r^2}\diffp{\left(r^2 F_r\right)}{r} +
    \frac{1}{r \sin θ}\diffp{\left(\sin θ\:  F_θ\right)}{θ}
    + \frac{1}{r \sin θ}\diffp{F_ϕ}{ϕ}
\end{equation}
where  $r$~denotes the  radius,  $θ$~the polar  angle,  and $ϕ$~the  azimuthal
angle. $F_r, F_θ$ and $F_ϕ$~denote  the corresponding components of the vector
field~$\mathbf{F}$.

The  co-ordinate  origin  at  each  electrode is  aligned  with  the  centre  of
its  representative  spherical  particle.  Due  to symmetry  in  the  polar  and
azimuthal axes, the  divergence becomes a function of only  the radial position.
\Cref{eq:fullsphericaldiv} therefore reduces to
\begin{equation}\label{eq:reducedsphericaldiv}
    ∇ ⋅ \mathbf{F} = \frac{1}{r^2}\diffp{\left(r^2 F_r\right)}{r}
\end{equation}
Applying     the     \gls{rhs}     of     the     divergence     operator     of
\cref{eq:reducedsphericaldiv} in \cref{eq:cartesiandiffusion} yields
\begin{equation}\label{eq:csdiffusioneqn}
    \diffp{c_\sj}{t} = \frac{1}{r^2}\diffp*{\left(r^2 D_\sj\, ∇ c_\sj \right)}{r}
\end{equation}
As   per  the   assumption  of   uniform   diffusivity  in   the  solid   phase,
\cref{eq:csdiffusioneqn} becomes
\begin{align}
    \diffp{c_\sj}{t} &= \frac{D_\sj}{r^2}\diffp*{\left(r^2 ∇ c_\sj \right)}{r}\label{eq:csdiffusionconstdiffusivity}
    \intertext{Applying the gradient operator of \cref{eq:csdiffusionconstdiffusivity} along
    the radial direction~$r$,}
    \diffp{c_\sj}{t} &= \frac{D_\sj}{r^2}\diffp*{\left(r^2 \diffp{c_\sj}{r} \right)}{r}\label{eq:csdiffusionfinal}
\end{align}
\Cref{eq:csdiffusionfinal} represents  a mass-balance equation  describing solid
phase diffusion  in each electrode  and is  identical to the  governing equation
\cref{eq:dfnsoliddiff} from \cref{tbl:dfneqns}. The  potential at each electrode
depends  on   the  solid  phase  surface   concentration~$c_\sjsurf$  \ie~the
\ch{Li^0} concentration~$c_\sj(r,t)$  evaluated at~$r=R_\pj$,~\jinnegpos{} where
$R_\pj$~represents  the  equivalent  radius  of  each  representative  spherical
particle.

Due to spherical symmetry,  flux at the centre of the  particle is considered to
be zero.
\begin{equation}\label{eq:csfluxcentre}
    \diffp{c_\sj}{r}{\mathrlap{r=0}} = 0
\end{equation}

Diffusion in the solid phase is driven by concentration gradients induced due to
intercalation flux  density at the  particle surface  \ie~the surface  of each
particle experiences a pore-wall flux density driven by reaction kinetics. Based
on  the  \gls{spm}  geometry discussed  in  \cref{subsec:basicspmgeometry},  the
spatial dependence of this molar  flux density~$j_\nj(x,t)$ is eliminated and it
can  therefore  be  represented  as~$j_\nj(t)$,~\jinnegpos{}. For  the  sake  of
brevity,  its  explicit temporal  dependence  is  also  omitted resulting  in  a
simplified notation~$j_\nj$. Hence, at the particle surface
\begin{equation}\label{eq:csfluxsurface}
    D_\sj\diffp{c_\sj}{r}{\mathrlap{r=R_\pj}} = -j_\nj
\end{equation}
The sign convention chosen here is such that pore-wall flux leaving the particle
surface is considered to be positive.

Charge   conservation   in   solid   phase    is   applied   to   evaluate   the
\gls{rhs}  in \cref{eq:csfluxsurface},   a  detailed  derivation  of   which  is
presented  in  Domenico~\etal~\cite{DiDomenico2010}.  In  summary,  by  assuming
a  uniform   charge  density   throughout  the   thickness  of   each  electrode
(see \cref{subsec:basicspmassumptions}), we get
\begin{align}
    j_\nj(t)                       &= ± \frac{I(t)}{A \, l_j a_\sj F}\label{eq:uniformcurrdensity}   \qquad \jinnegposordered
    \intertext{Substituting \cref{eq:uniformcurrdensity} in \cref{eq:csfluxsurface}}
    D_\sj\diffp{c_\sj}{r}{\mathrlap{r=R_\pj}} &= ∓ \frac{I}{A \, l_j a_\sj F}\label{eq:csfluxsurfacefinal} \qquad \jinnegposordered
\end{align}
wherein  the  load current~${I(t)  >  0}$  for discharge,  whose  explicit
time-dependence  has  been  omitted in  \cref{eq:csfluxsurfacefinal}  for  being
consistent notation with the \gls{lhs}. The positive and negative signs apply to
the negative  and positive  electrode respectively as  indicated by  the ordered
pair \jinnegposordered. It  should be noted that the term  involving the Faraday
constant in the \gls{rhs}  of \cref{eq:uniformcurrdensity} is~$nF$, where $n$~is
the number  of electrons  transferred during the  reaction. However,  since this
thesis only  discusses lithium-ion  chemistries where~$n=1$, this  is implicitly
conveyed and shall be omitted for all potential occurrences.

The \gls{soc} of the cell can be obtained from the bulk concentration of lithium
in  either the  negative  or  positive electrode.  By  convention, the  negative
electrode is used.
\begin{equation}\label{eq:socinitialdefn}
    z(t) = \frac{3}{c_\snegmax}∫_0^{R_\pneg}r^2 c_\sneg (r,t)\, dr
\end{equation}
Given an  initial cell  \gls{soc}~${z(0) = z_0}$  at rest,  the equilibrium
concentration of \ch{Li^0} in the two individual electrodes can be computed as
\begin{equation}\label{eq:csfluxinitialcondition}
    c_\sj(r,0) = c_\sjmax \, \bigg[z_0 \left(θ_\maxj - θ_\minj \right) + θ_\minj \bigg]
\end{equation}

\Cref{eq:csdiffusionfinal},          its         corresponding          boundary
conditions~\eqref{eq:csfluxcentre} and~\eqref{eq:csfluxsurfacefinal}  along with
initial   condition~\eqref{eq:csfluxinitialcondition},   provide  the   complete
description of  time-domain evolution of  lithium in the  conventional \gls{spm}
for  a  given  applied  current  profile~$I(t)$.  Considerations  for  efficient
numerical simulation of this system is presented next.

%  in \cref{subsec:basicspmgeometry}.


\subsubsection*{Further Reduction in Dimensionality}\label{subsec:basicspmfurtherdimensionalityreduction}

A naive approach to numerically solving the solid phase diffusion equation is to
discretise each of the two representative particles in the radial direction~$r$.
Given the elaborate  simplifications made to remove spatial  resolution from the
axial  direction, the  efficacy of  using  a radial  discretisation is  rendered
questionable, particularly within the scope of  embedding the model in an online
simulation and  state-estimation environment. Since diffusion  in each spherical
particle is modelled by the well-known Fickian dynamics~\cite{Fick1995}, several
attempts have  been made to  obtain an  approximate analytical solution  for the
solid phase concentration in both electrodes.
% In  the  context  of  \gls{spm}  modelling, the  earliest  such  work \ie~a
% comparison of  the discretised version  with an approximate  analytical solution
% was performed by Santhanagopalan~\etal~\cite{Santhanagopalan2006}.
In a  dimensionless analysis study,  Zhang and White~\cite{Zhang2007}  provide a
comparative  evaluation of  the various  approximation methods  for solid  phase
diffusion, a summary of which is presented in \cref{tbl:solidphaseapprox}.

% -*- root: ../main.tex -*-
%!TEX root = ../main.tex

\begin{table}[!htbp]
    \caption[Solid phase diffusion approximation methods]{Summary of approximation methods for solid phase diffusion}
    \label{tbl:solidphaseapprox}
    \centering
    \begin{tabular}{@{}ll@{}}\toprule
        Method                     & Introduced by                                             \\ \midrule
        Duhamel's superposition    & Doyle, Fuller \&  Newman~\cite{Doyle1993,Fuller1994}      \\
        Diffusion length           & Wang~\etal~\cite{Wang1998}                                \\
        Corrected Diffusion length & Wang and Srinivasan~\cite{Wang2002}                 \\
        Polynomial approximation   & Subramanian~\etal~\cite{Subramanian2001a,Subramanian2005} \\
        Pseudo steady-state        & Liu~\cite{Liu2006}                                        \\
        \bottomrule
    \end{tabular}
\end{table}


In the  aforementioned study, the computational  burden \ie~storage requirements
and CPU times of Duhamel's superposition method was found to be excessively high
to warrant further interest in it. The original diffusion length method proposed
by Wang~\etal{} is valid only after the  diffusion layer builds up to its steady
state, and hence  leads to significant errors in  transient conditions. Although
Wang and Srinivasan  introduced an empirical correction factor  to the diffusion
length to extend its validity to  short-time scale operations, this affected the
convergence of the  method for steady state conditions. The  pseudo steady state
solution proposed by Liu uses a finite integral transform technique to eliminate
the  radial  dependence  of  solid phase  concentration.  However,  this  method
uses computations  involving infinite summations, exponential  and trigonometric
quantities, which  in this thesis  author's view,  makes it less  attractive for
online implementations.

The         literature         on          polynomial         methods         by
Subramanian~\etal{}~\cite{Subramanian2005} provide  detailed derivations  of the
\engordnumber{2}~and   \engordnumber{4}~order  polynomial   approximations.  The
\engordnumber{2}~order solution was found to have poor performance for transient
behaviour, similar  to that  of the original  diffusion length  method. However,
higher order polynomial  approximations were found to  provide acceptable levels
of  performance  for  both  transient  and steady  state  conditions  and  shall
therefore be examined further.

The  polynomial   approximation  method  describes  the   dynamic  evolution  of
the  volume  averaged concentration
\begin{equation}
    c_\sjavg(t)  = \frac{1}{Ω}  ∫\limits_Ω c_\sj(r,t)\,  dΩ
\end{equation}
as  a function  of the  applied load  current~$I(t)$. Here,  $Ω$~represents the
volume  of  the spherical  particle.  For  notational brevity,  $c_\sjavg(t)$~is
shortened to~$\mean{c}_\sj$ whilst also dropping its explicit time dependence.

The \engordnumber{4}~order polynomial approximation assumes that the solid phase
concentration~$c_\sj(r,t)$ is a quartic function of the radial co-ordinate~$r$.
\begin{equation}\label{eq:fourthorderpoly}
    c_\sj(r,t) = a(t) + b(t)\left(\frac{r}{R_\pj}\right)^2 + d(t) \left(\frac{r}{R_\pj}\right)^4
\end{equation}

The    detailed   derivation    of   the    coefficients~${a(t),    b(t)
\text{ and } c(t)}$     is     provided    in     Subramanian~\etal{}~\cite{Subramanian2005}.
\Cref{eq:csmeanevolution,eq:qmeanevolution,eq:csurffromcsavg}    summarise   the
governing equations  obtained by applying the  \engordnumber{4}~order polynomial
approximation of \cref{eq:fourthorderpoly}  to the system of  equations given by
\cref{eq:csdiffusionfinal,eq:csfluxcentre,eq:csfluxsurfacefinal}.
\begingroup
\allowdisplaybreaks
\begin{align}
    \diff*{\mean{c}_\sj}{t} + 3\frac{j_\nj}{R_\pj}                                                &=0 \label{eq:csmeanevolution} \\
    \diff*{\mean{q}_j}{t} + 30\frac{D_\sj}{R_\pj^2}\mean{q}_j + \frac{45}{2}\frac{j_\nj}{R_\pj^2} &=0 \label{eq:qmeanevolution}\\
    35\frac{D_\sj}{R_\pj}\left(c_\sjsurf - \mean{c}_\sj\right) - 8D_\sj \mean{q}_j                &= -j_\nj \label{eq:csurffromcsavg}
\end{align}%
\endgroup
where $\mean{q}_j(t)$~represents  the volume  averaged concentration  flux, that
defines  the  average  change  of  concentration  with  respect  to  the  radial
position~$r$.

As   per  \cref{eq:uniformcurrdensity},   the   interfacial   flux  density   is
proportional to  the applied current. Hence  \cref{eq:csmeanevolution} implies a
simple linear relationship between the applied current and the rate of evolution
of average \ch{Li^0} concentration within  each spherical particle. This further
implies  that the  \gls{soc} of  the cell  has a  linear rate-dependence  on the
externally  applied  current. Furthermore,  due  to  the elimination  of  radial
discretisation, the computation of \gls{soc}, given by \cref{eq:socinitialdefn},
reduces to the task of first computing the ratio of bulk (average) concentration
to  surface concentration  and  then adjusting  it to  account  for the  useable
stoichiometry limits for the relevant electrode. Thus, the cell's \gls{soc}
% as predicted by the \gls{spm}
can be computed as
\begin{equation}\label{eq:soccomputation}
    z = \frac{\tfrac{\mean{c}_\sneg}{c_\snegmax} - θ_\minneg}{θ_\maxneg - θ_\minneg}
\end{equation}
where $\mean{c}_\sneg$~is obtained by  solving \cref{eq:csmeanevolution} for the
negative electrode.

In   the  views   of   this  author,   this  \engordnumber{4}~order   polynomial
approximation   proposed  by   Subramanian~\etal~\cite{Subramanian2005}  strikes
an  acceptable  balance  between   the  three  modelling  pivots ---
\begin{enumerate*}[label=\roman*)]
    \item computational complexity,
    \item mathematical  tractability, and
    \item numerical accuracy
\end{enumerate*}
and has therefore  been adopted for all \gls{spm} simulations  presented in this
work.

At   the   end   of    this   dimension-reduction   step,   spatial   dependence
is   completely  eliminated,   yielding   a  zero-order   (in  space)   physical
model    whose    dynamics   are    described    by    the   \gls{dae}    system
of \cref{eq:csmeanevolution,eq:qmeanevolution,eq:csurffromcsavg}.

% Thermodynamics, kinetics and state-space formulation  of the model are presented
% next in \cref{subsec:basicspmthermodynamics}.

\subsubsection*{Equilibrium Thermodynamics}\label{subsec:basicspmthermodynamics}

The equilibrium potential of a porous electrode is a thermodynamic property that
depends on the extent of lithiation in the outermost interstitial sites near the
\gls{sei} layer. This surface stoichiometry~$θ_j$  for an electrode is obtained
by  computing the  surface  concentration  (using \cref{eq:csurffromcsavg})  and
dividing by the maximum lithiation capacity of that electrode.
\begin{equation}
    θ_j = \frac{c_\sjsurf}{c_\sjmax}
\end{equation}

Although based upon the theoretical foundation  laid out by the Nernst equation,
owing  to a  multitude of  complex phase  transitions, the  potential of  porous
electrodes  (with respect  to metallic  lithium) is  usually given  as empirical
functions of its surface stoichiometry~\cite{Reddy2011,Rahn2013}.
\begin{equation}\label{eq:ocpstoichiometry}
    U_j(t) = \mathcal{U}_j\left(θ_j(t)\right)
\end{equation}
where  the  empirical  relationships~$\mathcal{U}_j$ are  typically  high  order
polynomials  or rational  functions  that  are fitted  to  relaxation data  from
\gls{gitt} experiments on half-cells~\cite{Birkl2015a,Ecker2015}.

In the \gls{spm},  the cell's \gls{ocp} is obtained by  subtracting the negative
electrode  equilibrium  potential~$U_\text{neg}$  from  its  positive  electrode
counterpart~$U_\text{pos}$.
\begin{equation}\label{eq:ocpdefinition}
    U_\text{ocp} = U_\text{pos} - U_\text{neg}
\end{equation}
Even though the  concept of \gls{ocp} is defined only  in equilibrium conditions
when no  current flows,  the individual electrode  potentials themselves  form a
significant component of the cell's terminal voltage~$V(t)$.

\subsubsection*{Reaction Kinetics}

In the \gls{spm}, the reaction kinetics in each spherical electrode is modelled
using the Butler-Volmer expression (see \cref{eq:butlervolmer}).
\begin{align}
    j_\nj   &= j_{0_j} \left[ \exp\left( \frac{\left(1-α\right) F η_j}{R T}\right) -  \exp\left( \frac{-α F η_j}{R T}\right)\right] \label{eq:bvwithalpha} \\
    \shortintertext{where}
    j_{0_j} &= k_\jr c_\text{e}^{1-α} c_\sjsurf^{α} \left(c_\sjmax - c_\sjsurf\right)^{1-α}
\end{align}

The equilibrium  rate of forward  and backward  reactions at both  electrodes is
assumed  to  be  equal.  With charge  transfer  coefficient~${α  =  0.5}$,
\cref{eq:bvwithalpha} simplifies to
\begin{equation}\label{eq:BVwithalphahalf}
    j_\nj = 2 k_\jr \sqrt{c_\text{e} c_\sjsurf \left(c_\sjmax - c_\sjsurf\right)} \sinh\left(\frac{F η_j}{2 R T}\right)
\end{equation}

The  expression   for  overpotential~$η_j$   can  be   obtained  rearranging by
\cref{eq:BVwithalphahalf}     whilst     substituting    for     $j_\nj$~from by
\cref{eq:uniformcurrdensity} and is given                                     by
\begin{equation}\label{eq:overpotential_j}
    η_j(t) =  \frac{2 R T}{F }\sinh^{-1} \left( \frac{± I(t)}{2 A \, l_j a_\sj F k_\jr \sqrt{c_\text{e} c_\sjsurf \left(c_\sjmax - c_\sjsurf\right)}}\right)
\end{equation}

\subsubsection*{Cell Terminal Voltage}\label{subsec:basicspmcellterminalvoltage}

The terminal voltage  of the cell under applied load  is obtained by subtracting
the potential of the negative electrode from its positive counterpart.

Starting from the definition of the overpotential of each electrode
\begin{align}
    η_\text{pos} &= ϕ_\spos - \cancelto{0}{ϕ_\epos} - U_\text{pos} \label{eq:posoverpotential} \\
    η_\text{neg} &= ϕ_\sneg - \cancelto{0}{ϕ_\eneg} - U_\text{neg} \label{eq:negoverpotential}
\end{align}
Within  each electrode  domain,  the contribution  of  electrolyte potential  is
neglected  (see \cref{subsec:basicspmgeometry,subsec:basicspmassumptions}  for a
brief discussion on the exclusion of electrolyte dynamics).

Subtracting \cref{eq:negoverpotential}   from \cref{eq:posoverpotential},
% whilst substituting for $U_j$ from \cref{eq:ocpstoichiometry}
\begin{align}
    η_\text{pos} - η_\text{neg} &= \underbrace{ϕ_\spos - ϕ_\sneg}_{V_\text{cell}} - U_\text{pos} + U_\text{neg}\label{eq:overpotentialdifference}\\
\shortintertext{whose rearrangement yields}
    V_\text{cell}               &= η_\text{pos} - η_\text{neg} + U_\text{pos} - U_\text{neg}\label{eq:cellterminalvoltagebasic}
\end{align}
% \mathcal{U}_\text{pos}\left(θ_\text{pos}\right) - \mathcal{U}_\text{neg}\left(θ_\text{neg}\right)
In the  basic \gls{spm},  \cref{eq:cellterminalvoltagebasic} is used  to compute
the cell's terminal voltage under  load. Although their explicit time-dependence
notation  is omitted  in  the notation  here,  it is  worth  reminding that  all
quantities in \cref{eq:cellterminalvoltagebasic} are indeed continuous functions
of time.

\subsubsection*{State Space Representation}\label{subsec:basicspmstatespace}

For control  oriented applications, it is  imperative to have a  classical state
space representation  that collates  all intermediate equations  and definitions
presented  thus  far into  a  single  system  of  equations that  describes  the
evolution of solid concentration and terminal  voltage over time, expressed as a
response to the  external current input~$I(t)$. However,  the non-linearities in
the equation for terminal voltage \ie~\cref{eq:cellterminalvoltagebasic} imply
that it is  not possible to represent  the \gls{spm} in the form  of a classical
\gls{lti}  system  of \cref{eq:LTIstatespace}.  Instead,  the  \gls{spm} can  be
summarised  by a  system  of linear  state equations  together  with the  single
non-linear output equation.

Rearranging  \cref{eq:qmeanevolution,eq:csurffromcsavg}, the  state equation  is
obtained as
\begin{equation}\label{eq:fourstatesmatrixvec}
    \setstackgap{L}{1.5\baselineskip}
    \fixTABwidth{T}
    \diff*{\parenMatrixstack{
            \vphantom{\frac{45}{2} \frac{1}{R_\ppos^2 A \, l_\text{pos} a_\spos F}}
            \mean{q}_\text{pos} \\
            \vphantom{\frac{45}{2} \frac{1}{R_\ppos^2 A \, l_\text{pos} a_\spos F}}
            % \vphantom{\frac{D_\sneg}{R_\pneg^2}}
            \mean{q}_\text{neg} \\
            \vphantom{\frac{45}{2} \frac{1}{R_\ppos^2 A \, l_\text{pos} a_\spos F}}
            % \vphantom{\frac{D_\spos}{R_\ppos^2}}
            \mean{c}_\spos \\
            \vphantom{\frac{45}{2} \frac{1}{R_\ppos^2 A \, l_\text{pos} a_\spos F}}
            % \vphantom{\frac{D_\sneg}{R_\pneg^2}}
            \mean{c}_\sneg
        }
    }{t}
    = \underbrace{\parenMatrixstack{
            -30\frac{D_\spos}{R_\ppos^2} & 0                            & 0 & 0 \\
            0                            & -30\frac{D_\sneg}{R_\pneg^2} & 0 & 0 \\
            \vphantom{\frac{45}{2} \frac{1}{R_\ppos^2 A \, l_\text{pos} a_\spos F}}
            0                            & 0                            & 0 & 0 \\
            \vphantom{\frac{45}{2} \frac{1}{R_\ppos^2 A \, l_\text{pos} a_\spos F}}
            0                            & 0                            & 0 & 0
    }}_{A}
    \parenMatrixstack{
        \vphantom{\frac{D_\spos}{R_\ppos^2}}
        \mean{q}_\text{pos} \\
        \vphantom{\frac{D_\sneg}{R_\pneg^2}}
        \mean{q}_\text{neg} \\
        \vphantom{\frac{D_\spos}{R_\ppos^2}}
        \mean{c}_\spos \\
        \vphantom{\frac{D_\sneg}{R_\pneg^2}}
        \mean{c}_\sneg
    }
    +
    \underbrace{\parenMatrixstack{
            \frac{45}{2} \frac{\hphantom{-}1}{R_\ppos^2 A \, l_\text{pos} a_\spos F} \\
            \frac{45}{2} \frac{-1}{R_\pneg^2 A \, l_\text{neg} a_\sneg F} \\
            \hphantom{\frac{45}{2}} \frac{\hphantom{-}3}{R_\ppos  A \, l_\text{pos} a_\spos F} \\
            \hphantom{\frac{45}{2}} \frac{-3}{R_\pneg  A \, l_\text{neg} a_\sneg F}
    }}_{B}
    I(t)
\end{equation}
which corresponds to the classical \gls{lti} form
\begin{equation}
    \dot{\mathbf{x}} = A\,\mathbf{x} + B\,\mathbf{u} \\
\end{equation}
where~${\mathbf{x}   =    \vect{\mean{q}_\text{pos},\mean{q}_\text{neg},
\mean{c}_\spos,  \mean{c}_\sneg},   \,  x  ∈  \mathbb{R}^{4   \times  1}}$  is
the  state  vector.  The  scalar system  input~$\mathbf{u}  ∈  \mathbb{R}$  is
the  applied  current~$I(t)$.  The  system matrix~${A  ∈  \mathbb{R}^{4  \times
4}}$  and  input  matrix~$B  ∈  \mathbb{R}^{4 \times  1}$  are  also  shown  in
\cref{eq:fourstatesmatrixvec}.

For state estimation and controller design purposes, it is important to keep the
number  of elements  in the  state vector  as small  as possible  by eliminating
redundant  variables.  For  instance,  Di~Dominico~\etal{}~\cite{DiDomenico2010}
noted that with  output voltage as the only measured  quantity, the observablity
of   the  four-state   model  of   \cref{eq:fourstatesmatrixvec}  is   adversely
affected.  To tackle  this issue,  a  state-reduction approach  was proposed  by
Di~Domenico~\etal~\cite{DiDomenico2010},  which  hinges  upon the  principle  of
material balance.

The total number of moles of lithium in the system is given by
\begin{equation}\label{eq:totallithiummoles}
    n_\text{Li} = \frac{ε_\spos \, l_\text{pos}\, A}{\frac{4}{3} π R_\ppos^3} ∫_0^{R_\ppos} 4 π r^2 c_\spos(r,t) \, dr
    +  \frac{ε_\sneg \, l_\text{neg}\, A}{\frac{4}{3} π R_\pneg^3} ∫_0^{R_\pneg} 4 π r^2 c_\sneg(r,t) \, dr
\end{equation}
Upon considering only the bulk concentration as per the dimensionality reduction
procedure    outlined   in \cref{subsec:basicspmfurtherdimensionalityreduction},
\cref{eq:totallithiummoles} reduces to
\begin{align}\label{eq:totallithiumsimplified}
    n_\text{Li}  &= \frac{ε_\spos \, l_\text{pos}\, A}{\frac{4}{3} π R_\ppos^3}\mean{c}_\spos ∫_0^{R_\ppos} 4 π r^2  \, dr
    + \frac{ε_\sneg \, l_\text{neg}\, A}{\frac{4}{3} π R_\pneg^3}\mean{c}_\sneg ∫_0^{R_\pneg} 4 π r^2  \, dr
                \\
                 &= ε_\spos \, l_\text{pos}\, A \, \mean{c}_\spos + ε_\sneg \, l_\text{neg}\, A \, \mean{c}_\sneg
\end{align}
Assuming    no    loss   of    cycleable    lithium    or   other    degradation
mechanisms,  the   total  number  of   moles  of   lithium  in  the   system  is
conserved   \ie~${\diff{n_\text{Li}}{t}   =  0}$.   Substituting   this   into
\cref{eq:totallithiumsimplified},
\begin{align}
    0                          &= \phantom{+} \diff*{ε_\spos \, l_\text{pos}\, A \, \mean{c}_\spos }{t} + \diff*{ε_\sneg \, l_\text{neg}\, A \, \mean{c}_\sneg }{t} \\
    \diff*{\mean{c}_\spos}{t}  &= -\diff*{\mean{c}_\sneg}{t} \label{eq:bulkconcrelationship}
\end{align}

As  per   \cref{eq:bulkconcrelationship},  the   time  evolution  of   the  bulk
concentration  of one  electrode can  be obtained  as a  function of  the other.
Furthermore, Di~Domenico~\etal{}~\cite{DiDomenico2010}  show that  the diffusion
dynamics of the  bulk concentrations can be algebraically  related through their
stoichiometric factors as
\begin{equation}\label{eq:csposbulkfromcsnegbulk}
    \mean{c}_\spos(t) = c_\sposmax \, \bigg[\frac{\mean{c}_\sneg(t)- θ_\minneg
    c_\snegmax}{\left(θ_\maxneg - θ_\minneg\right)c_\snegmax} \left(θ_\maxpos - θ_\minpos \right) + θ_\minpos \bigg]
\end{equation}

Hence, it  is possible  to eliminate the  bulk concentration of  any one  of the
electrodes from the state-equation to arrive at a three-state description of the
model dynamics. In  extant lithium-ion chemistries, owing to  its proclivity for
lithium deposition during  charging, the negative electrode is  considered to be
the limiting electrode (See  Arora~\etal~\cite{Arora1999}). Hence it is retained
in the state vector, thereby leading to  the final form of the state dynamics of
the conventional \gls{spm} as
\begin{equation}\label{eq:threestatesmatrixvec}
    \setstackgap{L}{1.5\baselineskip}
    \fixTABwidth{T}
    \diff*{\parenMatrixstack{
            \vphantom{\frac{45}{2} \frac{1}{R_\ppos^2 A \, l_\text{pos} a_\spos F}}
            \mean{q}_\text{pos} \\
            \vphantom{\frac{45}{2} \frac{1}{R_\ppos^2 A \, l_\text{pos} a_\spos F}}
            \mean{q}_\text{neg} \\
            \vphantom{\frac{45}{2} \frac{1}{R_\ppos^2 A \, l_\text{pos} a_\spos F}}
            \mean{c}_\sneg
        }
    }{t}
    = \underbrace{\parenMatrixstack{
            -30\frac{D_\spos}{R_\ppos^2} & 0                            & 0  \\
            0                            & -30\frac{D_\sneg}{R_\pneg^2} & 0  \\
            \vphantom{\frac{45}{2} \frac{1}{R_\ppos^2 A \, l_\text{pos} a_\spos F}}
            0                            & 0                            & 0
    }}_{A}
    \parenMatrixstack{
        \vphantom{\frac{D_\spos}{R_\ppos^2}}
        \mean{q}_\text{pos} \\
        \vphantom{\frac{D_\sneg}{R_\pneg^2}}
        \mean{q}_\text{neg} \\
        \vphantom{\frac{D_\spos}{R_\ppos^2}}
        \mean{c}_\sneg
    }
    +
    \underbrace{\parenMatrixstack{
            \frac{45}{2} \frac{\hphantom{-}1}{R_\ppos^2 A \, l_\text{pos} a_\spos F} \\
            \frac{45}{2} \frac{-1}{R_\pneg^2 A \, l_\text{neg} a_\sneg F} \\
            \hphantom{\frac{45}{2}} \frac{-3}{R_\pneg  A \, l_\text{neg} a_\sneg F}
    }}_{B}
    I(t)
\end{equation}

The  measured   variable~${y  ∈  \mathbb{R}}$  is   the  cell's  terminal
voltage~$V(t)$ and  is expressed as  a non-linear  scalar function of  the state
vector and the load current.
\begin{equation}\label{eq:spmoutputeqn}
    y = h\left(\mathbf{x}(t),u(t)\right)
\end{equation}
The output  equation given by \cref{eq:spmoutputeqn} includes  a non-zero direct
feedthrough dependency  of the voltage  on the input current,  thereby modelling
the resistive component of the cell's  impedance. The full expression for output
voltage is given by expanding \cref{eq:cellterminalvoltagebasic} as

\begin{multline}
    V_\text{cell}(t) = \frac{2 R T}{F }\sinh^{-1} \left( \frac{- I(t)}{2 A
    l_\text{pos} a_\spos F k_\posr \sqrt{c_\text{e} c_\spossurf(t)
    \left(c_\sposmax - c_\spossurf(t)\right)}}\right) \\
    - \frac{2 R T}{F }\sinh^{-1} \left( \frac{I(t)}{2 A \, l_\text{neg} a_\sneg F
    k_\negr \sqrt{c_\text{e} c_\snegsurf(t) \left(c_\snegmax - c_\snegsurf(t)\right)}}\right) \\
    + \mathcal{U}_\text{pos}\left(c_\spossurf(t)\right) -
    \mathcal{U}_\text{neg}\left(c_\snegsurf(t)\right)\label{eq:spmbasicoutputvoltagefinal}
\end{multline}
wherein the solid  phase surface concentration at  each electrode~$c_\sjsurf$ is
obtained  from its  corresponding bulk  concentration~$c_\sjavg$ by  rearranging
\cref{eq:csurffromcsavg} and is given by
\begin{align}
    c_\spossurf &= \mean{c}_\spos  + \frac{8R_\ppos}{35} \mean{q}_\text{pos}
    +\frac{R_\ppos}{35 D_\spos A \, l_\text{pos} a_\spos F} I(t)
    \label{eq:csurfposfromcavgpos}\\
    c_\snegsurf &= \mean{c}_\sneg  + \frac{8R_\pneg}{35} \mean{q}_\text{neg} -\frac{R_\pneg}{35 D_\sneg A \, l_\text{neg} a_\sneg F} I(t)\label{eq:csurfnegfromcavgneg}
\end{align}
where~${I(t) > 0}$ for discharge.

Given the initial  \gls{soc} of the cell~$z(0)$, the  initial bulk concentration
of   the  negative   electrode  at   equilibrium~$c_\sneg(0)$  is   obtained  by
\cref{eq:csfluxinitialcondition}.  The   initial  value   of  the   mean  radial
concentration flux  in both  electrodes is zero  \ie~${q_j(0) =  0}$. Therefore,
the  initial  state  vector  is~$\vect{0,0,c_\sneg(0)}$.  Thus,  the  system  of
equations given by \crefrange{eq:csposbulkfromcsnegbulk}{eq:csurfnegfromcavgneg}
form a complete  state-space representation of the  conventional \gls{spm}. This
state-space model  can be  simulated as  a standalone  \gls{ivp} or  embedded as
the  plant model  in control-oriented  applications  such as  for dynamic  state
estimation.


\section{Numerical Implementation}\label{sec:numericalimplementation}
% -*- root: ../../main.tex -*-
%!TEX root = ../../main.tex
% this file is called up by main.tex
% content in this file will be fed into the main document
% vim:textwidth=80 fo=cqt

The  equations  presented   in  \cref{sec:spmmodeldevelopment}  are  well-known,
self-sufficient and  fully descriptive so  as to implement the  basic \gls{spm}.
Although discrete-time numerical implementation  of circuit-oriented cell models
has been  considered~\cite{Plett2004,Plett2004a,Plett2004b,Plett2006}, there has
been  no such  treatment  of this  critical aspect  in  the \gls{spm}  modelling
literature. Since  this thesis has  a strong focus  towards enabling the  use of
physics-based models in an embedded  environment, at least the numerical aspects
of implementing  these equations  need to  be discussed.  The finer  details and
practical engineering considerations of real-time programming, in particular the
integration  of  the  cell  model  into  the  pack,  interaction  with  upstream
components and other such aspects  of a typical vehicular drivetrain controller,
are beyond  the scope of this  academic work. Nevertheless, the  discussion here
aims  to  lower  the  barrier  to  real-time  implementation  and  is  a  unique
contribution in the implementation context of cell models.

\subsection{Continuous-time Implementation}
\subsubsection*{Analytical solution}

Although  not   explicitly  given  in  \gls{spm}   literature,  using  \gls{lti}
system  theory,  the  analytical  solution for  continuous-time  state  equation
(\cref{eq:LTIstatespace}) with current input\footnote{The analytical closed form
solution cannot be obtained for constant  voltage operation. This is because the
boundary flux is implicitly determined  by the non-linear Butler-Volmer equation
\cref{eq:butlervolmer} and needs to be solved numerically with some variant of a
Newton-type iteration scheme.} is given by
\begingroup
\allowdisplaybreaks
% \setlength{\abovedisplayskip}{0.9765625ex}
% \setlength{\abovedisplayskip}{0.1765625ex}
% \setlength{\belowdisplayskip}{0ex}
\begin{align}
    % \SwapAboveDisplaySkip
    \mathbf{x}(t) &= e^{A (t-t_0)}\mathbf{x}(t_0) + \int_{t_0}^{t}e^{A (t-τ)}B \mathbf{u}(τ)\,dτ \label{eq:genanalyticctssoln}
    \\
    \shortintertext{With a standard \gls{ivp},~${t_0 = 0}$}
    \mathbf{x}(t) &= e^{A t}\mathbf{x}(0) + \underbrace{\int_{0}^{t}e^{A (t-τ)}B \mathbf{u}(τ)\,dτ}_{\text{convolution integral}}\label{eq:analyticalctssoln}
\end{align}
\endgroup

The matrix exponential~$e^{At}$ is known as the state-transition matrix and is
defined as
\begin{equation}
    e^{A t} ≜ \mathcal{L}^{-1}\left\{(s I - A)^{-1}\right\}
\end{equation}
although several methods exist for its efficient numeric
computation~\cite{Moler2003}.

The analytical solution  given by \cref{eq:analyticalctssoln} can  be applied to
obtain the  matrix-vector state  equation \cref{eq:threestatesmatrixvec}  of the
\gls{spm}. Once  the state variables  are obtained  at a given  time-step, after
evaluating the  surface concentrations as per  \cref{eq:csurfposfromcavgpos} and
\cref{eq:csurfnegfromcavgneg}, they may be  substituted into the output equation
of \cref{eq:spmbasicoutputvoltagefinal}  to obtain  the cell's  terminal voltage
for that time-step.

\subsubsection*{Numerical considerations for continuous time implementation}

The  procedure  described  thus  far  has  a  practical  limitation.  The  input
current~$I(t)$ to the  cell has been defined as a  continuous quantity. Although
for the  purpose of characterising the  cell's behaviour, it is  possible to use
pre-determined continuous-functions  as load profiles  (\eg~sinusoidal waveforms
for virtual \gls{eis} testing), it is desirable to evaluate the model's response
to typical real-life conditions. In a vehicular application, only \emph{samples}
of cell current  measured by sensors at discrete-time intervals  are reported to
the \gls{bms}. A \gls{zoh} operation is  used at the model's input \ie~the level
of current is assumed to be held constant between two successive measurements.

% Current profiles can be computed from standard drive cycle data.

It    is    tedious    to    hand-compute   the    convolution    integral    of
\cref{eq:analyticalctssoln}.   However,  a   variety   of  state   of  the   art
adaptive-time  solvers  employing  numerical  schemes  such  as  Dormand-Prince,
Runge-Kutta,  Collocation and  Backward  Differentiation  Formula are  available
to  efficiently  handle  such  \glspl{ode}. Since  lithium  concentrations  vary
smoothly  over  time  without   abrupt  discontinuities,  a  standard  non-stiff
solver  of  moderate   order  shall  suffice.  A  single  line   of  code  using
\textsc{MATLAB}'s  ode45  solver  can  implement this  time  integration  \eg~\\
\mintinline{matlab}{[~,x_new] =  ode45(@(t,x) stateEqn(x,Ik,spm_params), t_span,
x_old); }

Since  the direction  of applied  current  is susceptible  to sudden  reversals,
(\eg~due  to  acceleration  and  braking events  in  a  vehicular  application),
the  solver   needs  to   be  stopped  and   re-started  every   sample  period.
\Cref{alg:ctstimespm} shows the sequence of  operations for a desktop simulation
of  the  continuous time  \gls{spm}  on  a  digital  computer. The  source  code
listing  of  an  example  implementation   in  \textsc{MATLAB}  is  provided  in
\cref{sc:ctstimespm}.

% \vspace{3ex}
% -*- root: ../main.tex -*-
%!TEX root = ../main.tex
% this file is called up by main.tex
% content in this file will be fed into the main document
% vim:textwidth=80 fo=cqt

\vspace{-1em}
\noindent\makebox[\textwidth][c]{%
    \begin{customalgo}[1.0\textwidth]{Continuous time \gls{spm}}{alg:ctstimespm}
        \Require Load profile \Comment{\eg{} a \texttt{csv} file of $t$ vs. C-rate}
        \Require \gls{spm} parameter set  \Comment{\eg{} stored in a struct \texttt{params}}
        \Userdata $z[0], t_\text{f,user}$, $t_\text{f,condition}$, cell capacity $I_\text{1C}$, sample rate $T_s$ \Comment{$t_\text{f,condition} \in  \left\{\texttt{max}, \texttt{min}\right\}$}
        \Ensure  $z[0], V_\text{cell}[0]$ within limits
        \Procedure{Simulate\gls{spm}}{}%{$z[0],t_\text{f,desired},T_s,I_\text{1C},\texttt{params}$}
        \State {$t_\text{f,desired} =
            % \scriptstyle
            \begin{cases}
                % \scriptstyle
                \max(t_\text{f,user},t_\text{f,profile}),
                &\text{%\scriptsize
            if $t_\text{f,condition}$ == \texttt{max};}\\
            % \scriptstyle
            \min(t_\text{f,user},t_\text{f,profile}),
            &\text{%\scriptsize
        otherwise.}
    \end{cases}$} \Comment{\parbox[t]{0.25\textwidth}{may terminate early due to
cut-off violations}}
% }}
        \FullComment{\scriptsize Flexible end time. Extrapolate last C-rate from
        profile until $t_\text{f,desired}$ if necessary.}
        \State $N_\text{max} \gets \ceil*{\frac{t_\text{f,desired}}{Ts}} + 1$
        \Comment{max iterations assuming no cut-offs}
        \State Allocate storage of size $\mathbb{R}^{N_\text{max}\times 1}$ for each simulation variable
        \State Compute $\mean{c}_\sneg$[0] as per~\cref{eq:csfluxinitialcondition}
        \State $I[0] \gets I_\text{1C} \times  \text{C-rate}[0], \quad \mathbf{x}[0] \gets \vect{0,0, \mean{c}_\sneg[0]}$ \Comment{ $\text{C-rate}[0]$ from profile}
        % \State $\mathbf{x}[0] \gets \vect{0,0, \mean{c}_\sneg[0]}$
        \State $V_\text{cell}[0] \gets
        \textsc{ComputeCellVoltage}(\textbf{x}[0],I[0],\texttt{params})$
        \Comment{from direct feedthrough}
        \For{$k \gets 2 : N_\text{max}$}
        \State $I[k] \gets $ interpolate from profile using \gls{zoh}
        \State Solve continuous-time equation~\cref{eq:threestatesmatrixvec}
        \Comment{solver IC set to $x[k-1]$}
        \State $\mathbf{x}[k] \gets $ last time-entry  vector of
        soln.\  matrix
        \Comment{from an adaptive solver \eg{} MATLAB's \texttt{ode45}}
        \State Compute $z[k]$ as per~\cref{eq:soccomputation}
        \State $V_\text{cell} \gets \textsc{ComputeCellVoltage}(\textbf{x}[k],I[k],\texttt{param}) $
        \If {$z[k] \text{or} V_\text{cell}[k]$ exceeded cut-off limits}
            \State $k \gets k - 1$ \Comment{data from last  step is invalid}
            \State \textit{break};
            \EndIf
        \EndFor
        \EndProcedure

        \OutputEqn{\textbf{x},I,\texttt{params}}
        \State Compute $c_\snegsurf$ as per~\cref{eq:csurfnegfromcavgneg}
        \Comment{consider saturating \ie{} $c_\snegmin \le c_\snegsurf \le
        c_\snegmax$}
        \State Compute $\mean{c}_\spos$ as per~\cref{eq:csposbulkfromcsnegbulk}
        \State Compute $c_\spossurf$ as per~\cref{eq:csurfposfromcavgpos}
        \State Compute $V_\text{cell}$ as per~\cref{eq:spmbasicoutputvoltagefinal}
        \EndOutputEqn%
    \end{customalgo}
}%


In  the author's  view,  the  continuous time  \gls{spm}  algorithm has  limited
practical  use.  Computing  the  convolution  integral  or  deploying  \gls{ode}
solver codes  onto a microcontroller  is challenging and  introduces substantial
computational burden. Although the continuous time model can be used for desktop
simulation, more sophisticated \glspl{pbm} are  already available for this task.
Therefore,  for online  deployment  in  state estimation  and  control tasks,  a
discrete-time  version of  the model  suitable for  real-time implementation  is
needed.

% Nevertheless,  the  continuous time  \gls{spm}  has  one important  application,
% \viz~estimation of physical parameters.
% which will  explored  in \cref{sec:spmparameterestim}.

\subsection{Conceptual Overview of Real-Time Processing}

The   equations   in    \cref{subsec:basicspmgoverningeqns}   are   derived   in
continuous   time   form.  In   particular,   the   state  equation   given   by
\cref{eq:threestatesmatrixvec} describes  the continuous time  dynamic evolution
of  quantities such  as the  bulk concentration  and rate  of mean  radial flux.
However,  a  typical embedded  controller  such  as  that  used in  a  vehicular
\gls{bms}  operates   in  discrete-time~\cite{Andrea2010}.  This   implies  that
\emph{samples} of voltage, current and  temperature measurements are obtained at
a periodic time interval~$T_s$. The updating of solution variables are performed
between two successive data acquisition events from the sensors.

% The  computations of  the  model  equations and  updates  of solution  variables
% (such as  bulk concentrations  and terminal voltage)  are performed  between two
% successive data acquisition events from the sensors.


Execution of  control-oriented reduced-order  \glspl{pbm} such as  the \gls{spm}
and  their  associated  computations  are   modular  sub-tasks  of  a  vehicular
\gls{bms}.  A   single  \gls{bms}   often  provides  a   whole  host   of  other
auxiliary  functionality such  as  cell~balancing,  protection, diagnostics  and
data-logging~\cite{Plett2016}. Although  thermal management tasks  are typically
delegated to dedicated controllers, the \gls{bms} software routines often handle
data exchanged  between various subsystems  on the vehicular  communication bus.
While some of these tasks such  as book-keeping and diagnostics can be performed
at  a low  execution-rate, others  such as  cell measurements  and model-related
computations need to be performed with high priority.

\begin{figure}[!tbp]
    \savebox{\algboxA}{%
        \begin{varwidth}[b]{0.65\linewidth}
            \begin{flushleft}
                \vspace*{1.5ex}
                \begin{algorithmic}[0]

                    \Initialise \gls{soc} \& other global variables
                    \Ensure voltage, current \& temperature limits
                    \Procedure{Main}{$ $}

                    \State configure interrupts
                    \State enable timers
                    \State $\vdots$
                    \While{\textproc{True}} \Comment[\scriptsize]{until ``key off" or shutdown}
                    \State background task \#1 \Comment[\scriptsize]{diagnostics/protection}
                    \State background task \#2 \Comment[\scriptsize]{\textproc{canbus} communication}
                    \State $\vdots$
                    \If{\texttt{needs\textunderscore balancing == 1}}
                    \Function{PackBalance}{$n_\text{cells}$,$\text{\gls{soc}}_i$,$v_i$}
                    \State \textit{subroutine for pack balancing}
                    \State $\vdots$
                    \EndFunction
                    \EndIf
                    \State $\vdots$
                    \State background task \#$n$ \Comment[\scriptsize]{supervisory reporting}
                    \EndWhile
                    \EndProcedure
                \end{algorithmic}
            \end{flushleft}%
        \end{varwidth}%
    }%
    \savebox{\algboxB}{%
        \begin{varwidth}[b]{0.65\linewidth}
            \begin{flushright}
                \vspace*{2em}
                \begin{algorithmic}[0]
                    \ISR[]{}
                    \State read new sensor data from ADC
                    \Function{ComputeSPM}{$i_{k-1}$, params}
                    \State evaluate spm model equations
                    \State $\vdots$
                    \State compute model output voltage
                    \Function{SOCEstimator}{$v_\text{model}$,$v_\text{meas}$}
                    \State \textit{state estimation subroutine}
                    \State $\vdots$
                    \EndFunction
                    \Function{ICEControl}{$ $}
                    \State $\vdots$
                    \State write control outputs to DACs
                    \EndFunction
                    \EndFunction
                    \END
                \end{algorithmic}
            \end{flushright}
        \end{varwidth}%
    }
    \centering
    \framebox[\textwidth]{
        \begin{subfigure}[t]{\wd\algboxA}
            \subcaption{\uline{background process (low priority)}}\label{subfig:bgRTprocess}
            \usebox{\algboxA}
        \end{subfigure}
        \hfill
        \begin{subfigure}[t]{\wd\algboxB}
            \subcaption{\uline{foreground processes (high priority)}}\label{subfig:fgRTprocess}
            \raisebox{\dimexpr.5\ht\algboxA-.5\ht\algboxB}{%
                \usebox{\algboxB}%
            }%
        \end{subfigure}
    }
    \caption[Overview of real-time software implementation of a typical
    \glsfmtshort{bms}]{Overview of the real-time software implementation of a typical
        \gls{bms}. Through an interrupt-driven architecture for time-critical tasks as
        as state estimation and control, the same processor can be
        efficiently utilised by employing its idle CPU cycles for background
    tasks such as diagnostics, fault logging and book-keeping.}
    \label{fig:basicRTCsoftwarearch}
\end{figure}

\Cref{fig:basicRTCsoftwarearch} shows  an example of a  plausible implementation
of  a \gls{bms}  software  in an  embedded microcontroller.  The  vast array  of
functionality  performed by  the \gls{bms}  can be  grouped and  managed as  two
separate processes ---
\begin{enumerate*}[label=\roman*)]
    \item a background thread, and
    \item a foreground thread.
\end{enumerate*}
The  background  thread runs  continuously  within  the main  loop  sequentially
processing instructions. \Cref{subfig:bgRTprocess} shows an example illustration
of typical  background tasks that  a \gls{bms} handles. All  high-priority tasks
are  triggered  by  an  interrupt  and the  supervisory  control  loop  suspends
the  presently executing  background  task  for later  resumption.  As shown  in
\cref{subfig:fgRTprocess}, a typical example of such an interrupt driven process
is the  evaluation of the \gls{spm}  model equations and computation  of control
outputs and is discussed next.

\begin{figure}[!tbp]
    \centering
    \includegraphics[width=\textwidth]{timing_diagram_large}
    \caption[Timing diagram of a real-time software loop of a \glsfmtshort{bms}]
    {Timing diagram of a real-time software loop of a \gls{bms}. The sequence of
        events within one sample period~$T_s$ in relation to the base clock of
        the controller is shown. Particular emphasis is placed on depicting the
        handling of \gls{isr} requests pertinent to cell models. Other
        background tasks performed by the CPU are de-emphasised. Moreover, the
        integration of the  \gls{bms} software loop within  the larger  scope of
        a master  vehicular controller is not shown. Illustration adapted from
    Southward~\cite{Southward2011}.}
    \label{fig:timingdiagramBig}
\end{figure}


\Cref{fig:timingdiagramBig}  depicts  an exploded  view  of  the timing  aspects
of  the  \textsc{Interrupt  Service  Routine}  that  was  briefly  presented  in
\cref{subfig:fgRTprocess}.  Upon  the  expiry  of an  on-chip  timer  calibrated
against a  baseline precision-clock,  hardware interrupts are  raised by  one or
more \glspl{adc} associated  with voltage/current sensors mounted  on cells. The
\gls{isr}  disables  the interrupt  and  reads  the  samples  of data  from  the
\glspl{adc} into software. At the end  of this process, the \gls{isr} rearms the
interrupts  and  simultaneously sends  and  acknowledgement  to the  appropriate
sensor which reloads its timer. The \gls{spm} model equations are then evaluated
in  software and  the  resulting  computational variables  such  as voltage  and
concentrations are  employed in  other pertinent tasks  such as  state estimator
subroutines. If  the \gls{bms}  also performs  control tasks  \eg~regulating the
coolant's  flow rate  or  \gls{ice}  state-toggling such  as  in the  hysteresis
control of a series hybrid drivetrain,  these control outputs are written to the
relevant \glspl{dac}.


% \FloatBarrier

\subsection{Sample Delay Considerations}

\Cref{fig:timingdiagramSmall}  shows a  vertically  compressed view  of all  CPU
activities across a larger  time horizon. The CPU's load factor  is the ratio of
time spent in foreground  requests to its idling time. While  a high load factor
is beneficial in terms of efficient usage of resources, it adversely affects the
power efficiency of the chip.

\begin{figure}[!htbp]
    \centering
    \includegraphics{timing_diagram_small}
    \caption[Timeline of \glsfmtshort{bms} activities over multiple CPU cycles of a real-time
    controller]{A compressed timeline of CPU execution cycles showing details of
        activities within each sample interval. The execution sequence is shown
        over a larger horizon so as to illustrate the proportion of `activity
        time' relative to the `idle time'. The vast majority of the CPU cycle is
        spent in idling or background tasks. The servicing of the \gls{isr}
        occupies a relatively small fraction of each CPU cycle. Diagram
    reproduced from Plett~\cite{PlettECE5540_02}.}
    \label{fig:timingdiagramSmall}
\end{figure}

For  Li-ion  cell  modelling,  a  sampling  interval  of~$T_s  =  1$\si{\second}
is   commonly  used,   thus  aiming   to   capture  the   cell  dynamics   below
\SI{500}{\milli\hertz}\footnote{In  the ideal  case,  according  to the  Nyquist
sampling theorem. In practice, the frequency range is smaller.}. The CPU's clock
used is of  several \si{\MHz}, a vast  majority of which is  spent in background
tasks or in sleep mode. Furthermore, a low-latency \gls{isr} code is employed in
the tasks  of reading the  \gls{adc} value,  evaluating the model  equations and
writing any control outputs to the  \gls{dac}. Using a simplified \gls{pbm} such
as the \gls{spm} helps in achieving a low-latency throughput for the \gls{isr}.

The overall implication of such a scheme  is that any \emph{delays} owing to the
sample and  hold process  at the model  input and outputs  can be  neglected. In
conventional sampled-data systems, control delays may be analysed by considering
a multiplicative factor of~$e^{-sλ}$ in the Laplace domain transfer function of
the system. Delay parameters of~${λ = 0.5  T_s}$ or ${λ = 1 T_s}$ are commonly
employed  as  conservative estimates.  However,  owing  to  the small  CPU  load
factors, in this thesis this delay term is omitted for discrete-time formulation
of the \gls{spm}.

\subsection{Discrete-Time \glsfmtshort{spm} Formulation}

% \Cref{fig:blockdiagctsdisc} shows  a block  diagram representation of  the plant
% model its associated input and output signals.

Due  to the  sampling and  \gls{zoh} operations  at the  \gls{adc} input  to the
system,  the input  to the  \gls{spm} is  transformed from  a simple  continuous
time  signal to  discrete-time sequences  \ie~${u(t) \mapsto  u[k]}$ and  ${y(t)
\mapsto y[k]}$,  where ${k =  0,1,\dots,∞}$~is the sample  index corresponding
to  the  continuous time  instant~${t_k  =  kT_s}$.  The continuous  time  plant
model  represented by  the  \gls{ode}  system of  \cref{eq:threestatesmatrixvec}
is  therefore   replaced  by   a  discrete-time  process   and  modelled   by  a
\emph{difference} equation which is to be determined.

% \begin{figure}[!htbp]
%     \centering
%     % show block diagram cross-referencing equation and a question mark for the
%     % dt system
%     \includegraphics{placeholder_images/example-image-golden.pdf}
%     \caption[Block-diagram of continuous and discrete-time systems]{Block-diagram of the plant model and associated signals of a continuous-time system and its discrete-time counterpart}
%     \label{fig:blockdiagctsdisc}
% \end{figure}

Consider       the       general      continuous-time       solution       given
by \cref{eq:genanalyticctssoln}. Let ${t_0 = k T_s}$ and ${t = (k+1)T_s}$, where~${k  = 0,1,\dots,∞}$. Therefore,
\begin{alignat}{2}
    x(t_0) & = x(kT_s)     & & \equiv x[k] \\
    x(t)   & = x((k+1)T_s) & & \equiv x[k+1]
\end{alignat}
Substituting these relationships into \cref{eq:genanalyticctssoln},
\begin{equation}
    \mathbf{x}[k+1] = e^{A T_s}\mathbf{x}[k] + \int_{k T_s}^{(k+1)T_s}e^{A ((k+1)T_s-τ)}B \mathbf{u}(τ)\,dτ \label{eq:intermediatediscrete}
\end{equation}
With  the  \gls{zoh} scheme  discussed  here,  $u(\tau)$~remains  constant  from~$k  T_s$  to~$(k+1)T_s$,  and  is  equal  to  $u(kT_s)$ \ie~$u[k]$.  Consider  a
change-of-variable definition  for the dummy  variable of integration  $\tau$ as~${\eta =  (k+1)T_s -  \tau}$. Thus,  $\tau = (k+1)T_s  - \eta$.  Hence, $d  \tau =
-d\eta$. Substituting these into \cref{eq:intermediatediscrete},
\begin{align}
    \mathbf{x}[k+1] &= e^{A T_s}\mathbf{x}[k] + \left[\int_{T_s}^{0}e^{A \eta }B \right] u[k]\,{-d\eta}\\
    \shortintertext{Reversing the order of integration leads to}
    \mathbf{x}[k+1] &= e^{A T_s}\mathbf{x}[k] + \left[\int_{0}^{T_s}e^{A \eta }B \,{d\eta} \right] u[k] \label{eq:disctimefulleqn}
\end{align}

\Cref{eq:disctimefulleqn} represents a discrete-time state-space representation
of the dynamics of the system whose generic representation is given by the
difference equation
\begin{equation}\label{eq:discgenericLTI}
    \mathbf{x}[k+1] = A_d x[k] + B_d u[k]
\end{equation}
where ${A_d = e^{A T_s}}$ and ${B_d = \int_{0}^{T_s}e^{A \eta}B
\,{d\eta}}$.
If the continuous-time system matrix~$A$ is invertible, a closed form
expression for~$B_d$ is obtained as
\begin{align}
    B_d &= A^{-1}(A_d - I_n)B && \text{(if $A^{-1}$ exists)}
\end{align}
For the continuous time system matrix~$A$ of the \gls{spm}, its determinant is
zero.
\begin{equation}
\begin{vsmallmatrix}
    -30\frac{D_\spos}{R_\ppos^2} & 0                            & 0 \\
    0                            & -30\frac{D_\sneg}{R_\pneg^2} & 0 \\
    0                            & 0                            & 0
\end{vsmallmatrix} = 0
\end{equation}
and hence  is not invertible.  This necessitates  an explicit evaluation  of the
integral in \cref{eq:disctimefulleqn} for computation of the discrete-time input
matrix~$B_d$.

Since  the only  non-zero entries  of  the matrix  lie along  its main  diagonal
\ie~its modes are decoupled, the matrix exponential reduces to a diagonal matrix
whose elements are simply the scalar exponentials of the original entries. The
discrete-time input matrix~$B_d$ can be obtained by evaluating \cref{eq:B_dinit}
as shown below.
\begin{gather}
    A_d = e^{A T_s} = \exp\left(
        \begin{bmatrix}
            -30\frac{D_\spos}{R_\ppos^2} & 0                            & 0 \\
            0                            & -30\frac{D_\sneg}{R_\pneg^2} & 0 \\
            0                            & 0                            & 0
    \end{bmatrix} T_s \right)
    % \end{equation}
    % \begin{equation}\label{eq:A_d}
    % A_d
    =
    \begin{bmatrix}
        e^{-30\frac{D_\spos}{R_\ppos^2} T_s} & 0                                    & 0 \\
        0                                    & e^{-30\frac{D_\sneg}{R_\pneg^2} T_s} & 0 \\
        0                                    & 0                                    & 1
    \end{bmatrix} \label{eq:A_dinit} \\
    % \end{equation}
    % \begin{equation}
    B_d = \int_{0}^{T_s}e^{A \eta}B \,{d\eta}
    % \end{equation}
    % \begin{equation}\label{eq:B_dintermediate}
    % B_d
    =\bigint_{0}^{T_s} \left( \begin{bmatrix}
            e^{-30\frac{D_\spos}{R_\ppos^2} \eta} & 0                                    & 0 \\
            0                                    & e^{-30\frac{D_\sneg}{R_\pneg^2} \eta} & 0 \\
            0                                    & 0                                    & 1
        \end{bmatrix}\cdot
        \begin{bmatrix}
            \frac{45}{2} \frac{\hphantom{-}1}{R_\ppos^2 A \, l_\text{pos} a_\spos F} \\
            \frac{45}{2} \frac{-1}{R_\pneg^2 A \, l_\text{neg} a_\sneg F} \\
            \hphantom{\frac{45}{2}} \frac{-3}{R_\pneg  A \, l_\text{neg} a_\sneg F}
    \end{bmatrix} \right) \, d\eta \label{eq:B_dinit} \\
% \end{gather}
% \begin{equation}
    B_d = \begin{bmatrix}
        \hphantom{-}\frac{3}{4} \frac{1 - \exp\left(-30\frac{D_\spos}{R_\ppos^2}\right)T_s}{D_\spos A \, l_\text{pos} a_\spos F} \\[1em]
        -\frac{3}{4} \frac{1 -
        \exp\left(-30\frac{D_\sneg}{R_\pneg^2}\right)T_s}{D_\sneg A \, l_\text{neg} a_\sneg F} \\[1em]
        \hphantom{-}\hphantom{\frac{3}{4}} \frac{-3 T_s}{R_\pneg  A \, l_\text{neg} a_\sneg F}
    \end{bmatrix} \label{eq:B_d}
\end{gather}

The  discrete-time  matrix-vector  system  presented  in  \cref{eq:A_dinit}  and
\cref{eq:B_d} have  not been presented in  existing literature, but is  vital to
understanding  the  implementation  of  the \gls{spm}  in  digital  controllers.
Although,  simpler alternatives  such  as Forward  Euler  methods are  available
to  approximate  the time-derivative  of  the  state  vector, they  suffer  from
problems  such  as  a  growth  in   the  rate  of  local  truncation  error  per
time-step~\cite{Ascher1997}, necessitating  the use  of very high  sample rates,
which increases  the burden on  the embedded controller. The  matrix exponential
approach is superior in  terms of accuracy and stability across  a wide range of
sample rates.

For  a  pre-determined  sample-rate,  the   matrix  exponential  and  hence  the
$A_d$  and   $B_d$  matrices  can   be  computed   offline  on  a   desktop  and
stored  into  the   non-volatile  memory  of  the  embedded   controller  to  be
loaded  onto  RAM  during  operation.   The  vectorised  implementation  of  the
state  dynamics  presented  here  is  highly  efficient  and  directly  amenable
for  use  in   classical  state-vector  algorithms.  For   the  cell's  terminal
voltage   computation,   the   basic   structure  and   form   of   the   output
equation  given  by  \cref{eq:spmoutputeqn}  remains  intact,  except  that  the
continuous  time   variables~$\left(\mathbf{x}(t),  u(t)\right)$  need   to  be
replaced  by  their discrete-time  counterparts  in  the corresponding  equation
set  \ie~\crefrange{eq:spmbasicoutputvoltagefinal}{eq:csurfnegfromcavgneg}.  The
discrete-time output function~$h_d$ is evaluated \emph{after} updating the state
vector (through \cref{eq:discgenericLTI}).
\begin{equation}\label{eq:discspmoutputeqn}
    y[k+1] = h_d(\mathbf{x}[k+1],u[k+1])
\end{equation}

The complete sequence of steps for implementing the discrete-time variant of the
\gls{spm} is given in \cref{alg:disctimespm}. In particular, it can be seen that
the  discrete-time system  and input  matrices,~$A_d$ and~$B_d$  can be  easily
pre-computed from  the parameter set  using the matrix exponential  approach. In
\textsc{MATLAB}, this  can be achieved  by passing  the arguments of  the matrix
exponential to the  `\verb+expm+' command. The vectorised  implementation of the
discrete-time  state equation  given in  line~\nolink{\ref{algLine:discstateEq}}
of  \cref{alg:disctimespm}  is a  set  of  efficient linear  algebra  operations
consisting of simple matrix-vector product and vector-addition routines.

% -*- root: ../main.tex -*-
%!TEX root = ../main.tex
% this file is called up by main.tex
% content in this file will be fed into the main document
% vim:nospell

\begin{algorithm}[!htbp]
    \caption{Discrete-time \glsfmtshort{spm}}\label{alg:disctimespm}
    \begin{algorithmic}[1]
        \Require Load profile \Comment{\eg{} a \texttt{csv} file of $t$ vs. C-rate}
        \Require \gls{spm} parameter set  \Comment{\eg{} stored in a struct \texttt{params}}
        \Userdata $z[1], t_\text{f,user}$, $t_\text{f,condition}$, cell capacity $I_\text{1C}$, sample rate $T_s$ \Comment{$t_\text{f,condition} \in  \left\{\texttt{max}, \texttt{min}\right\}$}
        \Ensure  $z[1], V_\text{cell}[1]$ within limits \Comment{index $[k=1] \wedgeq \text{time } (t=0) $}
        \Procedure{Simulate\gls{spm}}{}%{$z[1],t_\text{f,desired},T_s,I_\text{1C},\texttt{params}$}
            \State {$t_\text{f,desired} =
                \begin{cases}
                   \max(t_\text{f,user},t_\text{f,profile}),
                        &\text{%\scriptsize
                    if $t_\text{f,condition}$ == \texttt{max};}\\
                    \min(t_\text{f,user},t_\text{f,profile}),
                    &\text{%\scriptsize
                otherwise.}
                \end{cases}$} \Comment{\parbox[t]{0.25\textwidth}{may terminate early due to cut-off violations}}
                    \FullComment{\scriptsize Flexible end time. Extrapolate last C-rate from profile until $t_\text{f,desired}$ if necessary.}
            \State $N_\text{max} \gets \ceil*{\frac{t_\text{f,desired}}{Ts}} + 1$ \Comment{max iterations assuming no cut-offs}
            \State Allocate storage of size $\mathbb{R}^{N_\text{max}\times 1}$ for each simulation variable
            \State Compute $\mean{c}_\sneg$[1] as per \cref{eq:csfluxinitialcondition}
            \State $I[1] \gets I_\text{1C} \times  \text{C-rate}[1], \quad \mathbf{x}[1] \gets \vect{0,0, \mean{c}_\sneg[1]}$ \Comment{ $\text{C-rate}[1]$ from profile}
            \State \ColorLine{Compute $A_d$ and $B_d$ \Comment{as per \cref{eq:A_d} and \cref{eq:B_d}}}\label{algLine:computeAdBd}
            \State $V_\text{cell}[1] \gets \textsc{ComputeCellVoltage}(\textbf{x}[1],I[1],\texttt{params})$ \Comment{from direct feedthrough}
            \For{$k \gets 2 : N_\text{max}$}
                \State $I[k] \gets $ interpolate from profile using \gls{zoh}
                \State \ColorLine{$x[k] \gets A_d x[k-1] + B_d u[k-1]$ \Comment{\cref{eq:discgenericLTI}}}\label{algLine:discstateEq}
                \State Compute $z[k]$ as per \cref{eq:soccomputation}
                \State $V_\text{cell} \gets \textsc{ComputeCellVoltage}(\textbf{x}[k],I[k],\texttt{param}) $
                \If {$z[k] \text{ or } V_\text{cell}[k]$ exceeded cut-off limits}
                    \State $k \gets k - 1$ \Comment{data from last  step is invalid}
                    \State \textit{break};
                \EndIf
            \EndFor
        \EndProcedure

        \OutputEqn{\textbf{x},I,\texttt{params}} \Comment{uses discrete-time variants of eqs, \ie{} at index $k$}
            \State Compute $c_\snegsurf$ as per \cref{eq:csurfnegfromcavgneg}
            \Comment{consider saturating \ie{} $c_\snegmin \le c_\snegsurf \le
            c_\snegmax$}
            \State Compute $\mean{c}_\spos$ as per \cref{eq:csposbulkfromcsnegbulk}
            \State Compute $c_\spossurf$ as per \cref{eq:csurfposfromcavgpos}
            \State Compute $V_\text{cell}$ as per \cref{eq:spmbasicoutputvoltagefinal}
        \EndOutputEqn%
    \end{algorithmic}
\end{algorithm}


This thesis strives  for an inclusive approach by taking  into account that some
battery researchers whose  focus is on fundamental aspects of  lithium ion cells
\eg~those  specialising in  electrochemistry,  might not  be  familiar with  the
nuances  of the  matrix exponential  and discrete-time  matrix computations  (in
line~\nolink{\ref{algLine:computeAdBd}}  of \cref{alg:disctimespm}).  Therefore,
a  snippet   of  \textsc{MATLAB}   code  clarifying   the  computation   of  the
discrete-time   system   and   input   matrices,~$A_d$   and~$B_d$   is   given
in   \cref{codesnippet:computeAdBd}.  A   full  code   listing  of   an  example
discrete-time  \gls{spm}  implementation  in   \textsc{MATLAB}  is  provided  in
\cref{sc:disctimespm}.

\begin{listing}[!htbp]
\begin{minted}[mathescape,autogobble,bgcolor=mintedbg,escapeinside=||,texcomments=true]{matlab}
% Returns $A_d$ and $B_d$ matrices
A_cts = [-30*Ds_pos/(R_pos^2),                    0, 0; ...
                            0, -30*Ds_neg/(R_neg^2), 0; ...
                            0,                    0, 0];
% $A_d = e^{A T_s}$ \fontfamily{libertinus}\selectfont(see \cref{eq:A_dinit})
A_disc = expm(A_cts*Ts); % $\mathtt{expm}$ command computes the matrix exponential

B_cts = [ (45/2)/(R_pos^2*a_pos*L_pos*F*A); ...
         (-45/2)/(R_neg^2*a_neg*L_neg*F*A); ...
             (-3/(R_neg*a_neg*L_neg*F*A))];

% $B_d = \int_{0}^{T_s}e^{A \eta}B \,{d\eta}$ \fontfamily{libertinus}\selectfont(see \crefrange{eq:B_dinit}{eq:B_d})
B_disc = nan(size(B_cts));
B_disc(1) = B_cts(1)*(exp(A_cts(1,1)*Ts)-1)/A_cts(1,1);
B_disc(2) = B_cts(2)*(exp(A_cts(2,2)*Ts)-1)/A_cts(2,2);
B_disc(3) = B_cts(3)*Ts;
\end{minted}
\caption{Computation of discrete-time matrices~$A_d$ and $B_d$ in
\textsc{MATLAB}}
\label{codesnippet:computeAdBd}
\end{listing}

Thus,  a  discrete-time model  of  the  basic  \gls{spm}  is now  available  for
implementation  in  an embedded  \gls{bms}.  Further  analysis of  discrete-time
issues  such  as  aliasing,  quantisation  noise,  signal  pre-conditioning  and
discrete fourier analysis  lies in the specialised engineering  domain of signal
processing and falls outside  the scope of the thesis. The  results of the basic
\gls{spm} are presented next in \cref{sec:basicspmsimresults}.

% \FloatBarrier

% % https://tex.stackexchange.com/questions/113719/cleveref-fails-to-reference-algorithms

% % https://tex.stackexchange.com/questions/110412/numbering-in-algorithmicx
% % https://tex.stackexchange.com/questions/65993/algorithm-numbering

% % https://tex.stackexchange.com/questions/203713/how-can-i-typeset-function-names-as-they-appear-in-algorithmic-environments
% % https://tex.stackexchange.com/questions/100346/typesetting-listofalgorithms-like-listoffigures-and-listoftables-using-titletoc
% % https://tex.stackexchange.com/questions/30363/how-do-i-define-a-new-command-in-algorithmicx

% % https://tex.stackexchange.com/questions/67908/customizing-the-algorithmic-package-break-and-loop-labels

% % https://tex.stackexchange.com/questions/69449/avoid-putting-statements-on-the-same-line-with-algorithmicx

% % \usepackage{float}
% % \newfloat{algorithm}{t}{lop}
% % Add \floatname{algorithm}{Algorithm} to capitalise the float name


\section{Desktop Simulation}\label{sec:basicspmsimresults}
% -*- root: ../main.tex -*-
%!TEX root = ../main.tex
% this file is called up by main.tex
% content in this file will be fed into the main document
% vim:textwidth=80 fo=cqt

In this  section, the performance  of the  basic \gls{spm} is  discussed through
desktop  simulation and  by comparison  against a  standard \gls{dfn}  benchmark
model incorporating the full \gls{p2d} dynamics.

\subsection{Cell Parametrisation}\label{subsec:spmp2dparametrisation}
% -*- root: ../main.tex -*-
%!TEX root = ../main.tex
% this file is called up by main.tex
% content in this file will be fed into the main document
% vim:nospell

\begin{table}[!htbp]
    \small
    \caption[Simulation parameters of an \protect{\gls{lco}} cell]{Complete set of parameters for simulating the \protect{\gls{p2d}} and \protect{\gls{spm}} implementations of an \protect{\gls{lco}} cell (\ch{LiCoO_2}--\ch{LiC_6} electrode pair with \ch{LiPF_6} electrolyte),\quad \protect{$j \in \{\text{pos},\text{sep},\text{neg}\}$}}
    \label{tbl:LCOSimParamsSPMP2D}

    \begin{threeparttable}
        \centering
        \begin{varwidth}[t]{0.48\linewidth}
            \begin{tabular*}{\textwidth}{l @{\extracolsep{\fill}} r}
                \multicolumn{2}{c}{\textbf{System Conditions}} \\
                \toprule
                \multicolumn{1}{l}{Parameter} \\
                \midrule

                Lower cutoff cell voltage, $V_\text{min}$ \si{(V)} & \tnote{a}\num{2.50}   \\
                Upper cutoff cell voltage, $V_\text{max}$ \si{(V)} & \tnote{b}\num{4.30}   \\
                Cell temperature, $T_\text{cell}$ \si{(K)}         & \tnote{c}\num{298.15} \\

                \bottomrule
            \end{tabular*}
        \end{varwidth}
        \hfill
        \begin{varwidth}[t]{0.48\linewidth}
            \begin{tabular*}{\textwidth}{l @{\extracolsep{\fill}} r}
                \multicolumn{2}{c}{\textbf{Other Constants}} \\
                \toprule
                \multicolumn{1}{l}{Parameter} \\
                \midrule

                Faraday constant, $F$ \si{(C.mol^{-1})}                   & \num{96487}         \\
                Universal gas constant, $R$ \si{(J.mol^{-1}.K^{-1})}      & \num{8.314}         \\
                Init. electrolyte conc., $c_\text{e,0}$ \si{(mol.m^{-3})} & \tnote{c}\num{1000} \\

                \bottomrule
            \end{tabular*}
        \end{varwidth}

        \bigskip

        \begin{tabularx}{\textwidth}{ l L C R }
            \multicolumn{4}{c}{\textbf{Thermodynamic, Kinetic, Geometric and Transport Parameters}} \\
            \toprule
            \multicolumn{1}{l}{Parameter} & \multicolumn{1}{l}{Pos} & \multicolumn{1}{c}{Sep} & \multicolumn{1}{r}{Neg}\\
            \midrule

            \rowcolor{imperiallightgray} Filler vol.\ fraction, ${\varepsilon}_{\text{fi}_j}$               & \tnote{c}\num{0.025}    & ---                      & \tnote{c}\num{0.033}    \\
            \rowcolor{imperiallightgray} Material vol.\ fraction, $\varepsilon_\sj$                         & \tnote{d}\num{0.590}    & \tnote{d}\num{0.276}     & \tnote{d}\num{0.482}    \\
            \rowcolor{imperiallightgray} Bruggeman coefficient, $\text{brugg}_j$                            & \tnote{c}\num{4}        & \tnote{c}\num{4}         & \tnote{c}\num{4}        \\
            \rowcolor{imperiallightgray} Electrolyte diffusivity, $D_j$ \si{(m^2.s^{-1})}                   & \tnote{g}\num{3.22e-10} & \tnote{g}\num{3.22e-10}  & \tnote{g}\num{3.22e-10} \\
            \rowcolor{imperiallightgray} Electrolyte conductivity, $\kappa_j$ \si{(S.m^{-1})}               & \tnote{h}\num{26.24e-3} & \tnote{c}\num{328.15e-3} & \tnote{c}\num{66.08e-3} \\
            \rowcolor{imperiallightgray} \ch{Li^+} transference number, $t^0_\text{+}$                      & \tnote{c}\num{0.363}    & \tnote{c}\num{0.363}     & \tnote{c}\num{0.363}    \\
            \rowcolor{imperiallightgray} Electronic conductivity, $\sigma_j$ \si{(S.m^{-1})}                & \tnote{c}\num{100.00}   & ---                      & \tnote{c}\num{100.00}   \\
                                         Thickness, $l_j$ \si{(m)}                                          & \tnote{c}\num{88e-6}    & \tnote{c}\num{25e-6}     & \tnote{f}\num{72e-6}    \\
                                         Particle radius, $R_\pj$ \si{(m)}                                  & \tnote{c}\num{2e-6}     & ---                      & \tnote{c}\num{2e-6}     \\
                                         Specific interfacial surface area, $a_\sj$ \si{(m^{2}.m^{-3})}     & \tnote{e}\num{885e3}    & ---                      & \tnote{e}\num{723.6e3}  \\
                                         Electrode diffusivity, $D_{\text{s}_j}$ \si{(m^2.s^{-1})}          & \tnote{c}\num{1e-14}    & ---                      & \tnote{c}\num{3.9e-14}  \\
                                         Stoichiometry, 0\% SOC, ${\theta}_{\text{min}_j}$                  & \tnote{i}\num{0.9917}   & ---                      & \tnote{i}\num{0.0143}   \\
                                         Stoichiometry, 100\% SOC, ${\theta}_{\text{max}_j}$                & \tnote{i}\num{0.4955}   & ---                      & \tnote{i}\num{0.8551}   \\
                                         Max concentration, ${c_\text{s,max}}_j$ \si{(mol.m^{-3})}          & \tnote{c}\num{51554}    & ---                      & \tnote{c}\num{30555}    \\
                                         Reaction rate coefficient, $k_\jr$ \si{(m^{2.5}mol.^{-0.5}s^{-1})} & \tnote{c}\num{2.33e-11} & ---                      & \tnote{c}\num{5.03e-11} \\
                                         Open circuit potential, $U_j$ \si{(V)}                             & \tnote{k}see table note & ---                      & \tnote{m}see table note \\
            \bottomrule
        \end{tabularx}

        \bigskip
        %%%%%%%%%%%%%%%%%%%%%%%%%%%% SIMULATION PARAMS TABLE %%%%%%%%%%%%%%%%%%%%%%%%%%%%%
        \begin{tabularx}{\textwidth}{ l L C R }

            \multicolumn{4}{c}{\textbf{Spatial Discretisation}} \\
            \toprule
            \multicolumn{1}{l}{Parameter} & \multicolumn{1}{l}{Pos} & \multicolumn{1}{c}{Sep} & \multicolumn{1}{r}{Neg}\\
            \midrule

            \rowcolor{imperiallightgray} Nodes, through-thickness (axial), $N_{\text{a}_j}$          & \num{15} & \num{15} & \num{15} \\
            \rowcolor{imperiallightgray} Nodes, within spherical particle (radial), $N_{\text{r}_j}$ & \num{10} & ---      & \num{10} \\

            \bottomrule
        \end{tabularx}

        \medskip

        \begin{tablenotes}[para,flushleft]
            \begin{footnotesize}
                \noindent\begin{tabular}{@{} l l @{}}
                \item[a]\, Ref.~\cite{Northrop2011}                      & \item[b]\, Set to $\approx $\SI{100}{\milli\volt} above the cell's \gls{ocp} at \SI{100}{\percent} cell \gls{soc}                                        \\
                \item[c]\, Ref.~\cite{Subramanian2009}                   & \item[d]\, Computed as $1-\varepsilon_j - \varepsilon_{\text{fi}_j}$, where $\varepsilon_j$ is the electrolyte porosity from Ref.~\cite{Subramanian2009} \\
                \item[e]\, Computed as $\frac{3 \varepsilon_\sj}{R_\pj}$ & \item[f]\, Set up for capacity balance of electrodes such that $l_\text{neg} = 1.22 \times l_\text{pos}$ here.                                           \\
                \end{tabular}
            \end{footnotesize}
            \begin{footnotesize}
                \noindent\begin{tabular}{@{} l @{}}
                \item[g]\, Computed at $T_\text{cell} = T_\text{ref} = \SI{298.15}{\kelvin}$ using coefficients from table \rom{2} in Ref.~\cite{Valoen2005} \\
	            \item[h]\, Computed at $T_\text{cell} = T_\text{ref} = \SI{298.15}{\kelvin}$ using coefficients from table \rom{3} in Ref.~\cite{Valoen2005}\\
                \item[i]\, Obtained as residual stoichiometries after a C/\num{500} simulated discharge from \SI{100}{\percent} cell \gls{soc} to \SI{2.7}{V} \\
                \end{tabular}
            \end{footnotesize}
            \noindent\begin{tabular}{@{} l @{}}
            \item[k]\, $ \mathcal{U(\theta_\text{pos})} = \displaystyle \frac{-4.656 + 88.669\theta_\text{pos}^2 - 401.119\theta_\text{pos}^4 + 342.909\theta_\text{pos}^6 - 462.471\theta_\text{pos}^8 + 433.434\theta_\text{pos}^{10}}{-1 + 18.933\theta_\text{pos}^2 - 79.532\theta_\text{pos}^4 + 37.311\theta_\text{pos}^6 - 73.083\theta_\text{pos}^8 + 95.96\theta_\text{pos}^{10}}$ \\[1em]
                \begin{footnotesize}
                \item[m]\, $\mathcal{U(\theta_\text{neg})} = 0.7222 + 0.1387\theta_\text{neg} + 0.029\theta_\text{neg}^{0.5} - \frac{0.0172}{\theta_\text{neg}} + \frac{0.0019}{\theta_\text{neg}^{1.5}} + 0.2808 e^{(0.9 - 15\theta_\text{neg})} - 0.7984 e^{(0.4465\theta_\text{neg} - 0.4108)}$
                \end{footnotesize}
            \end{tabular}
        \end{tablenotes}

    \end{threeparttable}
\end{table}

% Electrolyte diffusivity, $D_j$ \si{(m^2.s^{-1})}                         & \multicolumn{3}{c} {\Vhrulefill{} refer to equation in \tnote{g} \Vhrulefill{}} \\
% \item[g] $ D_j = 10^{-4} \times 10^{-4.43 - \frac{54}{T_\text{cell} - 229 - 5\times10^{-3} c_\text{e}(x,t)} - 0.22\times10^{-3} c_\text{e}(x,t)}, \quad \jinpossepneg $\\



\Cref{tbl:lcoSimParamsSPMp2d} lists  the simulation  parameters of  an \gls{lco}
cell  whose positive  and negative  electrodes are  \ch{LiCoO_2} and  \ch{LiC_6}
respectively.  The  electrolyte in  this  system  consists of  \ch{LiPF_6}  salt
in  a   solution  of   \gls{ec}/\gls{dmc}/\gls{emc}  in   a  1:1:1   ratio.  The
standard  set  of  \gls{dfn}  parameters have  been  extensively  described  and
documented  in  literature.  The   detailed  characterisation  of  the  physical
properties  of  lithium-ion  cells  falls  outside the  scope  of  this  thesis.
Here, the  vast majority  of electrochemical  parameters, \viz{}  the geometric,
thermodynamic, kinetic  and transport properties  of the cell have  been sourced
from  Subramanian~\etal{}~\cite{Subramanian2009}.   The  significance   of  each
of  the  parameters  in  the  context of  the  modelling  assumptions  discussed
in~\cref{subsec:basicspmassumptions} is examined.


The simulation parameters that are applicable exclusively to the \gls{p2d} model
are shown as  highlighted text in~\cref{tbl:lcoSimParamsSPMp2d}. It  can be seen
that only a  subset of the isothermal \gls{dfn} model's  parameters are required
for the the \gls{spm}. In particular, there is no requirement to estimate any of
the electrolyte-related  parameters in  each electrode region.  Furthermore, the
properties of the separator material which  are necessary in the \gls{dfn} model
are also  not considered in the  \gls{spm} computations. A brief  enumeration of
the additional  \gls{p2d}-specific parameters in the  context of parametrisation
requirements is provided here.

\begin{enumdescriptnum}[leftmargin=!,itemsep=1ex,labelwidth=\widthof{$\symbfit{\text{brugg}_j}\ \scriptstyle (\times 3)$abc}
    ,partopsep=0pt
    ,topsep=0pt
    ]

    \customenum{\text{brugg}_j}{3}  The  empirical Bruggeman  coefficient  helps
    to   define   the  effective   values   of   conductivity  and   diffusivity
    of   the  electrolyte.   Although  an   identical   value  of   4  is   used
    in~\cref{tbl:lcoSimParamsSPMp2d}, in  principle all  three regions  can have
    different values of brugg and need to be parametrised separately.

    \customenum{D}{1}  The  intrinsic  electrolyte   diffusivity  of  a  typical
    electrolyte  consisting  of  \ch{LiPF_6}  salt in  an  organic  solvent  was
    experimentally  characterised   and  provided  as  a   table  of  polynomial
    coefficients  by  Valøen  and  Reimers~\cite{Valoen2005}.  Evaluating  this
    polynomial  at a  cell  temperature of  \SI{298.15}{\kelvin}  results in  an
    intrinsic diffusivity of \SI{3.22e-10}{\meter\squared\per\second}. Since the
    intrinsic  diffusivity is  a material  property  and is  independent of  the
    region within the cell, it needs to be parametrised only once.

    \customenum{\scalebox{1.35}{$\kappa$}}{1}   Like    the   diffusivity,   the
    intrinsic electrolyte conductivity is also a material property and its value
    is independent  of the region within  the cell. Unlike the  diffusivity, the
    electrolyte conductivity  is a strong  function of its  ionic concentration.
    Thus, the polynomial proposed by Valøen and Reimers~\cite{Valoen2005} needs
    to  be evaluated  at  $T_\text{cell}= \SI{298.15}{\kelvin}$  and  has to  be
    updated during the  simulation as salt concentration  within the electrolyte
    changes over time. A discussion on the choice of initial concentration is
    provided in~\cref{subsec:basicspmsimsetup}.

    \customenum{\scalebox{1.25}{$t$}_+^0}{1}  The  cationic transference  number
    measures the relative  mobility of the \ch{Li^+} ion in  the organic solvent
    and is  independent of  the region  within the  cell. Hence,  this intrinsic
    property is to be parametrised only once (per solvent).

    \customenum{\scalebox{1.25}{$\sigma$}_j}{2} The intrinsic conductivity of the solid
    phase depends on the material used in the porous electrodes. Although a
    simplified assumption of equal conductivity is used for the two electrodes,
    in practice, this property needs to be characterised for each of the two
    electrodes.

\end{enumdescriptnum}

\sisetup{detect-weight=true}   Thus,   neglecting   Arrhenius-type   temperature
dependence of  physical properties and their  corresponding activation energies,
the basic  \gls{spm} facilitates the  ability to afford  physics-based modelling
capabilities with  \emph{eight} fewer parameters than  the equivalent isothermal
\gls{p2d}  model. With  the  naive assumption  of  equal parametrisation  effort
per  physical  property,  this implies  a  \textbf{\SI{20}{\percent}}  reduction
in  parametrisation  requirements  for  the basic  \gls{spm}  when  compared  to
its  \gls{dfn} counterpart.  However,  considering the  fact that  parametrising
the  electrolyte's transport  properties requires  apparatus and  infrastructure
typically available  only in specialised chemical/materials  labs, the reduction
in parametrisation  overhead for  system-level engineering stakeholders  is more
pronounced.

Prima facie, it may seem that electrolyte porosities and filler volume fractions
do  not  influence the  \gls{spm}  model.  However,  they  do have  an  indirect
bearing  on arriving  at a  critical parameter,  \viz{} the  solid phase  volume
fraction, $\varepsilon_\sj$. This parameter is  required to compute the specific
interfacial  surface  area  of  the  electrodes  $a_\sj$,  \ie{}  the  effective
electrode area exposed  to reaction and is an important  entity in the \gls{spm}
model  equations  presented  in~\cref{subsec:basicspmgoverningeqns}.  The  solid
phase volume fractions are also required in simulating the \gls{p2d} model owing
to  the  need for  computing  the  effective  electronic conductivities  of  the
electrodes.\fxnote{cross-reference to  specifc newman equations here}.  They are
calculated as
\begin{equation}
\varepsilon_\sj = 1 - \varepsilon_j - \varepsilon_{\text{fi}_j}
\end{equation}
where $\varepsilon_j$  and $\varepsilon_{\text{fi}_j}$  are the  electrolyte and
filler  volume-fraction  within  the  respective electrode  regions.  Using  the
values from~\cref{tbl:lcoSimParamsSPMp2d} results in $\varepsilon_\spos = 0.590$
and  $\varepsilon_\sneg  =  0.482$  for the  positive  and  negative  electrodes
respectively. The specific interfacial surface areas are then calculated as
\begin{equation}\label{eq:specificsurfarea}
    a_\sj = \varepsilon_\sj \frac{4 \pi R_\pj^2}{\frac{4}{3} \pi R_\pj^3} = \frac{3\varepsilon_\sj}{R_\pj}
\end{equation}

As    discussed   in    the   assumptions    made   during    model   derivation
(see~\cref{subsec:basicspmassumptions})  and consistent  with the  assumed model
geometry,  the  parameters  not  covered  by  the  \gls{spm}  pertain  to  those
describing electrolyte dynamics and distribution  of electronic charge along the
axial  thickness  direction  of  the  cell. Properties  such  as  the  intrinsic
diffusivities  and conductivities  of  the electrolyte,  transference number  of
\ch{Li^+} in  the organic  solvent are thus  completely redundant  for \gls{spm}
simulation. The assumption of uniform charge density along the through-thickness
length of each electrode implies that the intrinsic electronic conductivities of
the two electrodes  do not play any  role in the model  dynamics. The porosities
and Bruggeman  coefficients in~\cref{tbl:lcoSimParamsSPMp2d} serve  as modifying
factors  of  the  intrinsic  conductivities  and  diffusivities  leading  to  an
effective value  within each  region of the  electrochemical layer.  Thus, their
relevance is also  rendered void in the case of  the basic \gls{spm} simulation.
The  thickness of  the  separator  material only  plays  a  role in  electrolyte
behaviour.  By the  model geometry  presented in~\cref{subsec:basicspmgeometry},
this parameter also falls outside the scope of the basic \gls{spm}.

The  thicknesses  of the  electrodes  are  optimised  for equal  loading,  \ie{}
to  achieve  a   balance  in  their  individual  capacities   to  store  \ch{Li}
atoms.  The thickness  of the  positive  electrode is  the chosen  as the  value
from~Subramanian~\etal{}~\cite{Subramanian2009}. The  thickness of  the negative
electrode region  is then  computed with  the goal of  equalising the  volume of
active material  in each electrode  for every  electrochemical layer in  a pouch
cell.
\begin{equation}\label{eq:basiccapacitybalance}
    A_{\text{elec}_\text{pos}}\varepsilon_\spos l_\text{pos} = A_{\text{elec}_\text{neg}}\varepsilon_\sneg l_\text{neg}
\end{equation}

In a  lithium-ion pouch cell, the  electrodes are designed such  that the layers
can be  overlaid on  top of  one another  and finally  encapsulated in  a pouch.
Geometrical considerations  then imply  that the  cross-sectional area  (or face
area) of the two electrodes must be  the same. However, due to the consideration
of  avoiding  plating  at  the  edges  due  to  microscopic  malformations,  the
design  is done  such as  to have  a small  overhang of  the negative  electrode
layer,  \approx\SI{2}{\milli  \meter} with  respect  to  the positive  electrode
layer\fxnote{citation  needed}  as  shown  in~\cref{fig:anodeoverhangpouchcell}.
However, the  active surface area, is  just the common overlap  area between the
two electrodes, and  thus, $A_\text{elec}$ is equal to  the cross-sectional area
of the positive electrode.

\begin{figure}[h]
    \centering
    \includegraphics[width=\textwidth]{placeholder_images/example-image-golden.pdf}
    \caption[Stacking of layers within a pouch cell]
    {Image showing the stacking of layers within a pouch cell. For the negative
        electrodes, there is a small overhang (\approx\SI{2}{mm}) at the
    edges with respect to the positive electrodes. This design choice
helps to avoid plating of lithium at the edges.}
    \label{fig:anodeoverhangpouchcell}
\end{figure}

Thus,~\cref{eq:basiccapacitybalance} reduces to
\begin{equation}
    \frac{l_\text{neg}}{l_\text{pos}} = \frac{\varepsilon_\text{pos}}{\varepsilon_\text{neg}} = 1.22
\end{equation}
yielding $l_\text{neg} = \SI{72}{\micro\meter}$.

At first, it  may be surprising to  note that the values of  the particle radius
$R_\pj$ used in  both the \gls{p2d} and \gls{spm} remain  identical. However, it
is important  to note that  the \gls{p2d} equations  of the \gls{dfn}  model are
cast in  a normalised  form, \ie{}  already set  up to  account for  the overall
capacity of the  cell under consideration implicitly through usage  of a current
density  (per unit  area) for  its  simulation. Furthermore,  this explains  why
increasing the  number of  discretisation nodes does  not increase  the modelled
capacity, but  instead serves to improves  the simulation accuracy owing  to the
enhanced spatial resolution.

The  overall  active  surface  area  $A  =  n  A_\text{elec}$  is  the  combined
cross-sectional  area  of all  layers  ($n$  is  the number  of  electrochemical
(pos,sep,neg) triplets) stacked  into the pouch cell. In both  the \gls{p2d} and
\gls{spm}  models, this  parameter serves  to  scale the  external load  current
down  to  the current  density  experienced  by  each electrochemical  layer.  A
value  of  \approx\SI{30}{\ampere \per  \meter  \squared}  was reported  in  the
results section of  Subramanian~\etal{}~\cite{Subramanian2009}. When considering
a \SI{60}{\ampere\hour} pouch cell with \SI{10}{\milli\meter} exterior thickness
and  using  the  parameters  reported  in~\cite{Subramanian2009},  this  results
in  a cross-sectional  area of  \SI{2.053}{\meter\squared}\fxnote{to add:  refer
to  layer  optimisation  chapter   for  detailed  derivation}.  Considering  the
equalisation of capacity  loading, with the newly chosen thickness  value of the
negative  electrode, the  1C-rate  capacity  of the  cell  has  been revised  to
\SI{29.23}{\ampere\per\meter\squared}.

On  simulating  the \gls{p2d}  model  with  a  trickle bleeding  type  discharge
corresponding   to   a  current   of   C/500   and   logging  the   data   every
\SI{1}{\milli\second}, after reaching \SI{2.7}{\volt}\footnote{From manufacturer
datasheets  for \protect{\gls{lco}}  chemistries,  this value  is considered  to
correspond to \SI{0}{\percent} \protect{\gls{soc}}.}, the remnant concentrations
in  the two  electrodes  were noted.  The  corresponding residual  stoichiometry
values  are reported  in~\cref{tbl:lcoSimParamsSPMp2d}  and  is consistent  with
typical values reported in literature for \gls{lco} chemistries.\fxnote{citation
needed here.}  This validation is  important, since below  certain stoichiometry
thresholds, the spinel/olivine  structure of the electrodes  can become unstable
and  collapse\fxnote{citation needed?}.

Since   the   cell   behaviour   is   considered   isothermal,   the   parameter
table     in~\cref{tbl:lcoSimParamsSPMp2d}     omits     activation     energies
for    the   various    diffusivities    and    conductivities   of    materials
(see~\cref{subsec:basicspmsimsetup}  for  further thermal  considerations).  For
this  reason,~\cref{tbl:lcoSimParamsSPMp2d}  does  not  include  other  material
properties such  as specific  heats, and  thermal conductivities.  No properties
of  the  current  collectors  appear  in  the  isothermal  model  equations  for
both  the \gls{p2d}  and  \gls{spm}  models and  hence  are  omitted. All  other
electrochemical  properties,   \viz{}  stoichiometries   at  \SI{100}{\percent},
maximum concentrations, diffusivities, reaction  rate coefficients and \gls{ocp}
of  the two  electrodes remain  invariant  between the  \gls{p2d} and  \gls{spm}
models.

\subsection{Simulation Setup}\label{subsec:basicspmsimsetup}

For  reproducibility of  results, it  is important  to discuss  the system-level
parameters influencing simulation setup.

The  lower cutoff  voltage of  the cell  is chosen  to be  \SI{2.5}{\volt}. This
is  deliberately  kept  lower  than  the voltage  corresponding  to  the  cell's
\SI{0}{\percent},  \ie{}  \SI{2.7}{\volt}.  If set  above  \SI{2.7}{\volt}  even
at  infinitesimally small  discharge  currents, the  cell  would cut-off  before
achieving complete discharge. Choosing a value lower than \SI{2.7}{V} means that
the cell gets a chance to recover its terminal voltage, despite spikes in highly
dynamic  load  currents  that  might  bring the  voltage  below  this  threshold
momentarily. If a low-enough value is  not chosen, a system-level shutdown shall
be  initiated  despite possessing  the  ability  to  continue to  operate  after
recovery  of terminal  voltage.  In  this case,  choosing  a  cutoff voltage  of
\SI{2.5}{\volt}  does  not  damage  the  cell  since  checks  are  in  place  to
monitor  the \gls{soc}  and  trigger cutoff  in the  event  of charge  depletion
(see~\cref{alg:ctstimespm,alg:disctimespm}).  Northrop~\etal~\cite{Northrop2011}
use this value, although no explanation is given for the choice.

The  upper  cutoff voltage  of  the  cell  is  chosen at  \SI{4.3}{\volt}  \ie{}
\approx\SI{100}{\milli  \volt}   higher  than   the  equilibrium   \gls{ocp}  at
\SI{100}{\percent} \gls{soc}. There are several  reasons for this smaller margin
at the upper end of the voltage spectrum
\begin{description}[leftmargin=!,labelwidth=\widthof{\bfseries low
    probabilities},itemsep=1ex]

    \item[safety] li-ion  cells are less  tolerant to overcharging and  can pose
    fire hazards.

    \item[degradation]  overcharging  li-ion  cells  can  lead  to  plating  and
    accelerate other degradation mechanisms.

    \item[low  C-rates]  charging C-rates  are  typically  lower than  discharge
    C-rates.

    \item[CCCV charging]  For on-board  chargers, taper charging  (such as  in a
    \gls{cccv} profile) is activated, which  ensures that charging current drops
    off  rapidly, leading  to a  lower overvoltage  towards the  upper \gls{soc}
    range.

    \item[low probabilities] The only charging event when an electrified vehicle
    is in motion is during regenerative braking. The vehicular \gls{bms} manages
    the operating window such that the starting \gls{soc} is much lower than the
    overvoltage  that  could  be  caused due  to  braking.  Furthermore,  during
    operation, the net discharge events  occur more frequently than regenerative
    braking events.

\end{description}

For  both the  \gls{p2d} model  and the  \gls{spm} model,  the cell  temperature
is   kept   constant   at   its   initial   value   of   \SI{25}{\degreeCelsius}
(\SI{298.15}{\kelvin}). This implies that the  operation of the lithium ion cell
is assumed  to be isothermal.  While this  is not true  in general, it  is worth
noting that thermal gradients in  the through-thickness direction is negligible.
\fxnote{citation needed} Furthermore, for  the C-rates considered (<5C), typical
in a \gls{bev} application and  for short-duration transient loads studied, this
is a reasonable assumption. Detailed modelling of thermal dynamics is not within
the scope of this thesis, as the primary goal is to obtain a physics based model
incorporating  electrochemical  principles  amenable for  embedded  application.
Thus, thermal dependence of parameters through an Arrhenius-type relationship is
not  considered.    Future   work  could   include  performing   thermally  coupled
simulations, incorporating  thermally relevant  parameters and use  a simplified
heat generation expression,  \eg{} a lumped thermal model in  both the \gls{spm}
and \gls{p2d} and compare their performances.

To  understand  the parametrisation  of  initial  concentration, the  expression
for   electrolyte   conductivity   needs    to   be   examined.   As   discussed
in~\cref{subsec:spmp2dparametrisation}, the intrinsic conductivity of a specific
type  of  electrolyte  is  a  material   property  that  depends  on  the  local
concentration of \ch{Li^+} ions and  temperature. In the characterisation of the
electrolyte in Valøen and  Reimers~\cite{Valoen2005}, the polynomial expression
in~\cref{eq:kappavsCeandT} was obtained for the electrolyte conductivity.
\begin{multline}\label{eq:kappavsCeandT}
    \kappa_j =  10^{-4} c_\text{e}(x,t) \bigl(-10.5 + \num{0.668e-3} c_\text{e}(x,t) + \num{0.494e-6}  c_\text{e}(x,t)^2\\
        + (0.074 - \num{1.78e-5}) c_\text{e}(x,t) - \num{8.86e-10}
    c_\text{e}(x,t)^2 \bigr)T_\text{cell}\\
	+ \left(\num{-6.96e-5} + \num{2.8e-8} c_\text{e}(x,t))T_\text{cell}^2\right)^2
\end{multline}

For  the  isothermal  case,  $T(t)  =  T_\text{cell}$.  At  equilibrium  initial
condition,  the   electrolyte  concentration   is  uniform  over   space.
Hence,~\cref{eq:kappavsCeandT} reduces to
\begin{multline}\label{eq:kappavsCeinitandTcell}
    \kappa_j =  10^{-4} c_\text{e,0} \bigl(-10.5 + \num{0.668e-3} c_\text{e,0} + \num{0.494e-6}  c_\text{e,0}^2\\
        + (0.074 - \num{1.78e-5}) c_\text{e,0} - \num{8.86e-10}
    c_\text{e,0}^2 \bigr)T_\text{cell}\\
	+ \left(\num{-6.96e-5} + \num{2.8e-8} c_\text{e,0})T_\text{cell}^2\right)^2
\end{multline}

\begin{figure}[h]
    \centering
    \includegraphics[width=\textwidth]{placeholder_images/example-image-golden.pdf}
    \caption[Surface plot of electrolyte conductivity]
    {Electrolyte conductivity as a function of cell temperature and initial
    concentration}
    \label{fig:kappavsCeandT}
\end{figure}

\Cref{fig:kappavsCeandT} shows a surface plot of the electrolyte conductivity as
a function of initial  concentration  $c_\text{e,0}$ and  cell
temperature $T_\text{cell}$.\fxnote{comment on how the variation in the
direction of temp is less significant to that around conc}. Since the cell's
temperature is fixed at \SI{298.15}{\kelvin}, this relationship can be better
visualised as a one-dimensional plot as shown in~\cref{fig:kappavsce}

\begin{figure}[h]
    \centering
    \includegraphics[width=\textwidth]{placeholder_images/example-image-golden.pdf}
    \caption[]
    {Electrolyte conductivity as a function of equilibrium concentration at
    $T_\text{cell} = \SI{298.15}{\kelvin}$}
    \label{fig:kappavsce}
\end{figure}

From~\cref{fig:kappavsce},  it  is  evident that  the  electrolyte  conductivity
attains its maximum value at $c_\text{e} = \SI{1000}{\mole\per\meter\cubed}$. It
is advantageous  to operate  the cell  around this salt  concentration so  as to
minimise  the  cell's  overall  resistance.  Hence,  the  initial  concentration
$c_\text{e,0}$ is  chosen to  be \SI{1000}{\mole\per\meter\cubed}. It  should be
noted that while  the electrolyte concentration in the  \gls{p2d} model exhibits
both spatial  and temporal  variations during the  simulation, in  the \gls{spm}
model, it remains constant throughout.

The      reduction      in      parametrisation      requirements      discussed
in~\cref{subsec:spmp2dparametrisation}    is   only    one   of    the   factors
contributing  to   the  simplicity   and  ease   of  simulation.   As  discussed
in~\cref{subsec:basicspmgeometry},   an   important  computational   requirement
that   is  present   in   the  \gls{p2d}   model,   but  completely   eliminated
from  the   \gls{spm}  is  the   requirement  of  discretisation.   As  reported
in~\cref{tbl:lcoSimParamsSPMp2d},  with  15 nodes  per  region  along the  axial
direction  and  with 10  shells  per  electrode  in  the radial  direction,  the
\gls{p2d}  model under  simulation  achieves mesh  independence  to a  tolerance
of  \approx   \SI{2}{\percent}  for  the   range  of  C-rates   considered.  For
higher  C-rates,  coupling  a  thermal   model  is  of  higher  importance  than
incorporating further meshing refinements. The discretisation-related parameters
are  specific to  the  \gls{p2d}  model and  is  hence, highlighted  accordingly
in~\cref{tbl:lcoSimParamsSPMp2d}.

With  the cell  parametrisation discussed  and the  simulation setup  presented,
the  simulation   results  are  fully   reproducible  and  are   presented  next
in~\cref{subsec:simresultsbasicspm}.

\subsection{Simulation Results}\label{subsec:simresultsbasicspm}

% electrolyte resistance

% % Table comparing simulation speeds for cts and disc for constant current and dynamic current

% \section{Enhancing the Conventional \glsfmtshort{spm}}\label{sec:electrolyteinclusion}
% % -*- root: ../main.tex -*-
%!TEX root = ../main.tex
% this file is called up by main.tex
% content in this file will be fed into the main document
% vim:textwidth=80 fo=cqt

As  evidenced by  the  results of  the constant  current  charge, discharge  and
dynamic  simulation runs,  the basic  \gls{spm} suffers  from a  \emph{critical}
drawback. The lack of electrolyte dynamics in the conventional \gls{spm} results
in  poor voltage  accuracy  even at  moderate C-rates.  This  renders the  model
unsuitable for  observer design in  \gls{soc} estimation applications  since the
output voltage from the model maps  to a radically different \gls{soc} operating
point. A number of candidate solutions have been proposed in literature in order
to mitigate this drawback. Their salient aspects are briefly evaluated here.

Rahimian~\etal{}~\cite{KhaleghiRahimian2013} discuss  the usage of  a polynomial
approximation  for   electrolyte  concentration  and  potentials.   However,  no
restriction was imposed on the order  of the polynomials chosen to represent the
electrolyte concentration within  each porous electrode region.  In the standard
\gls{dfn} model, the  number of equations and  corresponding boundary conditions
describing electrolyte charge and mass transport within the cell is insufficient
to uniquely solve for all  unknown coefficients of the polynomial approximation.
The  challenges  posed  due  to  this equation  deficiency  shall  be  discussed
in~\cref{temp:eqndeficiency}. Although  the original  \gls{spm} did  not involve
solving  for  the  electrolyte  concentrations  or  potentials,  The  polynomial
approximation of the single


Rahimian~\etal{} adopted  a cubic  polynomial for approximating  the electrolyte
concentration within  the porous electrodes.  To overcome the issue  of equation
deficiency, they adapted a scheme wherein one additional spatial location in the
interior  of  each  electrode  was  used. The  coefficients  of  the  polynomial
approximation  were then  obtained  by iteratively  solving  a relatively  large
coupled  system of  algebraic equations,  embedding within  them the  additional
equations evaluated at  the interior point. An additional  complicating issue is
the specific positioning of this  additional interior point. An online numerical
optimisation  was performed  to obtain  the optimal  placement of  this interior
node. Although it serves as a proof of concept towards implementing higher order
polynomial approximations,  the author of this  thesis deems this method  as too
complex for online implementations.


% A common  characteristic of  these proposals  is that  they seek  to incorporate
% electrolyte  dynamics  of  varying  degrees  of  complexity  directly  into  the
% \gls{spm}. But I did something different.  The high accuracy of the \gls{soc} in
% the basic  \gls{spm} leads to  the author's  hypothesis that if  the electrolyte
% concentration  can  be  solved  as an  independent  subsystem  and  incorporated
% into  the  terminal  voltage,  this  can lead  to  an  improved  \gls{spm}  with
% electrolyte  dynamics. Such  a decoupled  electrolyte inclusion  into the  basic
% \gls{spm}\fxnote{fix this sentence}



% \section{Quadratic Approximation of Ionic Spatial Concentration}\label{sec:quadraticapprox}
% % -*- root: ../main.tex -*-
%!TEX root = ../main.tex
% this file is called up by main.tex
% content in this file will be fed into the main document
% vim:textwidth=80 fo=cqt

In this  section, the  quadratic approximation  of ionic  spatial concentration,
that underpins the electrolyte model  in many improved \gls{spm} formulations is
presented. An analysis of  the weakness of this model is  performed based on the
results  from applying  this model.  Mitigation of  this critical  drawback lead
to  this  author's  decoupled spatio-temporal  electrolyte  concentration  model
structure which is presented next in~\cref{sec:newelectrolytemodel}.

\begin{figure}[!htb]
    \captionsetup{singlelinecheck=off}
    \centering
    \includegraphics{placeholder_images/example-image-golden.pdf}
    \caption[Co-ordinate systems for quadratic approximation of
    electrolyte concentration]{Schematic diagram of the electrochemical sandwich
        consisting of
        \begin{enumerate*}[label=\itshape\alph*\upshape)]
            \item negative electrode,
            \item separator, and
            \item positive electrode
        \end{enumerate*} depicting the co-ordinate system used in deriving the
        quadratic approximation profile. The global spatial co-ordinate is $x
        \in \{0,l_\text{tot}\}$, where $l_\text{tot} = l_\text{neg} +
        l_\text{sep} + l_\text{pos}$. Local co-ordinate systems specific to each
        region are also defined. It should be noted that the positive
        electrode's local co-ordinate axis direction is reversed.}
    \label{fig:coordsquadapprox}
\end{figure}

The  schematic  in~\cref{fig:coordsquadapprox}  shows   the  definition  of  the
co-ordinate  systems  used  in  deriving the  polynomial  approximation  of  the
electrolyte concentration  profile. The globally defined  $x$ co-ordinate starts
at  the negative  current  collector  interface ($x=0$)  and  terminates at  the
positive  current  collector  interface  ($x =  l_\text{tot},\,  l_\text{tot}  =
l_\text{neg} +  l_\text{sep} +  l_\text{pos}$). Three local  co-ordinate systems
$z_\mu$  valid  only  within  their  respective regions  are  also  defined.  In
particular, it  must be  noted that  the direction  of the  local $z_\text{pos}$
co-ordinate axis is opposite to that of  the other two local co-ordinate axes as
well as the global co-ordinate axis. In subsequent usages, the suffix in $z_\mu$
is dropped and  the reader is advised  to infer the region of  validity from the
usage  context  which are  unambiguous  as  they  occur in  separate  equations.
Furthermore, the  notation of  the three regions  $\{\text{neg, sep,  pos}\}$ is
abbreviated  to $\{n,s,p\}$  respectively in  all mathematical  expressions. The
author  is convinced  that this  notation does  not detract  from following  the
derivations, but rather aids it by keeping the notations compact.

A  standard  quadratic expression  is  chosen  a  priori for  approximating  the
electrolyte concentration profile within each region.
\begin{alignat}{2}
    c_\ensub &= a_2(t) z^2 + a_1(t) z + a_0(t),\quad &&0 \le z \le l_\text{n}\label{eq:cenquadstart} \\
    c_\essub &= a_5(t) z^2 + a_4(t) z + a_3(t),\quad &&0 \le z \le l_\text{s}\label{eq:cesquadstart} \\
    c_\epsub &= a_8(t) z^2 + a_7(t) z + a_6(t),\quad &&0 \le z \le l_\text{p}\label{eq:cepquadstart}
\end{alignat}
where  the  coefficient  vector  $\vect{a_0(t),a_1(t),  \dots  ,a_8(t)}$  is  to
be   determined   at   each   time-step\footnote{For   brevity,   in   rest   of
the   equations,   the   time-dependence   is   dropped   from   the   notation.
However,  it   must  be   implicitly  understood   that  all   coefficients  are
indeed  time-varying.}.   Applying  boundary   conditions  of   the  electrolyte
diffusion  equation  of   the  \gls{dfn}  model~(refer  \cref{eq:dfnliquiddiff})
to~\crefrange{eq:cenquadstart}{eq:cepquadstart}, it is clear  that $a_1 = 0$ and
$a_7 = 0$. Thus, ~\crefrange{eq:cenquadstart}{eq:cepquadstart} become
\begin{alignat}{2}
    c_\ensub &= a_2(t) z^2 + a_0(t),\quad &&0 \le z \le l_\text{n}\label{eq:cenquadreduced} \\
    c_\essub &= a_5(t) z^2 + a_4(t) z + a_3(t),\quad &&0 \le z \le l_\text{s}\label{eq:cesquadreduced} \\
    c_\epsub &= a_8(t) z^2 + a_6(t),\quad &&0 \le z \le l_\text{p}\label{eq:cepquadreduced}
\end{alignat}

% -*- root: ../../main.tex -*-
%!TEX root = ../../main.tex
% this file is called up by main.tex
% content in this file will be fed into the main document
% vim:nospell textwidth=180 foldlevelstart=3 foldlevel=3 conceallevel=0

\begin{table}[!htbp]
    \centering
    \caption[Electrolyte equations \& boundary conditions of \glsfmtshort{dfn} model in separator]{Electrolyte-specific governing equations and boundary conditions of the \glsfmtlong{dfn}~(\glsfmtshort{dfn}) model within the separator domain.}
    \label{tbl:dfnelectrolyteeqnsinsep}
    \begingroup
    \makeatletter\def\f@size{9.25}\check@mathfonts
    \addtolength{\jot}{0.875em}
    \begin{tabular*}{\textwidth}{@{} l c r l r @{}}
        \toprule
        \multicolumn{1}{c}{\small Region} & \small Governing equations & \multicolumn{2}{c}{\small Boundary conditions } & {} \\
        {} & {} & \multicolumn{2}{c}{\scriptsize $(l_\text{neg} \coloneqq l_\text{n},\, l_\text{sep} \coloneqq l_\text{s},\, l_\text{pos} \coloneq l_\text{p})$} \\
        \midrule
        \multicolumn{1}{l |}{{\rotatebox[origin=c]{90}{\makecell{\footnotesize Separator\\ \scriptsize $\delta \in \{\text{sep}\}$}}}} &
        $\begin{aligned}
            \vphantom{D_{\text{\tiny eff}_\text{n}}\!\! \! \!\, \diffp{c_\text{e}}{x}{\mathrlap{x = l^{-}_\text{n}}}} \varepsilon_\delta \diffp{c_\text{e}}{t} &=D_\effdelta  \diffp[2]{c_\text{e}}{x} \\[-0.75em]
            \vphantom{D_{\text{\tiny eff}_\text{s}}\!\! \! \!\, \diffp{c_\text{e}}{x}{\mathrlap{x=(l_{\text{n}} + l_\text{s})^{-}}}}\\[1.25em]
            \vphantom{\kappa_{\text{\tiny eff}_\text{n}}\!\! \! \!\, \diffp{c_\text{e}}{x}{\mathrlap{x = l^{-}_\text{n}}}\hspace{5mm} =\kappa_{\text{\tiny eff}_\text{s}}\!\!\!\!\,\diffp{c_\text{e}}{x}{\mathrlap{x = l^{+}_\text{n}}}} \frac{I}{A} &= \overline{\kappa}_\effdelta \left( \diffp[2]{\phi_\text{e}}{x} + \frac{2 R T}{F} (t^0_{+}-1)\diffp[2]{ \ln c_\text{e}}{x}\right) \\[-0.75em]
            \vphantom{\kappa_{\text{\tiny eff}_\text{s}}\!\! \! \!\, \diffp{c_\text{e}}{x}{\mathrlap{x=(l_{\text{n}} + l_\text{s})^{-}}}} \\
        \end{aligned}$ &
        $\begin{aligned}
    \vphantom{D_{\text{\tiny eff}_\text{n}}\!\! \! \!\, \diffp{c_\text{e}}{x}{\mathrlap{x = l^{-}_\text{n}}}} \qquad c_\text{e}\Bigr\rvert_{\mathrlap{x=l^{-}_\text{n}}}\hspace{5mm} &= c_\text{e}\Bigr\rvert_{\mathrlap{x=l^{+}_\text{n}}},\\[-0.75em]
     \vphantom{\kappa_{\text{\tiny eff}_\text{n}}\!\! \! \!\, \diffp{c_\text{e}}{x}{\mathrlap{x = l^{-}_\text{n}}}\hspace{5mm} =\kappa_{\text{\tiny eff}_\text{s}}\!\!\!\!\,\diffp{c_\text{e}}{x}{\mathrlap{x = l^{+}_\text{n}}}} c_\text{e}\Bigr\rvert_{\mathrlap{x=(l_{\text{n}} + l_\text{s})^{-}}}\hspace{5mm} &= c_\text{e}\Bigr\rvert_{\mathrlap{x=(l_{\text{n}} + l_\text{s})^{+}}},\\[1.25em]
 \vphantom{\kappa_{\text{\tiny eff}_\text{n}}\!\! \! \!\, \diffp{c_\text{e}}{x}{\mathrlap{x = l^{-}_\text{n}}}} \vphantom{\left( \diffp[2]{\phi_\text{e}}{x} + \frac{2 R T}{F} (t^0_{+}-1)\diffp[2]{ \ln c_\text{e}}{x}\right)} \phi_\text{e}\Bigr\rvert_{\mathrlap{x=l^{-}_\text{n}}}\hspace{5mm} &= \phi_\text{e}\Bigr\rvert_{\mathrlap{x=l^{+}_\text{n}}},\\[-0.75em]
 \vphantom{\kappa_{\text{\tiny eff}_\text{s}}\!\! \! \!\, \diffp{c_\text{e}}{x}{\mathrlap{x=(l_{\text{n}} + l_\text{s})^{-}}}} \phi_\text{e}\Bigr\rvert_{\mathrlap{x=(l_{\text{n}} + l_\text{s})^{-}}}\hspace{5mm} &= \phi_\text{e}\Bigr\rvert_{\mathrlap{x=(l_{\text{n}} + l_\text{s})^{-}}},\\
    \end{aligned}$ &
    $\begin{aligned}
        \quad D_{\text{\tiny eff}_\text{n}}\!\! \! \!\, \diffp{c_\text{e}}{x}{\mathrlap{x = l^{-}_\text{n}}}\hspace{5mm} &=D_{\text{\tiny eff}_\text{s}}\!\!\!\!\,\diffp{c_\text{e}}{x}{\mathrlap{x = l^{+}_\text{n}}}\\[-0.75em]
        D_{\text{\tiny eff}_\text{s}}\!\! \! \!\, \diffp{c_\text{e}}{x}{\mathrlap{x=(l_{\text{n}} + l_\text{s})^{-}}}\hspace{5mm} &=D_{\text{\tiny eff}_\text{p}}\!\!\!\!\,\diffp{c_\text{e}}{x}{\mathrlap{x=(l_{\text{n}} + l_\text{s})^{+}}}\\[1.25em]
        \vphantom{\left( \diffp[2]{\phi_\text{e}}{x} + \frac{2 R T}{F} (t^0_{+}-1)\diffp[2]{ \ln c_\text{e}}{x}\right)} \kappa_{\text{\tiny eff}_\text{n}}\!\! \! \!\, \diffp{c_\text{e}}{x}{\mathrlap{x = l^{-}_\text{n}}}\hspace{5mm} &=\kappa_{\text{\tiny eff}_\text{s}}\!\!\!\!\,\diffp{c_\text{e}}{x}{\mathrlap{x = l^{+}_\text{n}}}\\[-0.75em]
        \kappa_{\text{\tiny eff}_\text{s}}\!\! \! \!\, \diffp{c_\text{e}}{x}{\mathrlap{x=(l_{\text{n}} + l_\text{s})^{-}}}\hspace{5mm} &=\kappa_{\text{\tiny eff}_\text{p}}\!\!\!\!\,\diffp{c_\text{e}}{x}{\mathrlap{x=(l_{\text{n}} + l_\text{s})^{+}}}\\
    \end{aligned}$ &
    $\begin{aligned}
        \vphantom{D_{\text{\tiny eff}_\text{n}}\!\! \! \!\, \diffp{c_\text{e}}{x}{\mathrlap{x = l^{-}_\text{n}}}} \quad \refstepcounter{equation}(\theequation)\label{eq:liquiddiffnsep} \\[-0.75em]
        \vphantom{D_{\text{\tiny eff}_\text{s}}\!\! \! \!\, \diffp{c_\text{e}}{x}{\mathrlap{x=(l_{\text{n}} + l_\text{s})^{-}}}}\\[1.25em]
        \vphantom{\kappa_{\text{\tiny eff}_\text{n}}\!\! \! \!\, \diffp{c_\text{e}}{x}{\mathrlap{x = l^{-}_\text{n}}}} \vphantom{\left( \diffp[2]{\phi_\text{e}}{x} + \frac{2 R T}{F} (t^0_{+}-1)\diffp[2]{ \ln c_\text{e}}{x}\right)} \refstepcounter{equation}(\theequation) \label{eq:liquidpotentialsep}\\[-0.75em]
        \vphantom{\kappa_{\text{\tiny eff}_\text{s}}\!\! \! \!\, \diffp{c_\text{e}}{x}{\mathrlap{x=(l_{\text{n}} + l_\text{s})^{-}}}}
    \end{aligned}$
    \\
    \bottomrule
\end{tabular*}
\endgroup
\end{table}



\Cref{tbl:dfnelectrolyteeqnsinsep}    lists   the    equations   and    boundary
conditions   for  phenomena   describing   electrolyte   diffusion  and   charge
balance  within  the   separator  domain.  Essentially,~\cref{eq:liquiddiffnsep}
and~\cref{eq:liquidpotentialsep}  are  obtained  by applying  the  corresponding
electrolyte  equations  of  the \gls{dfn}  model  (see  ~\cref{eq:dfnliquiddiff}
and~\cref{eq:dfnliquidpotential}) respectively to the separator region.


% \section{A New Electrolyte Model through System Identification}\label{sec:newelectrolytemodel}
% % -*- root: ../main.tex -*-
%!TEX root = ../main.tex
% this file is called up by main.tex
% content in this file will be fed into the main document
% vim:textwidth=80 fo=cqt

Having performed a  comprehensive analysis of the state of  the art in \gls{spm}
modelling, this section presents the  author's unique contribution to the field.
Firstly, the  scope of the  contribution is identified. The  methodology adopted
and corresponding results are presented thereafter.

\subsection{Scope and motivation}\label{subsec:scopenewelectrolyte}

This subsection is intended as a capstone summary helping to briefly recount the
discussion so far  and to provide a  context for the author's work  in the wider
realm  of the  \gls{spm}  modelling art.  In  the same  vein  as the  discussion
in~\cref{sec:electrolyteinclusion}, the scope of the proposed enhancement to the
\gls{spm} concerns entirely  with improving the electrolyte subsystem  as it has
already been established in~\cref{subsec:simresultsbasicspm} that the simplified
representation of  the solid-phase subsystem  through a fourth  order polynomial
approximation method  for diffusion of  \ch{Li^0} in  the solid particle,  is of
sufficiently high accuracy.

Inspecting the  electrolyte domain,  the electrolyte  overpotential contribution
to  terminal  voltage   consists  of  a  diffusion   overpotential  in  addition
to   the  time-dependent   ohmic  losses   that  originates   from  differential
concentration  gradients  that  is   indirectly  dependent  upon  concentration.
Hence,  accurate  determination  of   spatio-temporal  concentration  takes  the
centre  stage.  For   the  computation  of  overpotential   in  the  electrolyte
phase,~\cref{eq:electrolytepdwithce}  proposed  by  Prada~\etal~\cite{Prada2012}
may be used.

There exists a  subtle detail in the  use of~\cref{eq:electrolytepdwithce} which
is discussed here upfront before proceeding  ahead to the refined context of the
author's  work.  The  intrinsic  conductivity  of  electrolyte,  $\kappa$  is  a
function of  the ionic concentration  (refer~\cref{subsec:basicspmsimsetup}). If
the ionic  concentration at  the corresponding current  collectors are  used for
$\kappa_\text{neg}$  and  $\kappa_\text{pos}$, this  would  lead  to a  lopsided
computation of the overpotential in electrolyte. Furthermore, under this scheme,
the computation of electrolyte conductivity shall be rendered ambiguous since it
is unclear which  separator interface shall be chosen for  the separator's ionic
concentration. Although this  has not been discussed clearly  in literature, the
author  of this  thesis chose  to use  the mean  concentration within  each cell
region, defined as
\begin{equation}
    \mean{c}_{\text{e},j}(t) = \frac{1}{l_j}\int_0^{l_j} c_{\text{e}_j}(z,t)\, dz = \frac{Q_{\text{e}_j}(t)}{\varepsilon_j l_j}
\end{equation}
although other measures  of central tendency might be equally  valid. Hence, the
results of this section have the associated variability in them depending on how
the electrolyte concentration computations are  used in evaluating the intrinsic
conductivity of electrolyte.

As  the  ionic  concentration  has  both  a  direct  and  indirect  contribution
in~\cref{eq:electrolytepdwithce}, its spatio-temporal  computation is a critical
aspect. As discussed  in~\cref{sec:quadraticapprox}, the quadratic approximation
is a widely used spatio-temporal model for electrolyte concentration which makes
the best  use of available physical  constraints. As established in  the results
of~\cref{subsec:quadraticsimresultsanalysis}, while  the spatial  performance of
the quadratic approximation approach is acceptable, its time-domain performance,
particularly at the crucial current collector locations is mediocre at best.

Thus,   the  \emph{scope}   of  the   author's  work   is  to   obtain  suitable
alternate   expressions   for   improving  the   computation   of   \textbf{time
evolution}  of  the electrolyte  concentration  whilst  retaining the  quadratic
approximation   approach   for  describing   its   spatial   profile.  Such   an
approach  is motivated  by  the  keen observation  that  the baseline  quadratic
approximation  model  has  a  natural  `pause'  in  its  model  description.  To
clarify,~\crefrange{eq:cecontinuitynegsep}{eq:Qepbyintegration}  form a  tightly
coupled set of  seven linear equations in seven unknowns.  The time evolution of
$Q_{\text{e}_j}$  are described  through  a system  of  first order  \glspl{ode}
given by~\crefrange{eq:negliionmolesquadratic}{eq:posliionmolesquadratic}.  In a
practical  implementation,  these  \glspl{ode}  are solved  independently  in  a
decoupled  manner,\ie{}  by using  the  coefficients  obtained from  the  linear
system of~\crefrange{eq:cecontinuitynegsep}{eq:Qepbyintegration} in the previous
time-step. The  author's hypothesis is that  by taking advantage of  the natural
break  in the  operational  sequence which  involves  two separate  computations
between  two independent  subsystems (for  all practical  purposes), it  must be
possible  to  replace  the  underperforming time-evolution  equations  from  the
baseline quadratic approximation with a superior alternate model.

\subsection{Methodology --- System identification}

This section presents the methodology adopted in obtaining an improved model for
the rate of evolution  of overall moles per unit area of  \ch{Li^+} ions in each
of the three regions of the cell.

\subsubsection*{Background}

This  section presents  the  background and  thought  process in  systematically
arriving  at  the  choice of  the  methodology  that  was  adopted for  the  new
time-evolution model of the electrolyte concentration.

Based  upon the  experience  gained  in dealing  with  the literature  presented
in~\cref{sec:electrolyteinclusion}, it is  the author's view that,  owing to the
complex behaviour of electrolyte, a  naive top-down approach \ie{} including all
the physics upfront  followed by a systematic simplification, might  result in a
model that is  mathematically intractable for adoption in  an embedded \gls{bms}
environment.  The baseline  quadratic  approximation method  has  proven that  a
bottom-up  approach, \ie{}  pre-assuming a  simplified structure  for the  final
model  and adapting  its coefficients  to physical  constraints yields  a viable
candidate for inclusion in the conventional \gls{spm}.

Upon  a   closer  examination   of  the  rubrics   of  the   baseline  quadratic
approximation  model, it  comes  to  light that  the  natural `pause'  discussed
towards  the  end  of~\cref{subsec:scopenewelectrolyte}  permeates  to  a  level
more  than  merely  having  to  operate  sequentially  on  two  pseudo-decoupled
subsystems  --- it  goes to  the extent  of rendering  the operating  philosophy
of  fitting  physical  equations  semi-void.   To  clarify  this  statement  and
to  substantiate   the  claim,  while  there   is  no  doubt  that   the  linear
algebraic   equations  of~\crefrange{eq:cecontinuitynegsep}{eq:Qepbyintegration}
do    incorporate    physical    principles   from    the    \gls{dfn}    model,
the     same     does     not      hold     true     for     the     \glspl{ode}
of~\crefrange{eq:negliionmolesquadratic}{eq:posliionmolesquadratic}.   In  fact,
all   the   boundary   conditions   from   the   \gls{dfn}   model   have   been
exhausted   by  this   stage  (refer~\cref{subsec:quadraticsimresultsanalysis}).
Although~\crefrange{eq:negliionmolesgen}{eq:posliionmolesgen}     are    derived
from     the      \gls{dfn}     model,      the     coefficients      of     the
diffusivities   in    the   \gls{rhs}   of    the   next   set    of   equations
\ie{}~\crefrange{eq:negliionmolesquadratic}{eq:posliionmolesquadratic},   merely
involve  substitutions  of the  spatial  derivatives  of the  assumed  quadratic
expression.

Herein lies the weakness of the  baseline quadratic approach. Unlike the spatial
algebraic equations, which are tightly bound by the continuity and flux boundary
conditions at the  separator interfaces, there is no equality  constraint on the
spatial  derivative, which  is free  to grow  or shrink  without any  explicitly
imposed  bounds.  The  onus  of  being accurate  is  therefore  on  the  spatial
derivative evaluation which in-turn depends  on the correctness of the quadratic
functions  (\crefrange{eq:cenquadreduced}{eq:cepquadreduced}) themselves.  It is
not feasible to quantify the magnitude of error introduced in the time-evolution
of concentration  given a small-signal  perturbation in the coefficients  of the
quadratic spatial  computation,\ie{} the implicit  coupling between them  is not
transparent. Since the quadratic approximation itself is not perfect, \ie{} does
not capture the  spatial gradient \emph{exactly} as the \gls{p2d}  model as seen
in~\cref{fig:spatialionicconc1C}, the internal coupling of coefficients leads to
errors in time-evolution computation.

The author's  approach is to  therefore break this detrimental  coupling between
spatial  derivative of  concentration  and its  temporal evolution  counterpart.
Inspired by the fact that the  quadratic approximation model had almost achieved
the desired goals with
\begin{enumerate}[label=\emph{\alph*})]
    \item a bottoms-up approach, \ie{} assuming some model structure apriori, and
    \item not bound by any physical considerations due to the exhaustion of governing equations
\end{enumerate}
have led  the author  to broach  a modelling concept  that exhibit  these common
traits,  yet  of  a  completely  different nature  and  hitherto  unexplored  in
physics-based  battery  modelling  in   general  and  electrolyte  modelling  in
particular --- \emph{black-box system identification}.

\subsubsection*{Brief introduction to system identification}\label{subsubsec:introsysid}

An in-depth  coverage of the topic  of system identification is  well beyond the
scope of  this thesis.  However, keeping  in mind the  interests of  the battery
modelling community  who might not be  familiar with this subject  area, a brief
overview of the core ideas that  are essential for tackling the specific problem
at hand, is presented. For readers  further interested in this topic, the author
suggests the textbook by  Ljung~\cite{Ljung1999} for a comprehensive theoretical
treatment of the foundation topics in system identification.

System identification aims  to provide a mathematical model  of the input-output
mapping  of  a system\footnote{The  precise  definition  of what  constitutes  a
`system' is detailed  in Ljung's textbook. However, for  all practical purposes,
in this  thesis the word `system'  stands for any unknown  entity whose terminal
behavioural model  is being  sought for ---  primarily from  input-output data.}
under consideration. The three categories of system identification are:


\begin{enumdescriptnum}[leftmargin=!,itemsep=1ex,labelwidth=\widthof{$\symbf{\text{brugg}_j}\ \scriptstyle (\times 3)$abc}
    ,partopsep=0pt
    ,topsep=0pt
    ]

\item[White box] wherein underlying physical equations are completely known. The
numerical value  of coefficients of  governing equations are to  be parametrised
from input-output data.

\item[Black box]  wherein no  governing equations are  available for  the system
under  consideration. The  model formulation  is facilitated  by a  rich set  of
system theory  which proceeds by exciting  the system with input  waveforms with
certain desirable  properties and correlating characteristics  from the response
in order  to draw  conclusions about viable  mathematical structures  capable of
emulating  the  terminal  behaviour  of the  system  under  generalised  inputs.
Black  box system  identification was  employed for  the specific  problem under
consideration and hence all future descriptions will pertain to this class.

\item[Grey box] is a hybrid of the  two approaches wherein a part of the model's
governing  physics is  known a  priori, \eg{}  the structure  of a  well-defined
subsystem  that is  part  of a  large,  complex  system may  be  known ahead  of
time,  where the  task  is  to characterise  the  full  system. Grey-box  system
identification tasks can  often be reduced to a single  sub-problem of black-box
system  identification  by  removal  of  the known  physics  and  tackling  them
separately.

\end{enumdescriptnum}

\subsubsection*{Overview of black-box system identification}
Black-box system identification techniques can be broadly classified into ---
\begin{enumerate*}[label=\emph{\alph*})]
     \item non-parametric methods, and
     \item parametric methods.
 \end{enumerate*}

 Non-parametric methods do not seek a pre-assumed mathematical structure for the
 system. They  aim to directly  estimate the  very kernel of  what characterises
 every system \viz{} the Markov parameters  in the time-domain and the \gls{frf}
 in  the frequency  domain,  thereby requiring  \emph{infinite}  number of  data
 points for their representation. The major non-parametric system identification
 methods include:
 \begin{itemize}[topsep=0pt]
     \item Identification in time domain
         \begin{itemize}

             \item Direct  estimation of the system's  Markov parameters through
             statistical correlation of its response to an unit-pulse input

         \end{itemize}
     \item Identification in frequency domain,  \ie{} of the \gls{frf}
         \begin{enumerate}

             \item   Direct   estimation    through   input-output   statistical
             cross-correlation

             \item  \gls{etfe} using \glspl{dft} of input and output sequences

             \item  Smoothed periodogram estimates using Welch's method

             \item  Blackman-Tukey  estimation   method  using  standard  filter
             windows (such as Hamming, Hanning, Bartlett, Boxcar etc.)

         \end{enumerate}
 \end{itemize}

 % It is important to recognize that in model str
 % in the absence of physical equations describing the phenomena, some equation
 % structure is assumed.




