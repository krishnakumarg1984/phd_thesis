% -*- root: ../main.tex -*-
%!TEX root = ../main.tex
% this file is called up by main.tex
% content in this file will be fed into the main document
% vim:textwidth=80 fo=cqt

% As  discussed in  \cref{sec:classificationscheme},\fxnote{may need  to cross-ref
% the relevant subsection}

Reducing  the   number  of   computational  dimensions   in  a   physical  model
helps  in  formulating  their  low order  approximations,  thereby  facilitating
fast   computations.   The  \gls{spm},   originally   used   in  modelling   the
Metal-Hydride    chemistry~\cite{Haran1998}   and    later   on    adapted   for
Li-ion   cells~\cite{Ning2004},  represents   the  canonical   apogee  of   such
dimension-reduction strategies.


During the initial  years following its inception, the formulation  of the basic
\gls{spm} has  been discussed  extensively within  application-specific contexts
such  as  \gls{soc}  estimation~\cite{Santhanagopalan2006a,Santhanagopalan2008},
parameter   estimation~\cite{Santhanagopalan2007},   life   cycle   and   ageing
predictions~\cite{Santhanagopalan2008a,Safari2009}.   There   have   also   been
detailed    stand-alone    presentations    of    various    facets    of    the
basic   \gls{spm},   such   as    its   inherent   assumptions   and   governing
equations~\cite{Santhanagopalan2006,Chaturvedi2010}. The basic \gls{spm} suffers
from  certain major  drawbacks  which  are discussed  in  the results  presented
in \cref{subsec:simresultsbasicspm}. Since  the turn of the  decade, researchers
have attempted to tackle many of these  issues and such efforts are discussed in
\cref{sec:electrolyteinclusion}.


A  survey of  the  most recent  literature  in all  \gls{spm}  family of  models
reveals  a  diminishing rate  of  advancement  in quantifiable  improvements  to
the  underlying  plant  model.  This   nearly-static  trend  can  be  attributed
to  the  general consensus  within  the  research  community that  these  models
may  be  too  simplistic  and  not  of  suitable  accuracy  to  warrant  further
studies.   Other  than   a  small   minority   of  papers   that  propose   core
modelling  improvements  to  tackle  their modelling  inaccuracies  or  add  new
enhancements  such as  mechanical-stress physics~\cite{Li2017a,Li2018b},  latest
work  in   this  family   of  models  predominantly   pertains  to   the  topics
of   state   estimation~\cite{Chaochun2018,Lin2017,Tran2017,Moura2017,Zou2016a},
optimal    charging~\cite{Perez2015},    cycling    performance~\cite{Maia2017},
conversion           to            equivalent           circuits~\cite{Li2017b},
parametrisation~\cite{Li2018,Rajabloo2017,Bizeray2017,Namor2017}, pack-balancing
studies~\cite{Docimo2014}   and  observer   design  for   joint  state-parameter
estimation~\cite{Ascencio2016}. The \gls{spm} approach  was also extended to the
case  of  composite  electrodes,  leading  to a  state  estimator  design  after
basic  observablity  analysis~\cite{Bartlett2015}.  Owing to  their  simplicity,
this  author believes  that  \glspl{spm}  hold the  highest  potential to  bring
a  physics  based  model  to  embedded \glspl{bms}.  With  this  goal  in  view,
this  thesis seeks  to  resurrect  interest in  this  field  by addressing  this
paucity in fundamental  model improvements. A proposed enhancement  to the basic
\gls{spm}  is the  main  contribution of  this chapter  and  shall be  discussed
in \cref{sec:newelectrolytemodel}.

