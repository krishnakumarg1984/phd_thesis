% -*- root: ../main.tex -*-
%!TEX root = ../main.tex
% this file is called up by main.tex
% content in this file will be fed into the main document
% vim:textwidth=80 fo=cqt

\fxnote{uncomment and edit the intro in the latex source file later on}

% Firstly, the contrasting nature of this modelling objective is presented.

% Next, the drawbacks of this family of models is discussed in detail.

% The state of the art implementation  for tackling these drawbacks is presented
% and their inadequacies are discussed.

Based  on the  comparison  of  the strengths  and  weaknesses  of the  modelling
families   in   the   literature  considered   (see~\cref{ch:littreview}),   the
overarching simplicity of the \gls{spm}  coupled with its immediate potential to
bring the  power of  physics-based prediction  to an  embedded environment  is a
strong  motivation to  pursue  an  in-depth exploration  of  its horizons.  This
chapter discusses the  performance of multiple \gls{spm}  variants, ranging from
the  basic to  the  sophisticated  from an  implementation  point  of view.  The
governing  equations  of the  conventional  \gls{spm}  is first  introduced  and
its  baseline  performance is  evaluated.  Next,  an  in-depth analysis  of  the
basic  \gls{spm}'s drawbacks  is  presented. Various  attempts  to mitigate  the
current challenges towards implementability  is presented and their inadequacies
discussed. The state of the art method in enhancing the performance of the basic
\gls{spm} through the means of introducing electrolyte dynamics is presented and
a  comprehensive analysis  of  its  strengths and  weaknesses  is performed.  In
continuation  of the  insights gained  from such  analyses, the  thesis author's
attempts  to  surpass the  performance  of  the  current pinnacle  in  modelling
art  in  the  form  of  a  new  electrolyte  concentration  model  is  presented
in~\cref{ch:newelectrolytemodel}.


% presented  in \ldots.  The electrolyte  concentration and  potential fixes  is
% presented in  \ldots. Finally,  results and discussion.  \fxnote{REWRITE after
% finishing up everything}


\fxnote{Write the chapter. Finally come back to summarize this}


% The following efforts/trials were done (failures)
% \begin{itemize}
%     \item first attempt
%     \item second attempt
% \end{itemize}
% The following successes were achieved.
% \begin{itemize}
%     \item first attempt
%     \item second attempt
% \end{itemize}


% At the  end of this  chapter, we have a  control oriented reduced  order battery
% model amenable for use in real-time applications for SOC, SOH etc.\ estimations.


