\fxnote{uncomment and edit the intro in the latex source file later on}

% Firstly, the contrasting nature of this modelling objective is presented.

% A suitable  family of models from  the broad category of  reduced-order models
% is  identified  as  a  promising  candidate  for  implementation  in  controls
% applications. Next,  the drawbacks of  this family  of models is  discussed in
% detail. The  state of the art  implementation for tackling these  drawbacks is
% presented and their inadequacies are discussed.

% Prima Facie, based  on a preliminary comparison of the  strengths and weaknesses
% of  the  modelling  families  in  the  literature  considered,  the  overarching
% simplicity of  \glspl{spm} and their immediate  potential to bring the  power of
% physics-based prediction  to an embedded  environment is a strong  motivation to
% pursue their in-depth exploration. The  equations describing the single particle
% model  is introduced  in  \ldots. An  in-depth analysis  of  their drawbacks  is
% presented in \ldots. Various attempts to  fix this issue is presented in \ldots.
% The  electrolyte  concentration and  potential  fixes  is presented  in  \ldots.
% Finally, results and discussion. \fxnote{REWRITE after finishing up everything}


\fxnote{Write the chapter. Finally come back to summarize this}


% The following efforts/trials were done (failures)
% \begin{itemize}
%     \item first attempt
%     \item second attempt
% \end{itemize}
% The following successes were achieved.
% \begin{itemize}
%     \item first attempt
%     \item second attempt
% \end{itemize}


% At the  end of this  chapter, we have a  control oriented reduced  order battery
% model amenable for use in real-time applications for SOC, SOH etc.\ estimations.


